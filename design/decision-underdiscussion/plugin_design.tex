%% This is a document discussing a proposed design for the VT plugin
%% that we need to execute and track VT's during design.
%%
%% This document is the main document: compile with xelatex to get the complete pdf.
%%

\documentclass[a4paper,11pt,final]{article}

\usepackage[english]{babel}
\usepackage{graphicx}
\usepackage{hyperref}
\hypersetup{ colorlinks = true, citecolor = blue,linkcolor = blue }
\usepackage[a4paper]{geometry}
\usepackage{titlesec}% added to change section headers, see newcommand definition.
\usepackage{xspace}%
\usepackage[obeyFinal]{todonotes}%

\bibliographystyle{alpha}

\author{ABI team 33\\(Stefan Versluys, Guus Bonnema, Jeroen Kleijn)}

\date{21/02/2015}

\title{\color{blue}VT plugin design proposal}

\newcommand{\w}[1]{\texttt{#1}\xspace}
\newcommand{\xmas}{\textsc{xmas}\xspace}%
\newcommand{\xmd}{\textsc{xmd}\xspace}%
\newcommand{\vt}{\textsc{vt}\xspace}%
\newcommand{\qt}{\textsc{Qt}\xspace}%
\newcommand{\qtquick}{\textsc{QtQuick}\xspace}%
\newcommand{\qml}{\textsc{Qt}\xspace}%
\newcommand{\cpp}{\textsc{C++}\xspace}%
\newcommand{\code}[1]{\texttt{#1}\xspace}%

\begin{document}
\selectlanguage{english}

\maketitle

%% Modification history

\begin{tabular}{|l|l|p{20em}|}
\hline 
\multicolumn{3}{|c|}{\bf Modification History}\\\hline
Guus Bonnema & 21-02-2015 & Initial proposal document \\\hline
\end{tabular} 


\maketitle

\tableofcontents

\section{Problem background and description}

\subsection{Why do we need a plugin?}

The \xmd subsystem was designed to be independent from the \vt subsystem. 
Still they do have to cooperate in order to show \vt output during 
graphical design of a network.

In order to have a loose coupling between the two subsystems the idea
is to have a plugin structure where the \vt tool designer is able to 
integrate a verification tool in a dynamic manner. 

\subsection{What should the plugin do?}

The plugin should run one or more of the verification tools and
redirect standard output to the graphical designer, that will
display the information in a console like fashion.

The plugin should also be able to return specific information
of any level (info, warning or error) to the graphical tool. This
could be information about one specific component, a channel, or it
could be information about a group of components and channels.

Implementation of the text warning and error interface is the minimum
requirement. 

\section{VT plugin function interface}

\subsection{What interfaces do we need?}

\begin{description}
	\item[text information]	Lines of information in a default color. This is the
	minimum interface requirement for tracking the running verification tool.

	\item[Info, warning, error]	Lines of information with a specified label. This is an
	extra and optional requirement for tracking the running verification tool.

	\item[Component or Channel] Lines of information concerning a component. The effect could
	translate into a color of the object on screen plus a line a text. This is a nice to have
	requirement for tracking the running verification tool.

\end{description}

\subsection{How to define the interfaces?}

Each interface is a class containing pure virtual methods. Each method is connected with
the user interface through signals and slots. The initiative for issuing a message originates
from the verification tool. The user interface should show these messages in text or in graphics.

The proposal for an interface will be methods with the following signature:

\begin{verbatim}
	virtual QString message() = 0;
	virtual QString info() = 0;
	virtual QString warning() = 0;
	virtual QString error() = 0;
	virtual std::pair<QString, QString> componentInfo() = 0;
	virtual std::pair<QString, QString> componentWarning() = 0;
	virtual std::pair<QString, QString> componentError() = 0;
\end{verbatim}

A plugin should at least implement message() and optionally the other methods.
The methods returning a \code{std::pair} concern a component or a
channel and could lead to both textual and graphical output on the
designer window.

The concept of definig a type of message like \code{componentWarning()} 
is easily extendable to different types of messages, like for instance
\code{componentIncomplete()} or \code{FunctionMissing()}.

An alternative to having all messages in one interface is to define
different types of interfaces so one can have multiple plugins and only
implement the interfaces one is interested in. For instance, one could
distinguish between a \code{ComponentInterface} and a \code{SyntaxInterface}.
Initially using just one interface \code{MessageInterface} seems the
wise thing to do. In time we could extend.

\subsection{Plugin definition}

The plugin is a \qt program that implements one or more interfaces. The
verification tool connects to the interface methods by calling them from the
code. 
\todo[inline]{How in heavens are we going to connect the plugin interface implementation to 
the user interface? Using signal and slot? Can we do that accross thread boundaries?}
\todo[inline]{Are the plugins running by thread or by processes? If so, how do we keep track
of each? And can we restart plugins to run repeatedly during an editing session?}

\subsection{Changes in the verification tool}

The verification tools are classes that we can either extend through subclasses  or 
use as-is. In both cases, the verification tools do not need any change to 
redirect standard output or standard error to the plugin. They do need change 
for component directed messages.

\todo[inline]{Find out if using poco will solve the \qt dependancy and still give us some
sort of ipc facility that is cross platform.}

\section{Qt Plugin architecture}

\subsection{General description}

The \qt system supports two forms of plugin, one that supports a \qt extension through plugin, and
the other that supports general extension of an application through a plugin. It is the second form
that we are discussing in this document. 

\subsection{Plugin structure}

An application plugin using \qt is based on defining an interface and a registration with the \qt
system using a \code{QObject} derived class. After registration the \qt system provides a class
\code{} that will load a program and determine whether it is a plugin, and if so, determine
whether it is the plugin we are looking for.

The class implementing the plugin has the following structure:

\begin{verbatim}
	class <type>Plugin : public QObject, <type>Interface
	{
	    Q_OBJECT
	    Q_PLUGIN_METADATA(IID "unique.prefix.<type>Interface" FILE "<type>plugin.json")
	    Q_INTERFACES(<type>Interface)
	
	public:
	    QString <method>(const QString &message) Q_DECL_OVERRIDE;
	};
\end{verbatim}

In this code one can replace \verb=<type>= by the type of interface we are creating. The 
example this code is started from is an echo server where the interface was called 
\code{EchoInterface}. The \verb=<method>= should be defined as pure virtual in the interface.
The interface should contain a virtual destructor and only pure virtual methods or the
moc-compiler will complain.

Following this structure one can implement a plugin loader. For example the following code was 
copied from the example of an echo plugin in the \code{QtCreator}:

\begin{verbatim}
	bool EchoWindow::loadPlugin()
	{
	    QDir pluginsDir(qApp->applicationDirPath());
	#if defined(Q_OS_WIN)
	    if (pluginsDir.dirName().toLower() == "debug" || pluginsDir.dirName().toLower() == "release")
	        pluginsDir.cdUp();
	#elif defined(Q_OS_MAC)
	    if (pluginsDir.dirName() == "MacOS") {
	        pluginsDir.cdUp();
	        pluginsDir.cdUp();
	        pluginsDir.cdUp();
	    }
	#endif
	    pluginsDir.cd("plugins");
	    foreach (QString fileName, pluginsDir.entryList(QDir::Files)) {
	        QPluginLoader pluginLoader(pluginsDir.absoluteFilePath(fileName));
	        QObject *plugin = pluginLoader.instance();
	        if (plugin) {
	            echoInterface = qobject_cast<EchoInterface *>(plugin);
	            if (echoInterface)
	                return true;
	        }
	    }
	
	    return false;
	}
\end{verbatim}

It first determines for a file whether it is a plugin and subsequently which interface
it implements.

\end{document}
