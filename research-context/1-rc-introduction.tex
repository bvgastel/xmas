\section*{Introduction}

\paragraph{Ad 1. Network-on-Chip research area}
 Network-on-Chips (NoCs) have emerged as the backbone for the inter-core communication of a chip-multiprocessor (CMP). (Power-aware performance increase via core/uncore reinforcement control for chip-multiprocessors (ACM))
 
Computer chips consist of core and uncore. uncore = communication fabrics and on chip cache. it is the NoC infrastructure that connect the cores.

\subparagraph{Research problem} 

errors -> money drain -> uncore tests -> test costs NoC -> significant part of total test costs -> desire to automate tests 

Much research effort on test from different angles.

on-chip communication fabric and shared Last-Level Caches (LLC), which we term uncore here (Up by their bootstraps: Online learning in Artificial Neural Networks for CMP uncore power management \verb^http://ieeexplore.ieee.org/xpl/login.jsp?tp=&arnumber=6225465&tag=1&url=http%3A%2F%2Fieeexplore.ieee.org%2Fxpls%2Fabs_all.jsp%3Farnumber%3D6225465%26tag%3D1)^

However, unlike other formalisms this approach leads to a precise yet intuitive graphical notation for microarchitecture that captures timing and functionality in sufficient detail to be useful for reasoning about correctness and for communicating microarchitectural ideas to RTL and circuit designers and validators.
(same paper)

Other emphasis (core-uncore performance optimization):
Power-aware performance increase via core/uncore reinforcement control for chip-multiprocessors

Pre-production test using test NoC implementation -> (NoC-Test.pdf)

\textit{Contribution of our project to research goals}

xmas: identify a richer set of microarchitectural primitives that allows us to describe complete systems by composition alone. (xMAS: Quick Formal Modeling of Communication Fabrics to Enable Verification)

Goal: prevent errors in a design stage long before pre production tests, excluding types of errors.
Goal: automate error detection during design, prevent test cost later on (when physical pre-production starts
Goal: minimize time to test -> test cost

\textit{What is the state of the art?}




Input Stefan:
{\small
\begin{lstlisting}
dingen die voor de research context in m'n hoofd komen zijn :

    een chip fabrikant wenst verschillende onderdelen op een chip 
    		(bvb 1 chip = goedkoper, energie zuiniger, 
    		plaatsbesparend en heeft een betere kwaliteit)

    een chip fabrikant wenst onderdelen van andere te hergebruiken 
    		(bvb spraakchip van fabrikant A geïntegreerd op chip 
    		van fabrikant B), time to market , 
    		financiele besparingen , core business

    verschillende onderdelen communiceren via een standaard interface 
    		& protocol (NoC) 
    		--> herbruikbaarheid en eenvoudigere integratie

    door deze vereenvoudiging neemt de mogelijkheid tot nog meer integratie toe 
    		alsook de complexiteit. Het risico op een ontwerpfout eveneens met 
    		ernstige imago schade voor de chip fabrikant tot gevolg

    Van zo'n netwerk, met dezelfde eigenschappen doorheen de onderdelen 
    		op de chip die, kan een model gemaakt worden. 
    		Intel --> XMAS primitieven
    		
    Met geavanceerde verificatie tools op zo'n model kunnen in een vroeg ontwerp 
    		stadium fouten voorkomen worden
    		
    Een kwalitatieve, multiplatform design tool voor het maken van zulke modellen 
    		stimuleert het gebruik en het nut er van.
    		
    Door de verificatie tools toegankelijk te maken vanuit de design tool met 
    		directe feedback zal het voorgaande nog versterkt worden

de meerwaarde van de nieuwe tools (xmd,vt,plugins,...

    multiplatform
    standaard interface voor de verificatie tools d.m.v. plugins
    1 bestandsformaat
    gui en vt integratie
    sources en ontwikkelomgeving op basis van de nieuwste technologiën
    documentatie 
    onderhoudbaarheid
    gebruiksgemak

op basis van een vluchtige brainstorm in m'n eentje, nu koffie en nieuws ;-)
\end{lstlisting}
}

