\documentclass[a4paper,11pt,twoside,draft]{article}

\usepackage{ucs}
\usepackage{subfigure}
\usepackage[dutch,english]{babel}
\usepackage{graphicx}

\usepackage[dvips,pdftex]{hyperref}
\hypersetup{%
   pdfcreator=%
}

\author{Guus Bonnema}
\date{24/09/14}

\title{Plan van aanpak voor xmas ontwerp tool}

\begin{document}

\section{Opdracht}

\subsection{Inleiding}

\paragraph{Factoren}

\begin{enumerate}
\item geografisch gespreid team
\item technisch product
\item high level functionele requirements bekend bij opdrachtgever.
\item Kwalitatieve requirements nog niet expliciet
\item 3 teamleden
\item Vaste tijd voor volledige uitvoering (8 maanden +/- 1 maand)
\item Weinig externe risico's
\end{enumerate}



\paragraph{Alternatieve aanpakken}

\begin{enumerate}
\item Volledig plangedreven zoals SDM2 (System Development Method 2)
\item Volledig agile zoals XP (Extreme programming)
\item Hybride aanpak zoals UP (Unified Process)
\item Scrum aanpak (additief aan een agile of hybride aanpak): iteratief
\end{enumerate}

\paragraph{Overwegingen}

\begin{enumerate}
\item Klein team met deels bekende requirements
\item geografische spreiding
\item teamleden hebben geen ervaring met agile methoden
\end{enumerate}

\paragraph{Conclusies}
\begin{description}
\item XP vergt fysieke nabijheid en grote gebruikers betrokkenheid: valt af
\item SDM2 heeft een rigide requirements Engineering process
\item UP: Iteratief met agile constructie komt het dichtst in de buurt:
\begin{enumerate}
 \item Plangedreven voorbereiding
 \item iteratieve sprints tijdens de uitvoering
\end{enumerate}
\end{description}

Scrum is het meest bruikbaar voor project management tijdens de constructie omdat
het gemakkelijk bij agile en bij incrementeel cq iteratief ontwikkelen te gebruiken is.

\subsection{Business case}

Wat maakt deze wijziging nodig? Wie helpt het? Hoe helpt het?
Hoe groot is de benefit?

\subsection{Risicos}

Wat zijn de belangrijkste risicos? Welke moeten we accepteren, en welke wat aan doen?


\subsection{Challanges}

Wat zijn de belangrijkst project uitdagingen? Als we dit goed formuleren, dan komen hier
requirements uit voort.

\subsection{Vision}

Wat zien we voor ons als het systeem klaar is?

\subsection{High level Requirements}

Wat zijn de high level requirements?

\subsection{Succesfactoren}
Wanneer is het project een succes?
Wanneer is het project mislukt?

\section{Architectuur}
Aan welke eisen moet de architectuur voldoen?
Op welke platformen draait het?
Is het multi user? Is het client/server? Is er een repository?
Is het taalonafhankelijk?

\section{Domeinanalyse}

Welke domeinanalyse doen we en wie doet wat? Hoeveel tijd kunnen er aan besteden?

\section{Aanpak}

\paragraph{Volgorde van fasen}

\begin{enumerate}
 \item Planning en aanpak
 \item Business case
 \item Risico analyse
 \item Vision en challanges
 \item Requirements
 \item Domeinanalyse
 \item Architectuur (incl GUI framework)
 \item Uitvoering (3 weekse iteraties)
 \begin{enumerate}
  \item a. iteratie 1 -- hoogste risico dempen
  \item b. iteratie 2 -- v1
  \item c. iteratie 3 -- v2
  \item d. iteratie 4 -- v3
  \item e. iteratie 5 -- v4
 \end{enumerate}
\item Afronding en presentatie
\end{enumerate}

\paragraph{De basis principes}
\begin{itemize}
 \item De uitvoering is agile en tijdgedreven.
 \item Elke iteratie heeft een vaste hoeveelheid tijd
 \item Elke iteratie begint met een analyse (requirements engineering, prioriteiten en selectie). Soms wijzigen requirements of komen er nieuwe bij.
 \item Elke iteratie eindigt met een evaluatie
 \item Dit moeten we valideren bij Bernard en Freek. Gebruikers betrokkenheid is
 beperkt tot het begin van de iteratie, de prioriteitsstelling van requirements, bijstelling van
 requirements, nieuwe requirements en uiteindelijk de selectie van wat we gaan bouwen.
\end{itemize}

\paragraph{Tijdsindeling}

\begin{description}
 \item[Fase 1] Het bepalen van de toolkit en ontwikkel omgeving. Duur 3 weken.
 \item[Fase 2-4] Business case, risico's, vision en challanges. Duur 4 weken.
 \item[Fase 5-7] Requirements engineering uitsluitend high level requirements. Duur 4 weken.
 \item[Fase 8] De uitvoering in 5 iteraties. Duur 5 x 3 weken is 15 weken
 \item[Fase 9] De afronding en de presentatie. Duur 4 weken.
\end{description}

Netto duur 27 weken, bruto 8 maanden 1 maand speling voor vakantie, ziekte en onvoorzien.

Project start 20 sept 2014, Eind 20 mei 2015. Neem de exacte datum met een korrel zout.


\subsection{Agile aanpak}
------------

\begin{itemize}
 \item Gebruiken git voor versiebeheer van documenten en software
 \item Passen TDD toe (binnen redelijke)
 \item Passen refactoring toe
 \item Elke iteratie:
 \begin{itemize}
   \item Analyse (requirements aanpassen, prioriteitsstelling, selectie)
   \item Ontwikkeling prototype
   \item Evaluatie
 \end{itemize}
\end{itemize}


\subsection{Open punten}

\begin{enumerate}
 \item Wat zijn de domeinanalyses precies? Welke onderwerpen? Hoe diepgaand? Hoeveel tijd kost het?
 \item Aan welke eisen moet het scriptieverslag voldoen?
 \item Aan welke eisen moet de presentatie voldoen?
 \item Hoe gaan we agile precies invullen?
 \item Hoe valt Freek in de iteratie tijdens de uitvoering? Welke rol speelt Freek precies tijdens de andere fases?
\end{enumerate}

\end{document}
