\chapter{Reflection}
\paragraph{Process} We started out largely unsure about how to continue. The first few months
planning we structured internal and external communication. Additionally, we decided to 
work in time boxes (iterations) where we successively worked toward a finished project. 
The internal communication was intense (three times a week, sometimes more) where one of our
members routinely worked more hours at the office in order to join our call. The division of work 
was partly pre-agreed upon and partly just happened.

Right after the first iteration we encountered the first problem: our mentor for the project 
and the customer both ceased to respond to our mails. Being December, we figured they were 
on an unscheduled holiday. About a month later (half of January) the customer explained 
what was happening. The mentor was sent off to the U.S. and our customer's workload had been 
increased to a level where he hardly had the chance to join us in a call. This was all due to 
a reorganization within the Open University. The OU was retiring approximately 10 - 30\% of their
personnel\footnote{When we commented on this during the midterm review, the OU people present,
did not feel the students needed to know. Apparently, our team was the only team impacted 
by the change. Still we wonder why such a large reorganization was not communicated broadly, 
even if most of the students should not notice anything, after all: it's a big change.}.
 
We realized we had to change our approach. Although initially we talked about rescheduling,
we decided not to. One of the advantages of agile working is the principle of the time-box. 
The consequence of the decrease of communication with the customer would be a decrease in speed
of deciding. We agreed with the customer to have at minimum one skype session per iteration. 
This we added to our planning. Informally, we could request a quick answer to urgent questions
by email.

In conclusion we finished our product more or less as we planned to. We had initially
hoped to have covered more ground, but in the circumstances we are quite content with what 
we produced. The customer and the mentor expressed contentment as well.

\paragraph{Product} The product turned out more or less as we envisioned at the start. The one
thing we might do differently in hindsight, is to spend a little more attention to the decision
of using a tight integration with the existing xmas programs that were developed before we started.

The issue was whether to use a text interface causing complete segregation of user interface and
xmas programs, or, to use a tight interface to the xmas programs. The customer chose the tight 
interface fearing the consequences of a second parser for maintenance. We chose to integrate
\qml and \cpp which probably cost about 2 iterations (6 weeks) extra in comparison to a text
interface. Although this is partly conjecture, if the team and the customer had known the extent 
of the consequences, we might have chosen differently.

\paragraph{Requirements}
%\begin{tcolorbox}[colback=white]
The product satisfies  the main requirements as we set out to meet. 
The benefits of the new xMAS designer toolkit are that
\begin{itemize}
\item \textbf{it is platform independent.}
\item \textbf{it's maintainability is improved.}
\item \textbf{it integrates verification tools through
			 a standard plug-in interface}
\end{itemize}
%\end{tcolorbox}

\paragraph{Project} When it comes to the end of a project like this, it is always a good practice to
step back and ask ourselves: did we meet the requirements of the customer? In the end we strive
to satisfy the customer in addition to our own ambitions.

\subparagraph{Platform independence} One of the main requirements was to have a platform independent xMAS
toolkit. The new xMAS toolkit was developed and tested in a multi platform
environment. It was exhaustively tested on Linux and Windows while MacOs was used to
try-out the releases\footnote{Bernard offered to take on the responsibility for the Macintosh 
because the team did not have a Macintosh machine. We owe him our gratitude for that.}.

\subparagraph{Maintenance} To meet the maintainability requirement we have chosen Qt as integrated
development environment. Qt is a powerful open source platform independent IDE.
The QML declarative UI language from Qt, enforces separation of view and control. The
Qt IDE is an all-in-one solution with a lot of support available. We chose \qt because
of its ease of use, speed and power of development. 

\subparagraph{Design} In retrospect and referring to the integration design decision 
in chapter~\ref{sec:integration}, we are strongly convinced that a loose coupling between 
designer and verification back-end would ease maintenance, increase development
speed and make development of verification tools more independent from the 
user interface.

Additionally, separating UI code from pure c++ model and verification code is 
a good idea because UI tools are much more exposed to changes, e.g. 
Qt version update. The tighter these are coupled the bigger the chance that 
modification ripple through the whole system crossing presentation / data model
boundaries.

Also the benefit of updating the back-end verification model with every canvas
action doesn't make sense. Why would a designer start a verification tool if the
canvas model is still under creation? This implies that it isn't necessary to
keep the back-end model constantly up to date with every canvas action. Once the
designer thinks the model is ready he or she can start a verification tool.
Details of this proposal can be found in report chapter~\ref{sec:integration}

\subparagraph{Functionality} Integration of formal verification tools was implemented 
through a standard plug-in interface. This way it will be possible to refactor or create 
verification tools independently and without the need of touching the designer source code. 
Once a verification tool has the right plug-in interface it can be used in the xMAS designer.

\paragraph{Guus} I felt very happy working this project with a very intense communication pattern,
a positively constructive team and supportive and helpful mentor and customer. I started out 
unfamiliar with the agile way of working and ended up liking the way it works. 
Especially the time box turned out very useful when events 
turned against us. In a planned environment, where functionality is fixed and time variable, I 
am convinced we would have had to reschedule our project. I am very glad we chose to go agile.

Learning \cpp was a real pleasure, although my work managing stuff took a large part of my time,
especially during planning and preparing for our meetings. Additionally, once we started programming,
all time I wanted to spend on learning \cpp evaporated. My connotation of \cpp has significantly improved
due to the working with live \cpp programs.

I also learned the impact of strictness in an agile project. Too much strictness stiffles initiative 
and creativity. However, some strictness is necessary to constantly know where we are and what we 
still need cover in order to succeed.

Agile and strictness is a troubled marriage at best. I am very glad to have learned this valuable lesson and
I am grateful to my team mates who endured my overload of control and finally corrected me when I was 
enforcing too much.

\paragraph{Jeroen}
The xMAS-project has been an educational experience to me. Of the three competences
that I formulated at the start of the project, I've been able to develop most to a
satisfying degree. The primary reason to opt for this project was the programming
language C++. Although I was already familiar with this language, experience with it
was limited to small spare time projects. This project provided the chance to
enhance my skills of C++ in a larger and more complex code base. C++ is a language
that tries to achieve a compromise between low-level, performance critical code while at
the same it provides programming constructs of high enough level to ease programming.
The latest revisions of the standard (C++11/14) are a big improvement with regards
to the latter goal. While writing code for this project, I could apply quite a few
of the new features provided by the newer standards, including template programming
and the more value driven programming style.

\paragraph{TDD}
Another goal I wanted to achieve was the application of Test Driven Development.
Several courses in the bachelor curriculum address the use of test code in development
and in a few assignments writing test code has been practised. At my day time job,
I've been trying to integrate test code in the development process but to date
this has not been very successful.

Using TDD has been a mixed success. The technique worked pretty well to write code
to serialize data to a JSON file\footnote{This code has not been used in the
final product but it was useful as a programming exercise}. The JSON file format
is well documented and as such it was easy to generate test cases. Both cases that
should succeed as well as those that should fail can be derived from the specification.

Applying TDD to implement the flattener algorithm proved to be more challenging.
The algorithm was not defined in advance. Multiple approaches were tried out
before the final algorithm was reached. Therefore, writing test code first was
hard because the expected behaviour was not known at that time. As such, I could not
resist the urge to write (parts of) the algorithms implementation before the test
code. The main goal, however, was actually using test code as part of the development
process, whether using TDD or not. I was able to reach that goal. The test code that
was shipped as part of the verification tools has been helpful as an example.

\paragraph{Member of a project team}
The third goal was to experience working as a member of a project team in a more
formal setting. At the office, I work as one of a team of two programmers. We both
have our own areas of expertise with just a small overlap. Therefore, I'm used
to work mainly independently and on my own code. Furthermore, communication about
requirements, etc. involves only colleagues and is informal and with few, if any, artifacts
or documents.

Initially, we planned to follow the agile way with strict fixed meetings with
the customer. Unfortunately, the customer wasn't able to spend as much time
on our project as initially envisioned. As such, I wasn't able to fully experience
the agile way to develop software.

Using an agile software process was a good idea, especially considering the limited
time available to the project. Fixed iterations and duration of the project guarantee
that the project is finished in time. However, it also leaves the impression that
we've created a somewhat unfinished product. Some functionality is 'almost there'
but it would have required another iteration to complete.

\paragraph{Teamwork}
Working with my teammates did turn out to work well. Although large distances
made it infeasible to arrange face-to-face meetings, frequent communication
using Skype and Teamviewer was a good substitute. Contrary to my daily job, I
didn't master all code of the project. That is, of some parts of the code, I
couldn't immediately figure out how to make adjustments. At these moments,
cooperation using software like Teamviewer (or face-to-face meetings) was
of great value. While one team member was able to guide the others through
code he was more familiar with, the others could do the same in converse.

\paragraph{}
Several times during the project, we didn't fully agree on some aspects.
Even after exchanging arguments we sometimes still had different opinions.
Some issues were minor and not critical to the project in the sense that
they did not have a large impact. If necessary, rewriting these code fragments
later on would not require much effort. As the project continued I
learned to refrain from unnecessary long discussions on these issues.



\paragraph{}
Overall, the project has been very instructive. Both from a technical point
of view as well as a social point of view. In some areas it has been a
contrasting experience compared to my daily job. I hope to be able to
apply the knowledge gained during this project to new projects in the future.




\paragraph{Stefan}
From time to time I thought that it would be difficult to reach our targets so
we could meet the customer requirements. After all we had to start from scratch
and with a limited amount of time. It took some time to find the right platform
independent tools plus we had to understand the complex context and think about
solutions for new features. So we had to do a lot of research before we could
start writing a single line of code. Even though we weren't familiar with some
technologies we always tried to pick the best options instead of picking quick
\& cheap.

Meeting our goals and learning new technologies was one thing but I also wanted
to try out an agile approach. I can't say if DAD was the right choice or not,
but I would recommend DAD to anyone who want to start practicing agile methods.
This because DAD can be tailored by the needs and it does not prescribe strict
rules but is rather people and learning oriented.

One of the suggested DAD practices that was important for our distributed team
was how to ``visualize work''. For that purpose we chose Agilefant which is an
open source web based agile management tool. I'm very satisfied about this kind
of tools, it has a little bit of learning curve but you get quickly much in
return. A tool like Agilefant reduces the disadvantage of a distributed team
compared to a co-located teams. Many times we've used it to lead our on-line
meetings so we could align our thoughts. Together with on-line communication
tools I think this is a must have for distributed agile teams.

As for anyone else we also had to deal with ups and downs, technically and
personally. The most important thing is that we all were there to help or
listening whenever it was necessary. We communicated a lot and it was rarely
that someone was not available and in my opinion this was also our strength that
we managed to quickly find a solution and kept motivation.

After nine months of hard working, I can say that I'm very happy about the
result and the experience. I could increase my skills in new technologies more
as expected. I could also practice agile methods and learn how these contribute
in reaching goals. Finally and last but not least, all of this wouldn't be possible
without the support and effort of Guus and Jeroen.
