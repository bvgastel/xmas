\chapter{Reflection}
\begin{tcolorbox}[colback=yellow!30]
Freek $\rightarrow$ \\ 
proces report , team en individuele reflecties samenpakken in 1 reflectie: zijn jullie tevreden met het resultaat? Met de gekozen procesvorm? Met de gekozen technieken? Vervolgens dan jullie individuele reflectie.
\end{tcolorbox}

\todo[inline]{process \& team reflection}

\paragraph{Guus: }
todo \\
\todo[inline]{reflection guus}


\paragraph{Jeroen: }
todo \\ 
\todo[inline]{reflection jeroen}



\paragraph{Stefan: }
For my job related tasks I'm stitched to Microsoft products. This is because the
policy of the company I work for. As a student of informatics I want to gain
experience in other kinds of products, especially in open source and the tools
used for this. Also the agile approach is something that I'm really interested
in and wanted to practice. I have some experience as part of a team that support
software projects but although the lead talks about agile and waterfall, my
experience is that when it comes to practice it always ends in a bunch of
inconsistent text based to-do lists. Same for the documentation, important but
again when it comes to creating those, nobody seams to like it and are seen as a
second-rate activity. So this school project has no economically side effects
and ideally to try methods by the book. Therefor at the start of this project
I've made myself a list of competences that I wanted to develop:
\begin{itemize}
 \item practice agile methods e.g. DAD.
 \item using agile tools.
 \item acquire software development in a distributed team.
 \item gain experience of non Microsoft, platform independent tools.
\end{itemize}

The choice of an appropriate agile method was not straightforward. None of our
team members could go to their desk, plan and start to build. This because of
the high level requirements and we had to do a lot of investigation in advance.
For me it was not clear how to fit this into an agile method. I was wondering
how can you plan things like ``...get knowledge of something you don't know
yet...''. The DAD method is learning-oriented and a cherry picking framework. In
my opinion the main benefit of DAD for us was the idea of ``don't worry about
strict rules'' ``pick whatever you're team or project fits'', so people first,
learn \& adopt.

The tool we used to fulfill a minimum of the DAD approach was Agilefant.
I'm am very satisfied about the benefits that such a tool gives. 
\\
\textbf{still working on it.....}
