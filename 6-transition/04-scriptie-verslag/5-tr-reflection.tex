\chapter{Reflection}
\paragraph{Project} We started out largely unsure about how to continue. The first few months
planning we structured internal and external communication. Additionally, we decided to 
work in time boxes (iterations) where we successively worked toward a finished project. 
The internal communication was intense (three times a week, sometimes more) where one of our
members routinely worked more hours at the office in order to join our call. The division of work 
was partly pre-agreed upon and partly just happened.

Right after the first iteration we encountered the first problem: our mentor for the project 
and the customer both ceased to respond to our mails. Being December, we figured they were 
on an unscheduled holiday. About a month later (half of January) the customer explained 
what was happening. The mentor was sent off to the U.S. and our customer's workload had been 
increased to a level where he hardly had the chance to join us in a call. This was all due to 
a reorganization within the Open University. The OU was retiring approximately 10 - 30\% of their
personnel\footnote{When we commented on this during the midterm review, the OU people present,
did not feel the students needed to know. Apparently, our team was the only team impacted 
by the change. Still we wonder why such a large reorganization was not communicated broadly, 
even if most of the students should not notice anything, after all: it's a big change.}.
 
We realized we had to change our approach. Although initially we talked about rescheduling,
we decided not to. One of the advantages of agile working is the principle of the time-box. 
The consequent of the decrease of communication with the customer would be a decrease in speed
of deciding. We agreed with the customer to have at minimum one skype session per iteration. 
This we added to our planning. Informally, we could request a quick answer to urgent questions
by email.

In conclusion we finished our product more or less as we planned to. We had initially
hoped to have covered more ground, but in the circumstances we are quite content with what 
we produced. The customer and the mentor expressed contentment as well.

\paragraph{Product} The product turned out more or less as we envisioned at the start. The one
thing we might do differently in hindsight, is to spend a little more attention to the decision
of using a tight integration with the existing xmas programs that were developed before we started.

The issue was whether to use a text interface causing complete segregation of user interface and
xmas programs, or, to use a tight interface to the xmas programs. The customer chose the tight 
interface fearing the consequences of a second parser for maintenance. We chose to integrate
\qml and \cpp which probably cost about 2 iterations (6 weeks) extra in comparison to a text
interface. Although this is partly conjecture, if the team and the customer had known the extent 
of the consequences, we might have chosen differently.

\paragraph{Guus} I felt very happy working this project with a very intense communication pattern,
a positively constructive team and supportive and helpful mentor and customer. I started out 
unfamiliar with the agile way of working and ended up liking the way it works. 
Especially the time box turned out very useful when events 
turned against us. In a planned environment, where functionality is fixed and time variable, I 
am convinced we would have had to reschedule our project. I am very glad we chose to go agile.

Learning \cpp was a real pleasure, although my work managing stuff took a large part of my time,
especially during planning and preparing for our meetings. Additionally, once we started programming,
all time I wanted to spend on learning \cpp evaporated. My connotation of \cpp has significantly improved
due to the working with live \cpp programs.

I also learnt the impact of strictness in an agile project. Too much strictness stiffles initiative 
and creativity. However, some strictness is necessary to constantly know where we are and what we 
still need cover in order to succeed.

Agile and strictness is a troubled marriage at best. I am very glad to have learnt this valuable lesson and
I am grateful to my team mates who endured my overload of control and finally corrected me when I was 
enforcing too much.

\paragraph{Jeroen}
todo \\ 
\todo[inline]{reflection jeroen}

\paragraph{Stefan: }
From time to time I thought that it would be difficult to reach our targets so
we could meet the customer requirements. After all we had to start from scratch
and with a limited amount of time. It took some time to find the right platform
independent tools plus we had to understand the complex context and think about
solutions for new features. So we had to do a lot of research before we could
start writing a single line of code. Even though we weren't familiar with some
technologies we always tried to pick the best options instead of picking quick
\& cheap.

Meeting our goals and learning new technologies was one thing but I also wanted
to try out an agile approach. I can't say if DAD was the right choice or not,
but I would recommend DAD to anyone who want to start practicing agile methods.
This because DAD can be tailored by the needs and it does not prescribe strict
rules but is rather people and learning oriented.

One of the suggested DAD practices that was important for our distributed team
was how to ``visualize work''. For that purpose we chose Agilefant which is an
open source web based agile management tool. I'm very satisfied about this kind
of tools, it has a little bit of learning curve but you get quickly much in
return. A tool like Agilefant reduces the disadvantage of a distributed team
compared to a co-located teams. Many times we've used it to lead our on-line
meetings so we could align our thoughts. Together with on-line communication
tools I think this is a must have for distributed agile teams.

As for anyone else we also had to deal with ups and downs, technically and
personally. The most important thing is that we all were there to help or
listening whenever it was necessary. We communicated a lot and it was rarely
that someone was not available and in my opinion this was also our strength that
we managed to quickly find a solution and kept motivation.

After nine months of hard working, I can say that I'm very happy about the
result and the experience. I could increase my skills in new technologies more
as expected. I could also practice agile methods and learn how these contribute
in reaching goals. Finally and last but not least, all of this wouldn't be possible
without the support and effort of Guus and Jeroen.
