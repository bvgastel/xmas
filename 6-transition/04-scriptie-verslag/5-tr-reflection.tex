\chapter{Reflection}
\paragraph{Project} We started out largely unsure about how to continue. The first few months
planning we structured internal and external communication. Additionally, we decided to 
work in time boxes (iterations) where we successively worked toward a finished project. 
The internal communication was intense (three times a week, sometimes more) where one of our
members routinely worked more hours at the office in order to join our call. The division of work 
was partly pre-agreed upon and partly just happened.

Right after the first iteration we encountered the first problem: our mentor for the project 
and the customer both ceased to respond to our mails. Being December, we figured they were 
on an unscheduled holiday. About a month later (half of January) the customer explained 
what was happening. The mentor was sent off to the U.S. and our customer's workload had been 
increased to a level where he hardly had the chance to join us in a call. This was all due to 
a reorganisation within the Open University. The OU was retiring approximately 10 - 30\% of their
personnel\footnote{When we commented on this during the midterm review, the OU people present,
did not feel the students needed to know. Apparantly, our team was the only team impacted 
by the change. Still we wonder why such a large reorganisation was not communicated broadly, 
even if most of the students should not notice anything, after all: it's a big change.}.
 
We realised we had to change our approach. Although initially we talked about rescheduling,
we decided not to. One of the advantages of agile working is the priniciple of the time-box. 
The consequent of the decrease of communication with the customer would be a decrease in speed
of deciding. We agreed with the customer to have at minimum one skype session per iteration. 
This we added to our planning. Informally, we could request a quick answer to urgent questions
by email.

In conclusion we finished our product more or less as we planned to. We had initially
hoped to have covered more ground, but in the circumstances we are quite content with what 
we produced. The customer and the mentor expressed contentment as well.

\paragraph{Product} The product turned out more or less as we envisioned at the start. The one
thing we might do differently in hindsight, is to spend a little more attention to the decision
of using a tight integration with the existing xmas programs that were developed before we started.

The issue was whether to use a text interface causing complete segregation of user interface and
xmas programs, or, to use a tight interface to the xmas programs. The customer chose the tight 
interface fearing the consequences of a second parser for maintenance. We chose to integrate
\qml and \cpp which probably cost about 2 iterations (6 weeks) extra in comparison to a text
interface. Although this is partly conjecture, if the team and the customer had known the extent 
of the consequences, we might have chosen differently.

\paragraph{Guus} I felt very happy working this project with a very intense communication pattern,
a positively constructive team and supportive and helpful mentor and customer. I started out 
unfamiliar with the agile way of working and ended up liking the way it works. 
Especially the time box turned out very useful when events 
turned against us. In a planned environment, where functionality is fixed and time variable, I 
am convinced we would have had to reschedule our project. I am very glad we chose to go agile.

Learning \cpp was a real pleasure, although my work managing stuff took a large part of my time,
especially during planning and preparing for our meetings. Additionally, once we started programming,
all time I wanted to spend on learning \cpp evaporated. My connotation of \cpp has significantly improved
due to the working with live \cpp programs.

I also learnt the impact of strictness in an agile project. Too much strictness stiffles initiative 
and creativity. However, some strictness is necessary to constantly know where we are and what we 
still need cover in order to succeed.

Agile and strictness is a troubled marriage at best. I am very glad to have learnt this valuable lesson and
I am grateful to my team mates who endured my overload of control and finally corrected me when I was 
enforcing too much.

\paragraph{Jeroen}
todo \\ 
\todo[inline]{reflection jeroen}

\paragraph{Stefan: }
For my job related tasks I'm stitched to Microsoft products. This is because the
policy of the company I work for. As a student of informatics I want to gain
experience in other kinds of products, especially in open source and the tools
used for this. Also the agile approach is something that I'm really interested
in and wanted to practice. I have some experience as part of a team that support
software projects but although the lead talks about agile and waterfall, my
experience is that when it comes to practice it always ends in a bunch of
inconsistent text based to-do lists. Same for the documentation, important but
again when it comes to creating those, nobody seams to like it and are seen as a
second-rate activity. So this school project has no economically side effects
and ideally to try methods by the book. Therefor at the start of this project
I've made myself a list of competences that I wanted to develop:
\begin{itemize}
 \item practice agile methods e.g. DAD.
 \item using agile tools.
 \item acquire software development in a distributed team.
 \item gain experience of non Microsoft, platform independent tools.
\end{itemize}

The choice of an appropriate agile method was not straightforward. None of our
team members could go to their desk, plan and start to build. This because of
the high level requirements and we had to do a lot of investigation in advance.
For me it was not clear how to fit this into an agile method. I was wondering
how can you plan things like ``...get knowledge of something you don't know
yet...''. The DAD method is learning-oriented and a cherry picking framework. In
my opinion the main benefit of DAD for us was the idea of ``don't worry about
strict rules'' ``pick whatever your team or project fits'', so people first,
learn \& adopt.

The tool we used to fulfill a minimum of the DAD approach was Agilefant.
I'm am very satisfied about the benefits that such a tool gives. 
\\
\textbf{still working on it.....}
