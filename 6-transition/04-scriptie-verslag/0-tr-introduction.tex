\chapter*{Introduction}
\addcontentsline{toc}{chapter}{Introduction}
\todo[inline]{introductie}
\begin{tcolorbox}[colback=yellow!30]

Freek $\rightarrow$ \\ 
Introductie: Drie pagina's introductie is inderdaad ruim, omdat jullie research context al veel behandelt. Het lijkt me voldoende om in 1 paragraaf context te schetsen (verificatie van NoCs), dan het probleem en de vragen die daarbij kwamen. Het is ``common practice'' om de intro te eindigen met een paragraaf die de structuur van het document beschrijft. Ik zou zelf de research context na de introductie doen, zodat je kan zeggen ``In het volgende hoofdstuk lichten we de relevantie van ons product toe tov de state-of-the-art'' of iets dergelijks.

\end{tcolorbox}

Formal verification is a pre-production activity to prove the correctness of a 
design. Designers need to use verification technology, but often are not 
acquainted with formal verification methods. A designer tool with built-in 
verification tools that provides formal verification is of value for pre-production
design of an \Noc. 

The following chapter illustrates why formal verification methods play a crucial 
role in the design of new technologies. Consequently it shows the relevance of the
XMAS design tool to the designing process in the context of verification enabling 
progress in the state of the art.

This document contains the research context, report, product manual and, system documentation.



%%\newpage