\chapter*{Introduction}
\addcontentsline{toc}{chapter}{Introduction}
\todo[inline]{introductie}
\begin{tcolorbox}[colback=yellow!30]

Freek $\rightarrow$ \\ 
Introductie: Drie pagina's introductie is inderdaad ruim, omdat jullie research context al veel behandelt. Het lijkt me voldoende om in 1 paragraaf context te schetsen (verificatie van NoCs), dan het probleem en de vragen die daarbij kwamen. Het is ``common practice'' om de intro te eindigen met een paragraaf die de structuur van het document beschrijft. Ik zou zelf de research context na de introductie doen, zodat je kan zeggen ``In het volgende hoofdstuk lichten we de relevantie van ons product toe tov de state-of-the-art'' of iets dergelijks.

\end{tcolorbox}


%%\newpage