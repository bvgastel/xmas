\chapter*{Introduction}
\addcontentsline{toc}{chapter}{Introduction}

In modern chip architectures a System on Chip (\Soc) contains a processor (core) and 
on-chip communication fabrics (uncore). The communication fabrics are often referred 
to as Network on Chip (\Noc).

Production of an \Noc may take years to prepare (pre-production phase) taking large financial investments
before production of the chip. Errors in the \Noc could lead to both financial and reputation 
damage that could linger years after. Verification of correctness is necessarily part of the 
pre-production phase of each \Noc.

Correctness verification is a specialized area of expertise often unfamiliar to designers of chips.
Additionally, manual verification is a complex, time intensive activity in arbitrary designs.
Research to verify correctness with respect to cycles, deadlocks and livelocks is ongoing. 

Intel introduced a high level design language for communication fabrics called xMAS (eXecutable 
Micro Architectural Specification). This model prevents many errors by 
construction\cite{DBLP:journals/dt/ChatterjeeKO12}. The limitation of construction to eight 
carefully designed primitives enables automatic verification of networks, increasing the 
speed of designing a verifiable correct \Noc and \Soc.

The previous installment of an xMAS design tool in C\# (WickedXMAS) was hindered by platform 
dependencies and \cpp integration problems. The program was limited to Microsoft platforms. The
managed C\# environment did not integrate well with the unmanaged \cpp environment.
To solve these issues our project developed a program that works on multiple platforms and 
roughly supports the same networks as the original program. It integrates properly with the existing \cpp 
programs. Because both the \xmas designer and the verification tools are written 
using \cpp, maintenance is of both parts is easier to combine.

