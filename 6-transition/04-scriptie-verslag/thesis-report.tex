%%
%% Dit is het hoofddocument: compileer dit met latex of xelatex en je krijgt de gehele pdf
%%
\documentclass[a4paper,11pt]{report}
%
%% option = draft will generate black except with \color, replace images by a rectangle
%% option = final will generate color
%

\usepackage{color}
\usepackage{tcolorbox}
\usepackage[all]{nowidow}
\usepackage{bold-extra}
\usepackage{footmisc}% granting the ability to use label for a footnote
\usepackage{subfig}
\usepackage{wrapfig}% product wrapfigure and wraptable
\usepackage{array}% additions to tabular
%\usepackage{supertabular}% multiple pages tabular
\usepackage{longtable}% multiple pages table like tabular
\usepackage{rotating}% for the environment sidewaysfigure / sidewaystable
\usepackage[english,dutch]{babel}
\usepackage{graphicx}
\usepackage{url}
\usepackage{hyperref}% Load after biblatex
\hypersetup{
    colorlinks = true,
    citecolor = blue,
    linkcolor = blue
}
\usepackage[prependcaption,colorinlistoftodos,obeyFinal,textsize=small]{todonotes}% when generating final (documentclass option) skip notes
\usepackage{pdflscape}
\usepackage[a4paper]{geometry}
\usepackage{titlesec}% added to change section headers, see newcommand definition.
\usepackage{boxedminipage}
\usepackage{amssymb}% For \checkmark
\usepackage{pifont}% for \ding{'-code or "-code}
\usepackage{listings}
\usepackage{xspace}%
\usepackage[utf8]{inputenc}
\usepackage{fancyhdr}

\pagestyle{fancy}

\graphicspath{ {pictures/} }

\bibliographystyle{plain}% unsrt, plain, alpha, abbrv

\newcommand{\biburl}[1]{\hspace*{\fill}\\\url{#1} accessed oct/nov 2014}

\author{ABI team 33}
				
\date{03/03/2015}

\title{
	{\color{blue}XMAS Model Designer}\\
	{\large Open Universiteit Nederland, faculteit Informatica}\\
	{\small T61327 - Afstudeerproject bachelor informatica}\\
	\vspace{1cm}
	{\includegraphics[width=.25\textwidth]{xmd}}
}

\setlength\extrarowheight{2pt}% Adds a little space at the top of table rows

%% Document is in subdocumenten gesplitst.

\begin{document}

\selectlanguage{english}
%%\selectlanguage{dutch}

\hyphenation{func-tio-nal}

\nowidow% needs package nowidow

%%%%%%%%%%%%%%%%%%%%%%%%%%%%%%%%%%%%
\newcommand{\xmas}{x\textsc{mas}}%
\newcommand{\ok}{$\checkmark$}
\newcommand{\w}[1]{\textbf{\textsc{#1}}}
\newcommand\bw[1]{{\color{blue}#1}}
\newcommand{\Noc}{\textsc{NoC}\xspace}%
\newcommand{\cpp}{\textsc{C++}\xspace}%
\newcommand{\mybox}[1]{\begin{boxedminipage}[t]{\textwidth}#1\end{boxedminipage}}

%\definecolor{airforceblue}{rgb}{0.36, 0.54, 0.66}%%   This is color in hex #5D8AA8

%%%%%%%%%%%%%%%%%%%%%%%%%%%%%%%%%%%% different section format start %%%%%%%%%%%%%%%%%%%%%%%%%%%%%%%%%
%\newcommand\secformat[1]{%
%    {\fontsize{60}{60}\selectfont\thesection}%
%    \ifthenelse{\equal{\thesection}{}}{}{\quad\rule[-8pt]{2pt}{40pt}\quad}
%    \parbox[b]{.7\textwidth}{\filright\bfseries #1}}%
%\titleformat{\section}[block]
%    {\filright\normalfont\sffamily}{}{0pt}{\secformat}
%\titlespacing*{\section}{0pt}{*3}{*2}[1pc]
%%%%%%%%%%%%%%%%%%%%%%%%%%%%%%%%%%%% different section format end   %%%%%%%%%%%%%%%%%%%%%%%%%%%%%%%%%


\newcommand\smp[1]{%
	\marginpar{\color{blue}\small\bf\textsc#1}
}%
\newcommand\smpp[1]{\smp{#1}#1}


\maketitle

\begin{tcolorbox}[colback=yellow!50]

Hoi Stefan,

1.) OK
2.) OK

Ik vind de studienet beschrijving van het verslag niet erg goed, het is veel te specifiek en geeft de indruk dat jullie veel nieuwe dingen moeten schrijven. Ik hoop dat ik jullie met deze mail wat werk bespaar.

Introductie: Drie pagina's introductie is inderdaad ruim, omdat jullie research context al veel behandelt. Het lijkt me voldoende om in 1 paragraaf context te schetsen (verificatie van NoCs), dan het probleem en de vragen die daarbij kwamen. Het is ``common practice'' om de intro te eindigen met een paragraaf die de structuur van het document beschrijft. Ik zou zelf de research context na de introductie doen, zodat je kan zeggen ``In het volgende hoofdstuk lichten we de relevantie van ons product toe tov de state-of-the-art'' of iets dergelijks.
Chapter 1: copy-paste hier gewoon jullie systeem documentatie. Ik neem aan dat daar requirements in staan, en dingen als welke technieken (bijv. QT) gebruikt zijn.
Chapter 2: ik zou deze chapter gewoon overslaan, omdat ik denk dat dit allemaal al in Chapter 1 komt.
Chapter 3: copy-paste hier integraal de drie domeinanalyses
Chapter 4: copy-paste hier integraal de research context
Chapter 6+7: ik zou deze samenpakken in 1 reflectie: zijn jullie tevreden met het resultaat? Met de gekozen procesvorm? Met de gekozen technieken? Vervolgens dan jullie individuele reflectie.
Conclusie: Hier kun je onder andere wat ``future work'' kwijt. Hoe zouden jullie jullie tool verder ontwikkelen. Stel we hebben een nieuw ABI groepje, wat zou deze kunnen doen en wat raden jullie hun aan? Staar je niet blind op pagina nummers, 1 pagina is ook goed.

Succes met de laatste loodjes!

Groeten,

Freek

\end{tcolorbox}



\begin{tcolorbox}[colback=yellow!30]
  1) is het ok dat titelblad , team-info en abstract op een afzonderlijke pagina staan of liever niet? \\
  2) als logo gebruiken we de xmd metafoor (Bernard is er dol op). Deze komt van internet er 
  stond geen info van een artist bij dus we nemen aan dat we deze mogen gebruiken
  zolang er niemand over valt?
\end{tcolorbox}

\todo[inline]{enter studentnumber and check data}
\begin{flushleft}
    \begin{tabular}{p{2cm} l }
    \textbf{studenten:} \\

    & \begin{tabular}{p{3cm} p{2cm} p{3cm} l}
    \textbf{name} & \textbf{nb.} & \textbf{city} & \textbf{country} \\ \hline
    Guus Bonnema & 838523637  & Dieren  & Netherlands \\
    Jeroen Kleijn & ? & Den Haag & Netherlands \\
    Versluys Stefan & 850317700 & Sint-Laureins & Belgium \\
    \hline \break
    \end{tabular}
    \\ 
    \textbf{mentor:} & Freek Verbeek  \\
    \textbf{examiner:} & Marko van Eekelen \\
    \end{tabular}
\end{flushleft}


\begin{abstract}
%% abstract of about 250 words
Research of \Noc leads to problems of verification. Manual verification is time intensive and error prone.
Automatic verification of arbitrary networks causes state-space explosion.
Intel countered with the development of XMas (eXtensible M A S) which reduces the state-space using
eight primitive constructions the designer can glue together. This enables automatic verification of
certain desirable properties like freedom of cycles, deadlocks and maybe also livelocks.

The previous installment of XMAS in a C\# program turned out to miss the mark due to platform dependencies.
Additionally, using the non-managed environment of \cpp with the managed environment of C\# causes integration problems.
Our solution works on multiple platforms and roughly supports the same networks as the original program.
At the same time it integrates properly with the existing \cpp programs. Because all programs both the xmas
designer and the verification tools are written using \cpp maintenance is of both parts is easier to combine.

In this document we explain the research environment that the product will support, the requirements we followed and the 
functionality we developed using these requirements. We also provide hints for the following ABI project.

\end{abstract}

\newpage
\listoftodos   %% hidden when documentclass is set to final

\newpage
\tableofcontents
\newpage

\chapter*{Introduction}
\addcontentsline{toc}{chapter}{Introduction}
\todo[inline]{introductie, plusminus 3 A4 pagina's, gezamenlijk (mogelijke onderdelen: context, vraagstelling etc)}
\begin{tcolorbox}[colback=yellow!30]
  zelfde opmerking als bij conclusies , 3 pagina's aan introductie!? 
  
  Bedoeling hier is om het probleem te schetsen (klant) en de vragen die daarbij rijzen (klant-team), klopt dit?

\end{tcolorbox}
%%\newpage% include forces page break. Input does not.

\chapter{System Documentation}
\section{section1}
\todo[inline]{
requirements plusminus 2 pagina's
}
%%\newpage



\chapter{Overview}
\section{section1}
\todo[inline]{
algemeen overzicht domeinen en technieken plusminus 2 pagina's
}

\begin{tcolorbox}[colback=yellow!30]
  wat wenst men hier precies?  ontwikkelomgeving, gebruikte tools, agile methodiek, \dots ?

\end{tcolorbox}

%%\newpage



\chapter{Reports}
\section{Reports}
\todo[inline]{
individueel verslag onderzoek deeldomein en bijhorende technieken, plusminus 5 pagina's
}
%%\newpage



\chapter{Research context}
\section{Research Context}
\todo[inline]{
invoegen verslag analyse onderzoekcontext, plusminus 5 pagina's}

\begin{tcolorbox}[colback=green!30]
  in progress - stefan
\end{tcolorbox}


%%\newpage



\chapter{Product manual}
\todo[inline]{
beschrijving van het opgeleverde eindproduct, plusminus 15 pagina’s}
\begin{tcolorbox}[colback=green!30]
  in progress - stefan
\end{tcolorbox}

\section{Setup}

\section{Project}

\section{Tool}

\section{Remarks}
\begin{itemize}
\item	bugs
\begin{enumerate}
\item	No model modified flag when packet changed
\item	Packet dialog not cleared on new/open network
\item	Model modified flag set after new when canvas was not clear
\item	Invalid maximized main window restore from setup (Qt Bug)
\item	Group select has not always key focus (e.g. pressing delete has no effect)
\item	Clicking new if canvas is not clear aborts application under Linux.
\item	Model setup not restored on model open.
\item With composite on canvas and click new then dialog save first $+$ no does not
work
\item Sometimes a composite its mouse area becomes inactive. (start with a clean
canvas , add a composite in the library and drag one on the canvas. Save new
model and click new or open). This only occurs with new (nameless) networks,
adding a composite is probably not handled well in the xMAS networks list and
therefore do only exists on the canvas.
\item	Several memory leaks
\end{enumerate}

\item	fixes
\begin{enumerate}
\item	Grid snap on port center instead of component body left top
\item	Grid snap when drag a group of items
\item	Qt save as dialog not available in version 5.4.0
\item QML uses non blocking calls, cannot use yes/no dialogs feedback in a
proper way.
\item Use of destroy dialog to quit application from QML dialog to prevent
Windows warning (Qt bug).
\item	Tooltip for composites in library to show alias or filename
\item Allow entering source expression when not connected (xmas parser)
\item New page is now done by select all and delete, instead destroy page and
create new page.
\item Remove qml convert logic in network.cpp, instead send a add component
request to the canvas by its name and type. Signal the update to component.cpp
which already does the request to xMAS for the other properties like x,y,… This
to avoid duplicate conversion and checks.
\end{enumerate}
\item	features
\begin{enumerate}
\item	(x,y) anchor list to draw a channel via these points
\item	Pathfinder algorithm for channels (e.g. A*)
\item	Canvas scroll while drag items
\item	Canvas copy-cut-paste
\item	Canvas undo-redo
\item	Canvas rotate group selection
\item	Redirect standard output to the plug-in consoles
\item	Plug-in parameters change read-only to editable list
\item	Replace auto plug-in load from fixed folder to add/remove
\item	Model export
\item	Parametric composite
\item	Packet editor as editable key/value list
\item	Plug-in tools add progress value.
\item	Plug-in tools add stop process. 
\item	Deadlock checker with plug-in interface
\item	Plug-in structured text for “in model” feedback
\item Channel rewire via clicking a connected port. (no you need to delete and
create new channel)
\item	Multi page model editor
\item	Open composite sub network in main model (e.g. on a new page)
\item	Application help
\item A platform dependent binaries package of the Qt project so that user can
install the application without having Qt or other tools installed.
\end{enumerate}
\end{itemize}


%%\newpage



\chapter{Process report}
\section{section1}
\todo[inline]{
procesverslag, voor zover relevant voor het begrijpen waarom het eindproduct in deze context de beste oplossing is: stappen met de beslissingen en de argumenten daarbij, plusminus 2 pagina's. Wellicht is dit geen getrouwe geschiedschrijving, maar een manier om de opdrachtgever en de examinator te laten begrijpen waarom dit een goed product is gegeven de kaders en doelstellingen
}

\begin{tcolorbox}[colback=yellow!30]
  hier is het niet echt duidelijk waar ``goed'' op slaat. Of waarom goed voor de context (dit staat al in de research context toch?) of goed omwille van de gebruikte technologien zoals ``waarom Qt Quick2 en niet FLTK'' enz\dots...

\end{tcolorbox}

%%\newpage



\chapter{Reflections}
\begin{tcolorbox}[colback=yellow!30]
  we zouden hier onze gezamelijke dingen zetten waar we tegenaan zijn gelopen, of juist goed ging, zowel technisch als menselijk.
  Bij de individuele secties zouden we onze bevindingen zetten m.b.t. onze doelstellingen die we elk aan het begin van dit project vooropgesteld hadden. Zoals agile werken , c++ leren, \dots

\end{tcolorbox}

\section{Team}
\todo[inline]{
teamreflectie op het project en de gekozen aanpak: wie heeft welke rollen vervuld, hoe verliep het project, plusminus 2 pagina's
}



%%\newpage


\section{Individual reports}
\todo[inline]{
individueel verslag van de persoonlijke ervaringen en leermomenten
}
%%\newpage



\chapter*{Conclusion}
\todo[inline]{
conclusies en aanbevelingen, plusminus 2 pagina's, gezamenlijke tekst.
}
%%\newpage



\appendix
\chapter{Appendix}
%%\section{Bug list}
\begin{enumerate} \label{sec:bug-list}
\item	No model modified flag when packet changed
\item	Packet dialog not cleared on new/open network
\item	Model modified flag set after new when canvas was not clear
\item	Invalid maximized main window restore from setup (Qt Bug)
\item	Group select has not always key focus (e.g. pressing delete has no effect)
\item	Clicking new if canvas is not clear aborts application under Linux.
\item	Model setup not restored on model open.
\item With composite on canvas and click new then dialog save first $+$ no does not
work
\item Sometimes a composite its mouse area becomes inactive. (start with a clean
canvas , add a composite in the library and drag one on the canvas. Save new
model and click new or open). This only occurs with new (nameless) networks,
adding a composite is probably not handled well in the xMAS networks list and
therefore do only exists on the canvas.
\item	Several memory leaks
\end{enumerate}


\section{Fix list}
\begin{enumerate}\label{sec:fix-list}
\item	Grid snap on port center instead of component body left top
\item	Grid snap when drag a group of items
\item	Qt save as dialog not available in version 5.4.0
\item QML uses non blocking calls, cannot use yes/no dialogs feedback in a
proper way.
\item Use of destroy dialog to quit application from QML dialog to prevent
Windows warning (Qt bug).
\item Allow entering source expression when not connected (xmas parser)
\item New page is now done by select all and delete, instead destroy page and
create new page.
\item Remove qml convert logic in network.cpp, instead send a add component
request to the canvas by its name and type. Signal the update to component.cpp
which already does the request to xMAS for the other properties like x,y,… This
to avoid duplicate conversion and checks.
\item	\st{Tooltip for composites in library to show alias or filename}
\item	\st{Tooltip for primitives in toolbar}
\end{enumerate}


\section{Feature list}

\paragraph{features}
\begin{enumerate}
\item	(x,y) anchor list to draw a channel via these points
\item	Pathfinder algorithm for channels (e.g. A*)
\item	Canvas scroll while drag items
\item	Canvas copy-cut-paste
\item	Canvas undo-redo
\item	Canvas rotate group selection
\item	Redirect standard output to the plug-in consoles
\item	Plug-in parameters change read-only to editable list
\item	Replace auto plug-in load from fixed folder to add/remove
\item	Model export
\item	Parametric composite
\item	Packet editor as editable key/value list
\item	Plug-in tools add progress value.
\item	Plug-in tools add stop process. 
\item	Deadlock checker with plug-in interface
\item	Plug-in structured text for “in model” feedback
\item Channel rewire via clicking a connected port. (no you need to delete and
create new channel)
\item	Multi page model editor
\item	Open composite sub network in main model (e.g. on a new page)
\item	Application help
\item A platform dependent binaries package of the Qt project so that user can
install the application without having Qt or other tools installed.
\end{enumerate}


% Bibliography ---> need to include
%%\bibliography{tr}

\end{document} ;########################### end document ##################################;
