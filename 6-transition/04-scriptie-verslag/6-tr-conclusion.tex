\chapter{Conclusion}

In addition to the previous chapter we ponder improvements to formulate 
suggestions for the next project to work on. This chapter provides these considerations.

%\vspace{0.5 cm}
%\begin{tcolorbox}[colback=white]
Suggestions for future releases are:
\begin{itemize}
\item \textbf{Parametric composites} Add significantly to the expressive power
of the \Noc designs by implementing parametric composites.
\item \textbf{Decouple user interface} Refactor the designer as 
as a fully declarative UI and enforce independence from the back-end 
verification model, increasing the ease of development of both user interface
and verification tools.
\item \textbf{Deadlock verification} Add plug-in interface for deadlock verification tool
thus increasing the confidence of designers in their \Noc designs.
\item \textbf{Plugin tool} Improve and complete the plugin interface to a level where
the designer of a verification tool only has to write the verification tool and 
only needs to write some minimal plugin code. 
\end{itemize}
%\end{tcolorbox}

Some of the features are described in detail and can be found in the
reports section. The parametric composites is a proposal of how these
complex components could be designed and implemented (see chapter~\ref{sec:parametric}).

For release v0.6 in the appendix one can find the buglist (see~\ref{sec:bug-list}, a fixlist \ref{sec:fix-list}), 
and the feature list (see~\ref{sec:feature-list}).
The agilefant backlog\footnote{to be found in the repository} has these lists plus 
comments on how to reproduce the bugs.

Hopefully these ideas are useful for future releases and boost the xMAS designer
to the next level.
