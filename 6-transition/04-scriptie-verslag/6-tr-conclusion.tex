\chapter*{Conclusion}

When it comes to the end of a project like this, it is always a good practice to
resume and see if the requirements of the customer were met. It is also
important to take note of what is still missing or what improvement can we
suggest. So, it is not only about a nice and technically correct product but it
must satisfy the customer and match the expectations.


One of the main requirements was to have a platform independent xMAS
toolkit. The new xMAS toolkit is developed and tested in a multi platform
environment. It is exhaustive tested on Linux and Windows while MacOs is used to
try-out the releases.

To meet the maintainability requirement we've chosen Qt as integrated
development environment. Qt is a powerful open source platform independent IDE.
The QML declarative UI language from Qt, enforces separation of view and control. The
Qt IDE is an all-in-one solution with a lot of support available.


Integration of formal verification tools is implemented via a standard plug-in
interface. In this way it is possible to develop new or re-factor existing
verification tools independently and without the need of touching the designer
source code. Once a verification tool has the right plug-in interface it can be
used in the xMAS designer.


\vspace{0.5 cm}

\begin{tcolorbox}[colback=white]
According to the main requirements; The benefits of the new xMAS designer toolkit:
\begin{itemize}
\item \textbf{It is platform independent.}
\item \textbf{Maintainability is improved.}
\item \textbf{Integrate verification tools via a standard plug-in interface}
\end{itemize}
\end{tcolorbox}

The current product release v0.6 has still some bugs~\ref{sec:bug-list} or
things that can be fixed~\ref{sec:fix-list}. Both can be found in the appendix
of this document or in the Agilefant backlog. The latter has comments of how
these bugs can be reproduced.

In the appendix there's also a list of possible features~\ref{sec:feature-list}
that can be added. Some are described in detail and can be found in the
reports section. For example, report chapter~\ref{sec:parametric} is a proposal
of how parametric composites could be implemented.

We are still convinced that keeping the designer fully independent from the
verification model back-end makes the project much more maintainable.

Separating UI code from pure c++ model and verification code is also a good idea
because UI tools are much more exposed to changes, e.g. Qt version update. The
tighter these are coupled the bigger the chance that you have to make
modifications through the whole system.

Also the benefit of updating the back-end verification model with every canvas
action doesn't make sense. Why would a designer start a verification tool if the
canvas model is still under creation? This implies that it isn't necessary to
keep the back-end model constantly up to date with every canvas action. Once the
designer thinks the model is ready he or she can start a verification tool.
Details of this proposal can be found in report chapter~\ref{sec:integration}

\vspace{0.5 cm}
\begin{tcolorbox}[colback=white]
Some prior suggestions for future releases:
\begin{itemize}
\item \textbf{Parametric composites}
\item \textbf{Designer as full declarative UI enforces independency from back-end verification model}
\item \textbf{Add plug-in interface for deadlock verification tool}
\end{itemize}
\end{tcolorbox}

Hopefully these ideas are useful for future releases and boost the xMAS designer
to the next level.













