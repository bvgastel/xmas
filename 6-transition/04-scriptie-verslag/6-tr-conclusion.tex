\chapter*{Conclusion}

\begin{tcolorbox}[colback=yellow!30]
Freek $\rightarrow$ \\
 Conclusie: Hier kun je onder andere wat "future work" kwijt. Hoe zouden jullie jullie tool verder ontwikkelen. Stel we hebben een nieuw ABI groepje, wat zou deze kunnen doen en wat raden jullie hun aan? Staar je niet blind op pagina nummers, 1 pagina is ook goed.
\end{tcolorbox}

\todo[inline]{
Conclusion in progress by stefan
}

When it comes at the end of a project like this, it is always a good practice to
resume and see if the requirements of the customer were met or what is still
missing. It is not only important to handover a nice and technically correct
product but it must satisfy the customer and match the expectations.


One of our customer main requirements was to have a platform independent xMAS toolkit.
The new xMAS toolkit is developed and tested in a multi platform environment.
It is exhaustive tested on Linux and Windows while MacOs is used to try-out the releases.

To meet the maintainability requirement we've chosen Qt as integrated development
environment. Qt is a powerful open source platform independent IDE and its QML
declarative UI language enforces separation of view and control. The Qt IDE is
an all-in-one solution with a lot of support available.


Integration of formal verification tools is implemented via a standard plug-in interface.
In this way it is possible to develop new or re-factor existing verification tools independently.
Once a verification tool has the right plug-in interface it can be used in the xMAS designer.


\vspace{0.5 cm}

\begin{tcolorbox}[colback=white]
The main requirements of the new xMAS designer toolkit:
\begin{itemize}
\item \textbf{It is platform independent.}
\item \textbf{Maintainability is improved.}
\item \textbf{Integrate verification tools via a standard plug-in interface}
\end{itemize}
\end{tcolorbox}

The latest release, version 0.6, has still some bugs or things that can be fixed.
Both can be found in the appendix of this document or in the Agilefant backlog.
In the appendix there's also a list of possible features that can be added.


Hopefully this can boost future releases to the next level













