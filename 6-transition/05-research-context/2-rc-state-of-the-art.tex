\section{State of the art}

While SoC's are used from consumer to critical applications and the number of
IP's rises, new challenges are awaiting. Challenges where one has to compromise
between performance and reliability. E.g. if one could prove the absolute
correctness of a NoC than there would be no need to add communication fabric or
protocol overhead to recover or avoid errors. But that's not the case the
questions is finding the right balance between runtime and design-time
correctness achievements. While formal verification techniques have made a lot
of progress in proving deadlock freedom there is still some future work e.g. for
finding a deadlock freedom prove at bit level. The same hold to prove liveness,
although it is practical impossible to prove so, some experiments show that is
possible for a particular type of liveness called ``progress'' which is a form
of deadlock freedom. \cite{Ray:2012:SPV:2492708.2492936,itp}

Other verification techniques like those use by the ForEver framework addresses
the correctness of a NoC by attacking the problem both at design-time and at
runtime. With a minimal of overhead, this kind of approach is capable of
handling large networks which is difficult to do by formal verification.
\cite{Parikh:2014:FCF:2597868.2514871}

Other improvements like those of the communication fabrics will not only help to
gain bandwidth but facilitates in mechanisms to achieve correctness by avoidance
or recovery. Reconfigurable NoC of BiNoC, where channels can dynamically
reconfigure themselves so that these work bidirectional instead of the
unidirectional communication fabrics. The flow direction at each channel is
controlled by a channel direction control (CDC) algorithm. Implemented with a
pair of finite state machines, this CDC algorithm is shown to be high
performance, free of deadlock, and free of starvation \cite{5715603}

3D optical communication is also a promising technique it creates the
possibility for massive degree of integration in a single chip. Vertical optical
channels are used to interconnect the layers and have many improvements over
traditional copper-based SoC's. It is interesting to think about the relation
with formal verification. A question could be of how to deal with massive
networks, can these still be modelled by a human or what about the time it takes
to calculate.

To conclude, the importance of verification during design time remains and wil
always have benefits over realtime verification, no matter how massive the
challenge becomes. It is rather a question of how to deal with large networks in
formal verification. \cite{4542033,5306588}


\newpage

