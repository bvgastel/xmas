\section{State of the art}

As SoC's are used from consumer to critical applications and the number of IP's
rises, new challenges are awaiting. Chip designers have to compromise between
performance and reliability. For example, if one could prove the absolute
correctness of an NoC then there would be no need to add communication fabric or
protocol overhead to recover or avoid errors. The questions is finding the right
balance between runtime and design-time correctness achievements. While formal
verification techniques have made a lot of progress in proving deadlock freedom
there is still some future work e.g. for finding a deadlock freedom proof at bit
level. The same holds to prove liveness, although it is practically impossible to
prove so, some experiments show that is possible for a particular type of
liveness called ``progress'' which is a form of deadlock absence.
\cite{Ray:2012:SPV:2492708.2492936,itp}


\paragraph{Modelchecking (TODO)}
\textit{Modelchecking is een formele verificatie techniek door de volledige industrie
(hardware en software) is omarmd. Model checking heeft echter last van het
statespace explosion problem: vaak kan het grote problemen (in termen van het
aantal toestanden van het systeem) niet aan. Als jullie geen interessante
referenties kunnen vinden hierover, dan kan ik jullie daarbij helpen.}


\paragraph{ForEver}is a verification framework that addresses the correctness
of a NoC by attacking the problem both at design-time and at runtime. With a
minimal of overhead, this kind of approach is capable of handling large networks
which is difficult to do by formal verification.
\cite{Parikh:2014:FCF:2597868.2514871}

\paragraph{Improvements}of the communication fabrics will not only help to
gain bandwidth but facilitates in mechanisms to achieve correctness by avoidance
or recovery. Reconfigurable NoC of BiNoC, where channels can dynamically
reconfigure themselves so that these work bidirectionally instead of the
unidirectional communication fabrics. The flow direction at each channel is
controlled by a channel direction control (CDC) algorithm. Implemented with a
pair of finite state machines, this CDC algorithm is shown to be high
performance, free of deadlock, and free of starvation \cite{5715603}

\paragraph{3D optical communication}is also a promising technique. It creates the
possibility a massive degree of integration in a single chip. Vertical optical
channels are used to interconnect the layers and have many improvements over
traditional copper-based SoC's. It is interesting to think about the relation
with formal verification. Questions could be: How to deal with a massive
network? Can these still be modeled by a human? How long will it take to verify
such networks? \cite{4542033,5306588}

\paragraph{Conclusion: }The importance of verification during design time remains and will
always have benefits over real-time verification, no matter how massive the
challenge becomes. It is rather a challenge of how to deal with massive networks in
formal verification.


\newpage

