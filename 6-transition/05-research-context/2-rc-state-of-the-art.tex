\section{State of the art}

As SoC's are used from consumer to critical applications and the number of IP's
rises, new challenges are awaiting. Chip designers have to compromise between
performance and reliability. For example, if one could prove the absolute
correctness of an NoC then there would be no need to add communication fabric or
protocol overhead to recover or avoid errors. The questions is finding the right
balance between runtime and design-time correctness achievements. While formal
verification techniques have made a lot of progress in proving deadlock freedom
there is still some future work e.g. for finding a deadlock freedom proof at bit
level. The same holds to prove liveness, although it is practically impossible to
prove so, some experiments show that is possible for a particular type of
liveness called ``progress'' which is a form of deadlock absence.
\cite{Ray:2012:SPV:2492708.2492936,itp}


\paragraph{Model checking}\cite{baier2008principles}

Defects must be detected as soon as possible. For example, the costs of repairing
a software flaw during maintenance are roughly 500 times higher than a fix in an
early design phase. In a hardware design it is even more critical because it is
difficult or even impossible to repair an error afterward.

Formal verification methods can be used to check the correctness of a model.
This activity can be done during design time and is very suitable to detect
defects in an early stage of the development process. Exhaustive testing or peer
reviews can only prove an error its presence not its absence. Automated formal
verification methods can prove the absence or presence of a fomral property. For
example, if a model is checked on deadlock freedom, formal verification can
prove if this is true.
Some formal properties:

\begin{itemize}
\item deadlock freedom : A terminal state can be reached.
\item liveness : A state can be reachable whenever it is necessary.
\item fairness : No path will constantly exclude other paths leading to some
state. (starvation freedom)
\end{itemize} 

Model checking is a verification technique that explores all possible system
states in a brute-force manner. If the number of model components raise, the
number of possible states will increase exponential and so the space necessary
to hold these states. This is known as the state-space explosion problem.

The strengths of model checking:
\begin{itemize}
\item Can be used for a wide range of applications.
\item Supports partial verification to focus on essential properties first.
\item It proves absence of an error not the presence.
\item Provides diagnostic information.
\item Easy accessible ``push-button'' technology.
\item Interest by industry.
\item It can be easily integrated in existing development cycles.
\item It has a sound and mathematical underpinning.
\end{itemize} 

The weaknesses of model checking:
\begin{itemize}
\item State-space explosion problem, not applicable for data-intensive or
infinite-state systems.
\item Verification result is only as good as the system model.
\item Requires some expertise to interpret the results and to adopt the model.
\item The tools may contain defects itself.
\item It does not allow checking generalizations.
\end{itemize} 


\paragraph{ForEver}is a verification framework that addresses the correctness
of a NoC by attacking the problem both at design-time and at runtime. With a
minimal of overhead, this kind of approach is capable of handling large networks
which is difficult to do by formal verification.
\cite{Parikh:2014:FCF:2597868.2514871}

\paragraph{Improvements}of the communication fabrics will not only help to
gain bandwidth but facilitates in mechanisms to achieve correctness by avoidance
or recovery. Reconfigurable NoC of BiNoC, where channels can dynamically
reconfigure themselves so that these work bidirectionally instead of the
unidirectional communication fabrics. The flow direction at each channel is
controlled by a channel direction control (CDC) algorithm. Implemented with a
pair of finite state machines, this CDC algorithm is shown to be high
performance, free of deadlock, and free of starvation \cite{5715603}

\paragraph{3D optical communication}is also a promising technique. It creates the
possibility a massive degree of integration in a single chip. Vertical optical
channels are used to interconnect the layers and have many improvements over
traditional copper-based SoC's. It is interesting to think about the relation
with formal verification. Questions could be: How to deal with a massive
network? Can these still be modeled by a human? How long will it take to verify
such networks? \cite{4542033,5306588}

\paragraph{Conclusion: }The importance of verification during design time remains and will
always have benefits over real-time verification, no matter how massive the
challenge becomes. It is rather a challenge of how to deal with massive networks in
formal verification.


\newpage

