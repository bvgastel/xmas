\section{Introduction}

\paragraph{Background} \cite{SoC-market}

Today we all know that personal computers do not rule the chip industry's market
anymore but the more by small intelligent devices. If we look into the
near future upcoming technologies like the internet of things, intelligent cars
or drones will overrule all previous market demands for chip industries. Common
properties of such devices made chip producers already shifting their challenges
from transistor level to core level wich are:

\begin{enumerate}
\item Integration of multiple powerful functionalities into one small device
\item Energy efficiency, because of power dissipation in nanoscale VLSI circuits. 
\item Cost reduction due the growth of chip producing competitors in these markets.
\item Time to market improvement by combining intellectual properties into one product.
\end{enumerate}

This background illustrates the importance of further integration and the
related technologies that makes this possible. These technologies are introduced
in the next sections. 

\paragraph{SoC}

An integrated circuit that covers multiple cores or intellectual properties
blocks (IP's) which are linked to an internal bus system is called a system on
chip or SoC. IP's of several companies are integrated into a single chip. So the
SoC producer can focus on its core business instead of building all the
functionality by themselves and reduce time to market. But the more IP's attached
to a traditional bus system the wiring delay increases exponential and longer
wiring reduces the bus its bandwidth.\cite{SoC}

\paragraph{NoC}

The use of network communication inside a chip instead of a traditional bus
system is inevitable. One of the reasons is to cope with the increasing number
of cores into a single chip. It is also less complex to integrate IP's from
different companies. A so called network on chip or NoC has much less wiring
\cite{NoC-busses} and can handle more IP's without losing performance. For the
related business it has the same advantage as it does for network communication
in general:

\textit{``Replacement of SoC busses by NoCs will follow the same path of data
communications when the economics prove that the NoC either reduces SoC
manufacturing cost, SoC time to market, SoC time to volume, and SoC design risk
or increases SoC performance.''} \cite{NoC-busses} 

The success of the NoC design depends on the research of the interfaces between
processing elements of NoC and interconnection fabric.

\paragraph{xMAS}

If a NoC does not meet its specification or isn't reliable e.g. because of
deadlock situations in the network, the cost of production loss is much higher
than the costs spent for the extra design or verification effort. In worst case
a company can lose its market share and reputational damage. It is very
important to detect flaws in an early stage of NoC production process, therefor
researchers have developed a high level modelling language for communication
fabrics called xMAS or executable Micro Architectural Specifications. This high
level approach makes modelling less complex and gives researchers a way to gain
knowledge of how these fabrics behave under certain conditions so they can prove
the absence or presence of specific properties long before it's built on
silicon.

xMAS consists of only eight primitive components. Each component has one or more
ports. To create a valid model all ports must be connected by channels. Some
components have specific properties to set. These properties are used by the
verification tools. E.g. the queue has a size property. Once a model has been
created and all components are set up it can be verified to detect deadlock
situations or other flaws.

\begin{figure}[here]
\includegraphics[width=1.0\textwidth]{xmas-language}
\caption{Eight primitives of the xMAS language \cite{6225465}. Italicized letters indicate
parameters. Gray letters indicate ports.}
\label{fig:xmas-language}
\end{figure}

\paragraph{Design and verification tools}

To make use of the xMAS language and verification tools, computer scientists
have developed an application called WickedXMAS \cite{WickedXmas}.
With this application it is possible to draw a model and export it so it can be
verified. The verification tool reads the model and processes one or more
algorithms e.g. to detect network deadlocks or model syntax faults.

Although the WickedXMAS designer tool is still usefull and unique in its kind
there are some major shortcomings and had to be redesigned because:
\begin{enumerate}
\item It was written in csharp which makes it only available for researchers
working on a Windows platform.
\item Verification tools are not integrated which
makes it complicated to use.
\item The model must be exported before it can be
used by the verification tools.
\item It is difficult to maintain, verification tools are written in c++ while
designer is written in csharp and xaml
\item The tool aborts in several situations.
\item Lack of composite management.
\item Not working recursive composite feature.
\end{enumerate}

Therefor our team was asked to design and build a new tool. Our goal was to
create a maintainable, platform independent, xMAS modelling tool that integrates
the verification tools. The project is split into a designer that we call
``xmd'' or xMAS Model Designer and the verification process called ``xmv'' or
xMAS Model Verification. The project is based on Qt's latest technology where we
have written the GUI in QML (JavaScript) and the logic in c++.

Instead of implementing specific verification tools we have put our effort into
implementing a generic plugin interface. Via this interface, verification tools
can be easily plugged in and controlled. This interface can also send the
verification results to the console.
\begin{wrapfigure}{r}{0.55\textwidth}
  \vspace{-20pt}
  \begin{center}
    \includegraphics[width=0.50\textwidth]{console}
  \end{center}
  \vspace{-20pt}
  \caption{xmd syntax checker plugin console}
  \label{fig:console}
  \vspace{-10pt}
\end{wrapfigure}
A scientist can easily implement a new verification algorithm if it has the
plugin interface. We have provided the syntax checker of such a plugin interface
that can be used as an example for other verification tools. Each plugin
automatically gets its own console output and control with setup fields.
Starting a verification process is simply done by clicking the start button, no
conversion or exports are necessary anymore.


In the new designer ``xmd'' all actions on the canvas are now directly reflected
into the model network and can be verified immediately. Another improvement of
the designer is the management of composite components, which are subnetworks
that can be reused and make it possible to quickly build large models in an easy
way.
\begin{wrapfigure}{r}{0.55\textwidth}
  \vspace{-20pt}
  \begin{center}
    \includegraphics[width=0.50\textwidth]{composite-use}
  \end{center}
  \vspace{-20pt}
  \caption{xmd composite library and canvas use}
\label{fig:composite-use}
  \vspace{-10pt}
\end{wrapfigure}
A model can be setup with composite properties so it can be used just
like a primitive. A designer can do this by adding them to a model its composite
library and drag those into the canvas. The way that xmd implements composites
gives scientists the opportunity to extend these with a parametric expression.
With a parametric expression it is possible to call a composite in a recursive
way and avoid drawing large networks of already valid subnets.

\newpage