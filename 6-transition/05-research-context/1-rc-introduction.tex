\section{Introduction}

\paragraph{Background} \cite{SoC-market}

The chip industry's focus has shifted from personal computing to a plethora of
small, intelligent devices. The Internet of things, intelligent cars, smart
phones and drones are only a few examples of the devices that draw the chip industry's
attention more and more. Some common properties of such devices that force chip
producers to shift their attention from transistor level to core level are:

\begin{enumerate}
\item The need for integration of multiple powerful functionalities into a small device
\item The need for power efficiency, because of power dissipation in nanoscale VLSI circuits. 
\item The need for cost reduction due the growth of chip producing competitors in these markets.
\item The need for improvement of time to market using combined intellectual properties in a product.
\end{enumerate}

These issues illustrate the importance of further integration and the related
technologies that makes this possible. The next paragraphs introduce these
technologies.

\paragraph{SoC}

An integrated circuit that covers multiple cores or intellectual properties
blocks (IP's) linked to an internal bus system is called a system on chip or
SoC. A single chip often integrates IP's of several companies. It is this IP
integration that allows the SoC producer to focus on its core business, removing
the need to develop the functions. Additionally this decreases development risks
and reduces time to market.

However, with more IP's attached to a traditional bus system, the wiring delay increases 
and longer wiring reduces the bus's bandwidth.\cite{SoC}

\paragraph{NoC}

The use of network communication inside a chip instead of a traditional bus
system is inevitable. One of the reasons is to cope with the increasing number
of cores into a single chip. Using a NoC makes it less complex to integrate IP's from
different companies. A so called network on chip or NoC has much less wiring
\cite{NoC-busses} and can handle more IP's without losing performance. For the
related business it has the same advantage as it does for network communication
in general:

\textit{``Replacement of SoC busses by NoC's will follow the same path of data
communications when the economics prove that the NoC either reduces SoC
manufacturing cost, SoC time to market, SoC time to volume, and SoC design risk
or increases SoC performance.''} \cite{NoC-busses} 

The success of the NoC design depends on the research of the interfaces between
processing elements of NoC and interconnection fabric.

\paragraph{xMAS}

If an NoC does not meet its specification or isn't reliable e.g. because of
deadlock situations in the network, the cost of production loss is much higher
than the costs spent for the design or verification effort. Worst case a
company can lose its market share and incur reputational damage.
It is very important to detect flaws in an early stage of NoC production
process, therefore researchers have developed a high level modeling language for
communication fabrics called xMAS or executable Micro Architectural
Specifications. This high level approach makes modeling less complex and gives
researchers a way to gain knowledge of how these fabrics behave under certain
conditions so they can prove the absence or presence of specific properties long
before it's built on silicon.

xMAS consists of only eight primitive components. Each component has one or more
ports. To create a valid model all ports must connect to other ports in channels. Some
components have specific properties to set. These properties are used by the
verification tools. For example, the queue has a ``size'' property. Once a model
has been created and all components are set up it can be verified to detect
deadlock situations or other flaws.

\begin{figure}[here]
\includegraphics[width=1.0\textwidth]{xmas-language}
\caption{Eight primitives of the xMAS language \cite{6225465}. Italicized letters indicate
parameters. Gray letters indicate ports.}
\label{fig:xmas-language}
\end{figure}

\paragraph{Design and verification tools}

To make use of the xMAS language and verification tools, computer scientists
have developed an application called WickedXMAS \cite{WickedXmas}.
With this application it is possible to draw a model and export it so it can be
verified. The verification tool reads the model and processes one or more
algorithms e.g. to detect network deadlocks or model syntax faults.

Although the WickedXMAS designer tool is still useful and unique in its kind
there are some major shortcomings like:
\begin{enumerate}
\item It was written in csharp which makes it only available for researchers
working on a Windows platform.
\item Verification tools are not integrated which
makes it complicated to use.
\item The model must be exported before it can be
used by the verification tools.
\item It is difficult to maintain, verification tools are written in multiple languages
including c++ and Haskell while the designer is written in csharp and xaml
\item The tool aborts in several situations.
\item No management of composite components.
\item Recursive feature for the composite component is not working.
\item There is minimal and outdated documentation available.
\end{enumerate}

Therefore our team was asked to design new tool. Our goal was to
create a maintainable, platform independent, xMAS modeling tool that integrates
the verification tools.
By integrating formal verification into a new designer-friendly tool, a designer
can easily verify a design.

The project is split into a designer that we call
``xmd'' or xMAS Model Designer and the verification process called ``xmv'' or
xMAS Model Verification. The project is based on Qt's latest technology where we
have written the GUI in QML (JavaScript) and the logic in c++.

Instead of implementing specific verification tools we have put our effort into
implementing a generic plug-in interface. Via this interface, verification tools
can be easily plugged in and controlled. This interface can also send the
verification results to the console.
\begin{wrapfigure}{r}{0.55\textwidth}
  \vspace{-20pt}
  \begin{center}
    \includegraphics[width=0.50\textwidth]{console}
  \end{center}
  \vspace{-20pt}
  \caption{xmd syntax checker plug-in console}
  \label{fig:console}
  \vspace{-10pt}
\end{wrapfigure}
A scientist can easily implement a new verification algorithm if it has the
plug-in interface. We have provided the syntax checker of such a plug-in interface
that can be used as an example for other verification tools. Each plug-in
automatically gets its own console output and control with setup fields.
Starting a verification process is simply done by clicking the start button, no
conversion or exports are necessary anymore.


In the new designer tool all canvas actions are directly reflected into the
network model and can be verified immediately. Another improvement of the
designer is the management facility for composite components. These components
are subnetworks that can be reused. This makes the design of large models less
complicated and maintainable.

\begin{wrapfigure}{r}{0.55\textwidth}
  \vspace{0pt}
  \begin{center}
    \includegraphics[width=0.50\textwidth]{composite-use}
  \end{center}
  \vspace{-10pt}
  \caption{xmd composite library and canvas use}
\label{fig:composite-use}
  \vspace{0pt}
\end{wrapfigure}
A model can be setup with composite properties so it can be used just like a
primitive. A designer can do this by adding them to a model its composite
library and drag those into the canvas. The way that xmd implements composites
gives scientists the opportunity to extend these with a parametric expression.
With a parametric expression it is possible to call a composite component in a
recursive way. Instead of adding a composite component for each use in the model
it can be done with only one that has been provided with a parametric expression.

\newpage