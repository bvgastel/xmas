\section{Feedback interface}

While analyzing a network, verification tools can generate various types of
feedback messages. At this moment, verification tools present this feedback
as human readable text messages on the standard output. Although this form
is suitable for human consumption when running the verification tools as a
standalone application, they are hard to parse and interpret by software like
the designer.

Due to time constraints, presentation of feedback from the verification
tools has been kept simple. Similar to the standalone application, the
text messages are presented to the user in a text console integrated in
the designer. Future improvements could enhance the integration between
designer and verifications. As an example, problematic components and ports
could be highlighted on the design canvas.

\paragraph{}
To support these kinds of enhancements, a more structured way to pass feedback
from verification tools must be used. For this purpose, a feedback message
format has been defined along with several functions that can be used 
by verification tools to generate feedback messages.


\subsection{Message format}
Feedback messages are structured as follows:

\paragraph{}
[\textbf{sender}] **\textbf{type}**: \textbf{message}
(\textbf{component list}) (\textbf{port list})


\begin{itemize}
    \item \textbf{sender} is a string to identify the verification tool that
    sends the feedback message. Each verification tool should define a unique
    name to identify itself.
    
    \item \textbf{type} specifies the type of feedback. The following feedback
    types are defined:
    \begin{itemize}
        \item \textbf{INFO} - informational message, e.g. verification tool has started or finished
        \item \textbf{WARNING} - non-fatal warning, e.g. non-optimal network configuration detected
        \item \textbf{FAULT} - fault in the network detected, e.g. a combinatorial cycle
        \item \textbf{CRASH} - crash / unexpected exception thrown by the verification tool,
        this indicates a programming error, not a network fault
        \item \textbf{PROGRESS} - informational message containing progress of a verification tool
    \end{itemize}

    \item \textbf{message} contains the actual feedback message. Verification tools can pass
    custom text messages. For PROGRESS type feedback, the message field has a fixed format:
    ``progress/total'' where both progress and total are positive integers.
    
    \item \textbf{component list} is an optional list of component IDs to which the feedback
    applies. The components are listed inside brackets using comma-separated notation and are
    prefixed by ``Comp: '', e.g. Comp: [src1, join2, ...].

    \item \textbf{port list} is an optional list of ports (of components) to which the feedback
    applies. The ports are listed inside brackets using comma-separated notation and are
    prefixed by ``Port: '', e.g. Port: [src1.o, join2.a, ...].
    
    \item messages are terminated by a newline character
    
\end{itemize}


\subsection{Example}

\begin{verbatim}
[cycle-checker] **INFO**: Starting cycle checker
[cycle-checker] **FAULT**: Combinatorial cycle detected!  Port: [::frk0.a]
[cycle-checker] **INFO**: Finished! 
\end{verbatim}


\subsection{Current state \& future work}

The ``cycle checker'' verification tool has been adapted to output its feedback in the
described format. Optionally, this tool could be updated to emit progress information.
However, progress information is of little value due to the short runtime of this tool.

As mentioned earlier, the designer does not interpret the feedback message structure yet.
Extracting the information from the text string can be accomplished through the use of a
regular expression. All feedback formatting code is located in the feedback-interface.h/cpp
files. Changes to the message format and new code to parse the message string can be added
here.