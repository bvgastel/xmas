\section{Changes to the data model}

Due to requirements imposed by the designer, several changes and additions have
been made to the data model. Most of these changes are related to the additional
support for hierarchical networks in the designer. This section present an overview
of new and modified classes and functions.

\subsection{XMASNetwork}

XMASNetwork is a new class designed to represent an xMAS network. This class can be
used to model both flat and hierarchical networks. When used to model hierarchical
networks, an XMASNetwork represents a single level or subnetwork in the hierarchy.
Multiple networks are combined to form the complete network.

The main responsibility of XMASNetwork is to serve as a container of XMASComponent
instances. Internally, XMASNetwork stores the components in a map relating a
components name to its in-memory instance. Prior to the introduction of XMASNetwork,
the concept of an xMAS network was directly represented by such a map. Verification
tools still use this approach, although they could be easily adapted to use
XMASNetwork as well.

\paragraph{}
The 'promotion' of XMASNetwork to its own class definition has two primary reasons:
\begin{itemize}
 \item The designer uses additional network properties like the canvas size. These
 properties must be stored somewhere in the data model (i.e. in XMASNetwork).
 \item Support for hierarchical networks requires management of multiple network
 models. Using an explicit class to represent networks eases this task.
\end{itemize}


\subsection{XMASComposite}

\subsection{XMASProject}

\subsubsection{Loader}

\subsection{Flattening the network}

\subsection{Changes to the xmas file format}

\section{Feedback interface (not data model related, move to other file?)}

\newpage

