\section{XMAS network file format}\label{xmas-network-file-format}

This document describes the file format used to store XMAS network
models. Both the XMAS designer application and the standalone
verification tools use this file format. The structure is based on the
flat json format used by the checker application provided at the start
of this project. Several new fields have been added in order to support
hierarchical networks and information specific to the designer. As the
new file format is able to support both flat and hierarchical networks,
the (recommended) file extension has been changed from fjson to json.

\subsection{Root JSON object}\label{root-json-object}

Properties:

\begin{itemize}
\itemsep1pt\parskip0pt\parsep0pt
\item
  ``CANVAS'' : the \textsc{Canvas} object (optional) \emph{NEW}
\item
  ``COMPOSITE\_NETWORK'' : the \textsc{Composite\_Network} object (optional)
  \emph{NEW}
\item
  ``VARS'' : the \textsc{Vars} object
\item
  ``PACKET\_TYPE'' : the \textsc{Packet\_Type} object
\item
  ``NETWORK'' : an array of \textsc{Component} describes all components in
  the network
\end{itemize}

\subsection{CANVAS}\label{canvas}

\textbf{\emph{NEW}} Used by the XMAS designer application to define
properties of the designer canvas. When no \textsc{Canvas} information is
specified, default values are used.

Properties:

\begin{itemize}
\itemsep1pt\parskip0pt\parsep0pt
\item
  ``width'' : width of the canvas in logical units
\item
  ``height'' : height of the canvas in logical units
\end{itemize}

\subsection{COMPOSITE\_NETWORK}\label{compositeux5fnetwork}

\textbf{\emph{NEW}} Contains designer specific information to use this
network as a composite object. If this information is not present, the
network cannot be used as a composite object.

Properties:

\begin{itemize}
\itemsep1pt\parskip0pt\parsep0pt
\item
  ``alias'' : displayed inside a composite object to denote the objects
  type (e.g.~mesh)
\item
  ``image-name'' : name of graphical resource used as a symbol to denote
  the objects type (e.g.~mesh.ico)
\item
  ``boxed-image'' : set to 1 to use the image as a symbol inside a
  generic composite object (drawn boxed) or set to 0 to use the image as
  the graphical representation of the entire component, this can be used
  to draw common macro's like credit counters and delays using the
  symbols commonly used in literature to denote these macro's.
\item
  ``packet'' : The verification system does not use this information yet, 
  				although the user interface does allow entering a value.
  				It's value is reserved for later use.
\end{itemize}

\subsection{VARS}\label{vars}

\emph{TODO}\footnote{\label{fileformat-todo}The semantics of this field 
were not changed from the original file format.
It is up to the customer to define the contents of this field.}
Definition taken from original file format.

\subsection{PACKET\_TYPE}\label{packettype}

\emph{TODO}\footref{fileformat-todo}
Definition taken from original file format.

\subsection{COMPONENT}\label{component}

Describes a component in the network.

Properties:

\begin{itemize}
\itemsep1pt\parskip0pt\parsep0pt
\item
  ``id'' : \textsc{String} - unique identifier
\item
  ``type'' : \textsc{Component\_type} enum - type of the component
\item
  ``outs'' : array of \textsc{Out} - describes all connected output
  ports
\item
  ``fields'' : array of \textsc{Field} \emph{(optional)}
\item
  ``pos'' : \textsc{Position} - \emph{new} - position of the component
  on the canvas
\end{itemize}

\subsection{Component\_type}\label{componenttype}

Enumeration:

\begin{itemize}
\itemsep1pt\parskip0pt\parsep0pt
\item
  source
\item
  sink
\item
  function
\item
  queue
\item
  xfork
\item
  join
\item
  xswitch
\item
  merge
\item
  composite \textbf{\emph{NEW}}
\end{itemize}

\subsection{COMPONENT {[}type=``source'',
type=``sink''{]}}\label{component-typesource-typesink}

Properties:

\begin{itemize}
\itemsep1pt\parskip0pt\parsep0pt
\item
  ``required'' : \textbf{Boolean 1/0} - set to 1 when this source/sink
  is an interface port of a composite network that must be connected by
  the higher level network. set to 0 when this is optional.
\end{itemize}

\subsection{COMPONENT
{[}type=``composite''{]}}\label{component-typecomposite}

\textbf{\emph{NEW}}

Properties:

\begin{itemize}
\itemsep1pt\parskip0pt\parsep0pt
\item
  ``subnetwork'' : \textsc{String} - name of the subnetwork
\item
  (``parameters'' : (Not determined yet) - reserved property
  name, to be used in the future to parameterize the network)
\end{itemize}

Description: The subnetwork reference indicates the relative location of
the subnetwork on the filesystem.

E.g. ``mesh.json'' refers to the network defined in ``mesh.json'' in the
same directory as this network and ``spidergon/node.json'' refers to the
network defined in ``node.json'' in the subdirectory ``spidergon''.

Parameterization of composite objects has not been implemented yet. When
implemented, composite objects require an additional field in the
component description to store the parameters. The property name
``parameters'' has been reserved for this purpose.

\emph{Note: Currently the XMAS designer requires that all used composite
networks are defined in the same directory as the root network.}

\subsection{OUT}\label{out}

Description: Describes channels between the components in the network.
For each output port of a component, an \textbf{OUT} object describes to
which input port of which component it is connected.

Properties:

\begin{itemize}
\itemsep1pt\parskip0pt\parsep0pt
\item
  ``id'' : \textsc{String} - reference to the target component
\item
  ``in\_port'' : \textsc{Number} - index of the input port on the target
  component
\end{itemize}

\subsection{FIELD}\label{field}

\emph{TODO}\footref{fileformat-todo}
Definition taken from original file format

\subsection{POSITION}\label{position}

\textbf{\emph{NEW}} Stores positional data of the component on the
canvas.

Properties:

\begin{itemize}
\itemsep1pt\parskip0pt\parsep0pt
\item
  ``x'' : \textsc{Number} - x or horizontal position on the canvas
\item
  ``y'' : \textsc{Number} - y or vertical position on the canvas
\item
  ``orientation'' : \textsc{Number} - orientation of the component,
  measured in degrees clockwise.
\item
  ``scale'' : \textsc{Number} - scale factor of the component
\end{itemize}
