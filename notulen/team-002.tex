\documentclass{article}

\usepackage[dutch]{babel}
\usepackage{minutes}

\title{Notulen 002 overleg team 33}
\author{Jeroen}
\minutesstyle{header=list, vote=list, contents=true}

\begin{document}
%\selectlanguage{dutch}

\begin{Minutes}{Overleg 002 team 33}
\participant{Guus Bonnema, Stefan Versluys, Jeroen Kleijn, Freek Verbeek, Bernard van Gastel}
\minutesdate{20. September 2014}
\location{skype}

\maketitle% This is where LaTeX makes the title

\topic{Voorstellen teamleden en kennismaken opdrachtgever}

De leden van het projectteam en de opdrachtgevers stellen zich kort voor. De rolverdeling tussen Bernard en
Freek wordt duidelijk gemaakt. De uiteindelijke klant is Bernard, voor inhoudelijke zaken is hij het aanspreekpunt.
Freek is de begeleider van het project zelf. Met hem zal het team communiceren over planning en andere projectmatige zaken.

Doel is om het project in 8 maanden af te ronden, in ieder geval voor de zomervakantie.

\topic{Inleiding project}

Bernard geeft een inleiding op het project. In eerdere projecten is een tool ontworpen voor het ontwikkelen van zogenaamde
NoC processoren. Een moderne aanpak voor het ontwerpen van processoren met complexe communicatienetwerken tussen verschillende
componenten van de chip. Hiervoor wordt de door Intel ontwikkelde taal xMAS gebruikt. Met behulp van deze taal kan men op
hoog niveau een NoC ontwerp maken. De WickedXmas tool maakt het mogelijk deze ontwerpen te visualiseren, bewerken en analyzeren.
Voor deze laatste mogelijkheid zijn in de loop van de tijd enkele tools ontwikkeld.

Tijdens het gebruik van de tool zijn een aantal problemen/limitaties aan het licht gekomen. Het doel van het project is om
de WickedXmas tool te verbeteren door deze problemen aan te pakken. De wijze waarop is nog niet vastgelegd, dit kan zowel door
de huidige tool te refactoren of door op basis van de opgedane ervaringen met WickedXmas een geheel nieuwe tool te bouwen.

De WickedXmas tool in de huidige vorm is het resultaaat van meerdere projecten waarbij de functionaliteit steeds verder is
uitgebreid. Oorspronkelijk diende de tool enkel voor het visueel ontwerpen. Later zijn ook mogelijkheden toegevoegd voor het
analyzeren van deze ontwerpen. Integratie van deze nieuwe analysetools blijkt suboptimaal te zijn, uitbreiding van de ontwerptool
met nieuwe functionaliteit gaat lastig. De oorspronkelijke softwareachitectuur van de ontwerptool is kortom niet geschikt voor
toekomstige uitbreidingen en moet worden herzien.

\topic{Verbeterpunten}

Om de tool te verbeteren zijn onder andere de volgende problemen aangegeven:
\begin{itemize}
 \item 1. Een ontwerp kan op dit moment alleen in zijn geheel worden beschouwd.
Het analyseren van deelontwerpen is niet mogelijk. Dit zorgt ervoor dat het analyseren 
langer duurt dan nodig is en maakt het bestuderen van het ontwerp bovendien complexer
doordat de focus niet op een beperkt deel kan worden gelegd.
 \item 2. De editor ondersteunt niet alle gewenste mogelijkheden waardoor er soms nog met de hand JSON bestanden moeten worden bewerkt.
 \item 3. Integratie met analysetools kan beter, de tools zijn nu duidelijk losstaande programma's. Hechtere integratie waarbij direct
 feedback van de analysetools zichtbaar is in de UI van de ontwerptool is wenselijk.
 \item 4. De tool is gemaakt voor gebruik op het Windows platform. Veel onderzoekers werken echter met een Mac-systeem.
 Een cross-platform oplossing verdient dus de voorkeur.
 \item 5. WickedXmas is geschreven in C\# terwijl de analysetools in c/c++ zijn geschreven. Communicatie (bijv. datastructuren) tussen
 ontwerptool en analysetool is daardoor lastig. (Punten 3 en 4 zijn deels een gevolg van dit punt)
\end{itemize}

De functionaliteit van de ontwerptool zelf is voorlopig voldoende. Het resultaat van het project hoeft bij nieuwbouw geen exacte (pixel-perfect) kopie te zijn.
Het belangrijkste punt is de integratie met analysetools. Verder moet de documentatie op orde zijn en is een cross-platform oplossing wenselijk. Bij het opstellen
van de planning zal gekeken worden welke functionaliteit in de beschikbare tijd kan worden gerealiseerd.

\topic{Planning}

De eerstvolgende stap is het opstellen van een planning. Geprobeerd wordt om deze planning binnen een termijn van 2-3 weken af te hebben.
Gedurende deze 2-3 weken zal er ongeveer 3 keer per week een kort overleg (15 min.) plaatsvinden. In eerste instantie zal worden vastgesteld
welke projectaanpak zal worden gebruikt. In latere overleggen zal op basis van de eerder genomen besluiten de planning steeds verder worden uitgewerkt.
De invulling van de planning is vrij, het voorbeeld op studienet kan worden gebruikt, maar dit is niet verplicht. Freek geeft het advies om eerst
de domeinanalyse uit te voeren voordat begonnen wordt met het schrijven van code. Dit om te voorkomen dat de domeinanalyse als mosterd na de maaltijd komt.
Verder geeft hij aan dat het altijd mogelijk is bij problemen de begeleider(s) te raadplegen.


\topic{Besluiten}

\begin{itemize}
 \item Bernard zal relevante documentatie doorsturen naar de teamleden, ook zal hij toegang verlenen tot de source code van de tools
 \item Binnen 3 weken zal de planning worden gemaakt. Om de 2-3 dagen zal een kort overleg via skype worden gehouden. Tussendoor vindt contact via e-mail plaats.
 \item Het eerstvolgende overleg is zondag 21 september om 10:00, punten die dan aan de orde komen zijn:
 \begin{itemize}
  \item de projectaanpak: agile, hybride, anders
  \item de rolverdeling tussen de teamleden
  \item ideeën voor domeinanalyse
 \end{itemize}
 \item Verder zal er overleg via Skype zijn op:
 \begin{itemize}
  \item woensdag 24 september (19:00)
  \item zaterdag 27 september (10:00)
 \end{itemize}


\end{itemize}

\end{Minutes}
\end{document}
