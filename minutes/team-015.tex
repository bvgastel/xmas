\documentclass{article}
\usepackage{subfig}
\usepackage[dutch]{babel}
\usepackage{minutes}
\usepackage{graphicx}
\usepackage{float}

\title{Notulen 015 overleg team 33}
\author{Stefan}
\minutesstyle{header=table, vote=list, contents=true, columns={1}}

\begin{document}
%\selectlanguage{dutch}

\begin{Minutes}{Overleg 015 team 33}
\participant{Guus Bonnema, Jeroen Kleijn, Stefan Versluys}
\minutesdate{24 november 2014}
\location{Skype}

\maketitle% This is where LaTeX makes the title

\topic{Stand van zaken}

Bespreking hoe we nu verder gaan , agile tool , ontwikkelomgeving instellen,
architectuur bepalen waarvoor we nog extra requirements moeten verzamelen.

\topic{Ontwikkelomgeving}

FTLK en Codeblock zijn getest op Linux, Window8 en XP met een klein voorbeeld.
Woensdag a.s. wordt er samen getest op uitwisselbaarheid van een code blocks project.
Omgeving dusdanig instellen dat platform specifieke zaken niet in de repo komen. 


\topic{Architectuur}

Donderdag wordt er gebrainstormd over de architectuur.
De bedoeling is om een idee te hebben over welke mogeijkheden er zijn
en dit mee te nemen op een van de volgende meetings met Freek en Bernard

Vragen die tijdens deze meeting aan bod kwamen zijn : Wat met multithreading,
layer of client-server?
Diagrammen tekenen in Umbrella?

\topic{Agile tool}

Idereen heeft een account gemaakt en er werd een aanzet gegeven tot het instellen van de tool.

\topic{Vragen}

\topic{Agenda}
\begin{itemize}

 \item Het volgende overleg is woensdag 26 november 2014 om 19h  (max 2h). 
 

\begin{itemize}
\item testen ontwikkelomgeving op portabiliteit
\item brainstormen architectuur
\end{itemize}

 \item  Meeting voorzien komende zaterdag met Freek en Bernard i.v.m. architectuur
\end{itemize}


\end{Minutes}
\end{document}
