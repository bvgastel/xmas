\documentclass[a4paper,final]{article}
\usepackage{subfig}
\usepackage[dutch]{babel}
\usepackage{hyperref}
\usepackage{minutes}
\usepackage{graphicx}
\usepackage{float}
\usepackage{color}

\title{Resultaten work meeting 001 architecture}
\author{Guus}
\minutesstyle{header=table, vote=list, contents=true, columns={1}}

\begin{document}
%\selectlanguage{dutch}

\newcommand{\Noc}{\textsc{NoC}\xspace}

\begin{Minutes}{Work meeting 001 Architecture Team 33}
\participant{Bernard van Gastel, Guus Bonnema, Stefan Versluys, Jeroen Kleijn}
\subtitle{Architectuur}
\minutetaker{Guus}
\minutesdate{06 december 2014}
\location{Skype}

\maketitle% This is where LaTeX makes the title

\newcommand{\w}[1]{\textsc{#1}}

\topic{Huidige versie WickedXMas} 

\subtopic{Versies van programmatuur}

\paragraph{Bernard} Het probleem van het huidig programma is dat er meerdere
versies zijn, die dezelfde data net een andere interpretatie geven. Dat wil ik
nu niet meer: 1 versie, 1 programma. Ook de data moet in slechts 1 versie
beschikbaar zijn. Er waren te veel eigenaren, waardoor de ontwikkeling uit
elkaar ging lopen. Ik ben nu de offici\"{e}le eigenaar van het pakket software.

\topic{Communicatie Freek en Bernard}

\paragraph{Bernard} Ik houd Freek op hoofdlijnen op de hoogte. Hij doet 
hetzelfde bij mij. Daarnaast houd ik ook mijn andere collega onderzoekers op
de hoogte.

\paragraph{Github}

Bernard heeft toegang tot github gekregen. Onze werk repo staat onder
gbonnema\footnote{http://www.github.com/gbonnema} en de repo naam is xmas.
Bernard gaat de repo met \verb%git clone% clonen en \verb%git pull% bijhouden.

\paragraph{Agilefant}

Guus heeft Bernard een uitnodiging gestuurd voor Agilefant, zodat hij mee kan
kijken met de stories en de voortgang. Hij moet hier nog naar kijken.

\topic{Data en data structuur}

\paragraph{Bernard} De domein analyse van Jeroen is bijzonder duidelijk over
wat de structuur van de data momenteel is (gezien vanuit de cycle checker).  Ik
was hier positief verrast over! 

Jullie (team 33) zijn vrij om de data structuur in te richten zoals je het zelf wilt.
Als het maar effectief is en gemakkelijk te gebruiken, zowel voor het ontwerp tool
als voor de ontwerper van de verificatie tools.

\topic{Main requirements en prioriteiten}

\paragraph{Bernard} Ik ben het eens met de prioriteiten van de main
requirements zoals ze in de presentatie staan. De volgorde klopt
ook.  Bijvoorbeeld performance speelt alleen een rol waar het usability in de
weg gaat zitten.  Een ander voorbeeld is als we ten behoeve van portability
iets moeten doen dat onderhoud iets moeilijker maakt dan zij dat zo.

\begin{description}

	\item[Portability] {\color{blue} \it 1st main goal:} Portability is a
		leading requirement.

	\item[Ease of use / install] {\color{blue} \it 2nd main goal:} Assist in
		the design of \Noc + Promote tools to outsiders

	\item[Maintainability] {\color{blue} \it 3rd main goal:} A primary
		requirement.

	\item[Performance] {\color{blue} \it Follows} For all acceptable
		performance (which is a subjective measure). 

\end{description}

\topic{Architectuur}

\paragraph{Bernard} Ik ben het op hoofdlijnen met de architectuur en de dynamiek eens.

Zie presentatie in github voor details. Guus gaat dat opnemen in een architectuur document 
onder de subdirectory \texttt{design}.

\subtopic{Proces architectuur}
\paragraph{Guus} Open staat of we de verificatie tool in hetzelfde proces als de applicatie draaien
of in een apart proces. Het verschil zit in de mogelijkheid om een aparte user interface thread
te maken.

\paragraph{Bernard} Als het maar onderdeel van dezelfde applicatie is.

\paragraph{Guus} Voor de gebruiker hoeft er geen verschil te zijn. Alleen technisch
kun je de verificatie tool terugkoppeling loskoppelen. Doordat het een aparte user
interface is, kan de gebruiker doorwerken aan het huidig model: wijzigingen aanbrengen.
De terugkoppeling staat op een ander scherm, zowel textueel als grafisch. Daarmee is de terugkoppeling
altijd \'{e}\'{e}nduidig.

\paragraph{Bernard} Ik moet terugkoppelen met mijn collega's of zij 
een aparte terugkoppelingsscherm willen. Ik kom hierop terug.

\paragraph{Guus} Wil je gefabriceerde prototype om te zien hoe dit er uit komt te zien?

\paragraph{Bernard} Dat is niet nodig.

\subtopic{Recursie} We moeten recursie wat uitgebreider bespreken. Hiervoor is een aparte 
sessie nodig.

\subtopic{Composite} 

\paragraph{Bernard} Het is de bedoeling dat de verificatie tools (VT) altijd werken met een 
volledig platgeslagen netwerk. Dat komt omdat de VT met de low level informatie werkt
($t_{rdy}$ en $i_{rdy}$). Als je composites niet platslaat, dan heeft het die informatie niet.

\paragraph{Guus} En als we de semantiek van composites en primitives vangen in een data structuur,
inclusief de $t_{rdy}$ en $i_{rdy}$?

\paragraph{Bernard} Dat zou met parametrisering te ingewikkeld worden. Is buiten scope.

\subtopic{Geparametriseerde composites}

\paragraph{Bernard} Bij het bestaande tool ligt het genereren van een \w{grid} moeilijk. Dat
komt door de opzet van het tool dat te ingewikkeld is geworden. Het zou mooi zijn als dit in
het nieuwe tool\marginpar{aparte sessie} wel gemakkelijk kan. 

\subtopic{Dynamiek}

\paragraph{Bernard} Ik ben het met de dynamiek zoals hier gepresenteerd eens. Het punt van
terugkoppeling van VT resultaten ga ik met mijn collega's opnemen. 

Hieronder een samenvatting van deze punten.

\subtopic{Plugin architectuur}

\paragraph{Guus} De bedoeling van de plugin architectuur is de boilerplate
acties binnen ons project op te lossen en de specifieke VT mechanismen binnen
het VT op te lossen.

Het aanroepen van de plugin gebeurt asynchroon zodat de VT's die draaien het
ontwerp tool niet in de weg zitten. Er komt een terugkoppeling vanuit de VT
naar een scherm met de resultaten en de voortgang (progress bar).

\subtopic{Documentatie}

\paragraph{Guus} De bedoeling is deze plaatjes in een document te verenigen, dat
het overdragen van architectuur en onderhoud gemakkelijker maakt.

Let er op, dat wij maar een deel van de documentatie opleveren, te weten 
het deel dat met de design tool te maken heeft inclusief plugin architectuur,
maar zonder de hele VT straat.

\subtopic{Demonstratie}

\paragraph{Bernard} Er komt een filmpje aan van \'{e}\'{e}n van mijn
collega's. Dat kan ongeveer een week duren vanwege ander werk.


\end{Minutes}
\end{document}
