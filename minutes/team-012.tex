\documentclass{article}
\usepackage{subfig}
\usepackage[dutch]{babel}
\usepackage{minutes}
\usepackage{graphicx}
\usepackage{float}

\title{Notulen 012 overleg team 33}
\author{Jeroen}
\minutesstyle{header=table, vote=list, contents=true, columns={1}}

\begin{document}
%\selectlanguage{dutch}

\begin{Minutes}{Overleg 012 team 33}
\participant{Guus Bonnema, Jeroen Kleijn}
\minutesdate{15 november 2014}
\location{Skype}

\maketitle% This is where LaTeX makes the title

\topic{Bespreken voortgang domeinanalyses}

Volgens planning zouden de domeinanalyses nu moeten zijn afgerond. Guus en Stefan
hebben een conceptversie klaar, Jeroen verwacht deze dit weekend af te hebben.
Voor het afmaken van de domeinanalyses is naar verwachting nog een hele week nodig.
De betekent dat er in de planning inmiddels een vertraging van 1 week is opgetreden.

\topic{Demonstratie WickedXmas}

De demonstratie van de WickedXMas tool zoals in de planning voor de domeinanalyses
staat aangegeven heeft nog niet plaatsgevonden. Guus zal met Bernard en eventueel
Freek hier alsnog een afspraak voor maken.

\topic{Git repository}

Inmiddels heeft de OU een Git server beschikbaar gemaakt. De vraag is of wij van
deze server gebruik zullen gaan maken of dat de ontwikkeling via github blijft.
Mogelijk hebben Freek en/of Bernard een voorkeur voor een private repository i.v.m.
de verspreiding van hun source code.

\topic{Source code verificatietools}

Tijdens het bestuderen van de source code van de verificatietools is gebleken dat
er vrij veel nieuwe c++11 features worden gebruikt. Mogelijk moeten de teamleden
hierdoor extra tijd besteden om zich deze nieuwe features eigen te maken.

\topic{Vragen en afspraken}

\begin{itemize}
 \item Het volgende overleg is woensdag 19 november 2014. 
\end{itemize}


\end{Minutes}
\end{document}
