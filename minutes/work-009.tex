\documentclass[a4paper,final]{article}
\usepackage{subfig}
\usepackage[dutch]{babel}
\usepackage{hyperref}
\usepackage{minutes}
\usepackage{graphicx}
\usepackage{float}
\usepackage{color}
\usepackage{caption}

\title{Resultaten work meeting 009 Update iteratie 4 }
\author{Guus}
\minutesstyle{header=table, vote=list, contents=true, columns={1}}

\begin{document}
%\selectlanguage{dutch}

\newcommand{\Noc}{\textsc{NoC}\xspace}%
\newcommand{\w}[1]{\textsc{#1}\xspace}%
\newcommand{\qml}{\textsc{Qml}\xspace}%
\newcommand{\qt}{\textsc{Qt}\xspace}%
\newcommand{\qtquick}{\textsc{QtQuick}\xspace}%
\newcommand{\cpp}{\textsc{C++}\xspace}%
\newcommand{\code}[1]{\texttt{#1}\xspace}%
\newcommand{\xmas}{\textsc{xmas}\xspace}%
\newcommand{\xmv}{\textsc{Xmv}\xspace}%
\newcommand{\xmd}{\textsc{Xmd}\xspace}%
\newcommand{\xmvtest}{\textsc{XmvTest}\xspace}%
\newcommand{\xmdtest}{\textsc{XmdTest}\xspace}%
\newcommand{\bitpowder}{\textsc{Bitpowder}\xspace}%
\newcommand{\datamodel}{\textsc{datamodel}\xspace}%
\newcommand{\vt}{\textsc{Vt}\xspace}%
\newcommand{\src}{\textsc{src}\xspace}%
\newcommand{\agilefant}{\textsc{AgileFant}\xspace}%
\newcommand{\een}{\'{e}\'{e}n\xspace}%
\newcommand{\svn}{\textsc{svn}\xspace}%
\newcommand{\git}{\textsc{git}\xspace}%
\newcommand{\github}{\textsc{Github}\xspace}%
\newcommand{\subversion}{\textsc{subversion}\xspace}%
\newcommand{\radboud}{\textsc{Radboud}\xspace}%
\newcommand{\uml}{\textsc{uml}\xspace}%


\begin{Minutes}{Work meeting 009}
\participant{Guus Bonnema, Stefan Versluys, Jeroen Kleijn, Bernard van Gastel}
\subtitle{Update iteratie 4 met Bernard}
\minutetaker{Guus}
\minutesdate{9 maart 2015}
\location{Skype}

\maketitle% This is where LaTeX makes the title

\topic{Mededelingen}

Komende woensdag (11 maart) is Guus afwezig door een priv\'{e} verplichting. Zie hierover 
eerdere mededeling in mail.

\topic{Midterm bijeenkomst 25 maart 2015 in Utrecht}

Stefan kan niet komen. Alleen Guus en Jeroen gaan. Jeroen presenteert.

\topic{Composite voortgang}

Het team spreekt het \uml diagram voor de composite door en heeft met de 
structuur geen problemen. Jeroen gaat werken aan \een of twee use cases
en wat scenarios. Het doel van het ontwerpen is om de opzet vooraf te doordenken
om zo het programmeren gemakkelijker te maken. Dat is belangrijker dan 
mooie documentatie produceren, wat uiteindelijk ook moet gebeuren maar op dit 
moment op de tweede plaats staat.

De opstaande punten zijn:

\begin{itemize}
	\item Hoe maak je van een netwerk een composite in de praktijk en wat zijn
			de gevolgen voor onze composites?
	\item Hoe vind je de composites als netwerk ontwerper?
\end{itemize}



\topic{Vraag aan Bernard: IN en OUT}
Vraag te stellen aan Bernard is het voorstel van Guus om IN en OUT poorten
in composites te vervangen door Source en Sink met eventueel een optie om een 
'echte' source of sink op te nemen. Voordeel is dat we niet zelf een primitief 
hoeven te maken en toe te voegen aan \code{xmas.h}.

\topic{Vraag aan Bernard: poorten in composite}
Bij het tonen van de composites is de vraag hoeveel invoer- en uitvoerpoorten
we kunnen verwachten maximaal en hoeveel hij wil dat we minimaal kunnen tekenen
voor we een symbolische invoer poort tekenen.

\topic{Tijdsproblemen plugin en qml}
De \qml oplossing (directe oplossing) kost veel meer tijd dan gedacht en Guus
geeft aan in problemen te komen met de voortgang als hij dit alleen moet doen.
Hij vraagt Jeroen om \een van de twee over te nemen. Jeroen heeft er de voorkeur
voor om eerst de composites verder uit te werken. Omdat de samenwerking \qml en
\cpp moeilijker is dan gedacht, blijft nu de plugin liggen.

\topic{Plugin code}
Guus loopt op verzoek met Stefan en Jeroen de plugin code door met als doel 
iets meer inzicht te verschaffen in hoe de code werkt.

\topic{Algehele voortgang: project issue}
Op 2 maart heeft Bernard de oplossing van een parser verworpen en wil \een data
model: de \code{xmas.h} structuren voor de \vt's. Stefan heeft een duidelijk
voorkeur voor de directe \qml verbinding, maar helaas kost de \qml-\cpp verbinding 
meer tijd dan gedacht en kost experimenteren om het goed werkend te krijgen ook 
meer tijd dan gehoopt. Stefan en Guus zijn de afgelopen week bezig geweest de 
verbinding tussen \qml en \cpp tot stand te brengen en zijn nog niet klaar.

Het experimenteren dat nodig is ziet Stefan als bedreiging van het opleveren van 
een werkend product. Stefan wijst ons er op, dat eind mei voor hem echt einde 
project is. Waarvan akte. 

\paragraph{Opmerking notulist} We wisten dat eind mei de streef datum was van Stefan, 
maar niet dat het na eind mei voor Stefan echt afgelopen moet zijn. Wat we begrepen 
hadden was dat het voor de zomer vakantie afgelopen moest zijn (dat is 1 juli). Na
de meeting realiseerde ik me pas hoeveel impact deze opmerking heeft.

\paragraph{Voor volgend overleg}
Als gevolg is het zaak te begrenzen wat we nog gaan doen en het verdere verloop goed 
te monitoren. Ook moeten we koers houden op de huidige richting en niet meer van richting 
veranderen. Komende zaterdag moeten we het vervolg in dit licht bespreken en vastleggen, 
zodat we weten eind mei het product af te hebben. Dit plaatst ook het verzoek om hulp
van Guus in een ander licht. Ook dit voor de volgende keer.

\topic{Actiepunten}

In figuur \ref{fig:openpunten} de openstaande punten. Deze punten werken we elke sessie af, en zetten de 
afgeronde akties in de lijst daaronder.

\vspace{2em}

\begin{figure}[p]
%\begingroup
\begin{tabular}{|l|l|p{25em}|}
\hline
{\bf datum} & {\bf wie} & {\bf Aktie}\\\hline
09-03-2015  & team      & Project issue voortgang consequenties bespreken.\\\hline
09-03-2015  & team      & Bernard vragen over maximaal aantal poorten te tekenen\\\hline
09-03-2015  & team      & Bernard vragen over IN/OUT vervangen door Source/Sink\\\hline
02-03-2015  & team      & Sturen Bernard een patch van de \vt verschillen\\\hline
02-03-2015  & Bernard   & Bernard komt terug op repo: git of svn?\\\hline
24-01-2015  & Bernard   & Filmpje van een collega, Bernard geeft een seintje\\\hline
24-01-2015  & Bernard   & Voorbeeld deadlock op 2 deadlock vrije composites\\\hline
\end{tabular}\label{fig:openpunten}
\captionof{table}{Aktielijst open punten}
\end{figure}
%\endgroup

\vspace{2em}

\begin{figure}[p]
%\begingroup
\begin{tabular}{|l|l|l|p{10em}|p{10em}|}
\hline
{\bf datum} & {\bf gereed} & {\bf wie} & {\bf Aktie}                                      & commentaar      \\\hline
24-01-2015  & 22-02-2015   & team      & We moeten hernieuwde contactmomenten afspreken   & 22 febr 2015 afspraken gemaakt\\\hline
07-02-2015  & 22-02-2015   & team      & Afspraak met Bernard voor 21 februari 2015       & 22 febr 2015 10:30\\\hline
07-02-2015  & 13-02-2015   & team      & Werkwijze opnieuw bezien                         & vastgeled in plan\\\hline
07-02-2015  & 13-02-2015   & team      & Vernieuwd plan uitbrengen.                       & Akkoord B. 19-02-2015\\\hline
24-01-2015  & 07-02-2015   & Bernard   & Voert bugfix op xmas.h uit en geeft signaal      & uitgevoerd\\\hline
24-01-2015  & 07-02-2015   & team      & splitsen header files afstemmen met Bernard      & n.v.t.\\\hline
26-01-2015  & 07-02-2015   & team      & Kan de licentie van \w{bitpower lib} naar GPLv3? & uitgevoerd\\\hline
\end{tabular}
\captionof{table}{Aktielijst gesloten punten}
\end{figure}
%\endgroup

\end{Minutes}

\end{document}
