\documentclass[a4paper,final]{article}
\usepackage{subfig}
\usepackage[dutch]{babel}
\usepackage{hyperref}
\usepackage{minutes}
\usepackage{graphicx}
\usepackage{float}
\usepackage{color}
\usepackage{caption}

\title{Meeting 010}
\author{Stefan}
\minutesstyle{header=table, vote=list, contents=true, columns={1}}

\begin{document}
%\selectlanguage{dutch}

\newcommand{\Noc}{\textsc{NoC}\xspace}%
\newcommand{\w}[1]{\textsc{#1}\xspace}%
\newcommand{\qml}{\textsc{Qml}\xspace}%
\newcommand{\qt}{\textsc{Qt}\xspace}%
\newcommand{\qtquick}{\textsc{QtQuick}\xspace}%
\newcommand{\cpp}{\textsc{C++}\xspace}%
\newcommand{\code}[1]{\texttt{#1}\xspace}%
\newcommand{\xmas}{\textsc{xmas}\xspace}%
\newcommand{\xmv}{\textsc{Xmv}\xspace}%
\newcommand{\xmd}{\textsc{Xmd}\xspace}%
\newcommand{\xmvtest}{\textsc{XmvTest}\xspace}%
\newcommand{\xmdtest}{\textsc{XmdTest}\xspace}%
\newcommand{\bitpowder}{\textsc{Bitpowder}\xspace}%
\newcommand{\datamodel}{\textsc{datamodel}\xspace}%
\newcommand{\vt}{\textsc{Vt}\xspace}%
\newcommand{\src}{\textsc{src}\xspace}%
\newcommand{\agilefant}{\textsc{AgileFant}\xspace}%
\newcommand{\een}{\'{e}\'{e}n\xspace}%
\newcommand{\svn}{\textsc{svn}\xspace}%
\newcommand{\git}{\textsc{git}\xspace}%
\newcommand{\github}{\textsc{Github}\xspace}%
\newcommand{\subversion}{\textsc{subversion}\xspace}%
\newcommand{\radboud}{\textsc{Radboud}\xspace}%
\newcommand{\uml}{\textsc{uml}\xspace}%


\begin{Minutes}{Work meeting 010}
\participant{Jeroen Kleijn}
\subtitle{Midweek meeting}
\minutetaker{Stefan}
\minutesdate{10 maart 2015}
\location{Skype}

\maketitle% This is where LaTeX makes the title

\topic{Builden project}

Jeroen kan ondertussen alles builden na een manuele clean maar xmd crasht.
Werd met debugger verder onderzocht en heeft vermoedelijk iets te maken
met het ontbreken van de plugin deploy. Jeroen ging dit verder bekijken.

\topic{Composite voortgang}

Overleg over wat qml nodig heeft om een composite voor te kunnen stellen.

Volgende punten:
\begin{itemize}
	\item Breedte van het blokje moet groter omdat de poortnamen erin
		moeten zodat de designer weet welke poort wat is.
	\item Voor qml is naast de standaard component properties
		een lijstje met poorten voldoende.
	\item Bijkomend kan er linkje meegegeven worden naar
		een pictogram dat dan midden in het blokje getoond
		kan worden.
\end{itemize}

Deze week wordt er een open dialog voorzien bij het klikken op het composit pictogram.

De te openen bestandsnaam wordt naar de xmas code gestuurd.
Deze moet het netwerk parsen als een composite, en levert een component
van het type composite met naam,pictogram en een lijst met poorten aan
qml.

\topic{Demo xmd}
Met teamviewer demo aan Jeroen welke features er in xmd toegevoegd zijn.

\topic{Flattener}
Jeroen werkt aan de flattener in een branch xmascomposite.



\end{Minutes}

\end{document}
