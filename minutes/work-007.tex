\documentclass[a4paper,final]{article}
\usepackage{subfig}
\usepackage[dutch]{babel}
\usepackage{hyperref}
\usepackage{minutes}
\usepackage{graphicx}
\usepackage{float}
\usepackage{color}

\title{Resultaten work meeting 007 Einde iteratie 3 }
\author{Guus}
\minutesstyle{header=table, vote=list, contents=true, columns={1}}

\begin{document}
%\selectlanguage{dutch}

\newcommand{\Noc}{\textsc{NoC}\xspace}%
\newcommand{\w}[1]{\textsc{#1}\xspace}%
\newcommand{\qml}{\textsc{Qml}\xspace}%
\newcommand{\qt}{\textsc{Qt}\xspace}%
\newcommand{\qtquick}{\textsc{QtQuick}\xspace}%
\newcommand{\cpp}{\textsc{C++}\xspace}%
\newcommand{\code}[1]{\texttt{#1}\xspace}%
\newcommand{\xmas}{\textsc{xmas}\xspace}%
\newcommand{\xmv}{\textsc{Xmv}\xspace}%
\newcommand{\xmd}{\textsc{Xmd}\xspace}%
\newcommand{\xmvtest}{\textsc{XmvTest}\xspace}%
\newcommand{\xmdtest}{\textsc{XmdTest}\xspace}%
\newcommand{\bitpowder}{\textsc{Bitpowder}\xspace}%
\newcommand{\datamodel}{\textsc{datamodel}\xspace}%
\newcommand{\vt}{\textsc{Vt}\xspace}%
\newcommand{\src}{\textsc{src}\xspace}%
\newcommand{\agilefant}{\textsc{AgileFant}\xspace}%
\newcommand{\een}{\'{e}\'{e}n\xspace}%
\newcommand{\svn}{\textsc{svn}\xspace}%


\begin{Minutes}{Work meeting 007}
\participant{Guus Bonnema, Stefan Versluys, Jeroen Kleijn, Bernard van Gastel}
\subtitle{Einde iteratie 3, demo Bernard}
\minutetaker{Guus}
\minutesdate{22 februari 2015}
\location{Skype}

\maketitle% This is where LaTeX makes the title

\topic{Voortgang}

\subtopic{Interfacing designer en VT datastructuur}

Naar aanleiding van de mail van het team aan Bernard ontstaat hier
een discussie over de koppeling tussen de user interface en
de VT datastructuur. Bernard is niet direct overtuigd van de 
ene of de andere alternatief en heeft tijd nodig. Daarnaast
wil hij overleggen met Freek (momenteel in Amerika).
Volgende week maandag 2 maart schuift Bernard weer aan bij de skype sessie
om een uitspraak te doen.

\paragraph{Voortschrijdend inzicht} Uit deze discussie komt naar voren,
dat van origine het beperken van dubbele code een doel is 
geweest. Dit komt bij ons project onder de primaire requirement
van onderhoudbaarheid.

\paragraph{Concurrency} Het is de bedoeling dat alle concurrency bij
aanroepen van de VTs \textit{binnen} de VTs plaatsvindt. Het is dus niet
de bedoeling dat de plugin concurrency regelt: dat start \een proces op.

\subtopic{Demo user interface}

\paragraph{Hierarchische structuur} De bedoeling is schaalbaarheid
uit hierarchie te halen. Momenteel nog niet, maar dat komt er wel aan.
Conclusie is de hierarchie in de data structuur achter de user interface
te laten bestaan, maar platslaan op verzoek mogelijk te maken.

\paragraph{IN/OUT poorten} Dit is niet iets van Intel of \xmas, maar iets
wat de VT onderzoekers hebben verzonnen. De theorie rond \xmas heeft een focus
op gesloten netwerken. Voor ons in de user interface kunnen we de \textsc{IN/OUT}
poorten opnemen alsof het een primitief is.

\paragraph{Join varianten} De join kent 2 vormen: restricted en unrestricted. De
unrestricted join is de meest algemene. In deze join bepaalt een functie hoe
de input packets naar de output port gaat. Een restricted join heeft \een control
channel en \een data channel. Het is de bedoeling een standaard functie op te 
nemen zoals : \code{take(1)} of \code{take(2)}. Voor deze join kan men dan 
geen functie invoeren. Men moet in de userinterface kunnen aangeven of iets 
restricted is en zo ja, welke poort de data packet bevat en welke de 
\code{control} packet bevat: dat kan nu al. Nog geregeld moet: unrestricted en de standaard
functie text voor restricted. 

Het is specifiek de deadlock checker die gebruik maakt van de restricted join.

\paragraph{Functie beschrijving} De beschrijvende tekst die in de huidige versie boven
de functie van een primitive staat is oud. In de \svn repo staat de nieuwe tekst.
Om de functietext te kunnen parsen heeft de \vt code de classes:

\begin{itemize}
	\item \code{ParseSourceExpression}
	\item \code{ParseFunctionExpression}
	\item \code{ParsePackageFunction}
\end{itemize}

Wij kunnen die vanuit onze interface gebruiken.


\subtopic{Aanvullende communicatie afspraken op plan}
De contactmomenten zijn tot nu toe beperkt tot einde iteratie. Dit is noodzakelijk
maar voor onze discussies tijdens de iteratie onvoldoende. Bernard stelt voor om
op het moment dat er vragen zijn ofwel een zeer korte mail te sturen die hij direct 
kan beantwoorden ofwel een verzoek tot een skype sessie te doen. Omdat Bernard
in de avonduren moeite heeft tijd te vinden, is zijn voorkeur om overdag een skype 
sessie te regelen. Response tijd is afhankelijk van de lopende onderwijstaken van 
Bernard.

\topic{Actiepunten}

In figuur \ref{fig:openpunten} de openstaande punten. Deze punten werken we elke sessie af, en zetten de 
afgeronde akties in de lijst daaronder.

\begin{figure}
\begin{tabular}[!h]{|l|l|p{25em}|}
\hline
{\bf datum} & {\bf wie} & {\bf Aktie}\\\hline
24-01-2015  & team      & Filmpje van een collega, Bernard geeft een seintje\\\hline
24-01-2015  & Bernard   & Voorbeeld deadlock op 2 deadlock vrije composites\\\hline
\end{tabular}
\caption{Aktielijst open punten}\label{fig:openpunten}
\end{figure}


\begin{figure}
\begin{tabular}[!h]{|l|l|l|p{10em}|p{10em}|}
\hline
{\bf datum} & {\bf gereed} & {\bf wie} & {\bf Aktie}                                      & commentaar      \\\hline
24-01-2015  & 22-02-2015   & team      & We moeten hernieuwde contactmomenten afspreken   & 22 febr 2015 afspraken gemaakt\\\hline
07-02-2015  & 22-02-2015   & team      & Afspraak met Bernard voor 21 februari 2015       & 22 febr 2015 10:30\\\hline
07-02-2015  & 13-02-2015   & team      & Werkwijze opnieuw bezien                         & vastgeled in plan\\\hline
07-02-2015  & 13-02-2015   & team      & Vernieuwd plan uitbrengen.                       & Akkoord B. 19-02-2015\\\hline
24-01-2015  & 07-02-2015   & Bernard   & Voert bugfix op xmas.h uit en geeft signaal      & uitgevoerd\\\hline
24-01-2015  & 07-02-2015   & team      & splitsen header files afstemmen met Bernard      & n.v.t.\\\hline
26-01-2015  & 07-02-2015   & team      & Kan de licentie van \w{bitpower lib} naar GPLv3? & uitgevoerd\\\hline
\end{tabular}
\caption{Aktielijst gesloten punten}
\end{figure}

\end{Minutes}

\end{document}
