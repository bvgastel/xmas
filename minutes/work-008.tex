\documentclass[a4paper,final]{article}
\usepackage{subfig}
\usepackage[dutch]{babel}
\usepackage{hyperref}
\usepackage{minutes}
\usepackage{graphicx}
\usepackage{float}
\usepackage{color}

\title{Resultaten work meeting 008 Update iteratie 4 }
\author{Guus}
\minutesstyle{header=table, vote=list, contents=true, columns={1}}

\begin{document}
%\selectlanguage{dutch}

\newcommand{\Noc}{\textsc{NoC}\xspace}%
\newcommand{\w}[1]{\textsc{#1}\xspace}%
\newcommand{\qml}{\textsc{Qml}\xspace}%
\newcommand{\qt}{\textsc{Qt}\xspace}%
\newcommand{\qtquick}{\textsc{QtQuick}\xspace}%
\newcommand{\cpp}{\textsc{C++}\xspace}%
\newcommand{\code}[1]{\texttt{#1}\xspace}%
\newcommand{\xmas}{\textsc{xmas}\xspace}%
\newcommand{\xmv}{\textsc{Xmv}\xspace}%
\newcommand{\xmd}{\textsc{Xmd}\xspace}%
\newcommand{\xmvtest}{\textsc{XmvTest}\xspace}%
\newcommand{\xmdtest}{\textsc{XmdTest}\xspace}%
\newcommand{\bitpowder}{\textsc{Bitpowder}\xspace}%
\newcommand{\datamodel}{\textsc{datamodel}\xspace}%
\newcommand{\vt}{\textsc{Vt}\xspace}%
\newcommand{\src}{\textsc{src}\xspace}%
\newcommand{\agilefant}{\textsc{AgileFant}\xspace}%
\newcommand{\een}{\'{e}\'{e}n\xspace}%
\newcommand{\svn}{\textsc{svn}\xspace}%
\newcommand{\git}{\textsc{git}\xspace}%
\newcommand{\github}{\textsc{Github}\xspace}%
\newcommand{\subversion}{\textsc{subversion}\xspace}%
\newcommand{\radboud}{\textsc{Radboud}\xspace}%


\begin{Minutes}{Work meeting 008}
\participant{Guus Bonnema, Stefan Versluys, Jeroen Kleijn, Bernard van Gastel}
\subtitle{Update iteratie 4 met Bernard}
\minutetaker{Guus}
\minutesdate{2 maart 2015}
\location{Skype}

\maketitle% This is where LaTeX makes the title

\topic{Midterm bijeenkomst 25 maart 2015 in Utrecht}

Freek heeft zich afgemeld bij Harrie Passier. Bernard kan die dag helaas niet
aanwezig zijn vanwege priv\'{e} verplichtingen. Van ons team gaan Jeroen en Guus en 
indien mogelijk ook Stefan. 

\topic{Testen op de Mac}

Wij hebben geen beschikking over een Mac. Wel heeft Stefan een gehackte virtuele machine met
iets dat lijkt op de Mac. Bernard laat weten dat de Mac buiten scope is. Als we via de virtuele machine
incidenteel ons programma kunnen compileren en draaien dan zou dat mooi zijn, maar het is
geen onderdeel van onze opdracht.

\topic{Open punten vorige meeting}

Bernard geeft aan deze uit het oog te zijn verloren en gaat hier naar kijken.

\topic{Data model}

\subtopic{Interne datamodel in het geheugen}

Intern zijn er altijd twee data modellen: het \cpp model en het \qml model, dat is inherent aan
het karakter van \qml. In de oplossing die wij het vorige overleg hadden voorgesteld zou
interfacing plaatsvinden vanuit \qml met json, gegenereerd vanuit de \qml. 

Bernard wil geen tweede parser in \qml en wil het datamodel in de vorm van het \cpp model 
in het geheugen hebben op basis van de code in \code{xmas.h}.
Onderhoud beperkt zich dan tot het xmas model en er is maar \een parser. Ook is het dan gemakkelijker 
om in de main thread eenvoudige controles te doen zonder te hoeven serialiseren.

\subtopic{File typen}

Bernard streeft naar \een filetype.
Hij benadrukt nog een keer dat zowel Freek als hijzelf duplicatie van code willen vermijden.
Hij is beducht voor het onderhoud als er twee parsers zijn. Bernard accepteert de toename in 
complexiteit die te verwachten is als gevolg van koppelen met de \cpp structuur. Hij geeft aan
ook voordelen te zien (zie verderop bij uitvoering plugins).

\subtopic{Koppelen \qml en \cpp}

We zien een mogelijkheid om \qml en \cpp met \qml \w{properties} te verbinden. 
We hebben dan geen aparte signalen en slots nodig (die genereert het systeem intern).
Waarschijnlijk heeft het wel impact op de performance, maar hoe veel dat weten we nog niet.  
We implementeren volgens deze strategie (ijs en weder dienende).

\subtopic{Json file formaat}

We mogen in plaats van de flat json een hierarchische json opslaan. Het platslaan vindt
dan in core plaats.

Scheiding van netwerk en composite zoals Guus en Jeroen bespraken, is niet nodig. 
De hierarchische modellen mogen direct verwijzen naar andere netwerken als composite.

\subtopic{Serialiseren van datamodel}

Voor communicatie naar processen of threads is het beter textstreams of strings te gebruiken, 
dan de \cpp structuur uit \code{xmas.h}. Bernard is het hier mee eens.
We parsen dan twee keer: 1. (main thread) Om van \cpp structuur naar json text te komen. 2. (Process thread) 
Om van json text weer naar de \cpp structuur te komen (in het \vt process). Dit kan voor grote structuren de 
user interface kort bevriezen: dat is geen probleem. Als we binnen de main thread blijven, 
dan kunnen we de structuur gebruiken zonder naar json om te zetten.

Bernard merkt op, dat de huidige structuur de basis is en verbetering en uitbreiding kan gaan 
krijgen. Zoals de mogelijkheid om een XMASComponent te kopi\"{e}ren.

\topic{Plugins}

\subtopic{plugin werking}

Bij het runnen van plugins is het zaak de verhoudingen in de gaten te blijven houden. Zo kost
het draaien van een syntax check voor een spidergon 1024 op dit moment ongeveer 5 ms. Een 
spidergon 1024 heeft ca 18000 objecten. Dit netwerk is een json file van ca 5.5 MB en het parsen kost
ongeveer 500 ms. Dus 2x parsen zoals in ons geval kost vele malen meer dan het uitvoeren 
van een syntax check. Ook de cycle checker kost ongeveer 50 ms voor hetzelfde netwerk. 
Het lijkt verstandig beide controles in de main thread uit te voeren.

We mogen de standaard output van de verificatie tools aanpassen als dit nuttig is voor 
``parsing'' door de designer.

P.S. Een syntax check houdt in channel validiteit, functie definitie en source definitie. Onze huidige
versie controleert alleen de channels. Dit gaan we aanpassen.

\subtopic{Main plugin}

Voor het uitvoeren van de main plugin, waar we alle \vt's uitvoeren, hebben we voor een process gekozen.
Bernard is het hier mee eens in verband met de hoeveelheid tijd die een complete run kost. Hij wijst ons
er op, dat we in de main uit moeten gaan van de \code{-json} optie, en niet van de \code{-mesh} of 
\code{-spidergon} optie. Die main is geschreven onder tijdsdruk van de deadline voor papers en staat 
geen model voor hoe het moet worden.

\topic{Designer issues}

\subtopic{Real time online controles}

In de designer hebben we een dialoog voor het invoeren van text zoals expressies. Bernard
geeft aan, dat we vrij eenvoudig de correctheid van die expressies kunnen verifi\"{e}ren met
kant en klare classes uit de verificatie tools. Het gaat om de classes \code{ParseSourceExpression}
en \code{ParsePacketFunction}\footnote{redactie: Dit stond ook in vorige notulen.}. Het zou mooi
zijn om vrijwel direct commentaar te krijgen als iets met de expressies niet in orde is. We gaan 
hier naar kijken.

\topic{Source repository}

We bespreken de verschillende repo's die we nu gebruiken: \git op \github en \subversion aan de \radboud. 
Het probleem is dat de sources van de \vt's nu al uit elkaar zijn gaan lopen. Het lijkt verstandig 
vanuit 1 repo te gaan werken. Bernard komt hier op terug. 

We beloven Bernard het verschil als een patch te sturen en Bernard gaat nadenken over de repo waar hij in wil werken. 
De structuur zal niet overeenkomen, omdat wij de programmatuur uit elkaar hebben getrokken: test bij test, main apart 
van library classes. De wijzigingen zijn functioneel beperkt, maar zijn er wel veel, met name om warnings te voorkomen
en in het platform onafhankelijk maken.

\topic{Actiepunten}

In figuur \ref{fig:openpunten} de openstaande punten. Deze punten werken we elke sessie af, en zetten de 
afgeronde akties in de lijst daaronder.

\begin{figure}
\begin{tabular}[!h]{|l|l|p{25em}|}
\hline
{\bf datum} & {\bf wie} & {\bf Aktie}\\\hline
02-03-2015  & team      & Sturen Bernard een patch van de \vt verschillen\\\hline
02-03-2015  & Bernard   & Bernard komt terug op repo: git of svn?\\\hline
24-01-2015  & Bernard   & Filmpje van een collega, Bernard geeft een seintje\\\hline
24-01-2015  & Bernard   & Voorbeeld deadlock op 2 deadlock vrije composites\\\hline
\end{tabular}
\caption{Aktielijst open punten}\label{fig:openpunten}
\end{figure}


\begin{figure}
\begin{tabular}[!h]{|l|l|l|p{10em}|p{10em}|}
\hline
{\bf datum} & {\bf gereed} & {\bf wie} & {\bf Aktie}                                      & commentaar      \\\hline
24-01-2015  & 22-02-2015   & team      & We moeten hernieuwde contactmomenten afspreken   & 22 febr 2015 afspraken gemaakt\\\hline
07-02-2015  & 22-02-2015   & team      & Afspraak met Bernard voor 21 februari 2015       & 22 febr 2015 10:30\\\hline
07-02-2015  & 13-02-2015   & team      & Werkwijze opnieuw bezien                         & vastgeled in plan\\\hline
07-02-2015  & 13-02-2015   & team      & Vernieuwd plan uitbrengen.                       & Akkoord B. 19-02-2015\\\hline
24-01-2015  & 07-02-2015   & Bernard   & Voert bugfix op xmas.h uit en geeft signaal      & uitgevoerd\\\hline
24-01-2015  & 07-02-2015   & team      & splitsen header files afstemmen met Bernard      & n.v.t.\\\hline
26-01-2015  & 07-02-2015   & team      & Kan de licentie van \w{bitpower lib} naar GPLv3? & uitgevoerd\\\hline
\end{tabular}
\caption{Aktielijst gesloten punten}
\end{figure}

\end{Minutes}

\end{document}
