\documentclass[a4paper,final]{article}
\usepackage{subfig}
\usepackage[dutch]{babel}
\usepackage{hyperref}
\usepackage{minutes}
\usepackage{graphicx}
\usepackage{float}
\usepackage{color}
\usepackage{caption}

\title{Resultaten work meeting 011 Update iteratie 5 }
\author{Guus}
\minutesstyle{header=table, vote=list, contents=true, columns={1}}

\begin{document}
%\selectlanguage{dutch}

\newcommand{\Noc}{\textsc{NoC}\xspace}%
\newcommand{\w}[1]{\textsc{#1}\xspace}%
\newcommand{\qml}{\textsc{Qml}\xspace}%
\newcommand{\qt}{\textsc{Qt}\xspace}%
\newcommand{\qtquick}{\textsc{QtQuick}\xspace}%
\newcommand{\cpp}{\textsc{C++}\xspace}%
\newcommand{\code}[1]{\texttt{#1}\xspace}%
\newcommand{\xmas}{\textsc{xmas}\xspace}%
\newcommand{\xmv}{\textsc{Xmv}\xspace}%
\newcommand{\xmd}{\textsc{Xmd}\xspace}%
\newcommand{\xmvtest}{\textsc{XmvTest}\xspace}%
\newcommand{\xmdtest}{\textsc{XmdTest}\xspace}%
\newcommand{\bitpowder}{\textsc{Bitpowder}\xspace}%
\newcommand{\datamodel}{\textsc{datamodel}\xspace}%
\newcommand{\vt}{\textsc{Vt}\xspace}%
\newcommand{\src}{\textsc{src}\xspace}%
\newcommand{\agilefant}{\textsc{AgileFant}\xspace}%
\newcommand{\een}{\'{e}\'{e}n\xspace}%
\newcommand{\svn}{\textsc{svn}\xspace}%
\newcommand{\git}{\textsc{git}\xspace}%
\newcommand{\github}{\textsc{Github}\xspace}%
\newcommand{\subversion}{\textsc{subversion}\xspace}%
\newcommand{\radboud}{\textsc{Radboud}\xspace}%
\newcommand{\uml}{\textsc{uml}\xspace}%
\newcommand{\json}{\textsc{Json}\xspace}%

\begin{Minutes}{Work meeting 011}
\participant{Guus Bonnema, Stefan Versluys, Jeroen Kleijn, Bernard van Gastel}
\subtitle{Update iteratie 5}
\minutetaker{Guus}
\minutesdate{16 maart 2015}
\location{Skype}

\maketitle% This is where LaTeX makes the title

\topic{Agilefant}

Zie mail over agilefant. het bedrijf heeft besloten de 5 vrije plaatsen te verminderen tot 1 vrije plaats (de account houder).
Van gratis naar \$ 5 per persoon per maand is een behoorlijke stap. Het gevoel heerst dat Agilefant ons in een fuik heeft geleid 
en dat nu wil verzilveren. De gevoelens hierover zijn gemengd, maar we zitten einde traject en hebben Agilefant vanaf einde code 
maken, niet zo hard meer nodig.

In de mail staan alternatieven. We besluiten nader te overwegen wat we gaan doen. Guus krijgt van Stefan een lijst met alternatieven
om nader te bekijken. Jeroen gaat kijken of hij agilefant op de OU server kan installeren: maximaal 2 uur. In het ergste geval 
gebruiken we agile fant nog alleen voor de eind presentatie om een screenprint te laten zien.

Woensdag avond nemen we een besluit.

\topic{Midterm bijeenkomst 25 maart 2015 in Utrecht}

Jeroen besluit uiterlijk 17 april aan de presentatie voor de midterm te gaan werken. We realiseren ons,
dat dit erg dicht bij eind project is (10 mei).

\topic{Composite voortgang}

Jeroen werkt momenteel aan de flattening functie, maar heeft de \json parser en de \xmas structuren
nog niet aangepast. We besluiten dat we de aanpassing op structuur en \json parser eerst nodig hebben,
omdat xmd die ook moet gaan gebruiken. De flattening functie kan dan als stap 2.

\topic{Vraag aan Bernard: IN en OUT}
Vraag te stellen aan Bernard is het voorstel van Guus om IN en OUT poorten
in composites te vervangen door Source en Sink met eventueel een optie om een 
'echte' source of sink op te nemen. Voordeel is dat we niet zelf een primitief 
hoeven te maken en toe te voegen aan \code{xmas.h} en dat we de parser niet hoeven
aan te passen.

Deze vraag staat nog van 9 maart, is verstuurd, maar nog geen antwoord van Bernard ontvangen.
Guus gaat Bernard hier op wijzen en dat het project op antwoord hier wacht.

\topic{Vraag aan Bernard: poorten in composite}
Bij het tonen van de composites is de vraag hoeveel invoer- en uitvoerpoorten
we kunnen verwachten maximaal en hoeveel hij wil dat we minimaal kunnen tekenen
voor we een symbolische invoer poort tekenen.

Deze vraag staat nog van 9 maart, is verstuurd, maar nog geen antwoord van Bernard ontvangen.
We hebben hier minder haast mee, omdat de designer veel poorten (ca 50) zonder problemen aan kan.

\topic{Expressie- en functie parser}
De expressie en functie parser zijn af. Dit weekend heeft Bernard zijn \vt programmatuur
gecorrigeerd en aangevuld met een export functie naar een \json file.

De unrestricted join kan nog geen functie opslaan, omdat dit niet in \xmas is opgenomen.
Hiervoor is aanpassing van de extensies (een extra extensie) en aanpassing van de parser
nodig. Guus heeft Bernard gevraagd of hij dit zelf wil aanpassen of niet. Nog geen antwoord.

De volgende stap is connectie van componenten. Het \xmas deel hiervoor is eenvoudig,
maar Guus ziet niet direct hoe dit in \qml af te handelen. Morgenavond spreken 
Stefan en Guus af dit via teamviewer en skype op te lossen.

\topic{Plugin}
De plugin blijft nog steeds liggen. Pas na bouwen afmaken van \xmas integratie met
\qml kan Guus aan de plugin werken. Het doorlopen van de plugin code de vorige keer
hebben Stefan en Jeroen als nuttig ervaren.

\topic{Algehele voortgang: project einde nadert}
Guus wijst op de nadering van project einde. Iteratie 5 is gisteren begonnen en eindigt
op 5 april. Daar komt transitie van 10 april - 10 mei. Dat betekent dat we 5 april klaar 
moeten zijn en een werkende designer moeten hebben inclusief plugin en composites. Dat is
aardig ambitieus gezien de korte tijd (3 weken). We kunnen ons geen vertragingen meer veroorloven.
Wel hebben we na de einddatum van 10 mei nog een paar weken respijt, maar alleen als het
echt niet anders kan. We richten ons op 5 april de applicatie af. Tijdens de transitie 
maken de we software klaar voor release en oplevering aan Bernard.

Guus merkt op, dat de huidige programmatuur nog geen componenten kan verbinden en ook nog geen
bestanden kan exporteren. Verder is Bernard ongeveer begraven onder werk. Het is beter voor
de beeldvorming hem geen "half" product te tonen en pas als onze designer meer kan, een demo
te geven. 

We spreken af de tijdslijnen met Bernard te bespreken en hem voor te stellen in de week voor
5 april een final demonstratie te doen. Hij krijgt dan ook een versie van de programmatuur
zoals die dan is om mee te experimenteren.

\paragraph{Voor volgend overleg}
Komende woensdag wordt een heel kort overleg met alleen updaten waar we staan. Komende zaterdag
moet de xmas integratie af zijn en als alles meezit kunnen we ook \json bestanden bewaren.

\topic{Actiepunten}

In figuur \ref{fig:openpunten} de openstaande punten. Deze punten werken we elke sessie af, en zetten de 
afgeronde akties in de lijst daaronder.

\vspace{2em}

\begin{figure}[p]
%\begingroup
\begin{tabular}{|l|l|p{25em}|}
\hline
{\bf datum} & {\bf wie} & {\bf Aktie}\\\hline
16-03-2015  & Jeroen    & Is Agilefant zelf installeren een optie?\\\hline
16-03-2015  & Guus      & Alternatieven voor Agilfant bekijken\\\hline
16-03-2015  & team      & Wat te doen met Agilefant geld fuik?\\\hline
16-03-2015  & team      & Tijdslijnen plus voorstel demo naar Bernard.\\\hline
16-03-2015  & team      & Bernard herinneren aan IN/OUT kwestie: we wachten erop\\\hline
09-03-2015  & team      & Bernard vragen over maximaal aantal poorten te tekenen\\\hline
09-03-2015  & team      & Bernard vragen over IN/OUT vervangen door Source/Sink\\\hline
02-03-2015  & team      & Sturen Bernard een patch van de \vt verschillen\\\hline
02-03-2015  & Bernard   & Bernard komt terug op repo: git of svn?\\\hline
24-01-2015  & Bernard   & Filmpje van een collega, Bernard geeft een seintje\\\hline
24-01-2015  & Bernard   & Voorbeeld deadlock op 2 deadlock vrije composites\\\hline
\end{tabular}
\caption{Aktielijst open punten}\label{fig:openpunten}
\end{figure}
%\endgroup

\vspace{2em}

\begin{figure}[p]
%\begingroup
\begin{tabular}{|l|l|l|p{10em}|p{10em}|}
\hline
{\bf datum} & {\bf gereed} & {\bf wie} & {\bf Aktie}                                      & commentaar      \\\hline
09-03-2015  & 09-03-2015   & team      & Project issue voortgang consequenties bespreken. &     \\\hline
24-01-2015  & 22-02-2015   & team      & We moeten hernieuwde contactmomenten afspreken   & 22 febr 2015 afspraken gemaakt\\\hline
07-02-2015  & 22-02-2015   & team      & Afspraak met Bernard voor 21 februari 2015       & 22 febr 2015 10:30\\\hline
07-02-2015  & 13-02-2015   & team      & Werkwijze opnieuw bezien                         & vastgeled in plan\\\hline
07-02-2015  & 13-02-2015   & team      & Vernieuwd plan uitbrengen.                       & Akkoord B. 19-02-2015\\\hline
24-01-2015  & 07-02-2015   & Bernard   & Voert bugfix op xmas.h uit en geeft signaal      & uitgevoerd\\\hline
24-01-2015  & 07-02-2015   & team      & splitsen header files afstemmen met Bernard      & n.v.t.\\\hline
26-01-2015  & 07-02-2015   & team      & Kan de licentie van \w{bitpower lib} naar GPLv3? & uitgevoerd\\\hline
\end{tabular}
\captionof{table}{Aktielijst gesloten punten}
\end{figure}
%\endgroup

\end{Minutes}

\end{document}
