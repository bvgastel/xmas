\documentclass{article}
\usepackage{subfig}
\usepackage[dutch]{babel}
\usepackage{minutes}
\usepackage{graphicx}
\usepackage{float}

\title{Notulen 013 overleg team 33}
\author{Stefan}
\minutesstyle{header=table, vote=list, contents=true, columns={1}}

\begin{document}
%\selectlanguage{dutch}

\begin{Minutes}{Overleg 013 team 33}
\participant{Guus Bonnema, Jeroen Kleijn, Stefan Versluys}
\minutesdate{19 november 2014}
\location{Skype}

\maketitle% This is where LaTeX makes the title

\topic{Bespreken voortgang domeinanalyses}

DA van stefan en Guus zijn klaar rest enkel nog feedback op de laatste paar kleine aanpassingen.
Jeroen zijn DA zal volgende week klaar zijn.

\topic{Git repository}
(open item van 15/11)
Inmiddels heeft de OU een Git server beschikbaar gemaakt. De vraag is of wij van
deze server gebruik zullen gaan maken of dat de ontwikkeling via github blijft.
Mogelijk hebben Freek en/of Bernard een voorkeur voor een private repository i.v.m.
de verspreiding van hun source code.

\topic{Ontwikkelomgeving}
Guus en Stefan gaan alvast aan de slag met Codeblocks en FLTK om te zien of er een
``Hello world'' app kan gebouwd worden zonder fouten.
Stefan doet dit voor Windows en Guus voor Linux.
Bedoeling is om dit tegen zaterdag rond te hebben.
Aan Bernard vragen of hij de MAC OS tester kan zijn(?)

\topic{Repo Merging DA}
Guus gaat de DA branches van Stefan en Guus mergen met de master.

\topic{Agile tool}
Stefan stuurt de links door van de eerder getestte agile tools naar Guus. Bedoeling is om te zien
of Agilefant een goede keuze is. Indien gekozen, maken alle teamleden een account aan
zodat de agile tool ingesteld kan worden.


\topic{Demonstratie WickedXmas}

Door het DA onderzoek van stefan heeft deze voldoende kennis van de WickedXMas tool 
om een demo te geven aan de andere teamleden (zie ook DA composite objects).
Indien blijkt dat er nog te veel onduidelijkheden zijn is een eventuele bijkomende demo
van Freek of Bernard noodzakelijk. In dat geval kan het team zich eerder op
de specifieke zaken richten tijdens deze observatie.


\topic{Vragen en afspraken}

\begin{itemize}
 \item Het volgende overleg is zaterdag 22 november 2014. 
\end{itemize}


\end{Minutes}
\end{document}
