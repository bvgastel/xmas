\documentclass[a4paper,final]{article}
\usepackage{subfig}
\usepackage[dutch]{babel}
\usepackage{hyperref}
\usepackage{minutes}
\usepackage{graphicx}
\usepackage{float}
\usepackage{color}

\title{Resultaten work meeting 005 Voortgang }
\author{Guus}
\minutesstyle{header=table, vote=list, contents=true, columns={1}}

\begin{document}
%\selectlanguage{dutch}

\newcommand{\Noc}{\textsc{NoC}\xspace}
\newcommand{\w}[1]{\textsc{#1}\xspace}
\newcommand{\qml}{\textsc{Qml}\xspace}
\newcommand{\qt}{\textsc{Qt}\xspace}
\newcommand{\qtquick}{\textsc{QtQuick}\xspace}
\newcommand{\tv}{\textsc{TeamViewer}\xspace}
\newcommand{\cpp}{\textsc{C++}\xspace}
\newcommand{\fltk}{\textsc{Fltk}\xspace}
\newcommand{\ou}{\textsc{OU}\xspace}
\newcommand{\signal}{\textsc{signal}\xspace}
\newcommand{\slot}{\textsc{slot}\xspace}

\begin{Minutes}{Work meeting 005}
\participant{Guus Bonnema, Stefan Versluys, Jeroen Kleijn}
\subtitle{Voortgang en hoe nu verder?}
\minutetaker{Guus}
\minutesdate{26 januari 2015}
\location{Skype}

\maketitle% This is where LaTeX makes the title

\topic{Voortgang}

\paragraph{Status} We hebben de subdirectories voor model en test verwijderd
naar aanleiding van de opmerkingen van Bernard afgelopen zaterdag. Indien nodig kunnen
we altijd teruggrijpen op de sources.

\topic{Hoe nu verder?}
We zijn het er over eens dat integratie van xmas.h in onze sources en verbinden met
onze \qml2 user interface de volgende stappen zijn. Na een discussie over hoe dit
precies te verbinden met elkaar, maken we de volgende afspraken:

\begin{description}

	\item[integratie xmas.h] Dit moet de eerste stap zijn. xmas.h is verweven met 
	de bestaande VT's die Bernard tegelijkertijd met xmas.h heeft ontwikkeld.
	Het idee is om de header files te splitsen zodat we de data laag in een 
	aparte subdirectory kunnen opslaan. Omdat we geen functies veranderen zal
	Bernard hier geen bezwaar tegen hebben. \textit{Af te stemmen met Bernard}.
	
	\item[koppeling met user interface] Na integratie moeten we de \qml2 user
	interface koppelen met de data laag. Na wat discussie kwamen we uit op
	een koppeling met \w{signal} en \w{slot} dat waarschijnlijk efficient 
	vanuit javascript in \qml2 kan. De andere richting -- van data naar \qml2 --
	zou op dezelfde manier ook geen problemen mogen opleveren.
	
	\item[dubbele data] Stefan opperde het bezwaar dat we de data nu twee
	keer opslaan. Gezien de keuze voor \qml2 en de enorme voordelen die ontwikkelen
	met \qml2 met zich mee brengt, lijkt dit onvermijdelijk. Als je echter
	het systeem op hoog niveau ziet, dan komt de data laag overeen met het \w{model}
	in het \w{mvc} patroon en de gehele \qml2 applicatie als de presentatie laag.
	
	\item[connections] In de user interface van Stefan zijn de connections geen
	objecten van zichzelf. Dat kan leiden tot problemen wanneer we connecties als
	objecten moeten gaan benaderen. Stefan heeft hier iets over gevonden. Hier horen
	we later meer van.
	
\end{description}

De volgende stap is nu het integreren van xmas.h. Komende woensdag bekijken we de voortgang
en de daarop volgende stap.

\topic{Naschrift}

Omdat de tijd van Bernard beperkt is en we weinig contactmomenten gaan hebben, heb ik
besloten om de uitstaande acties van en naar Bernard in een lijst te zetten.
Dat voorkomt dat we punten kwijtraken. De lijst werken we elke sessie even af, 
en halen afgeronde akties eraf.

\begin{figure}
\begin{tabular}{|l|l|l|}
\hline
{\bf datum} & {\bf wie} & {\bf Aktie}\\\hline
24-01-2015  & Bernard   & Voert bugfix op xmas.h uit en geef signaal\\\hline
24-01-2015  & team      & We moeten hernieuwde contactmomenten afspreken\\\hline
24-01-2015  & Bernard   & Voorbeeld deadlock op 2 deadlock vrije composites\\\hline
26-01-2015  & team      & splitsen header files afstemmen met Bernard\\\hline
\end{tabular}
\caption{Aktielijst open punten}
\end{figure}

\end{Minutes}

\end{document}
