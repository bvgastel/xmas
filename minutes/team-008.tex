\documentclass{article}

\usepackage[dutch]{babel}
\usepackage{minutes}

\title{Notulen 008 overleg team 33}
\author{Stefan}
\minutesstyle{header=list, vote=list, contents=true}

\begin{document}
%\selectlanguage{dutch}

\begin{Minutes}{Overleg 008 team 33}
\participant{Guus Bonnema, Stefan Versluys, Jeroen Kleijn}
\minutesdate{8 Oktober 2014}
\location{Skype}

\maketitle% This is where LaTeX makes the title

\topic{Planning}

Bedoeling van overleg was om te zien of de planning klaar is om naar Freek te sturen, met volgende finale stappen:
\begin{itemize}
 \item Iedereen bekijkt nog eens grondig de pdf en past aan waar nodig en zet het in de repo.
 \item Open vragen die niet meer relevant zijn mogen verwijderd worden.
 \item Iedereen bevestigd via email dat het document klaar is om aan Freek te sturen , of indien blijkt dat er toch nog overleg noodzakelijk is dit aan te geven. 
 \item Planning wordt voor het weekend naar Freek gestuurd
\end{itemize}

\paragraph{DAD}
Er wordt een Skype sessie voorzien om DAD toe te lichten zodat iedereen dit zo juist mogelijk kan toepassen.

\paragraph{Markdown}
Verder is er nog gesproken geweest over Markdown en pandoc , het idee is om dit te gebruiken voor eenvoudige documenten en Latex voor het grotere werk.

\paragraph{Teamviewer}
Teamviewer is uitgeprobeerd voor het delen van een bureaublad , dit werkt prima, maar de kwaliteit van VoIP is beduidend minder dan deze van Skype. Bedoeling is om in het vervolg Skype als communicatie middel te gebruiken en teamviewer voor het white board en bureaublad 


\end{Minutes}
\end{document}
