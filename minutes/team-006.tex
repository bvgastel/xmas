\documentclass{article}

\usepackage[dutch]{babel}
\usepackage{minutes}

\title{Notulen 006 overleg team 33}
\author{Guus}
\minutesstyle{header=list, vote=list, contents=true}

\begin{document}
%\selectlanguage{dutch}

\begin{Minutes}{Overleg 005 team 33}
\participant{Freek Verbeek, Guus Bonnema, Stefan Versluys, Jeroen Kleijn (na vertrek Freek)}
\minutesdate{3 Oktober 2014}
\location{Skype}

\maketitle% This is where LaTeX makes the title

\topic{Inleiding}

Deze sessie gaat over domeinanalyse. Voor komende maandag maken we een vervolg afspraak om te praten over onderzoekscontext.

\topic{Domeinanalyse}

Freek lichtte in deze sessie toe wat domein analyse is, wat valkuilen zijn en aan welke criteria een goede domein analyse voldoet. Bij deze een samenvatting
van de toelichting.

\subtopic{Onderscheid onderzoekscontext en domeinanalyse}
De domeinanalyse gebeurt individueel, de onderzoekscontext is een team effort. De onderzoekscontext komt in een volgende sessie aan de orde.

\subtopic{Wat is een domeinanalyse?}
De domeinanalyse ligt op het domein van de klant (Bernard): het betreft een deelprobleem. Met een domeinanalyse licht je
\'e\'en deel van het probleem uit het geheel en bestudeer je wat gedetailleerder en wetenschappelijker dan de andere deelproblemen.

Freek benadrukte dat het belangrijk is voor het project om de domeinanalyses eerst te doen. Dat steunt het project bij de uitvoering omdat belangrijke beslissingen
dan al genomen zijn. Stefan merkte later op om de onderdelen met de meeste onzekerheid eerst te doen, waarvan akte.

\subtopic{Omvang en diepgang van de domeinanalyse}
Twee belangrijke kenmerken van een domeinanalyse zijn dat het een beperkt deel van het onderwerp is en dat het daadwerkelijk ondersteunend is aan het project.
De valkuil is om een te breed onderwerp te nemen.

De duur bepaalt grofweg de diepgang. De duur van de domeinanalyse is planmatig 10\% van de totale duur.
De werkelijke uitvoering kan meer of minder zijn.

\subtopic{Resultaat van een domeinanalyse}
De uitkomst van een domeinanalyse is een beschrijving van het probleem,
een beschrijving van de alternatieven, een beschrijving van de keuze criteria en een gewogen aanbeveling. Gewogen betekent dat de aanbeveling naar objectieve criteria plaatsvindt.

\subtopic{Aanbeveling en beslissing}
De domeinanalyse bevat een aanbeveling. De beslissing is een team effort, waarbij de klant (Bernard in dit geval) de doorslaggevende stem heeft.

\subtopic{Beoordelingscriteria}
De beoordeling van elke domeinanalyse is individueel evenals de uitwerking.
Om het resultaat en de beoordeling ervan wat beter te kunnen beschrijven gaf Freek
een oude versie van de beoordelingscriteria van een domeinanalyse. Het overzicht
toont de cumulatieve beoordelingscriteria. Bijvoorbeeld, voor een 6 moet de analyse aan alle
criteria tot en met 6 voldoen.

Freek noemde dat de examinator een lijst met alternatieven en de wetenschappelijk gemotiveerde aanbeveling bij
de beoordeling belangrijk vindt.

\begin{center}
\begin{tabular}{|c|p{30em}|}
{\bf cijfer} & {\bf beoordeling} \\
 4 - 5  & Er is een individuele domeinanalyse. \\
        & Het deelprobleem is duidelijk omschreven.\\
        & Er is een motivatie van het probleem.\\
 6 - 7  & Een overzicht van alternatieven. \\
 8      & Een passend advies op basis van een academisch gemotiveerde keuze.\\
 9 - 10 & Het onderwerp is interessant en complex. Het overzicht is aantoonbaar volledig.\\
\end{tabular}
\end{center}

\subtopic{Voorbeelden van domeinanalyse}

\begin{center}
    \begin{tabular}{|p{2.5cm}|p{10cm}|}
    \hline
        {\bf vb}		& {\bf beschrijving} \\\hline
        Integratie		& \textsc{Structureel}: wat wordt de interface met de analyse tools? \\
	met tools 		& \textsc{Dynamisch}: hoe gaan we de analyse tools integreren in het ontwerp tool?\\
				& \textsc{Functioneel}: hoe gaan we de analyse tools integreren in het ontwerp tool?\\\hline
        Platform onafhankelijk UI toolkit & Uitgaande van de high level requirements voor het ontwerp
					    tool, zoals platform onafhankelijkheid, uitbreidbaarheid
					    en hechte integratie: kies een UI toolkit die het
					    beste past bij dit project en dit probleem gebied.\\\hline
        combinatorische cycle checker & je zou de combinatorische cycle checker kunnen bestuderen inclusief de datastructuur die
					deze tool deelt met de andere analysetools en een domeinanalyse kunnen doen
					over hoe deze tool ge\"{i}ntegreerd kan worden met jullie tool.\\\hline
         combinatorial objects &  een onderzoek hoe om te gaan met combinatorial objects.
				    Bv hoe grafisch weer te geven, hoe op te slaan in de data structuur\\\hline
    \end{tabular}

\end{center}

\end{Minutes}
\end{document}
