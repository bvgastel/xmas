	\documentclass[a4paper,final]{article}
\usepackage{subfig}
\usepackage[dutch]{babel}
\usepackage{hyperref}
\usepackage{minutes}
\usepackage{graphicx}
\usepackage{float}
\usepackage{color}

\title{Resultaten work meeting 004 Contact Bernard }
\author{Guus}
\minutesstyle{header=table, vote=list, contents=true, columns={1}}

\begin{document}
%\selectlanguage{dutch}

\newcommand{\Noc}{\textsc{NoC}\xspace}
\newcommand{\w}[1]{\textsc{#1}\xspace}
\newcommand{\qml}{\textsc{Qml}\xspace}
\newcommand{\qt}{\textsc{Qt}\xspace}
\newcommand{\qtquick}{\textsc{QtQuick}\xspace}
\newcommand{\tv}{\textsc{TeamViewer}\xspace}
\newcommand{\cpp}{\textsc{C++}\xspace}
\newcommand{\fltk}{\textsc{Fltk}\xspace}
\newcommand{\ou}{\textsc{OU}\xspace}
\newcommand{\signal}{\textsc{signal}\xspace}
\newcommand{\slot}{\textsc{slot}\xspace}

\begin{Minutes}{Work meeting 004}
\participant{Guus Bonnema, Stefan Versluys, Jeroen Kleijn}
\subtitle{Voortgang, demo en issues}
\minutetaker{Guus}
\minutesdate{24 januari 2015}
\location{Skype}

\maketitle% This is where LaTeX makes the title

\topic{Voortgang}

\paragraph{Design met Qt} Ons laatste gesprek was op 6 december 2014. Sindsdien
is er het \'e\'en en ander gebeurd, waaronder wisseling van de \fltk library naar
\qt (die heeft Bernard nog net meegemaakt). Daarna hadden we diverse opties waaronder 
het gebruik van de standaard user interface op basis van \cpp, graphicsview, 
het gebruik van \qtquick1 (ook \qml1 genoemd) of \qtquick2 (ook \qml2 genoemd).
We hebben gekozen voor \qtquick2 omdat dit de toekomst heeft en omdat het beter
performt: het maakt gebruik van \textsc{OpenGL} voor hardware versnelling. Als een
computer geen \textsc{OpenGL} heeft dan is er een emulator beschikbaar. Wij hebben
die nog niet in ons project opgenomen.

\paragraph{Qml2} Het gebruik van \qml2 bracht de vraag of scheiding van data 
en presentatie hiermee in het gedrang zouden komen (Jeroen), en dat leidde tot 
de beslissing om het datamodel apart in \qml weer te geven, los van de user interface.
Dat is inmiddels gebeurd. Stefan heeft zich op de user interface geconcentreerd, Guus op
het data model.

\paragraph{Demo} Stefan doet een demo van de user interface zoals die er nu uitziet
met gebruik making van \qml2. Het enige \cpp code dat voorkomt is om het \qt systeem
te initialiseren en het \qml2 systeem op te starten. De gehele user interface is in 
\qml2 geschreven. Veel zaken zijn in \qml2 geregeld.

\topic{Contactmomenten}

Bernard legt uit waarom hij de afgelopen periode en de komende periode weinig tijd 
aan het project kan besteden. De \ou heeft sinds begin december een stormachtige 
ontwikkeling op reorganisatie gebied doorgemaakt. Alle medewerkers
boven 63 zijn met vervroegd pensioen gestuurd onder verdeling van het werk onder de
overige medewerkers. Dit is ongeveer 1/3e van de lesgevende medewerkers. De overige
2/3e moest dus vanaf begin december alle zeilen bijzetten om de lessen over te nemen.

Bernard is \'e\'en van de gelukkigen en mag zich verheugen in een 50\% docent 
aanstelling (waarvoor gefeliciteerd!) en 50\% promotie aanstelling. 
De laatste is nu over twee jaar uitgesmeerd met de verwachting dit 
na afloop om te zetten in een permanente aanstelling. Bernard moet nu ook op zeer 
korte termijn lessen voorbereiden van nieuwe vakken. Daarnaast is Freek de komende 
drie maanden naar Amerika. Hij is nog wel bereikbaar in skype en mail, maar weinig 
beschikbaar. 

Dit geheel heeft voor ons team als positief neven effect, dat de ontwikkelingen 
op het gebied van verificatie tools tot juni dit jaar stil komen te liggen. Wij 
kunnen ons verheugen in een stabiele omgeving wat betreft verificatie tools. 
Over de negatieve effecten hebben we het niet gehad. \textit{We moeten afspraken maken 
over contactmomenten met zowel Bernard als Freek.}

\topic{Data model}

E\'en van de vragen was of het data model zich aan de structuur van Bernard moest 
houden of een vervangende structuur kon invullen. Bernard heeft duidelijk aan dat
de verificatie tools werken vanuit zijn structuur. Als team kunnen wij de structuur
wijzigen en verbeteren, maar deze structuur is het uitgangspunt voor de VT's.

\paragraph{communicatie} We hebben afgesproken in kleine stappen te verbeteren, maar 
groot genoeg dat er ruimte is voor experimenteren. Een voorbeeld van zo'n stap is het 
toevoegen van composites aan de structuur (die momenteel geen composites kent). 

\paragraph{disk opslag} De manier van opslaan van de structuur is vrij met als 
restrictie dat we deze wel vanuit een console programma kunnen inlezen (zie volgend punt). 
Het meest logische is om uit te gaan van de huidige structuur en deze dan op schijf te 
zetten aangevuld met de mogelijkheid van composites. 

\paragraph{tooling} We kiezen er voor de structuur te kopi\"{e}ren van subversion naar
github, zodat we kunnen experimenteren zonder de verification tools in de weg te zitten.
Bovendien hebben we dan maar \'e\'en ontwikkel tool om mee te werken. \textit{Bernard moet nog
wel een wijziging aanbrengen op de structuur in verband met een kleine correctie. Hij
laat ons weten wanneer dit is gebeurd en welke source bestanden precies de VT structuur 
precies zijn.}

\topic{Consequenties en vragen} 

\begin{description}

	\item[\qml2 datamodel] Deze komt te vervallen. Hiervoor komt de structuur van Bernard
	in de plaats.
	
	\item[observer patroon] We blijven \signal en \slot gebruiken om contact tussen
	het design tool en de VT structuur te waarborgen. Hiermee kunnen we elke wijziging
	bij het ontwerpen in de design tool direct verwerken in de VT structuur.
	
	\item[On the fly controle] Het is gewenst om kleine en korte controles tijdens de 
	constructie van een network uit te voeren. Dit kan op de achtergrond in een aparte
	thread of process met terugkoppeling naar de user interface. Bernard denkt hierbij
	aan de syntax check en de combinatorische checker.
	
	\item[Platgeslagen structuur] Hoe we de VT structuur ook vormgeven, uiteindelijk
	moeten we de structuur platgeslagen aan kunnen bieden. Op dit moment kunnen VTs
	alleen nog met de platgeslagen structuur werken. Het streven is in de toekomst 
	met composites te kunnen werken.
	
	\item[Verificatie tools] Er zijn 5 stappen in het verificatie proces:
	
	\begin{description}
	
		\item[Syntax checker] Dit betreft eenvoudige controles van compleetheid en 
		correctheid.
		
		\item[Combinatorisch cycle checker] De cycle checker controleert op interne 
		cycles van afhankelijkheden. Zie ook de domein analyse van Jeroen.
		
		\item[Type inferentie] Dit gaat over het bepalen van de types van het data
		deel van elk channel.
		
		\item[Specificatie checker] Deze checker is innoverend en heeft een duidelijke 
		meerwaarde. Het controleert het pad dat een datapakket afloopt: waar komt het het 
		netwerk binnen en waar gaat het er weer uit?
		
		\item[Deadlock checker] Hiervan is momenteel een oudere versie van de hand van Freek
		in productie, die uitgaat van een \textit{brute force} benadering waarbij de breedte van 
		het datapad bepalend is voor de performance. Bij een breed datapad (32 bits) en veel 
		componenten explodeert de search space. Bernard is bezig een nieuw concept te bouwen.
		Tot juni dit jaar gaat hier niets aan gebeuren.
		
	\end{description}
	
	Dit proces kan later aanvulling vinden met andere verificatie tools, maar op dit moment
	is dit het gehele proces. Een voorbeeld van toekomstige aanvulling is een \textit{livelock} 
	checker. 
	
	\item[specificatie] Diverse onderdelen van de componenten (zoals specificatie) zijn voor
	de designer niets anders dan properties terwijl ze invoer zijn voor de VT's zoals de 
	specificatie checker en de type inferentie tool. Het is van belang dit op te nemen in
	het design tool, maar het tool hoeft er niets mee te doen.
	
	\item[uncore / core] Het onderzoek van deze research concentreert zich op de \w{uncore}. 
	De technieken zou men ook op de \w{core} kunnen toepassen, maar de huidige research richt 
	zich daar niet op.
	
	De \w{uncore} is alles wat niet bij de \w{core} hoort. De \w{uncore} bevat de \w{fabrics} 
	en andere elementen, zoals een \w{bus}. Bernard legt uit hoe de huidige hardware 
	verbetering vindt in het toepassen van \w{fabrics}. Momenteel moeten component ontwerpers 
	zich bewust zijn van alle andere componenten om die rechtstreeks te kunnen adresseren. 
	Met \w{fabrics} is het mogelijk de adressering aan de \w{fabrics} over te laten. Dit 
	levert productie voordelen en performance voordelen op.
	
	\item[Aantal componenten per composite] Een composite kan van ca 100 tot 1000 
	componenten bevatten.  Denk aan een spidergon van 64 nodes, dat levert 
	ongeveer 1000 componenten op.
	
	\item[Aantal componenten per netwerk] Bernard schat dat een groot netwerk $100.000$ 
	tot maximaal $1.000.000$ componenten kan bevatten. Dat heeft grote gevolgen voor de
	platgeslagen modellen, maar minder gevolgen voor de designer, omdat we daar met
	composites en geparametriseerde objecten kunnen werken.
		
	\item[component / transitor] Bernard schat 10 - 100 transistors per component, zodat
	een netwerk van de chipsets met circa 3 miljoen transitors (Intel oudere pentium chipset) 
	in onze termen uit $30.000$ - $300.000$ componenten bestaat. 
	
	\item[VT uitvoering] Bernard legt uit, dat men wel uitvoert vanaf een server
	waar geen \qt ge\"installeerd is. De console uitvoering van VT's moet dus vrij
	zijn van \qt.

	Op deze servers heeft men ook niet altijd \textsc{root} rechten. Dit houdt in, dat we 
	er niet van uit mogen gaan dat men \qt of andere hulp programmatuur kan installeren.
	We moeten uitgaan van standaard \cpp.
	
	\item[VT wijziging] Wie werken er nu aan de verificatie tools? Dit blijken Bernard zelf, 
	Bas Joosten, Freek Verbeek en soms Julien Schmalz. Zoals opgemerkt: pas weer vanaf juni.
	Wij krijgen een signaal zodra we de code voor de VT structuur kunnen kopi\"{e}ren.
	
	\item[composite voorbeeld] In een interview in het vorige ABI team gaf Freek een antwoord
	op de vraag waarom platslaan nog steeds nodig is. Hierin noemt hij het feit dat twee
	composite deadlock vrij kunnen zijn en samen toch een deadlock opleveren. Deze informatie
	heeft op ons project geen invloed. \textit{Bernard gaat dit bij Freek navragen}. 
	
\end{description}


\end{Minutes}
\end{document}
