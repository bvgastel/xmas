\documentclass{article}

\usepackage[dutch]{babel}
\usepackage{minutes}

\title{Notulen 010 overleg team 33}
\author{Jeroen}
\minutesstyle{header=table, vote=list, contents=true, columns={1}}

\begin{document}
%\selectlanguage{dutch}

\begin{Minutes}{Overleg 010 team 33}
\participant{Guus Bonnema, Stefan Versluys, Jeroen Kleijn, Freek Verbeek, Bernard van Gastel}
\minutesdate{18 Oktober 2014}
\location{Skype}

\maketitle% This is where LaTeX makes the title

\topic{Presentatie Plan}

Opmerkingen van Freek en Bernard: 
\begin{itemize}
\item Houdt er rekening mee dat wij uiteindelijk de documentatie als geheel opgeleveren.
\item We willen nauw betrokken worden, dus meer interactie met belanghebbende.
\item Onderzoekscontext is iets wat nadien mag, het beschrijft wat de oplossing
als wetenschappelijke meerwaarde biedt in dit domein of hoe kan de tool de aan
verdere ontwikkeling in dit domein bijdragen.
\item Maandag as. zal de verificatie source beschikbaar gesteld worden zodat we 
allen onder dezelfde git repo (xmas) werken.
Bedoeling is om deze source niet aan te passen maar om beter inzicht te krijgen
in de interface en omdat de verificatie tools bijgewerkt worden (on going
proces) is het op deze manier mogelijk om steeds de WickedxMas tool in
overeenstemming te brengen. Indien blijkt dat er aanpassingen aan de interface
noodzakelijk zijn is dit mogelijk. Een verificatie tool kan steeds via de
command line aangeroepen worden.
\end{itemize}


\topic{Open vragen - Antwoorden}

\begin{itemize}
\item Risico voor productiefouten: Produceren van hardware staat los van deze tool, dus risico vervalt.
\item De taal voor de tool is Engels, dit geldt zowel voor internationalisering als lokalisering.
\item Online updatemogelijkheid van de tool hoeft niet maar is een meerwaarde.
\item Dynamische controle zoals hier omschreven is belangrijk omdat je de analyse
	tools in ontwikkeling zijn. Je wilt experimenterend nieuwe tools ontwikkelen. Aan en uit
	kunnen zetten van de analyse tools is dus wel belangrijk.
		
\item Het is voldoende dat de tool standalone werkt, uitwisseling van een
ontwerp via email of het bestandje in een git repo kunnen plaatsen plaatsen is
voldoende. Client / Server werking mag, maar is extra\footnote{Ook al is C/S niet strikt
	nodig, het kan handig zijn om te zorgen dat de architectuur al op die manier is
	ingericht (dat kost niet per s\'{e} extra programmeer werk)}.
\end{itemize}

\topic{Besluiten}

\begin{itemize}
\item Bernard en Freek benadrukten duidelijk dat ze meer betrokken wensen te worden.
Ze lieten blijken veel belang te hechten aan een goede tool en niet alleen
als gebruiker van de tool maar zeker ook als ontwikkelaar. Ze wensen dus bij de
bouw van WickedxMas zo veel als kan mee te participeren opdat ze bij de
oplevering de code begrijpen en makkelijk verder kunnen ontwikkelen.
\item In de planning nemen we expliciet de formele momenten op wanneer we 
Bernard en Freek spreken.
\item Volgende Skype sessie met team 33 intern is maandag aanstaande.
\end{itemize}

\topic{Domein analyse}
\subtopic{Inhoud}
Freek is het eens met de inhoudelijke keuzes van de teamleden.
\subtopic{Proces}
Freek doet de begeleiding en wil graag tijdens het proces begeleiden.
In tegenstelling tot de instructie\footnote{waar de docent maar \'e\'en 
concept versie krijgt} wil Freek echt onderweg begeleiding geven. Wij mogen
onderweg bij de domein analyse review van stukken tekst vragen.

Qua inhoud moet de domein analyse het volgende bevatten:

\begin{itemize}
	\item mogelijke opties
	\item afweging van de opties
	\item aanbeveling
\end{itemize}

Het is een individuele opdracht. Maar onderling contact is toegestaan en wenselijk.
Bernard benadrukt graag bij het proces betrokken te willen zijn.

\end{Minutes}
\end{document}
