\documentclass{article}

\usepackage[dutch]{babel}
\usepackage{minutes}

\title{Notulen 010 overleg team 33}
\author{Jeroen}
\minutesstyle{header=table, vote=list, contents=true, columns={1}}

\begin{document}
%\selectlanguage{dutch}

\begin{Minutes}{Overleg 010 team 33}
\participant{Guus Bonnema, Stefan Versluys, Jeroen Kleijn, Freek Verbeek, Bernard van Gastel}
\minutesdate{18 Oktober 2014}
\location{Skype}

\maketitle% This is where LaTeX makes the title

\topic{Presentatie Plan}

Opmerkingen van Freek en Bernard: 
\begin{itemize}
\item Hou rekening dat uiteindelijk de documentatie als geheel geleverd wordt.
\item We willen nauw betrokken worden, dus meer interactie met belanghebbende.
\item Onderzoekscontext is iets wat nadien mag, het beschrijft wat de oplossing
als wetenschappelijke meerwaarde biedt in dit domein of hoe kan de tool de aan
verdere ontwikkeling in dit domein bijdragen.
\item Maandag as. zal de verificatie source beschikbaar gesteld worden zodat we 
allen onder dezelfde git repo (xmas) werken.
Bedoeling is om deze source niet aan te passen maar om beter inzicht te krijgen
in de interface en omdat de verificatie tools bijgewerkt worden (on going
proces) is het op deze manier mogelijk om steeds de WickedxMas tool in
overeenstemming te brengen. Indien blijkt dat er aanpassingen aan de interface
noodzakelijk zijn is dit mogelijk. Een verificatie tool kan steeds via de
command line aangeroepen worden.
\end{itemize}


\topic{Open vragen - Antwoorden}

\begin{itemize}
\item Risico voor productiefouten: Produceren van hardware staat los van deze tool, dus risico vervalt.
\item De taal voor de tool is Engels, dit geldt zowel voor internationalisering als lokalisering.
\item Online updatemogelijkheid van de tool hoeft niet maar is een meerwaarde.
\item Dynamische controle is geen prioriteit.
\item Het is voldoende dat de tool standalone werk , uitwisseling van een
ontwerp via email of het bestandje in een git repo kunnen plaatsen plaatsen is
voldoende.
\end{itemize}

\topic{Besluiten}

\begin{itemize}
\item Bernard en Freek benadrukten duidelijk dat ze meer betrokken wensen te worden.
Ze lieten blijken veel belang te hechten aan een goede tool en niet alleen
als gebruiker van de tool maar zeker ook als ontwikkelaar. Ze wensen dus bij de
bouw van WickedxMas zo veel als kan mee te participeren opdat ze bij de
oplevering de code begrijpen en makkelijk verder kunnen ontwikkelen.
\item In de planning nemen we expliciet op wanneer we Bernard en Freek spreken.
\item Volgende Skype sessie met team 33 intern is maandag aanstaande.
\end{itemize}

\end{Minutes}
\end{document}
