\documentclass[a4paper,final]{article}
\usepackage{subfig}
\usepackage[dutch]{babel}
\usepackage{hyperref}
\usepackage{minutes}
\usepackage{graphicx}
\usepackage{float}
\usepackage{color}

\title{Resultaten work meeting 003 voortgang user interface en model }
\author{Guus}
\minutesstyle{header=table, vote=list, contents=true, columns={1}}

\begin{document}
%\selectlanguage{dutch}

\newcommand{\Noc}{\textsc{NoC}\xspace}

\begin{Minutes}{Work meeting 003 Architecture and build Team 33}
\participant{Guus Bonnema, Stefan Versluys, Jeroen Kleijn}
\subtitle{User Interface en qml model}
\minutetaker{Guus}
\minutesdate{19 januari 2015}
\location{Skype}

\maketitle% This is where LaTeX makes the title

\newcommand{\w}[1]{\textsc{#1}\xspace}
\newcommand{\qml}{\textsc{QML}\xspace}
\newcommand{\qt}{\textsc{Qt}\xspace}
\newcommand{\qtquick}{\textsc{QtQuick}\xspace}
\newcommand{\tv}{\textsc{TeamViewer}\xspace}
\newcommand{\cpp}{\textsc{C++}\xspace}

\topic{User Interface} 

\subtopic{Voortgang}

\paragraph{Stefan} Heeft de GUI en de primitives in eigen qml elementen
opgebouwd. Er is nog geen connectie met het qml model van Guus. Dat komt er
wel, maar pas als het qml model ver genoeg is. Tot die tijd is er nog veel te
doen op grafisch gebied.

Stefan geeft een demo vanuit zijn machine (met \tv). De primitieven kunnen nu
overal neergezet, gedraaid, en je kunt uitzoomen en inzoomen.  Alle \cpp
programmatuur is nu uit de user interface: er is nog alleen \qml. Connecties
maken is de volgende stap.

Besproken requirements voor GUI:

\begin{description}

	\item[grid] Guus vraagt om een grid, zodat elke component op een vaste
		lokatie komt met logische co\"{o}rdinaten.

	\item[poorten] De plaatsen van de poorten zijn nodig in de tekening: waar
		komt een poort?  Het lijkt Guus wel onderdeel van het qml model te
		kunnen worden.  We bekijken dit in het definitief model nader.

\end{description}

\paragraph{Jeroen} Heeft gisteren een review van het qml model gedaan (nog
oppervlakkig). De komende week voortzetting van papers lezen om het model
straks goed te kunnen implementeren. Zodra de basis compleet is, kan Jeroen 
code voor zowel de user interface als de plugin aanleveren. Jeroen heeft 
er voor gekozen om aan beide onderdelen te willen werken. 

\paragraph{Guus} Heeft gisteren het qml model gemaakt en is daar vandaag meer 
verder gegaan. Een probleem met de code samen met Stefan en Jeroen opgelost!
Naar verwachting morgen een compleet model en in de dagen daarna de bibliotheek
met primitieven.

\paragraph{Bernard} Het is vandaag 19 januari, dus Bernard is weer terug. Guus
maakt een afspraak voor komende zaterdag. Het is niet zeker hoe ver de 
design tool dan is, maar we hebben elkaar al meer dan een maand niet gesproken,
dus het is wel nuttig om diverse zaken af te stemmen, zoals de wijzigingen in
\qt en \qml.

\end{Minutes}
\end{document}
