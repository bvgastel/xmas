\documentclass{article}

\usepackage[dutch]{babel}
\usepackage{minutes}

\minutesstyle{header = {table}}
\title{Notulen 009 overleg team 33}
\author{Jeroen}
\minutesstyle{header=list, vote=list, contents=true}

\begin{document}
%\selectlanguage{dutch}

\begin{Minutes}{Overleg 009 team 33}
\participant{Guus Bonnema, Stefan Versluys, Jeroen Kleijn}
\minutesdate{11 Oktober 2014}
\location{Skype}

\maketitle% This is where LaTeX makes the title

\topic{Planning}

In het overleg zijn de opmerkingen van Freek over de conceptversie van de planning
besproken. De opmerkingen zullen in de planning worden verwerkt. De
openstaande punten zullen samen met Freek in een Skype sessie worden besproken.
In het plan wordt verder nog het onderdeel van de domeinanalyses verbeterd. De
eerste iteratie krijgt in de tijdsplanning een extra week voor het configureren
van de ontwikkelomgeving, git repository, en dergelijken. Verder wordt een extra
iteratie ingepland.

\topic{Markdown}

Het idee is om systeemdocumentatie met behulp van markdown te realiseren. 
Het sterke punt van markdown is dat je het naar
heel veel andere formaten kunt vertalen, niet alleen papier, maar ook scherm formaten
zoals html, een pdf of zelfs wiki formaat. Het sterke punt van LaTeX is dat
goed kan typesetten. LaTeX is vooral op papier gericht en is iets complexer (en krachtiger)
dan markdown. 

De bedoeling is om vooral systeem gerichte zaken zoals systeem ontwerp en guides
in markdown te doen, zodat we de docs net zo gemakkelijk op het web kunnen zetten
in een wiki als in een pdf kunnen aanbieden. Merk op dat GitHub automatisch
markdown converteert naar het web.
Specifieke documenten kunnen we nog steeds in LaTeX aanbieden.

De komende dagen experimenteren Stefan en Jeroen met markdown met als doel om
te kijken of zij er comfortabel mee zijn om hier documentatie mee te maken.
Met name kijken we of we plaatjes op kunnen nemen (vergelijkbaar met LaTeX) en 
of vertaling naar wiki of html en naar pdf naar onze zin gaat. Als het redelijk
gemakkelijk te maken is en pandoc werkt ook goed onder ms windows (Stefan), dan
lijkt dit een goed alternatief voor systeem documentatie.

We spreken er de volgende keer over hoe we markdown gaan gebruiken.

\topic{Agile}

Als voorbereiding constructie zullen de teamleden nadenken over de werkwijze tijdens
de iteraties. Welke rolverdeling wordt er afgesproken? Is er een projectleider?
Wie is de architecture owner? Stefan zal de werkwijze in een document beschrijven en
kijken welke tools er beschikbaar zijn om deze werkwijze te ondersteunen. Guus en
Jeroen zullen zich inlezen in DAD.

\topic{Presentatie planning}

Het streven is om aanstaande donderdag de definitieve versie naar Bernard (en Freek)
op te sturen. De zaterdag daarop kunnen de teamleden dan via Collaborate de planning
toelichten. De rolverdeling zal dan zijn dat Guus globaal de planning toelicht en
Stefan en Jeroen zullen respectievelijk de ontwikkelwijze (DAD) en de schedule toelichten.

\topic{Schedule}

%%
%% formaat: \schedule{yyyy/mm/dd}[hh:mm]{Beschrijving}
%% 			de tijd is optioneel
%%
\schedule{2014/01/13}[19:00]{Reguliere meeting. Voortgang taken, zie takenlijst}
\schedule{2014/10/18}[10:00]{Reguliere meeting. Freek en Bernard uitnodigen voor plan}


\topic{Taken}

\task{Guus}{Commentaar van Freek verwerken in de planning}
\task{Guus}{Afspraak maken met Freek over openstaande vragen}
\task{Guus}{Afspraak maken met Freek en Bernard om plan te presenteren}
\task{Stefan}{Werkwijze bij de iteraties beschrijven in readme.md}
\task{Stefan}{Domein analyse aanpassen en specifiek maken}
\task{Jeroen}{Domein analyse aanpassen en specifiek maken}
\task{Jeroen}{Extra iteratie vanwege onderzoekscontext verwerken in planning}
\task{Jeroen}{Vooraf iteratie afstem momenten vastleggen (na akkoord planning)}
\task{Guus}{Inhoud eigen domein analyse afstemmen met Freek}
\task{Stefan}{Inhoud eigen domein analyse afstemmen met Freek}
\task{Jeroen}{Inhoud eigen domein analyse afstemmen met Freek}
\task{Jeroen}{Opvragen source code van de analyse tools}
\task{Stefan}{Downloaden en installeren pandoc, experimenteren met markdown}
\task{Jeroen}{Experimenteren met markdown}
\task{Guus}{Uitzoeken of images in markdown goed werken}
\task{Allen}{Besluit nemen over systeemdocumentatie}


\topic{Besluiten}
\decisiontheme{DA}{Theme for a decision}
\decision{DA}{Besluit wie welke domeinanalyse doet}[
Guus doet de toolkit. Jeroen neemt de integratie van verificatie modules
voor zijn rekening met de cycle checker als uitgangspunt. Stefan doet
de combinatorische objecten en werkt samen met Freek uit wat dit precies betekent.]


\end{Minutes}
\end{document}
