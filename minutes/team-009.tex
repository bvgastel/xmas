\documentclass{article}

\usepackage[dutch]{babel}
\usepackage{minutes}

\title{Notulen 009 overleg team 33}
\author{Jeroen}
\minutesstyle{header=list, vote=list, contents=true}

\begin{document}
%\selectlanguage{dutch}

\begin{Minutes}{Overleg 009 team 33}
\participant{Guus Bonnema, Stefan Versluys, Jeroen Kleijn}
\minutesdate{11 Oktober 2014}
\location{Skype}

\maketitle% This is where LaTeX makes the title

\topic{Planning}

In het overleg zijn de opmerkingen van Freek over de conceptversie van de planning
besproken. De opmerkingen zullen in de planning worden verwerkt. De
openstaande punten zullen samen met Freek in een Skype sessie worden besproken.
In het plan wordt verder nog het onderdeel van de domeinanalyses verbeterd. De
eerste iteratie krijgt in de tijdsplanning een extra week voor het configureren
van de ontwikkelomgeving, git repository, en dergelijken. Verder wordt een extra
iteratie ingepland.

\topic{Markdown}

Het idee is om systeemdocumentatie met behulp van markdown te realiseren. Mocht markdown
hier niet voor geschikt zijn dan zal de systeemdocumentatie waarschijnlijk met LaTeX
worden gemaakt. De komende dagen wordt met markdown ge\"experimenteerd (onder andere
het toevoegen van figures) om te kijken of markdown gebruikt kan worden.

\topic{Agile}

Als voorbereiding constructie zullen de teamleden nadenken over de werkwijze tijdens
de iteraties. Welke rolverdeling wordt er afgesproken? Is er een projectleider?
Wie is de architecture owner? Stefan zal de werkwijze in een document beschrijven en
kijken welke tools er beschikbaar zijn om deze werkwijze te ondersteunen. Guus en
Jeroen zullen zich inlezen in DAD.

\topic{Presentatie planning}

Het streven is om aanstaande donderdag de definitieve versie naar Bernard (en Freek)
op te sturen. De zaterdag daarop kunnen de teamleden dan via Collaborate de planning
toelichten. De rolverdeling zal dan zijn dat Guus globaal de planning toelicht en
Stefan en Jeroen zullen respectievelijk de ontwikkelwijze (DAD) en de schedule toelichten.



\topic{Besluiten}

\begin{itemize}
 \item Guus verwerkt de opmerkingen van Freek in de planning
 \item Guus maakt een afspraak met Freek om de openstaande vragen te bespreken
 \item Stefan beschrijft de werkwijze tijdens de iteraties
 \item Stefan en Jeroen passen de sectie van domeinanalyse aan
 \item Stefan werkt de indeling van activiteiten bij zoals besproken
 \item Jeroen werkt de chart bij
 \item Jeroen informeert bij Freek en Bernard wat handige vaste contactmomenten zijn.
 (Pas als de planning is goedgekeurd).
 \item Waar de inhoud/grenzen van de domeinanalyse nog niet duidelijk is zullen de
 teamleden dit (individueel) overleggen met Freek. 
 \item Vooruitlopend op de domeinanalyse zullen de teamleden alvast de voor de 
 domeinanalyse benodigde informatie opvragen (bijvoorbeeld de source code van de analysetools).
\end{itemize}


\end{Minutes}
\end{document}
