\documentclass{article}

\usepackage[dutch]{babel}
\usepackage{minutes}

\title{Notulen 001 overleg team 33}
\author{Guus}
\minutesstyle{header=list, vote=list, contents=true}

\begin{document}
%\selectlanguage{dutch}

\begin{Minutes}{Overleg 001 team 33}
\participant{Guus Bonnema, Stefan Versluys, Jeroen Kleijn}
\minutesdate{16. September 2014}
\location{skype}

\maketitle% This is where LaTeX makes the title

\topic{Overleg met opdrachtgever en begeleider}

Bernard van Gastel is onze opdrachtgever en Freek Verbeek is onze begeleider. Beiden zijn 
OU medewerkers. Freek werkt vermoedelijk ook bij de universiteit van Utrecht.

Doel van het overleg is om de achterliggende probleemstelling te horen en begrijpen en
om naar aanleiding daarvan een plan te formuleren. Wat we over de probleemstelling weten, staat
in de projectaanvraag:

\begin{quote}
"De huidige tool heeft een aantal problemen:
\begin{itemize}
\item niet modulair opgezet (waardoor uitbreidingen moeizaam gaan)
\item op Windows API gebaseerd (waardoor de onderzoekers die
			gebruik maken van Mac het lastig kunnen gebruiken)
\item moeizame integratie met tools: WickedXmas is nu geschreven
	in C\#, en lijkt moeilijk te integreren met onze C/C++ tools
\item geen documentatie
\end{itemize}
Deze problemen moeten opgelost worden, danwel door een grote
refactoring van de bestaande code, danwel door het opnieuw
bouwen."
\end{quote}

Uit deze beschrijving is duidelijk dat zowel C/C++ als C\# van belang zijn en dat we moeten streven naar 
enige vorm van platform onafhankelijkheid.

Met de komende meeting hopen we opheldering te krijgen over een meer precieze probleemstelling.
Open vragen zijn:

\begin{itemize}
\item In hoeverre is platform onafhankelijkheid van belang? 
\item Heeft de opdrachtgever voorkeuren?
\item Wat is de rolverdeling tussen opdrachtgever (Bernard) en begeleider (Freek)?
\end{itemize}

Verder nemen we aan dat we de beschikking krijgen over de source van het oude project\footnote{Dit moet
wel anders kun je geen refactor uitvoeren.}.

\subtopic{Decisions}
\decisiontheme{Opdracht}{Meeting met opdrachtgever}
\decision{Opdracht}{Eerste meeting}
De eerste meeting gaan we voorstellen op zaterdag van 10:00 - 11:00 (tijdsduur is een aanname).
\decision{Opdracht}{Notulen}
Jeroen maakt de eerste notulen en verspreidt die dezelfde dag.

\topic{Onderlinge samenwerking}
We weten nog niet precies wat we moeten doen, en kunnen nog geen taken verdelen. 
Vooralsnog is duidelijk dat we C++ en C\# kennis hebben bij 
Stefan en Guus wil graag vaardigheden met C++ krijgen. We wachten de opdracht beschrijving
af voor we taken gaan verdelen.

We hebben afgesproken om zaterdag na het overleg nog even na te praten, zodat we de week daarna
nuttig kunnen gebruiken.

Verder spreken we alvast af, dat we elke zaterdag om 10:00 uur 
overleg hebben, zodat we elke week aan de slag kunnen,
onderlinge voortgang kunnen afstemmen en eventueel bijsturen 
als nodig. Mochten er inhoudelijk problemen zijn, dan
kunnen we dat in het weekend altijd ad hoc bespreken.

\topic{Planning}
Planning opstellen heeft nog geen zin. Na het overleg 
kunnen we een raamwerk planning neerzetten en misschien de 
eerste paar weken invullen.

Doelstelling van Stefan is om voor de zomervakantie 
het project afgerond te hebben: dit is waar we naar streven.

\end{Minutes}
\end{document}
