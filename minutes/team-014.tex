\documentclass{article}
\usepackage{subfig}
\usepackage[dutch]{babel}
\usepackage{minutes}
\usepackage{graphicx}
\usepackage{float}

\title{Notulen 014 overleg team 33}
\author{Stefan}
\minutesstyle{header=table, vote=list, contents=true, columns={1}}

\begin{document}
%\selectlanguage{dutch}

\begin{Minutes}{Overleg 014 team 33}
\participant{Jeroen Kleijn, Stefan Versluys}
\minutesdate{22 november 2014}
\location{Skype}

\maketitle% This is where LaTeX makes the title

\topic{Bespreken voortgang domeinanalyses}

Jeroen heeft ondertussen feedback van Freek en zal dit weekend z'n DA afwerken.
Stefan z'n DA is door Freek, na de laatste aanpassingen van vorige week, definitief ok bevonden.

\topic{Git repository}
(open item van 15/11)
Inmiddels heeft de OU een Git server beschikbaar gemaakt. De vraag is of wij van
deze server gebruik zullen gaan maken of dat de ontwikkeling via github blijft.
Mogelijk hebben Freek en/of Bernard een voorkeur voor een private repository i.v.m.
de verspreiding van hun source code.
--> deze vraag moeten we stellen op de eerst volgende meeting met Freek en Bernard

\topic{SVN repository}
Bernard heeft ook de svn gegeven waar de C\# sources van de huidige tool te vinden
zijn. Stefan heeft deze met Vs2010 gebuild als test. 

\topic{Ontwikkelomgeving}

FLTK apps testen met VC2010 onder windows is geen probleem. 
Code Blocks daarentegen levert heel wat problemen op onder Windows XP als 8.1. 
Versie 13.12 kan niet zonder aanpassingen aan het fltk script gebruikt worden.
Bij het builden van de FLTk libraries met make van MinGW onder Code Blocks komen
er veel warnings. De documetatie README.MSWindows.txt beschrijft dat er aan
gewerkt wordt om dit op te lossen.



\topic{Repo Merging DA}
Guus heeft afgelopen week de DA branches van Stefan en zichzelf gemergt met de master.


\topic{Agile tool}
De eerder getestte agile tools werden kort overlopen en worden maandag verder besproken.  
In de loop van volgende week moeten we een keuze maken. Eens er een keuze is
maken we elk een account en gaan we experimenteren.

\topic{Demonstratie WickedXmas}

Tijdens deze demo ging het
vooral over de verificatie stappen en wat deze op de achtergrond doen.
Voor Guus kan er in de loop van volgende week een demo gegeven worden.

\topic{Vragen en afspraken}
\begin{itemize}
\item Jeroen stuurt een email naar Guus
\item maandag wordt er beslist om een afspraak te maken met Freek en Bernard i.vm.:
\begin{itemize}
\item afronden Domein analyses
\item voorstel Agile tool
\item vragen Architectuur of iteratie 0
\item ...
\end{itemize}
 \item Het volgende overleg is maandag 24 november 2014. 
\end{itemize}


\end{Minutes}
\end{document}
