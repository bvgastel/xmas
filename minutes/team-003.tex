\documentclass{article}

\usepackage[dutch]{babel}
\usepackage{minutes}

\title{Notulen 003 overleg team 33}
\author{Jeroen}
\minutesstyle{header=list, vote=list, contents=true}

\begin{document}
%\selectlanguage{dutch}

\begin{Minutes}{Overleg 003 team 33}
\participant{Guus Bonnema, Stefan Versluys, Jeroen Kleijn}
\minutesdate{27. September 2014}
\location{skype}

\maketitle% This is where LaTeX makes the title

\topic{Git}

Alle documenten worden voorlopig in een repository op github bewaard. Als de OU een git server klaar heeft staan
kan de centrale repository hiernaar worden verhuisd. Voor het werken aan de planning wordt een branch 'plan' aangemaakt.
Alle teamleden kunnen hier direct naar pushen. Later zullen meerdere branches worden gebruikt, bijv. per feature of per
teamlid. Voordat de ontwikkeling begint zal precies worden uitgewerkt hoe deze branches worden ingericht.

\topic{Plan}

Het plan wordt opgesplits in meerdere subdocumenten. De teamleden zullen voornamelijk in 'hun eigen' sectie
wijzigingen aanbrengen. Dit vermindert conflicten tussen wijzigingen van verschillende teamleden.

\topic{Taken verdelen}

Er moet worden nagedacht over het verdelen van de taken. Houdt elk teamlid zich met een bepaald onderdeel bezig
(UI, domeinklassen, integratie met analysetools)? Of werkt iedereen aan het hele project? Is het mogelijk om technieken
als peer review te gebruiken om elkaars code te controleren en om van elkaars code te leren?


\topic{Domeinanalyse}

In de planning moet ruimte worden gemaakt voor het uitvoeren van de domeinanaylse. In de mail van Freek zijn een aantal
ideeen voor domeinanalyse aangegeven. Als onderdeel van het maken van de planning moet een keuze worden gemaakt welke
domeinanalyses worden uitgevoerd en door wie.

\topic{Verwerken opmerkingen Freek}

In de planning moeten nog een aantal zaken verder worden uitgewerkt nu Freek antwoorden heeft gegeven op de vragen.
In het plan moet worden uitgewerkt wie de stakeholders zijn, hoe de domeinanalyse wordt aangepakt en hoe de architectuur
wordt ontworpen. Verder moet een gedetailleerdere tijdsplanning worden opgesteld.

\topic{Besluiten}

\begin{itemize}
 \item Guus splits het plan document op en maakt dit beschikbaar op github
 \item Guus vult het plan aan met de stakeholders en architectuur
 \item Stefan en Jeroen werken de tijdsplanning verder uit
 \item Aan Freek zal worden gevraagd hoe de Collaborate sessie kan worden gestart. Ondertussen wordt Teamviewer uitgeprobeerd.
 \item Naast contact via email kan ook de chat van Skype worden gebruikt. Dit is vooral voor interactieve communicatie. Belangrijke besluiten/opmerkingen worden op de mail gezet.
 \item Het volgende overleg is maandag 29 september 2014 om 19:00
\end{itemize}


\end{Minutes}
\end{document}
