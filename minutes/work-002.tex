\documentclass[a4paper,final]{article}
\usepackage{subfig}
\usepackage[dutch]{babel}
\usepackage{hyperref}
\usepackage{minutes}
\usepackage{graphicx}
\usepackage{float}
\usepackage{color}

\title{Resultaten work meeting 002 Architectuur en Build}
\author{Guus}
\minutesstyle{header=table, vote=list, contents=true, columns={1}}

\begin{document}
%\selectlanguage{dutch}

\newcommand{\Noc}{\textsc{NoC}\xspace}

\begin{Minutes}{Work meeting 002 Architecture and build Team 33}
\participant{Guus Bonnema, Stefan Versluys, Jeroen Kleijn}
\subtitle{Architectuur}
\minutetaker{Guus}
\minutesdate{12 januari 2015}
\location{Skype}

\maketitle% This is where LaTeX makes the title

\newcommand{\w}[1]{\textsc{#1}\xspace}
\newcommand{\qml}{\textsc{QML}\xspace}
\newcommand{\qt}{\textsc{Qt}\xspace}
\newcommand{\qtquick}{\textsc{QtQuick}\xspace}

\topic{Huidige versie WickedXMas} 

\subtopic{Voortgang}

\paragraph{Stefan} Heeft de build met aanwijzigingen van Guus uit Qt groep
en stack overflow omgezet, voorlopig nog naar \qtquick1. Het kan gemakkelijk
naar \qtquick2.

\paragraph{Jeroen} Is nog bezig C++ en \qt te leren. Heeft wat issues met
betrekking tot de huidige build in mail gezet (zie volgend punt). Ook
zien we compileer problemen met de classes van Bernard. Voor nu laten we die
er uit.

\paragraph{Guus} Is bezig de architectuur aan te passen op de meest recente
ontwikkelingen (vooral overgang naar \qt). Is bezig de requirements voor
de plugin te achterhalen. Nog geen contact gehad met Freek en Bernard is 
momenteel op vakantie. Moet wel concrete vragen hebben wil ik een afspraak
met Freek maken.

Het punt is dat de data structuren van Bernard behoorlijk veel complexer zijn 
dan wij nodig hebben. De vraag is of we deze complexiteit kunnen vervangen door
standaard library functionaliteit (C++ containers of \qt containers). Dit zijn
zaken waarvoor we Bernard nodig hebben.

Freek kan ons waarschijnlijk helpen met de requirements rond de verificatie tools
en wellicht toch meer inzicht in de volledige lijst met verificatie tools.

\paragraph{Besluit} Guus maakt een afspraak met Freek om de huidige stand
van zaken te bespreken en wat de plugin requirements zijn.
Voor nu kunnen we verder, maar uiteindelijk moeten we een beslissing
hebben over hoe met de data structuur om te gaan. Dit is waarschijnlijk
aan Bernard. We moeten hiervoor wachten tot Bernard terug is.

\subtopic{Gelaagde architectuur: scheiding data en presentatie laag}

Mail van Jeroen dd 11-01-2015 23:09 (Building xmas):

\begin{quote} Een van de nadelen van de aanpak met qml is dat de datalaag dan
	wordt gedefinieerd in GUI classes, in dit geval QDeclarativeItem. De
	verificatietools als standalone applicatie wordt hiermee afhankelijk van de
	designtool, in het bijzonder Qt en QtDeclarative/Quick om een netwerk in te
	laden en te verifieren. Of dit een probleem is weet ik niet, dat moeten we
	denk ik aan Bernard voorleggen.
\end{quote}

\paragraph{Guus} De architectuur is inderdaad gelaagd, maar anders dan web
applicaties.  De fundamentele lagen hier zijn designer en VTs, met de VT plugin
er tussen in.  Alle lagen hebben intern een repository architectuur (zie cursus
software engineering).  
De communicatie tussen de lagen vindt ook met repo's plaats: het model of het 
platgeslagen model. De scheiding tussen een data laag en een presentatie
laag speelt op architectureel niveau geen rol. Toch speelt scheiding tussen
presentatie en data wel een rol, op een wat lager niveau. De vraag is:

\begin{quote}
	Kunnen wij de data in \qml gebruiken in het programma voor de VT en VT plugin?
	Dat wil zeggen, gaat het gebruik van \qml het grafische subsysteem van \qt 
	meeslepen? Of is \qml daar vrij van?
\end{quote}

\paragraph{Stefan} Volgens mij is \qml onafhankelijk van het grafische
subsysteem van \qt. Ik kan er een proefprogramma voor maken.

\paragraph{Besluit} Stefan levert deze week een programma op waarin een data structuur vanuit 
een bestand via \qml in het geheugen komt, dat zowel (met qtquick2) in de
designer te tonen is, als (zonder grafische \qt) in een console programma 
komt.

\subtopic{Builden}

\paragraph{Jeroen} We hebben diverse problemen met de build van de applicatie. Met name
het linken van de eigen dynamische libraries levert problemen. Het maakt theoretisch
niet uit of we dynamisch of statisch linken, maar voor onze eigen libraries heeft
shared link (dynamisch) geen voordelen. Bij statisch weet je zeker dat je de juiste versie hebt.

\paragraph{Guus} De naamgeving van de subdirectory voor xmasmain is misleidend. Dat moet naar xmdmain.
Uiteindelijk is dat wat het draait. Dit houdt echter ook inhoudelijke wijzigingen in de naam van 
de \verb%.pro% file en binnen in deze en andere \verb%.pro% files.

\paragraph{Besluit} Jeroen gaat beide wijzigingen uitvoeren: statisch library en naamwijziging.
Stefan gaat de huidige programmatuur wijzigen van \qtquick1 naar \qtquick2.

\end{Minutes}
\end{document}
