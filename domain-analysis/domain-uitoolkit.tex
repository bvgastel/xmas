%%
%% Dit is het hoofddocument: compileer dit met latex of xelatex en je krijgt de gehele pdf
%%
\documentclass[a4paper,11pt,final]{article}

\usepackage{subfig}
\usepackage{wrapfig}% product wrapfigure and wraptable
\usepackage{array}% additions to tabular
\usepackage{supertabular}% multiple pages tabular
\usepackage{rotating}% for the environment sidewaysfigure / sidewaystable
\usepackage[dutch]{babel}
\usepackage{graphicx}
\usepackage{hyperref}
\hypersetup{
    colorlinks = true,
    citecolor = blue,
    linkcolor = blue
}
\usepackage[prependcaption,colorinlistoftodos,obeyFinal,textsize=tiny]{todonotes}% when generating final (documentclass option) skip notes
\usepackage{pgfgantt}
\usepackage{pdflscape}
\usepackage[a4paper]{geometry}
\usepackage{titlesec}% added to change section headers, see newcommand definition.
\usepackage{boxedminipage}
\usepackage{amssymb}% For \checkmark
\usepackage{pifont}% for \ding{'-code or "-code}

\bibliographystyle{alpha}

\author{Guus Bonnema}
\date{21/10/2014}

\title{Domain analysis for a user interface toolkit}

\setlength\extrarowheight{2pt}% Adds a little space at the top of table rows

%% Document is in subdocumenten gesplitst.

\begin{document}

\selectlanguage{english}

%%%%%%%%%%%%%%%%%%%%%%%%%%%%%%%%%%%%
\newcommand{\xmas}{x\textsc{mas}}%
\newcommand{\ok}{$\checkmark$}

\newcommand{\mybox}[1]{\begin{boxedminipage}[t]{\textwidth}#1\end{boxedminipage}}

%\definecolor{airforceblue}{rgb}{0.36, 0.54, 0.66}%%   This is color in hex #5D8AA8

%%%%%%%%%%%%%%%%%%%%%%%%%%%%%%%%%%%% afwijkend section formaat start %%%%%%%%%%%%%%%%%%%%%%%%%%%%%%%%%
\newcommand\secformat[1]{%
    {\fontsize{60}{60}\selectfont\thesection}%
    \ifthenelse{\equal{\thesection}{}}{}{\quad\rule[-8pt]{2pt}{40pt}\quad}
    \parbox[b]{.7\textwidth}{\filright\bfseries #1}}%
\titleformat{\section}[block]
    {\filright\normalfont\sffamily}{}{0pt}{\secformat}
\titlespacing*{\section}{0pt}{*3}{*2}[1pc]
%%%%%%%%%%%%%%%%%%%%%%%%%%%%%%%%%%%% afwijkend section formaat end   %%%%%%%%%%%%%%%%%%%%%%%%%%%%%%%%%

\maketitle

\begin{abstract}
    This document researches the available toolkits and develops an advise
    based on the requirements for the user interface of the design tool for xmas.

    The document starts out by detailing the required characteristics of the toolkit
    and using the as a basis for the recommendation. Failing a clear advantage in the
    \emph{must have} requirements, it compares the advantage in the \emph{should have}
    requirements and failing that in the \emph{could have} requirements.

    In case of multiple tools satisfying all requirements the final recommendation is made
    based on the teams wishes to learn a specific competence in this project.
\end{abstract}

%\listoftodos   %% commented out when creating final document

\section{Requirements for design tool}

The defined platforms are all ms windows, mac and linux platforms.

\begin{center}
    \label{fig: risico}
    \sf
    \tablecaption{Main requirements for design tool}
    \tablehead{\hline & \multicolumn{2}{c|}{\bf requirement class}\\\hline
    {} & \multicolumn{1}{c|}{\bf req name} & \multicolumn{1}{c|}{\bf req desc}\\\hline}
    \tabletail{\hline \multicolumn{3}{r}{\emph{Continue on the next page}}\\}
    \tablelasttail{\hline \multicolumn{3}{r}{\emph{Table end main requirements for design tool}}\\}
    \small\sf
    \begin{supertabular}{|c|p{23em}|p{13em}|}
	1	& \multicolumn{2}{c|}{\sf\emph{\large {\bf Must have} requirements}}
		\\\hline
		& Cross platform & The toolkit must run equally well on the
				    defined platforms with respect to all relevant features
		\\\hline
		& 2D drawing & The toolkit must be able to draw items on a
				canvas and treat graphical objects with composite
				behaviour or properties as a primitive object.
		\\\hline
		& C++ integration & The toolkit must integrate seamlessly with
				the existing C++ code for analysis and verification tools.
		\\\hline
	2	& \multicolumn{2}{c|}{\sf\emph{\large {\bf Should have} requirements}}
		\\\hline
		& expandability & the toolkit should have a means of supporting expandability
				    in the target toolkit.
	3	& \multicolumn{2}{c|}{\sf\emph{\large {\bf Could have} requirements}}
		\\\hline
		& maintainability & The toolkit could provide some aid in ease of maintenance
				    like a documentation aid.
        \end{supertabular}
\end{center}


\begin{center}
\begin{supertabular}{lll}
× & × & ×\\
× & × & ×\\
× & × & ×
\end{supertabular}
\end{center}

\section{Information on existing toolkits}
List of toolkits. Main characteristics according to requirements.
\section{Pre selection of toolkits}
Based on language. C++.
\section{Comparison of selected toolkits}
\section{Conclusion of comparison}
\section{Recommendation}
\section{Motivation}

%\appendix

%\bibliography{plan.bib}
%\section{Definities}
%\begin{description}
% \item[analyse module] synoniem met verificatie module
% \item[i18n] Internationalization, referring to translation of menu items, system documentation etc.
% \item[l10n] Localization, referring to country specific settings such as money, numbers, dates etc.
%\end{description}


\end{document} ;########################### end document ##################################;