\section{List of cross platform tools}

General source: see \cite{wiki:xplatf}. Left out non-free software or software
not for defined platforms. Due to the amount of toolkits, any obvious
disconnect with C++ is also reason to drop a choice\footnote{like Lazarus, that
in theory might be able to support the application, but has no direct
integration with C++.}.

\begin{description}
    \item[Cairo] is a library used to provide a vector graphics-based,
		device-independent API for software developers. It is designed
		to provide primitives for 2-dimensional drawing across a number
		of different backends. Cairo is designed to use hardware
		acceleration when available.
		The software is free software with a user license based on GPL
		and MPL.
		\hspace*{\fill}\\Ref: \cite{wiki:cairo}.

		Cairo has a wrapper \verb!cairomm! for C++.  Cairo is popular in the
		free software communicty for providing cross-platform support for
		advanced 2D drawing. Other GUI library (notably GTK+) already include
		Cairo. This library does not contain GUI widgets and so should be
		combined with \verb!SomeGUILibrary!. The verdict for Cairo is Partial.

    \item[FLTK] The Fast, Light Toolkit (FLTK, pronounced fulltick) is a
		cross-platform graphical control element (GUI)
		library developed by Bill Spitzak and others. Made to
		accommodate 3D graphics programming, it has an interface to
		OpenGL, but it is also suitable for general GUI programming.
		\hspace*{\fill}\\Ref: \cite{wiki:fltk}.

		FLTK is a GUI only library. For 2D a different library should be used.
		The toolkit is not included in the final comparison.

    \item[fpGUI] the Free Pascal GUI toolkit, is a cross-platform
		graphical user interface toolkit developed by Graeme Geldenhuys.
		fpGUI is open source and free software, licensed under a Modified LGPL
		license. The toolkit has been implemented using the Free Pascal
		compiler, meaning it is written in the Object Pascal language.
		fpGUI consists only of graphical widgets or components, and a
		cross-platform 2D drawing library.
		fpGUI is statically linked into programs and is licensed using a
		modified version of LGPL specially designed to allow static linking to
		proprietary programs. The only code you need to make available are
		any changes you made to the fpGUI toolkit - nothing more.
		\hspace*{\fill}\\Ref: \cite{wiki:fpgui}.

		fpGUI is a pascal library based on free pascal (lazarus).  The author
		is also the sole maintainer of the system. According to the WxWidget
		comparison Lazarus has no C++ integration to speak of. For this reason
		this toolkit is not included in the final comparison.
		\hspace*{\fill}\\Ref: \cite{Geldenhuys:fpgui} and
		\cite{wxwidget:comparison}.

	\item[GTK+] (previously GIMP Toolkit, sometimes incorrectly referred to as
		the GNOME Toolkit) is a cross-platform widget toolkit for creating
		graphical user interfaces. It is licensed under the terms of the GNU
		LGPL, allowing both free and proprietary software to use it. It is one
		of the most popular toolkits for the Wayland and X11 windowing systems,
		along with Qt.  \hspace*{\fill}\\Ref: \cite{wiki:gtk+}.

		GTK+ builds on Cairo and GDK and has a large developer and user
		community.  It has a wrapper for C++ (\verb!GTKMM!). It fulfills the
		main requirements and is included in the final comparison.
		\hspace*{\fill}\\Ref: \cite{gtkmm:gtk+}.

    \item[JUCE] is a free software, cross-platform C++ application framework, used
		for the development of GUI applications and plug-ins.
		The aim of JUCE is to allow software to be written such that
		the same source code will compile and run identically on Windows,
		Mac OS X and Linux platforms. It supports various development
		environments and compilers, such as GCC, Xcode and Visual Studio.
		\hspace*{\fill}\\Ref: \cite{wiki:juce}.

		A one man project with emphasis on audio.  It runs all defined
		platforms, written in C++. The user license is dual GPL and commercial.
		The verdict according to requirements is Maybe due to the size of the
		developer community. It is not included in the final comparison.

	\item[CEGUI] Crazy Eddie's GUI System is a free library providing windowing
		and widgets for graphics APIs / engines where such functionality is not
		natively available, or severely lacking. The library is
		object-oriented, written in C++, and targeted at game and application
		developers who should be spending their time creating great games and
		not on building GUI sub-systems!
		\hspace*{\fill}\\Ref: \cite{cegui:getting-started}.

		This toolkit satisfies the main requirements, but has a small developer
		community (\cite{wiki:cegui} and \cite{cegui:cegui}. The verdict for
		this toolkit is Maybe due to the size of the developer community. It is
		not included in the final comparison.

    \item[Mozilla Application Framework]
		 is a collection of cross-platform software components that
		 make up the Mozilla applications. It was originally known as
		 XPFE, an abbreviation of cross-platform front end.
		 While similar to generic cross-platform application
		 frameworks like GTK+, Qt and wxWidgets, the intent is to
		 provide a subset of cross-platform functionality suitable for
		 building network applications like web browsers, leveraging
		 the cross-platform functionality already built into the Gecko
		 layout engine.
		 \hspace*{\fill}\\Ref: \cite{wiki:mozilla_application_framework}.

		 The MAF is platform independent, web oriented with a large developer
		 and user community\footnote{All firefox users are part of the user
		 community}.  The library is meant to support a subset of the standard
		 GUI frameworks like GTK+, QT and WxWidgets, aimed at web-programs. Due
		 to the web orientedness and it's partial support for widgets this
		 toolkit is not selected for the final comparison.

    \item[OpenGL] is a cross-language, multi-platform application programming
		interface (API) for rendering 2D and 3D vector graphics.
		The API is typically used to interact with a graphics
		processing unit (GPU), to achieve hardware-accelerated
		rendering\footnote{OpenGL is a standard that has libraries based on free
		software and commercial libraries. Programmers never need a license
		to use an OpenGL library.}.
		\hspace*{\fill}\\Ref: \cite{wiki:opengl}.

		OpenGL gives hardware acceleration and is a powerful but complex 3D
		user interface toolkit. It is also a low level library and does not
		directly support any GUI widgets. Many GUI libraries support the use of
		an OpenGL library. This toolkit is not selected for final comparison.

    \item[Qt]  (/ˈkjuːt/ "cute", or unofficially as Q-T cue-tee) is a cross-platform
		application framework that is widely used for developing application
		software that can be run on various software and hardware platforms
		with little or no change in the codebase, while having the power and
		speed of native applications. Qt is currently being developed both
		by the Qt Company, a subsidiary of Digia, and the Qt Project under
		open-source governance, involving individual developers and firms
		working to advance Qt.

		Digia owns the Qt trademark and copyright. Qt is available with
		both proprietary and open source GPL v3 and LGPL v2 licenses.
		\hspace*{\fill}\\Ref: \cite{wiki:qt}.

		Qt has a large developer and user community. According
		to the WxWidget comparison with Qt, both are functionally comparable.
		\hspace*{\fill}\\Ref: \cite{wxwidget:comparison}.

    \item[Simple DirectMedia Layer] is a cross-platform software development
		library designed to provide a low level hardware abstraction
		layer to computer hardware components. Software developers
		can use it to write high-performance computer games and other
		multimedia applications that can run on many operating systems
		such as Android, iOS, Linux, Mac OS X, Windows and other platforms.

		SDL manages video, audio, input devices, CD-ROM, threads, shared
		object loading, networking and timers.[5] For 3D graphics it can
		handle an OpenGL or Direct3D context.

		The library is internally written in C and also provides the
		application programming interface in C, with bindings to other
		languages available. It is free and open-source software subject
		to the requirements of the zlib License since version 2.0 and with
		prior versions subject to the GNU Lesser General Public License.
		Because of zlib SDL 2.0 is freely available for static linking
		in commercial closed-source projects, unlike SDL 1.2. SDL is
		extensively used in the industry in both large and small projects.
		Over 700 games, 180 applications, and 120 demos have also been
		posted on the library website.

		It is often believed that SDL is a game engine, but this is not
		true. However, the library is well-suited for building an engine
		on top of it.
		\hspace*{\fill}\\Ref: \cite{wiki:sdl}.

		SDL is a low level library meant for 2D game development. It does not
		contain GUI widgets. It may be usable in combination with other
		libraries. It is not selected for final comparison.

	\item[Tk] is a free and open-source, cross-platform widget toolkit that
		provides a library of basic elements of GUI widgets for building a
		graphical user interface (GUI) in many different programming languages.
		\hspace*{\fill}\\Ref: \cite{wiki:tk}.

		Tk seems to contain everything necessary for both GUI and 2D drawing,
		but C++ integration is not clear (\cite{wiki:tk}, \cite{tcltk:tk}).
		According to the comparison on WxWidgets, for C++ it is better not to
		use Tk.  \hspace*{\fill}\\Ref: \cite{wxwidget:comparison}. 

    \item[Ultimate++] is a C++ cross-platform development framework which aims
		to reduce the code complexity of typical desktop applications
		by extensively exploiting C++ features.
		\hspace*{\fill}\\Ref: \cite{wiki:ultimate++}.

    \item[WxWidgets] (formerly wxWindows) is a widget toolkit and tools library
		for creating graphical user interfaces (GUIs) for cross-platform
		applications. wxWidgets enables a program's GUI code to compile
		and run on several computer platforms with minimal or no code
		changes. It covers systems such as Microsoft Windows, OS X
		(Carbon and Cocoa), iOS (Cocoa Touch), Linux/Unix (X11, Motif,
		and GTK+), OpenVMS, OS/2 and AmigaOS. A version for embedded
		systems is under development.
		\hspace*{\fill}\\Ref: \cite{wiki:wxwidget}

		The framework is complete and has no obvious disadvantages. According
		to their comparison to Qt, the two are functional comparable.
		\hspace*{\fill}\\Ref: \cite{wxwidget:comparison}.  

	\item[Gled]	is a C++ framework for rapid development of applications with
		support for GUI (using FLTK), 3D-graphics and distributed
		computing. It extends the ROOT framework (standard data-analysis
		tool in high-energy physics) with mechanisms for object collection
		management \& serialization, multi-threaded execution, GUI
		auto-generation (object browser \& editor) and dynamic visualization
		(OpenGL). Distributed computing model of Gled is a hierarchy of
		nodes connected via TCP/IP sockets. Gled provides authentication \&
		access control, data exchange, proxying of object collections and
		remote method-call propagation \& execution. Gled can be dynamically
		extended with library sets. Their creation is facilitated by a set
		of scripts for creation of user-code stubs. Simple tasks and
		application configuration can be efficiently done via the
		interactive C++ interpreter (CINT). Gled is used for development of
		programs in high energy physics and as a research tool in
		distributed and grid computing. 
		\hspace*{\fill}\\Ref: \cite{fltk:gled}

		This toolkit satisfies the main requirements, but has a small developer
		community (\cite{gled:gled}). It is not included in the final
		comparison.

    \item[GDK]  (GIMP Drawing Kit) is a library that acts as a wrapper around
		the low-level functions provided by the underlying windowing
		and graphics systems. GDK lies between the display server and
		the GTK+ library, handling basic rendering such as drawing
		primitives, raster graphics (bitmaps), cursors, fonts, as well
		as window events and drag-and-drop functionality.

		Like GTK+, GDK is licensed under the GNU Lesser General
		Public License (LGPL).
		\hspace*{\fill}\\Ref: \cite{wiki:gdk}

		GDK is a low level library that is ``An intermediate layer which
		isolates GTK+ from the details of the windowing system.''.  In the
		presence of GTK+ the toolkit is not included in the final comparison.
		\hspace*{\fill}\\Ref: \cite{gnome:gdk3}.

\end{description}

