\section{Requirements for UI toolkit}

Using the requirements for the application plus the competence requirements of
my team, the impact of all these demands is summarized in table
\ref{tab:uitool-requirements}. Only requirements directly relevant to the ui
toolkit have consequential requirements formulated.

In table \ref{tab:uitool-requirements} the term ``support for plugins'' is
meant to be a weak requirement in the sense that it does not hinder the design
of plugins. The ability to support plugins directly is a nice to have
requirement. The derived requirements ``ease of use'' and ``ease of learning''
are important requirements for the people maintaining the system.

\paragraph{Remark.} The term ``ui toolkit'' may indicate one set of tools (like qt) or
a multiple sets (like one for 2D and one for GUI elements).

\begin{center}
    \begin{longtable}{ll||cp{13em}}
	{\bf id: requirement}     & {\bf impact  } & {\bf id } & {\bf UI requirement}\\\hline\endhead
	\hline \multicolumn{4}{c}{UI toolkit requirements (Continue on next page)}\endfoot
	\hline\endlastfoot
	\hline
		M0: Free software    & full         &  0  & ui toolkit must be free software as defined by FSF.\\
        M1: Cross platform   & full         &  1  & ui toolkit runs equally well on the defined platforms for all features.\\
        M2: C++ integration  & full         &  2  & ui toolkit does not hinder C++ integration in any way.\\
        M3: 2D drawing       & full         &  3  & ui toolkit has equal 2D features on all defined platforms.\\
        M4: GUI Widgets      & full         &  4  & ui toolkit has relevant GUI widgets on all defined platforms.\\
        M5: xmas primitives  & none         &     & \\
        M6: xmas macros      & none         &     & \\
        S1: expandability    & limited      &  5  & ui toolkit does not hinder creating plugins.\\
        S2: installability   & limited      &  6  & ui toolkit does not hinder install procedure.\\
        S3: maintainability  & moderate     &  7  & ui toolkit is easy to use.\\
	                         &              &  8  & ui toolkit is easy to learn.\\
	                         &              &  9  & ui toolkit is well documented.\\
	                         &              & 10  & ui toolkit supports observer pattern.\\
	                         &              & 11  & ui toolkit dev. community should have enough mass.\\
	                         &              & 12  & ui toolkit user community should have enough mass.\\
        S4: plugability      & limited      & 13  & ui toolkit should not hinder creating plugins.\\
        S5: performance	     & moderate     & 14  & ui toolkit should support running concurrent processes for modules.\\
	                         &              & 15  & ui toolkit should support observer pattern from independent processes.\\
	                         &              & 16  & ui toolkit should support interruption during concurrent processing.\\
        N1: install util     & none    		&     & \\\hline
        %--------------------------------------------------------------------------------------------------------------------------
		GM0: Free Software  &              &     & Enclosed in requirement M0.\\
        GM1: UI tool	    &              &     & Enclosed in requirement M1.\\
        GS1: Agile	        &              &     & \\
        GS2: C++	        & compatible   & 20  & Prefer a ui toolkit that works with C++.\\
        GC1: versioncontrol &              &     & \\
        GC2: req. engineering
						    &              &     & \\\hline
        %--------------------------------------------------------------------------------------------------------------------------
        SS1: agile	        &              &     & \\
        SS2: online agile tools
							&              &     & \\
        SS3: distributed team
							&              &     & \\
        SC1: platform independent
							& compatible   & 21  & Prefer a platform independent toolkit.\\
    \end{longtable}
    \captionof{table}{UI Toolkit requirements derived.}
    \label{tab:uitool-requirements}
\end{center}
