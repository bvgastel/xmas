\section{Result and recommendation}
\label{sec:recommendation}

\subsection{Considerations and Motivation}

Considering the goals of the project, and the competence wishes, the team has
different options as a result of different viewpoints.  One option would be to
go with a toolkit that is ``\texttt{mean and lean}'' and add libraries only
where necessary. The other option would be to go with a tool kit that contains
``all++'' and thus is comfortable allbeit a little bloated. Whatever the
choice, it will affect the end product. 

In my recommendation I go with the tool kit \w{fltk} (pronounce
\emph{fulltick}) . The tool kit is targetted at both workstations and embedded
hardware. The focus is both on being \texttt{mean and lean} and fast and small,
as advertised on the \w{fltk} web site and user forum.

The recommendation is based on this information (also see appendix
\ref{sec:phase3listing}) and the personal assessment of the project in next
paragraph. Of course the final decision is up to the team.

The project emphasizes a Network on Chip design, with most user interface
effort going into the 2D drawing interface and the general user interface.
This makes the focus of \w{fltk} on user interface with respect to gui a good
choice. Should we need more emphasis on looks or more complex drawing than
\w{fltk} can provide, then we could substitute \w{OpenGL} or
\w{Cairo}\footnote{Integration of \w{OpenGL} or \w{Cairo} into \w{fltk} should
be trivial}.

\par The argument is solid, but with a shift of viewpoints a different choice
could be valid.  The team should think about what is important in this project
and decide on that basis. See appendix \ref{sec:phase3listing} on
page~\pageref{sec:phase3listing} for a summary of main characteristics of the
selected tool kits.

\subsection{Recommendation} 

I recommend using \w{fltk}.

