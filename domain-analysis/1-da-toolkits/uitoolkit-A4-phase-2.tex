\section{Phase 2}

\subsection{Phase 2 selection}

Table \ref{tab:compare-main-req} shows the remaining toolkits and their 
rating according to the high level requirements. The column {\sc Community} 
indicates the expected support for the toolkit. A large user 
commmunity could drive more people toward the developer community and thus
increase the life expectation of the toolkit and its support. 
This stimulates confidence in the toolkits' future.
For {\sf FLTK} and {\sf SDL} the community size is not clear from the home website,
so size is left out in table \ref{tab:compare-main-req}.

Among the toolkits qualified as {\sf\sc Maybe} are some unconventional, but
powerful frameworks. For instance {\sc Gled} is a framework that stems from
scientific work and features the easy distribution of nodes accross threads,
processes and machines. See \cite{greed:gled} for a discussion of the scientific
goals of the {\sc\sf Greed} project.


The toolkits qualified as {\sf\sc Partial} leave out either the GUI widgets or 
the 2D drawing capability. They could still be viable in combination with one
of the other toolkits.

\begin{center}
    \small\sf
    \begin{tabular}{c|lccc|c|ccc}
	\hline
	{\bf\sf nr} & {\bf\sf Toolkit} & {\bf\sf C++} & {\bf\sf 2D} & {\bf\sf GUI} &            & \multicolumn{3}{c}{\em\bf\sf Yay or Nay} \\
	         & {\bf\sf Name}    & 			 &          &            & {\bf\sf Community} 	& {\bf\sf Yay} & {\bf\sf Part} & {\bf\sf Nay} \\
	         &			     &         	 &          &            &         		&           & {\bf\sf Maybe}  &         \\
        \hline
%%%%%%%%%%%%%%%%%%  	  C++  2D  GUI  Comm  Yay   Part   Nay
%%%%%%%%%%%%%%%%%%                                  Maybe
1  &	GTK+		& 1   & 1 & 1 &  Large  & Yay &         &     \\
2  &	Qt	      	& 1   & 1 & 1 &  Large  & Yay &         &     \\
3  &	WxWidgets 	& 1   & 1 & 1 &  Large  & Yay &         &     \\
4  &	FLTK      	& 1   & 1 & 1 &         & Yay &         &     \\
\hline
5  &	JUCE      	& 1   & 1 & 1 &  Small  &     & Maybe   &     \\
6  &	CEGUI     	& 1   & 1 & 1 &  Small  &     & Maybe   &     \\
7  &	Gled		& 1   & 1 & 1 &  Small  &     & Maybe   &     \\
\hline
8  &	Cairo     	& 1   & 1 & 0 &  Large  &     & Partial &     \\
9  &	OpenGL 	  	& 1   & 1 & 0 &  Large  &     & Partial &     \\
10 &	SDL			& 1   & 1 & 0 &         &     & Partial &     \\\hline
11 &	Mozilla A.F.  	& 1   & 1 & 1 &  Large  &     & Web-oriented   &     \\\hline
12 &	Tk	        & 0.5 & 1 & 1 &  Large  &     &         & Nay \\
13 &	fpGUI     	& 0   &   &   &         &     &         & Nay \\
14 &	GDK       	& 1   & 1 & 0 &  Large  &     &         & Nay \\
\hline
    \end{tabular}
    \captionof{table}{Selected tools against must have product/project requirements}
	\label{tab:compare-main-req}
\end{center}

\subsection{Phase 2 analysis}

\paragraph{Toolkits included} The toolkits 1 through 3 satify the requirements. 
The {\sf FLTK} toolkit does not generate confidence due to unclear community size. 
The toolkits 5 through 7 all have a small community size and therefore are left out for further comparison. 
The toolkits 8 and 9 satisfy all but one requirement (the GUI) but could still function if they 
add sufficient function to a GUI oriented toolkit. 

\paragraph{Toolkits excluded}
Toolkit 10 ({\sf SDL} is not clear on the supporting community, but does seem to have a following. 
The comparison from WxWdidgets (\cite{wxwidget:comparison}) mentions {\sf SDL} as a viable addition 
to WxWidgets, but it is oriented towards gaming. The toolkits 12 through 13 do not satisfy the 
requirements fully, and toolkit 14 (GDK) is low level and incorporated in {\sf GTK+}. 
The Mozilla Application Framework ({\sf M.A.F.}) is oriented towards the web. The deciding feature 
is that {\sf MAF} contains a subselection of the regular GUI widgets geared towards html and css.

\paragraph{Next phase}
contains toolkits $1, 2, 3$ and 8 and 9: {\sf GTK+, Qt, WxWidgets, Cairo and OpenGL}. 
