\section{Requirements for design tool}

Table \ref{tab:appl-requirements} summarizes the high level requirements for
the chip design tool. Table \ref{tab:team-competence-req} shows the team
members' requirements in terms of competence goals. Each requirement may
transfer as full requirement to the UI tool, partially or not at all. Table
\ref{tab:uitool-requirements} summarizes the relationship of the project's
requirements with the user interface toolkit.

If multiple tools satisfy the complete set of requirements, then
team preference will rule the recommendation. Ultimately the customer
and the team decide depending on the recommendation and any external factors
like an other domain analysis.

\subsection{Free software requirement}

The first requirement $M0$ was specified to ensure the free availability of the
ui tool both now and in the future. The source of
this requirement is a remark from the Open University instructions
``Hulpmiddelen en bronnen'' (``tools and sources'') with the following text:

\vspace{1em}

\begin{tabular}[t]{ll}
\begin{minipage}{.45\textwidth}

	``Wij gaan ervan uit dat de producten van het team uiteindelijk veelal als
	open source beschikbaar gesteld worden, dus iedereen mag in principe de
	producten inzien. In de overeenkomst met de opdrachtgever dient op het punt
	van de rechten duidelijkheid gegeven te worden.''

\end{minipage}

&

\begin{minipage}{.45\textwidth}

	``We assume open source availability of the team's products so that the
	products are publicly available. The contract with the customer should be
	clear on this point.''

\end{minipage}

\end{tabular}

\vspace{1em}

\noindent Demanding free software follows from this specification and the fact
that non-free software could potentially lead to non-availability of the
software due to license fees or the author retracting the software. We specify
free software as a must-have requirement to be ``clear on this point''.

\subsection{The project and product requirements}

\begin{center}
    \small\sf
    \begin{tabular}{|l|p{7em}|p{23em}|}
	\hline
	        & \multicolumn{2}{c|}{\sf\em\large requirement class}\\\hline
	{}      & \multicolumn{1}{c|}{\bf req name} & \multicolumn{1}{c|}{\bf req desc}\\\hline
	M	& \multicolumn{2}{c|}{\sf\emph{\large Must have requirements}}
		\\\hline
	M0	& Free software  & The application is free software as defined by the FSF.
		\\\hline
	M1	& Cross platform & The application must run equally well on the
				    defined platforms with respect to all relevant features.
				    Linux, Mac and MS Windows are the defined platforms.
		\\\hline
	M2	& C++ integration & The application must integrate seamlessly with
				the existing C++ code for analysis and verification tools.
		\\\hline
	M3	& 2D drawing & The application must be able to draw items on a
				canvas and treat graphical objects with composite
				behaviour or properties as a primitive object.
		\\\hline
	M4	& GUI widgets & The application must be able to create GUI widgets
				on all defined platforms.
		\\\hline
	M5      & xmas primitives & The application must be able to draw the 8 xmas
				    primitives.
		\\\hline
	M6	& xmas macros     & The application must be able to define and draw
				    a macro of xmas primitives.
		\\\hline
	S	& \multicolumn{2}{c|}{\sf\emph{\large Should have requirements}}
		\\\hline
	S1	& expandability & the application should be expandable in terms of
				analysis and verification modules, xmas-primitives and
				network reporting.
		\\\hline
	S2	& installability & The application should be easy to install.
		\\\hline
	S3	& maintainability & The application should be as easily maintainable as possible.
		\\\hline
	S4	& plugability	& The application should use plugins for analysis and
				verification modules.
		\\\hline
	S5	& performance	& The application should be performant enough for the GUI
				    interface to be ``snappy''
		\\\hline
	C	& \multicolumn{2}{c|}{\sf\emph{\large Could have requirements}}
		\\\hline
	N	& \multicolumn{2}{c|}{\sf\emph{\large Nice to have requirements}}
		\\\hline
	N1	& install util	& It would be nice to have the application provide
				installable packages to the users.
		\\\hline
    \end{tabular}
    \captionof{table}{High level requirements for the design tool}
    \label{tab:appl-requirements}
\end{center}

\subsection{Team preferences}

%%%%% input from team competence tex file (a symbolic link) %%%%%
%%
%%  must be input: include is not permitted from an included file
%%
%%
%% This file contains the competence requirements for the team members.
%% It is part of other documents.
%%

The team has explicit wishes concerning improving or learning competences. Table
\ref{tab:team-competence-req} specifies these wishes per team member.

\begin{center}
    \begin{tabular}{cllp{17em}}
        \hline
        & {\bf Prio} & {\bf Competence}    & {\bf Goal}\\
        \hline
        \multicolumn{4}{c}{\sf\emph{Guus}}\\
        \hline
		GM0 & Must have    & Free software & Prefer to build free software.\\
        GM1 & Must have    & UI tool       & Learning to work with a cross platform UI tool like QT or GTK+.\\
        GS1 & Should have  & Agile		   & Learning to work in an agile environment.\\
        GS2 & Should have  & C++		   & Learning to program C++ (standard 2011).\\
	GC1 & Could have   & versioncontrol	   & Experiencing and learning teamwork with versioncontrol.\\
        GC2 & Could have   & Requirements Eng.   & Learning to apply Requirements engineering.\\
        \hline
        \multicolumn{4}{c}{\sf\emph{Stefan}}\\
        \hline
        SS1 & Should have  & Agile		   & Learning to work in agile environment, like DAD.\\
        SS2 & Should have  & Online Agile tools  & Learning to work agile tools.\\
        SS3 & Should have  & distributed team    & Learning to work in a distributed team.\\
        SC1 & Could have   & Non Microsoft tools & Learning to work with platform independent tools.\\
        \hline
        \multicolumn{4}{c}{\sf\emph{Jeroen}}\\
        \hline
                     &                     & \\
        \hline
    \end{tabular}
    \captionof{table}{Competence wishes of the team}
    \label{tab:team-competence-req}
\end{center}

%%%%%% input from team competence tex file (a symbolic link) %%%%%

\section{Requirements for UI toolkit}

Using the requirements for the application plus the competence requirements of
my team, the impact of all these demands is summarized in table
\ref{tab:uitool-requirements}. Only requirements directly relevant to the ui
toolkit have consequential requirements formulated.

In table \ref{tab:uitool-requirements} the term ``support for plugins'' is
meant to be a weak requirement in the sense that it does not hinder the design
of plugins. The ability to support plugins directly is a nice to have
requirement. The derived requirements ``ease of use'' and ``ease of learning''
are important requirements for the people maintaining the system.

Each requirement has a min-factor that reflects the type of requirement : 1.0
for a must-have requirements, 0.9 for should-have and 0.8 for a could-have
requirement. However, the factor could be less if the importance for the ui
tool is diminished.

The team requirements have a type downgraded by 0.1 indicating less importance
than the product requirements.

\paragraph{Remark.} The term ``ui toolkit'' may indicate one set of tools (like qt) or
a multiple sets (like one for 2D and one for GUI elements).

\begin{center}
    \begin{longtable}{ll||ccp{13em}}
	{\bf id: requirement}     & {\bf impact} & {\bf id } & {\bf min-factor} & {\bf UI requirement}\\\hline\endhead
	\hline \multicolumn{4}{c}{UI toolkit requirements (Continue on next page)}\endfoot
	\hline\endlastfoot
	\hline
		M0: Free software    & full         &  0  & 1.0 & ui toolkit must be free software as defined by FSF.\\
        M1: Cross platform   & full         &  1  & 1.0 & ui toolkit runs equally well on the defined platforms for all features.\\
        M2: C++ integration  & full         &  2  & 0.7 & ui toolkit does not hinder C++ integration in any way.\\
        M3: 2D drawing       & full         &  3  & 1.0 & ui toolkit has equal 2D features on all defined platforms.\\
        M4: GUI Widgets      & full         &  4  & 1.0 & ui toolkit has relevant GUI widgets on all defined platforms.\\
        M5: xmas primitives  & none         &     &     & \\
        M6: xmas macros      & none         &     &     & \\
        S1: expandability    & limited      &  5  & 0.9 & ui toolkit does not hinder creating plugins.\\
        S2: installability   & limited      &  6  & 0.9 & ui toolkit does not hinder install procedure.\\
        S3: maintainability  & moderate     &  7  & 0.9 & ui toolkit is easy to use.\\
	                         &              &  8  & 0.9 & ui toolkit is easy to learn.\\
	                         &              &  9  & 0.9 & ui toolkit is well documented.\\
	                         &              & 10  & 0.9 & ui toolkit supports observer pattern.\\
	                         &              & 11  & 0.9 & ui toolkit dev. community should have enough mass.\\
	                         &              & 12  & 0.9 & ui toolkit user community should have enough mass.\\
        S4: plugability      & limited      & 13  & 0.9 & ui toolkit should not hinder creating plugins.\\
        S5: performance	     & moderate     & 14  & 0.9 & ui toolkit should support running concurrent processes for modules.\\
	                         &              & 15  & 0.9 & ui toolkit should support observer pattern from independent processes.\\
	                         &              & 16  & 0.9 & ui toolkit should support interruption during concurrent processing.\\
        N1: install util          & none         &     &     & \\\hline
        %--------------------------------------------------------------------------------------------------------------------------
		GM0: Free Software  &              &     &     & Enclosed in requirement M0.\\
        GM1: UI tool	    &              &     &     & Enclosed in requirement M1.\\
        GS1: Agile	        &              &     &     & \\
        GS2: C++	        & compatible   & 20  & 0.8 & Prefer a ui toolkit that works with C++.\\
        GC1: versioncontrol &              &     &     & \\
        GC2: req. engineering
						    &              &     &     & \\\hline
        %--------------------------------------------------------------------------------------------------------------------------
        SS1: agile	        &              &     &     & \\
        SS2: online agile tools
							&              &     &     & \\
        SS3: distributed team
							&              &     &     & \\
        SC1: platform independent
							& compatible	 & 21  & 0.8 & Prefer a platform independent toolkit.\\
    \end{longtable}
    \captionof{table}{UI Toolkit requirements derived.}
    \label{tab:uitool-requirements}
\end{center}
