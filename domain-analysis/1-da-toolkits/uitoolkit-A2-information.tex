\section{Approach of the selection process}

The character of Graphical User Interface toolkits (GUI toolkits) is such that
a reasonably recent list of toolkits can only be found on internet. The best
start is wikipedia which links to toolkits, each often linking to other
toolkits. The set of toolkits studied here is a sound base for finding a
toolkit that suffices for our needs.

We selected the tools based on various web searches leading among others
leading to \cite{wiki:xplatf} and \cite{wiki:xplatf_gui}.  The amount of
toolkits available is staggering, so the first step is weeding out the toolkits
that do not support one of our defined platforms, are not free software of just
do not interact with C++ sufficiently.  For that reason this analysis does not
consider the following toolkits:

\section{Phase 1 Preselection of user interface toolkits}

\begin{center}
    \begin{tabular}{|p{7em}|p{25em}|}
	\hline
	{\bf name} & {\bf reason}\\\hline
	AppearIQ & non-free software. Ref: \cite{appear:appeariq}.\\
	Eclipse & a compiler, not a graphical framework. Ref: \cite{wiki:eclipse}.\\
	GeneXus & non-free software. Also not for linux
		    or mac. Ref: \cite{wiki:genexus}. \\
	Haxe & not a graphics platform. Ref: \cite{wiki:haxe}\\
	Max & not for linux, not free software. Ref: \cite{wiki:max}\\
	Mono & not a graphics environment, emulation
		of C\#. Ref: \cite{wiki:mono}.\\
	MonoCross & aimed at C\#. Ref: \cite{wiki:monocross}.\\
	MoSync & aimed at mobile platforms, no longer maintained. Ref: \cite{wiki:mosync}.\\
	Xojo & non-free software. Ref: \cite{wiki:xojo}.\\
	Smartface & non-free software\footnote{limited edition gratis available}. Ref: \cite{wiki:smartface} and \cite{smartface:license}.\\
	WebDev & aimed at creating websites. Ref: \cite{wiki:webdev}.\\
	WinDev & aimed at data centric apps with forms, works with webdev. Ref: \cite{wiki:windev}.\\
	XPower++ & Insufficient information, looks like an advertisment. Ref: \cite{wiki:xpower++}.\\
	Lazarus & A Pascal development environment. Ref: \cite{wiki:lazarus}.\\
	Ultimate++ & Does not support MacOS. Ref: \cite{wiki:ultimate++} and \cite{wxwidget:comparison}.\\
	\hline
    \end{tabular}
    \captionof{table}{Platforms not considered with reason}
	\label{tab:discarded-toolkits}
\end{center}


\section{Phase 2 Selection of user interface toolkits on main requirements}

Using the main requirements including free software (first column) and the
defined platforms (second column) Table \ref{tab: compare-main-req} shows the
remaining toolkits and their rating according to the main requirements
including free software and defined platforms.

Among the 3 toolkits qualified as ``Maybe'' are some pretty unconventional, but
powerful frameworks. For instance Gled is a framework that stems from
scientific work and features the easy distribution of nodes accross threads,
processes and machines.

The resulting preferred toolkits are GTK+, Qt and WxWidgets due to the size of
the developer and user community. 

\begin{center}
    \small\sf
    \begin{tabular}{c|lccc|c|ccc}
	\hline
	{\bf nr} & {\bf Toolkit} & {\bf C++} & {\bf 2D} & {\bf GUI} &            & \multicolumn{3}{c}{\em Yay or Nay} \\
	         & {\bf Name}    & 			 &          &            & {\bf Community} 	& {\bf Yay} & {\bf Part} & {\bf Nay} \\
	         &			     &         	 &          &            &         		&           & {\bf Maybe}  &         \\
        \hline
%%%%%%%%%%%%%%%%%%  	  C++  2D  GUI  Comm  Yay   Part   Nay
%%%%%%%%%%%%%%%%%%                                  Maybe
1  &	GTK+		& 1   & 1 & 1 &  Large  & Yay &         &     \\
2  &	Qt	      	& 1   & 1 & 1 &  Large  & Yay &         &     \\
3  &	WxWidgets 	& 1   & 1 & 1 &  Large  & Yay &         &     \\\hline
4  &	JUCE      	& 1   & 1 & 1 &  Small  &     & Maybe   &     \\
5  &	CEGUI     	& 1   & 1 & 1 &  Small  &     & Maybe   &     \\
6  &	Gled		& 1   & 1 & 1 &  Small  &     & Maybe   &     \\
\hline
6  &	Cairo     	& 1   & 1 & 0 &  Large  &     & Partial &     \\
7  &	FLTK      	& 1   & 0 & 1 &         &     & Partial &     \\
8  &	OpenGL 	  	& 1   & 1 & 0 &  Large  &     & Partial &     \\
9  &	SDL		& 1   & 1 & 0 &         &     & Partial &     \\\hline
10 &	Mozilla A.F.  	& 1   & 1 & 1 &  Large  &     & Web-oriented   &     \\\hline
11 &	Tk	        & 0.5 & 1 & 1 &  Large  &     &         & Nay \\
12 &	fpGUI     	& 0   &   &   &         &     &         & Nay \\
13 &	GDK       	& 1   & 1 & 0 &  Large  &     &         & Nay \\
\hline
    \end{tabular}
    \captionof{table}{Selected tools against must have product/project requirements}
	\label{tab:compare-main-req}
\end{center}

\paragraph{GTK+: Yay} builds on Cairo and GDK, has a large developer and user
community.  It fullfills the must-haves and the verdict is Yay. It has a
wrapper \verb!GTKMM! for C++. The verdict is Yay.

\paragraph{Qt: Yay} has a large developer and user community. According to the
wxwidget comparison with Qt, both are functionally comparable
(\cite{wxwidget:comparison}).

\paragraph{WxWidgets: Yay} The framework is complete and has no obvious
disadvantages. According to their comparison to Qt, the two are functional
comparable (\cite{wxwidget:comparison}).  The verdict is Yay.

\paragraph{JUCE: Maybe} A one man project with emphasis on audio
(\cite{juce:juce},\cite{wiki:juce}).  It runs all defined platforms, written in
C++. The user license is dual GPL and commercial.  The verdict according to
requirements is Maybe due to the size of the developer community.

\paragraph{CEGUI: Maybe} This toolkit satisfies the main requirements, but has a
small developer community (\cite{wiki:cegui} and \cite{cegui:cegui}. The
verdict for this toolkit is Maybe due to the size of the developer community.

\paragraph{Gled: Maybe} This toolkit satisfies the main requirements, but has a
small developer community (\cite{gled:gled}). The verdict is Maybe.

\paragraph{Cairo: Partial} has a wrapper \verb!cairomm! for C++.  Cairo is
popular
in the free software communicty for providing cross-platform support for
advanced 2D drawing. Other GUI library (notably GTK+) already include Cairo.
This library does not contain GUI widgets and so should be combined with
\verb!SomeGUILibrary!. The verdict for Cairo is Partial.

\paragraph{FLTK: Partial} This is a GUI only library. For 2D a different library
should be used and should be combined with \verb!Some2DLibrary!. The verdict is
Partial.

\paragraph{OpenGL: Partial} hardware acceleration and powerful but complex. Low
level library. It does not directly support any GUI widgets. Many GUI libraries
support the use of an OpenGL library. The verdict is Partial.

\paragraph{SDL: Partial} This is Low level library meant for 2D game
development. It does not contain GUI widgets. It may be usable in combination
with other libraries. The verdict is Partial.

\paragraph{Mozilla Application Framework: Maybe} The MAF is platform
independent, web oriented with a large developer and user
community\footnote{All firefox users are part of the user community}.  The
library is meant to support a subset of the standard GUI frameworks like GTK+,
QT and WxWidgets, aimed at web-programs. Due to the web orientedness and it's
partial support for widgets the verdict for this toolset is Maybe.

\paragraph{Tk: Nay} Seems to contain everything necessary for both GUI and 2D
drawing, but C++ integration is not clear (\cite{wiki:tk}, \cite{tcltk:tk}).
According to the comparison on WxWidgets, for C++ better not use Tk
(\cite{wxwidget:comparison}).  The verdict for that reason is Nay.

\paragraph{fpGUI: Nay} is a pascal library based on free pascal (lazarus)
(\cite{Geldenhuys:fpgui}).  The author is also the sole maintainer of the
system. According to \cite{wxwidget:comparison} Lazarus has no C++ integration
to speak of. For this reason we qualify this tool as Nay.

\paragraph{GDK: Nay} Is a low level library that is ``An intermediate layer
which isolates GTK+ from the details of the windowing system.'' according to
\cite{gnome:gdk3}. In the presence of GTK+ the verdict is Nay.

\section{List of cross platform tools}

General source: see \cite{wiki:xplatf}. Left out non-free software or software
not for defined platforms. Due to the amount of toolkits, any obvious
disconnect with C++ is also reason to drop a choice\footnote{like Lazarus, that
in theory might be able to support the application, but has no direct
integration with C++.}.

\begin{description}
    \item[Cairo] is a library used to provide a vector graphics-based,
		device-independent API for software developers. It is designed
		to provide primitives for 2-dimensional drawing across a number
		of different backends. Cairo is designed to use hardware
		acceleration when available.
		The software is free software with a user license based on GPL
		and MPL.
		\hspace*{\fill}\\Ref: \cite{wiki:cairo}.

    \item[FLTK] The Fast, Light Toolkit (FLTK, pronounced fulltick) is a
		cross-platform graphical control element (GUI)
		library developed by Bill Spitzak and others. Made to
		accommodate 3D graphics programming, it has an interface to
		OpenGL, but it is also suitable for general GUI programming.
		\hspace*{\fill}\\Ref: \cite{wiki:fltk}.

    \item[fpGUI] the Free Pascal GUI toolkit, is a cross-platform
		graphical user interface toolkit developed by Graeme Geldenhuys.
		fpGUI is open source and free software, licensed under a Modified LGPL
		license. The toolkit has been implemented using the Free Pascal
		compiler, meaning it is written in the Object Pascal language.
		fpGUI consists only of graphical widgets or components, and a
		cross-platform 2D drawing library.
		fpGUI is statically linked into programs and is licensed using a
		modified version of LGPL specially designed to allow static linking to
		proprietary programs. The only code you need to make available are
		any changes you made to the fpGUI toolkit - nothing more.
		\hspace*{\fill}\\Ref: \cite{wiki:fpgui}.

    \item[GTK+] (previously GIMP Toolkit, sometimes incorrectly referred to
		as the GNOME Toolkit) is a cross-platform widget toolkit for
		creating graphical user interfaces. It is licensed under the terms
		of the GNU LGPL, allowing both free and proprietary software to use
		it. It is one of the most popular toolkits for the Wayland and
		X11 windowing systems, along with Qt.
		\hspace*{\fill}\\Ref: \cite{wiki:gtk+}.

    \item[JUCE] is a free software, cross-platform C++ application framework, used
		for the development of GUI applications and plug-ins.
		The aim of JUCE is to allow software to be written such that
		the same source code will compile and run identically on Windows,
		Mac OS X and Linux platforms. It supports various development
		environments and compilers, such as GCC, Xcode and Visual Studio.
		\hspace*{\fill}\\Ref: \cite{wiki:juce}.

    \item[Mozilla Application Framework]
		 is a collection of cross-platform software components that
		 make up the Mozilla applications. It was originally known as
		 XPFE, an abbreviation of cross-platform front end.
		 While similar to generic cross-platform application
		 frameworks like GTK+, Qt and wxWidgets, the intent is to
		 provide a subset of cross-platform functionality suitable for
		 building network applications like web browsers, leveraging
		 the cross-platform functionality already built into the Gecko
		 layout engine.
		\hspace*{\fill}\\Ref: \cite{wiki:mozilla_application_framework}.

    \item[OpenGL] is a cross-language, multi-platform application programming
		interface (API) for rendering 2D and 3D vector graphics.
		The API is typically used to interact with a graphics
		processing unit (GPU), to achieve hardware-accelerated
		rendering\footnote{OpenGL is a standard that has libraries based on free
		software and commercial libraries. Programmers never need a license
		to use an OpenGL library.}.
		\hspace*{\fill}\\Ref: \cite{wiki:opengl}.

    \item[Qt]  (/ˈkjuːt/ "cute", or unofficially as Q-T cue-tee) is a cross-platform
		application framework that is widely used for developing application
		software that can be run on various software and hardware platforms
		with little or no change in the codebase, while having the power and
		speed of native applications. Qt is currently being developed both
		by the Qt Company, a subsidiary of Digia, and the Qt Project under
		open-source governance, involving individual developers and firms
		working to advance Qt.

		Digia owns the Qt trademark and copyright. Qt is available with
		both proprietary and open source GPL v3 and LGPL v2 licenses.
		\hspace*{\fill}\\Ref: \cite{wiki:qt}.

    \item[Simple DirectMedia Layer] is a cross-platform software development
		library designed to provide a low level hardware abstraction
		layer to computer hardware components. Software developers
		can use it to write high-performance computer games and other
		multimedia applications that can run on many operating systems
		such as Android, iOS, Linux, Mac OS X, Windows and other platforms.

		SDL manages video, audio, input devices, CD-ROM, threads, shared
		object loading, networking and timers.[5] For 3D graphics it can
		handle an OpenGL or Direct3D context.

		The library is internally written in C and also provides the
		application programming interface in C, with bindings to other
		languages available.[6] It is free and open-source software subject
		to the requirements of the zlib License since version 2.0 and with
		prior versions subject to the GNU Lesser General Public License.
		Because of zlib SDL 2.0 is freely available for static linking
		in commercial closed-source projects, unlike SDL 1.2. SDL is
		extensively used in the industry in both large and small projects.
		Over 700 games, 180 applications, and 120 demos have also been
		posted on the library website.

		It is often believed that SDL is a game engine, but this is not
		true. However, the library is well-suited for building an engine
		on top of it.
		\hspace*{\fill}\\Ref: \cite{wiki:sdl}.
    \item[Tk]	is a free and open-source, cross-platform widget toolkit that
		provides a library of basic elements of GUI widgets for building
		a graphical user interface (GUI) in many different programming
		languages.
		\hspace*{\fill}\\Ref: \cite{wiki:tk}.
    \item[Ultimate++] is a C++ cross-platform development framework which aims
		to reduce the code complexity of typical desktop applications
		by extensively exploiting C++ features.
		\hspace*{\fill}\\Ref: \cite{wiki:ultimate++}.

    \item[WxWidgets] (formerly wxWindows) is a widget toolkit and tools library
		for creating graphical user interfaces (GUIs) for cross-platform
		applications. wxWidgets enables a program's GUI code to compile
		and run on several computer platforms with minimal or no code
		changes. It covers systems such as Microsoft Windows, OS X
		(Carbon and Cocoa), iOS (Cocoa Touch), Linux/Unix (X11, Motif,
		and GTK+), OpenVMS, OS/2 and AmigaOS. A version for embedded
		systems is under development.
		\hspace*{\fill}\\Ref: \cite{wiki:wxwidget}

	\item[Gled]	is a C++ framework for rapid development of applications with
		support for GUI (using FLTK), 3D-graphics and distributed
		computing. It extends the ROOT framework (standard data-analysis
		tool in high-energy physics) with mechanisms for object collection
		management \& serialization, multi-threaded execution, GUI
		auto-generation (object browser \& editor) and dynamic visualization
		(OpenGL). Distributed computing model of Gled is a hierarchy of
		nodes connected via TCP/IP sockets. Gled provides authentication \&
		access control, data exchange, proxying of object collections and
		remote method-call propagation \& execution. Gled can be dynamically
		extended with library sets. Their creation is facilitated by a set
		of scripts for creation of user-code stubs. Simple tasks and
		application configuration can be efficiently done via the
		interactive C++ interpreter (CINT). Gled is used for development of
		programs in high energy physics and as a research tool in
		distributed and grid computing. 
		\hspace*{\fill}\\Ref: \cite{fltk:gled}

    \item[GDK]  (GIMP Drawing Kit) is a library that acts as a wrapper around
		the low-level functions provided by the underlying windowing
		and graphics systems. GDK lies between the display server and
		the GTK+ library, handling basic rendering such as drawing
		primitives, raster graphics (bitmaps), cursors, fonts, as well
		as window events and drag-and-drop functionality.

		Like GTK+, GDK is licensed under the GNU Lesser General
		Public License (LGPL).
		\hspace*{\fill}\\Ref: \cite{wiki:gdk}
\end{description}

