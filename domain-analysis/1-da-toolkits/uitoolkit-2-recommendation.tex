\section{Result and recommendation}

\subsection{Result}

The final phase results in 4 tool kits being selected: \w{gtkmm}, \w{qt},
\w{wxwidgets} and \w{fltk}. Each of these tool kits satisfies all requirements
either autonomously or in combination with other tool kits. In terms of
requirement based selection, the search is finished. 

There is no way to recommend any one of these tool kits without including other
preferences, like the target architecture, build tools, extra libraries to be
used or personal preference. The recommendation is based on my personal
assessment of the project. The final decision is up to the team.

\subsection{Recommendation} 

I recommend using \w{fltk}.

\subsection{Motivation}

The underlying reasoning is that the project emphasizes a Network on Chip
design, with most user interface effort going into the 2D drawing interface.
This makes the focus of \w{fltk} on user interface\footnote{``\texttt{lean and
mean}''} a fine choice. Should we need more emphasis on looks or more complex
drawing than \w{fltk} can provide, then we could substitute \w{OpenGL} or
\w{Cairo}\footnote{% Integration of \w{OpenGL} or \w{Cairo} into \w{fltk} is
trivial}.

The only toolkit that I would recommend against is \w{wxwidget} because the
resulting binaries seem overweight. This might hamper downloads and installs at
the user's site, or worse donate a bloated ``feeling'' to our app\footnote{This
is only perception, but it could hurt general use of the product.}. 

\paragraph{In conclusion}\,the argument is thin. All being equal, I stand by my
choice, but with a slight shift of arguments a different choice could be valid.
This requires the team to really think about what is important in this project
and choose on that basis.


