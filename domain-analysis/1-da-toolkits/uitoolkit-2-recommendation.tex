\section{Result and recommendation}

\subsection{Result}

The final phase results in 4 tool kits: \w{gtkmm}, \w{qt}, \w{wxwidgets} and
\w{fltk}. Each of these tool kits satisfies all requirements either
autonomously or in combination with other tool kits. In terms of requirement
based selection, the search is finished.

\subsection{Considerations}

Considering the goals of the project, and the competence wishes, the team
has different options as a result of different viewpoints.  One option would be
to go with a toolkit that is ``\texttt{mean and lean}'' and add libraries only
where necessary. The other option would be to go with a tool kit that contains
``all++'' and thus is comfortable allbeit a little bloated. Whatever the
choice, it will affect the end product. 
\subsection{Recommendation} 

I recommend using \w{fltk}.

\subsection{Motivation}

In my recommendation I go with \texttt{mean and lean}.  The recommendation is
based on extra information (appendix \ref{sec:phase3listing}) and a personal
assessment of the project. The final decision is up to the team.

The project emphasizes a Network on Chip design, with most user interface
effort going into the 2D drawing interface and the general user interface.
This makes the focus of \w{fltk} on user interface with respect to gui a good
choice. Should we need more emphasis on looks or more complex drawing than
\w{fltk} can provide, then we could substitute \w{OpenGL} or
\w{Cairo}\footnote{Integration of \w{OpenGL} or \w{Cairo} into \w{fltk} should
be trivial}.

\par The argument is solid, but with a shift of viewpoints a different
choice could be valid.  The team should think about what is
important in this project and decide on that basis. See appendix
\ref{sec:phase3listing} on page~\pageref{sec:phase3listing} for a summary of
main characteristics of the selected tool kits.

