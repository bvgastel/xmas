\section{Search approach}

\subsection{Search approach motivation}

Normally one searches literature for information on science subjects. 
For tool kits this may not be the most efficient or effective way of finding 
current tool kits. The problem with literature is that it is constantly
behind on the current state of affairs. Even though information on the internet
in general is less reliable than literature generally is, in this case it forms
a better basis to find a more complete list of tool kits. Additionally literature
supplements what we find on internet.

This analysis is able to determine a list of user interface tool kits starting 
from en.widipedia and fanning out through the links wikipedia provides. The dates 
on the sites provide information on how actual the data is. Cross referencing 
sites gives confidence in the correctness of the information. The next section 
lays out the criteria we used to select tool kits.

\subsection{Selection criteria}

Table \ref{tab:appl-requirements} contains a list of the main requirements for
the design tool. These requirements generate criteria which we can use to find
the most appropriate tool kits available. Table \ref{tab:team-competence-req} 
shows a list of the team's preferences. Table \ref{tab:uitool-requirements} 
shows the implications of the requirements resulting in specific user interface 
requirements. The next section shows what steps we followed and what the results 
were.

\subsection{Selection process}

Table \ref{tab:discarded-toolkits} shows the first wave of packages that we discarded
with motivation. Table \ref{tab:compare-main-req} shows the result of 
the second wave of selection. This table shows for each tool kit to what degree it
satisfies the first set of requirements. Table \ref{tab:final-comparison} shows
the final comparison. 
