\section{Analysis}

\subsection{Search apporach}

Normally one searches literature for information on subjects in domain
analysis.  For toolkits this may not be the most efficient or effective way of
finding the best toolkit. The problem with literature is that it is constantly
behind on the current state of affairs. Even though information on the internet
in general is less reliable than literature generally is, in this case it forms
a better basis to find a more complete list of toolkits. 

Using the en.wikipedia as entrance and fanning out through the links wikipedia
provides we were able to determine a list of user interface toolkits that in
practice is sufficient for our goal. The dates on the sites provide information
on how actual the data is. Crossreferencing sites gives confidence in the
correctness of the information. The next section lays out the criteria we used
to select toolkits.

\subsection{Selection criteria}

Table \ref{tab:appl-requirements} contains a list of the main requirements for
the design tool. These requirements generate criteria which we can use to find
the most appropriate toolkits available.

Table \ref{tab:team-competence-req} shows a list of the team's preferences.
Any ties\footnote{toolkits with equal qualification regarding requirements} can
be broken using the team members preferences. Table
\ref{tab:uitool-requirements} shows the implications of the requirements
resulting in specific user interface requirements.

The next section shows what the steps we followed and what the results were.

\subsection{Selection process}

Table \ref{tab:discarded-toolkits} shows the first wave of packages that did
not make the comparis due to not satisfying one of the important requirements
(being free software, running on the defined platforms).

Table \ref{tab:compare-main-req} shows the result of the second wave of
selection. These packages satisfy all main requirements and need further
selection. 

\todo[inline]{This should be a paragraph on the final selection.}
