\section{Phase 3}

\subsection{Phase 3 selection results} 
\label{sec:phase3appendix}

Table \ref{tab:uitool-requirements} in appendix~\ref{sec:requirement-impact}
shows the tool kit requirements.  The final comparison in
Table~\ref{tab:final-comparison} takes the remaining tool kit requirements into
account.

\vspace{1em}
\begin{minipage}{.95\textwidth}
	\begin{center}
		\small\sf
		\begin{tabular}{|c|p{9em}|p{8em}|cccc|}
			\hline
			{\bf id} & {\bf req}             & {\bf fitness}           & \w{gtkmm} & \w{qt} & \w{wx} & \w{fltk}\\
			\hline
			    7    & Ease of use           & reputation              & +                      & + & + & +\\
			    8    & Ease of learning      & rep \& tutorials        & +                      & + & + & +\\
			    9    & Documentation         & website \& rep          & +                      & + & + & + \\
			    10   & Observer pattern      & signal processing       & +                      & + & + & 
			    +\footnote{using standard C++ or another library like boost\label{fn:c++}} \\
			    14   & Concurrency           & support for concurrency & +                      & + & + & +\footref{fn:c++} \\
			    15   & Concurrent observer   & is it possible?         & +                      & + & + & +\footref{fn:c++} \\
				\hline
		\end{tabular}
		\captionof{table}{Final comparison of selected tool kits}
		\label{tab:final-comparison}
	\end{center}
\end{minipage}

\paragraph{Ease of use and ease of learning} We can only measure the
requirements for ease of use and learning subjectively. This is an experience
result and changes in time: any tool kit becomes easy to use and learn after
enough experience using it. 

So the only measure that we could use is reputation, plus tutorials for ease
of learning.  Although subjective, the cumulative criticism and opinions
indicate real quality. Neither of the systems have bad rep on any of the
subjectively measurable qualities (7, 8 and 9). Also, the main website for the
tool kit easily revealed the available documentation. The assumption that any
one of these tool kits lacks in ease of use, learning or in documentation can
be rejected on the basis of experience and reputation. 

\paragraph{Observer pattern.} The observer pattern is important in two ways.
First, all signals from the user interface should relay flawlessly to modules,
even if they execute concurrently. Secondly, any change of status or content in
the modules, even concurrent modules should relay flawlessly to the user
interface thread. Measuring this is a question of checking the docs and asking
around. 

\subsection{Phase 3 selection analysis}

Table \ref{tab:final-comparison} shows that tool kit requirements 7, 8 and 9
are satisfactory for all. The observer pattern and concurrent process
communication is not as simple. In summary, \w{qt}, \w{gtkmm} and \w{WxWdiget}
all suffice without modification. For \w{fltk} we need to use one of the
available cross platform libraries for concurrency. It turns out to be easy
to define ones own observer pattern, or use a library for this like \w{boost}.

As all packages can fulfil the requirements the question is, what distinguishes
these tool kits and on what basis should we choose? The short answer is: they
all suffice. So the choice is no longer one of user interface requirements, but
one of preference. 

