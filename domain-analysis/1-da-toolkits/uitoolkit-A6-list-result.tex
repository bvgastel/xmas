\section{Summary of applicable tool kits}

In order to support the final decision, in the following sections I summarize
the main characteristics of the tool kits plus I specify reasons why to choose
or not to choose that specific tool kit.

\subsection{\w{qt}} 

\w{qt} is a complete development environment based on an extension of C++ with
a \w{qt} pre-compiler. It supports many features among which are 2D,
concurrency and memory management. It also has many dependencies as shown in
\cite{qt:qt-dep}.

Main characteristic of \w{qt} is that it is uses a pre-compiler called
\texttt{moc} that extends standard C++ with facilities needed for signal and
slots. The owner is a company \w{Digia} that licenses the \w{qt} toolkit both
as free software and with a commercial license.

The tool kit \w{qt} supports the observer pattern with signals and slots.
Defining an independent observer and observable should also be possible.

\paragraph{Why to choose \w{qt}?} Because \w{qt} runs equally well on many
platforms, including netbooks, and other mobile computers, and because it has
many well known applications running on many platforms. Appealing is also its
simple signal and slot feature and support for it's user interface building is
renowned.  Finally, \w{qt} has an active community and very clear documentation
both online and in books. 

\paragraph{Why not to choose \w{qt}?} Because \w{qt} extends C++, uses macro's
heavily and thus needs an extra pre-processor to build the application. Also,
it provides a lot more than we need for our application.

\subsection{\w{gtkmm}}

\w{gtkmm} combines a graphical user interface with many non gui features like
signals and memory management. For 2D it builds on cairo or opengl. The library
for gtkmm has some dependencies on other libraries that have their own
dependencies as shown in \cite{gtkmm:gtk+-dep}.

Main characteristic of \w{gtkmm} is that it is primarily linux oriented and is
the main development environment for the \w{Gnome} window manager. It uses
standard C++ and additionally collections and other features from \w{Glib}.

The tool kit \w{gtkmm} supports the observer pattern using signals. Defining an
independent observer and observable should also be possible.

\paragraph{Why choose \w{gtkmm}?} Because \w{gtkmm} is C++ even if it is a
wrapper: it does not use macro's heavily. Because it supports cairo natively.
Finally \w{gtkmm} has an active community and clear documentation both online
and in books\footnote{\w{qt} has more literature than \w{gtk+}}.

\paragraph{Why not to choose \w{gtkmm}?} Because \w{gtkmm} is linux oriented
(developments starts from the \w{Gnome} window manager). The primary aim is
unix, not windows or mac.  Also, it provides a lot more than we need for our
application.

\subsection{\w{wxwidget}} 

\w{wxwidget} is a complete cross platform development environment. It is built
as multiple libraries by default. The features cover support for openGL and
concurrent execution. The \w{wxwidget} depends on different libraries for each
platform.

Main characteristic of \w{wxwidget} is that it has a separate underlying
interface to each platform. The result is an application with the look and feel
of the platform. The disadvantage is that applications run differently on
different platforms and may show platform dependent errors. Also, the binaries
of the application are usually bigger than for the other tool kits.

The tool kit \w{wxwidget} does not seem to support the observer pattern
directly, but some examples of how to define one are available like
\cite{wxwidget:observer-example}\footnote{The surrounding text is in Russion,
but the example is clear.}.

\paragraph{Why choose \w{wxwidget}?} Because it is cross platform. It is well
known and used in many applications.
 
\paragraph{Why not to choose \w{wxwidget}?} It produces overweight applications
that could make the application feel bloated.

\subsection{\w{fltk}}

\w{fltk} is a graphical user interface with 2D drawing and nothing more. For
concurrency we look to another cross platform tool kit, like \w{boost}. The
tool kit \w{fltk} depends on the window manager X11 or Wayland or Win32.

Main characteristic of \w{fltk} is that it has no additional facilities.  The
motto "Do one thing and do it right" is one that unix programmers often adhere
to. The binaries for programs using this tool kit are small in comparison to
other tool kits. 

The tool kit \w{fltk} does not support an observer pattern, but one can define
one as stated on the forum at
url{http://groups.google.com/forum/\#!forum/fltkgeneral}.

\paragraph{Why choose \w{fltk}?} Because it is focussed on graphical user
interface.  It does one thing and it does it well. It is easy to learn. It is
easy to combine with other libraries like \w{OpenGL}, \w{Cairo} or \w{boost}.
The community is active and willing to support users of their
library\footnote{I found their support to be both fast and objective (in an
\w{fltk} biased way of course).}. 

\paragraph{Why not to choose \w{fltk}?} Because it does not offer more than
graphical user interface. Also, the code shows it's age in using c-strings in
the API.

