\section{Problem domain}
\subsection{Description}

The customer's request is to build a graphical chip editor for Network on chip
designs that uses existing C++ program to interface with and that runs on all
defined platforms: Linux, ms Windows and Mac. This implies part of the solution
is a user interface tool that works from graphical representation of chip
networks. This document is the result of gathering information and analysing
existing user interface tools that we could use for this purpose.

\subsection{Approach}

\paragraph{Comparison.} The number of available user interface and 2D drawing
toolkits is large enough to justify a comparison on the basis of objective
criteria. The number of features each toolkit has, is large enough to warrant
limitation of criteria. The criteria derive from the project and product
requirements followed by team requirements and finally team preference.

\paragraph{Requirements.} The customer specified requirements -- both
non-functional and functional -- each of which could have implications for the
choice of user interface toolkit. The team members have their learning
requirements which could influence the choice as well.

\paragraph{Goal.} The primary requirement -- the one that we did not explicitly
write down when comparing the UI tools -- is that the UI tool \emph{suffices}.
It does not have to be perfect, it does not have to dazzle.
Anything more than sufficient is extra and not discriminating for this analysis
document. In the end -- all being equal -- the team will choose from the
candidates on basis of preference.