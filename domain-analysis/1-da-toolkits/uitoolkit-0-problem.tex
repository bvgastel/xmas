\section{Problem domain}
\subsection{Description}

The customer's request is to build a graphical chip editor for Network on chip
designs that uses existing C++ program to interface with and that runs on all
defined platforms: Linux, ms Windows and Mac. This implies part of the solution
is a user interface tool that works from graphical representation of chip
networks. This document is the result of gathering information and analysing
existing user interface tools that we could use for this purpose. Addiditonally
the customer has specified some qualitative requirements like maintainability
and installability.

\subsection{Approach}

The primary goal is that the UI tool suffices: it does not have to be perfect.
All being equal the team will choose from the candidates on the basis of
preference.

The number of available user interface toolkits is large.  Moreover the number
of features each toolkit has, is large. This warrants careful consideration
of toolkits where we start from the project's requirements and use the team's 
preference to break any ties.

\subsection{Search}

Normally ones searches literature for information on subjects in domain
analysis.  For toolkits this may not be the most efficient or in fact the most
effective way of finding the best toolkit. The problem with literature is that
it is constantly behind on the current, actual state of affairs. Even though
information on the internet in general is less reliable than literature
generally is, in this case it forms a better basis to find a more complete list
of toolkits. 

\section{Selection criteria}

Table (\todo{cite}) contains a list of the main requirements for the design tool.
These requirements are not directly applicable, but do generate criteria which we 
can use to find the most appropriate toolkits available. In order to do the team 
justice, any ties\footnote{toolkits with equal qualification as far as requirements are concerned} need to be broken using the team members preferences. Table \todo{cite} 
shows a list of the team's preferences.

Table \todo{cite} shows the implications of the requirements resulting in \todo{X} 
candidates. 
of the requirements.

