\section{Requirements for design tool}

The high level requirements for the project are aimed at the chip design tool. {\em see table \ref{tab: appl-requirements}}.
Student requirements in terms of competence goals.

Each requirement may transfer as full requirement to the UI tool, partially or not at all. {\em See table \ref{tab: uitool-requirements}}.
For analysis and comparison of UI tools we use the relevant requirements as defined in table \ref{tab: uitool-requirements}.
As multiple tools satisfy the high level requirements we use team preference and competence wishes to discriminate and select the tools.


\begin{center}
    \small\sf
    \begin{tabular}{|l|p{7em}|p{23em}|}
	\hline
	        & \multicolumn{2}{c|}{\sf\em\large requirement class}\\\hline
	{}      & \multicolumn{1}{c|}{\bf req name} & \multicolumn{1}{c|}{\bf req desc}\\\hline
	M	& \multicolumn{2}{c|}{\sf\emph{\large Must have requirements}}
		\\\hline
	M1	& Cross platform & The application must run equally well on the
				    defined platforms with respect to all relevant features.
				    Linux, Mac and MS Windows are the defined platforms.
		\\\hline
	M2	& 2D drawing & The application must be able to draw items on a
				canvas and treat graphical objects with composite
				behaviour or properties as a primitive object.
		\\\hline
	M3	& C++ integration & The application must integrate seamlessly with
				the existing C++ code for analysis and verification tools.
		\\\hline
	S	& \multicolumn{2}{c|}{\sf\emph{\large Should have requirements}}
		\\\hline
	S1	& expandability & the application should be expandable in terms of
				analysis and verification modules, xmas-primitives and
				network reporting.
		\\\hline
	S2	& installability & The application should be easy to install.
		\\\hline
	S3	& maintainability & The application should be as easily maintainable as possible.
		\\\hline
	S4	& plugability	& The application should use plugins for analysis and
				verification modules.
		\\\hline
	S5	& performance	& The application should be performant enough for the GUI
				    interface to be ``snappy''
		\\\hline
	C	& \multicolumn{2}{c|}{\sf\emph{\large Could have requirements}}
		\\\hline
	N	& \multicolumn{2}{c|}{\sf\emph{\large Nice to have requirements}}
		\\\hline
	N1	& install util	& It would be nice to have the application provide
				installable packages to the users.
		\\\hline
    \end{tabular}
    \captionof{table}{High level requirements for the design tool}
    \label{tab: appl-requirements}
\end{center}

%%%%%%%%%%%%%%%%%%%%%%%%%%%%%%%%%%%%%% input from team competence tex file (a symbolic link) %%%%%%%%%%%%%%%%%%%%%%%%%%%%%%%%%%
%%
%%  must be input: include is not permitted from an included file
%%
%%
%% This file contains the competence requirements for the team members.
%% It is part of other documents.
%%

The team has explicit wishes concerning improving or learning competences. Table
\ref{tab:team-competence-req} specifies these wishes per team member.

\begin{center}
    \begin{tabular}{cllp{17em}}
        \hline
        & {\bf Prio} & {\bf Competence}    & {\bf Goal}\\
        \hline
        \multicolumn{4}{c}{\sf\emph{Guus}}\\
        \hline
		GM0 & Must have    & Free software & Prefer to build free software.\\
        GM1 & Must have    & UI tool       & Learning to work with a cross platform UI tool like QT or GTK+.\\
        GS1 & Should have  & Agile		   & Learning to work in an agile environment.\\
        GS2 & Should have  & C++		   & Learning to program C++ (standard 2011).\\
	GC1 & Could have   & versioncontrol	   & Experiencing and learning teamwork with versioncontrol.\\
        GC2 & Could have   & Requirements Eng.   & Learning to apply Requirements engineering.\\
        \hline
        \multicolumn{4}{c}{\sf\emph{Stefan}}\\
        \hline
        SS1 & Should have  & Agile		   & Learning to work in agile environment, like DAD.\\
        SS2 & Should have  & Online Agile tools  & Learning to work agile tools.\\
        SS3 & Should have  & distributed team    & Learning to work in a distributed team.\\
        SC1 & Could have   & Non Microsoft tools & Learning to work with platform independent tools.\\
        \hline
        \multicolumn{4}{c}{\sf\emph{Jeroen}}\\
        \hline
                     &                     & \\
        \hline
    \end{tabular}
    \captionof{table}{Competence wishes of the team}
    \label{tab:team-competence-req}
\end{center}

%%
%%%%%%%%%%%%%%%%%%%%%%%%%%%%%%%%%%%%%% input from team competence tex file (a symbolic link) %%%%%%%%%%%%%%%%%%%%%%%%%%%%%%%%%%

\section{Requirements for UI toolkit}

Using the requirements for the application plus the competence requirements of mij team, the impact of
all these demands is summarized in table \ref{tab: uitool-requirements}. Only requirements directly relevant
to the ui toolkit have consequential requirements formulated.

\begin{center}
    \begin{longtable}{ll||cp{13em}}
	{\bf id: requirement}     & {\bf impact} & {\bf id } & {\bf UI requirement}\\\hline\endhead
	\hline \multicolumn{4}{c}{UI toolkit requirements (Continue on next page)}\endfoot
	\hline \multicolumn{4}{c}{UI toolkit requirements}\endlastfoot
	\hline
        M1: Cross platform        & full         & M1-ui1   & ui toolkit runs equally well on the defined platforms for all features.\\
        M2: 2D drawing            & full         & M2-ui1   & ui toolkit has equal 2D features on all defined platforms.\\
        M3: C++ integration       & full         & M3-ui1   & ui toolkit does not hinder C++ integration in any way.\\
        S1: expandability         & limited      & S1-ui1   & ui toolkit does not hinder creating plugins.\\
        S2: installability        & limited      & S2-ui1   & ui toolkit does not hinder install procedure.\\
        S3: maintainability       & moderate     & S3-ui1   & ui toolkit is easy to use.\\
	                          &              & S3-ui2   & ui toolkit is easy to learn.\\
	                          &              & S3-ui3   & ui toolkit is well documented.\\
	                          &              & S3-ui4   & ui toolkit supports observer pattern.\\
        S4: plugability           & limited      & S4-ui1   & ui toolkit should not hinder creating plugins.\\
        S5: performance	          & moderate     & S5-ui1   & ui toolkit should support running concurrent processes for modules.\\
	                          &              & S5-ui2   & ui toolkit should support observer pattern from independent processes.\\
	                          &              & S5-ui3   & ui toolkit should support interruption during concurrent processing.\\
        N1: install util          &              &          & \\
        \hline
        GM1: UI tool	          &              &          & Enclosed in requirement M1.\\
        GS1: Agile	          &              &          & \\
        GS2: C++	          & compatible   & GS2-ui1  & Prefer a ui toolkit that works with C++.\\
        GC1: versioncontrol       &              &          & \\
        GC2: req. engineering     &              &          & \\
        \hline
        SS1: agile	          &              &          & \\
        SS2: online agile tools   &              &          & \\
        SS3: distributed team     &              &          & \\
        SC1: platform independent & compatible	 & SC1-ui1  & Prefer a platform independent toolkit.\\
    \end{longtable}
    \captionof{table}{UI Toolkit requirements.}
    \label{tab: uitool-requirements}
\end{center}

In table \ref{tab: uitool-requirements} the term ``support for plugins''
is meant to be a weak requirement in the sense that it does not hinder
the design of plugins. The ability to support plugins directly is a nice
to have requirement. The derived requirements ``ease of use'' and ``ease of
learning'' are important requirements.
