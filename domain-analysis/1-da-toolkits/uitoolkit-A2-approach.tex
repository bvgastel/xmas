\section{Approach of the selection process}

The character of Graphical User Interface toolkits (GUI toolkits) is such that
a reasonably recent list of toolkits can only be found on internet. The best
start is wikipedia which links to toolkits, each often linking to other
toolkits. The set of toolkits studied here is a sound base for finding a
toolkit that suffices for our needs.

We selected the tools based on various web searches leading among others
leading to \cite{wiki:xplatf} and \cite{wiki:xplatf_gui}.  The amount of
toolkits available is staggering, so the first step is weeding out the toolkits
that do not support one of our defined platforms, are not free software of just
do not interact with C++ sufficiently.  For that reason this analysis does not
consider the following toolkits:

\section{Phase 1 Preselection of user interface toolkits}

\begin{center}
    \begin{tabular}{|p{7em}|p{25em}|}
	\hline
	{\bf name} & {\bf reason}\\\hline
	AppearIQ & non-free software. Ref: \cite{appear:appeariq}.\\
	Eclipse & a compiler, not a graphical framework. Ref: \cite{wiki:eclipse}.\\
	GeneXus & non-free software. Also not for linux
		    or mac. Ref: \cite{wiki:genexus}. \\
	Haxe & not a graphics platform. Ref: \cite{wiki:haxe}\\
	Max & not for linux, not free software. Ref: \cite{wiki:max}\\
	Mono & not a graphics environment, emulation
		of C\#. Ref: \cite{wiki:mono}.\\
	MonoCross & aimed at C\#. Ref: \cite{wiki:monocross}.\\
	MoSync & aimed at mobile platforms, no longer maintained. Ref: \cite{wiki:mosync}.\\
	Xojo & non-free software. Ref: \cite{wiki:xojo}.\\
	Smartface & non-free software\footnote{limited edition gratis available}. Ref: \cite{wiki:smartface} and \cite{smartface:license}.\\
	WebDev & aimed at creating websites. Ref: \cite{wiki:webdev}.\\
	WinDev & aimed at data centric apps with forms, works with webdev. Ref: \cite{wiki:windev}.\\
	XPower++ & Insufficient information, looks like an advertisment. Ref: \cite{wiki:xpower++}.\\
	Lazarus & A Pascal development environment. Ref: \cite{wiki:lazarus}.\\
	Ultimate++ & Does not support MacOS. Ref: \cite{wiki:ultimate++} and \cite{wxwidget:comparison}.\\
	\hline
    \end{tabular}
    \captionof{table}{Platforms not considered with reason}
	\label{tab:discarded-toolkits}
\end{center}


\section{Phase 2 Selection of user interface toolkits on main requirements}

Using the main requirements including free software (first column) and the
defined platforms (second column) Table \ref{tab:compare-main-req} shows the
remaining toolkits and their rating according to the main requirements
including free software and defined platforms.

Among the 3 toolkits qualified as ``Maybe'' are some pretty unconventional, but
powerful frameworks. For instance Gled is a framework that stems from
scientific work and features the easy distribution of nodes accross threads,
processes and machines.

The resulting preferred toolkits are GTK+, Qt and WxWidgets due to the size of
the developer and user community. 

\begin{center}
    \small\sf
    \begin{tabular}{c|lccc|c|ccc}
	\hline
	{\bf nr} & {\bf Toolkit} & {\bf C++} & {\bf 2D} & {\bf GUI} &            & \multicolumn{3}{c}{\em Yay or Nay} \\
	         & {\bf Name}    & 			 &          &            & {\bf Community} 	& {\bf Yay} & {\bf Part} & {\bf Nay} \\
	         &			     &         	 &          &            &         		&           & {\bf Maybe}  &         \\
        \hline
%%%%%%%%%%%%%%%%%%  	  C++  2D  GUI  Comm  Yay   Part   Nay
%%%%%%%%%%%%%%%%%%                                  Maybe
1  &	GTK+		& 1   & 1 & 1 &  Large  & Yay &         &     \\
2  &	Qt	      	& 1   & 1 & 1 &  Large  & Yay &         &     \\
3  &	WxWidgets 	& 1   & 1 & 1 &  Large  & Yay &         &     \\\hline
4  &	JUCE      	& 1   & 1 & 1 &  Small  &     & Maybe   &     \\
5  &	CEGUI     	& 1   & 1 & 1 &  Small  &     & Maybe   &     \\
6  &	Gled		& 1   & 1 & 1 &  Small  &     & Maybe   &     \\
\hline
6  &	Cairo     	& 1   & 1 & 0 &  Large  &     & Partial &     \\
7  &	FLTK      	& 1   & 0 & 1 &         &     & Partial &     \\
8  &	OpenGL 	  	& 1   & 1 & 0 &  Large  &     & Partial &     \\
9  &	SDL		& 1   & 1 & 0 &         &     & Partial &     \\\hline
10 &	Mozilla A.F.  	& 1   & 1 & 1 &  Large  &     & Web-oriented   &     \\\hline
11 &	Tk	        & 0.5 & 1 & 1 &  Large  &     &         & Nay \\
12 &	fpGUI     	& 0   &   &   &         &     &         & Nay \\
13 &	GDK       	& 1   & 1 & 0 &  Large  &     &         & Nay \\
\hline
    \end{tabular}
    \captionof{table}{Selected tools against must have product/project requirements}
	\label{tab:compare-main-req}
\end{center}

\paragraph{GTK+: Yay} builds on Cairo and GDK, has a large developer and user
community.  It fullfills the must-haves and the verdict is Yay. It has a
wrapper \verb!GTKMM! for C++. The verdict is Yay.

\paragraph{Qt: Yay} has a large developer and user community. According to the
wxwidget comparison with Qt, both are functionally comparable
(\cite{wxwidget:comparison}).

\paragraph{WxWidgets: Yay} The framework is complete and has no obvious
disadvantages. According to their comparison to Qt, the two are functional
comparable (\cite{wxwidget:comparison}).  The verdict is Yay.

\paragraph{JUCE: Maybe} A one man project with emphasis on audio
(\cite{juce:juce},\cite{wiki:juce}).  It runs all defined platforms, written in
C++. The user license is dual GPL and commercial.  The verdict according to
requirements is Maybe due to the size of the developer community.

\paragraph{CEGUI: Maybe} This toolkit satisfies the main requirements, but has a
small developer community (\cite{wiki:cegui} and \cite{cegui:cegui}. The
verdict for this toolkit is Maybe due to the size of the developer community.

\paragraph{Gled: Maybe} This toolkit satisfies the main requirements, but has a
small developer community (\cite{gled:gled}). The verdict is Maybe.

\paragraph{Cairo: Partial} has a wrapper \verb!cairomm! for C++.  Cairo is
popular
in the free software communicty for providing cross-platform support for
advanced 2D drawing. Other GUI library (notably GTK+) already include Cairo.
This library does not contain GUI widgets and so should be combined with
\verb!SomeGUILibrary!. The verdict for Cairo is Partial.

\paragraph{FLTK: Partial} This is a GUI only library. For 2D a different library
should be used and should be combined with \verb!Some2DLibrary!. The verdict is
Partial.

\paragraph{OpenGL: Partial} hardware acceleration and powerful but complex. Low
level library. It does not directly support any GUI widgets. Many GUI libraries
support the use of an OpenGL library. The verdict is Partial.

\paragraph{SDL: Partial} This is Low level library meant for 2D game
development. It does not contain GUI widgets. It may be usable in combination
with other libraries. The verdict is Partial.

\paragraph{Mozilla Application Framework: Maybe} The MAF is platform
independent, web oriented with a large developer and user
community\footnote{All firefox users are part of the user community}.  The
library is meant to support a subset of the standard GUI frameworks like GTK+,
QT and WxWidgets, aimed at web-programs. Due to the web orientedness and it's
partial support for widgets the verdict for this toolset is Maybe.

\paragraph{Tk: Nay} Seems to contain everything necessary for both GUI and 2D
drawing, but C++ integration is not clear (\cite{wiki:tk}, \cite{tcltk:tk}).
According to the comparison on WxWidgets, for C++ better not use Tk
(\cite{wxwidget:comparison}).  The verdict for that reason is Nay.

\paragraph{fpGUI: Nay} is a pascal library based on free pascal (lazarus)
(\cite{Geldenhuys:fpgui}).  The author is also the sole maintainer of the
system. According to \cite{wxwidget:comparison} Lazarus has no C++ integration
to speak of. For this reason we qualify this tool as Nay.

\paragraph{GDK: Nay} Is a low level library that is ``An intermediate layer
which isolates GTK+ from the details of the windowing system.'' according to
\cite{gnome:gdk3}. In the presence of GTK+ the verdict is Nay.

