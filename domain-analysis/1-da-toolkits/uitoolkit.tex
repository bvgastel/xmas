%%
%% Dit is het hoofddocument: compileer dit met latex of xelatex en je krijgt de gehele pdf
%%
\documentclass[a4paper,11pt,draft]{article}

\usepackage{subfig}
\usepackage{wrapfig}% product wrapfigure and wraptable
\usepackage{array}% additions to tabular
%\usepackage{supertabular}% multiple pages tabular
\usepackage{longtable}% multiple pages table like tabular
\usepackage{rotating}% for the environment sidewaysfigure / sidewaystable
\usepackage[english]{babel}
\usepackage{graphicx}
\usepackage{hyperref}% Load after biblatex
\hypersetup{
    colorlinks = true,
    citecolor = blue,
    linkcolor = blue
}
\usepackage[prependcaption,colorinlistoftodos,obeyFinal,textsize=tiny]{todonotes}% when generating final (documentclass option) skip notes
\usepackage{pgfgantt}
\usepackage{pdflscape}
\usepackage[a4paper]{geometry}
\usepackage{titlesec}% added to change section headers, see newcommand definition.
\usepackage{boxedminipage}
\usepackage{amssymb}% For \checkmark
\usepackage{pifont}% for \ding{'-code or "-code}

\bibliographystyle{plain}% unsrt, plain, alpha, abbrv

\newcommand{\biburl}[1]{\hspace*{\fill}\\\url{#1} accessed on 25 oct 2014}

\author{Guus Bonnema}
\date{21/10/2014}

\title{Domain analysis for a user interface toolkit}

\setlength\extrarowheight{2pt}% Adds a little space at the top of table rows

%% Document is in subdocumenten gesplitst.

\begin{document}

\selectlanguage{english}
\hyphenation{func-tio-nal}

%%%%%%%%%%%%%%%%%%%%%%%%%%%%%%%%%%%%
\newcommand{\xmas}{x\textsc{mas}}%
\newcommand{\ok}{$\checkmark$}

\newcommand{\mybox}[1]{\begin{boxedminipage}[t]{\textwidth}#1\end{boxedminipage}}

%\definecolor{airforceblue}{rgb}{0.36, 0.54, 0.66}%%   This is color in hex #5D8AA8

%%%%%%%%%%%%%%%%%%%%%%%%%%%%%%%%%%%% different section format start %%%%%%%%%%%%%%%%%%%%%%%%%%%%%%%%%
\newcommand\secformat[1]{%
    {\fontsize{60}{60}\selectfont\thesection}%
    \ifthenelse{\equal{\thesection}{}}{}{\quad\rule[-8pt]{2pt}{40pt}\quad}
    \parbox[b]{.7\textwidth}{\filright\bfseries #1}}%
\titleformat{\section}[block]
    {\filright\normalfont\sffamily}{}{0pt}{\secformat}
\titlespacing*{\section}{0pt}{*3}{*2}[1pc]
%%%%%%%%%%%%%%%%%%%%%%%%%%%%%%%%%%%% different section format end   %%%%%%%%%%%%%%%%%%%%%%%%%%%%%%%%%

\maketitle

\begin{abstract}
    This document researches the available toolkits and develops an advise
    based on the requirements for the user interface of the design tool for xmas.

    The document starts out by detailing the required characteristics of the toolkit
    and using the as a basis for the recommendation. Failing a clear advantage in the
    \emph{must have} requirements, it compares the advantage in the \emph{should have}
    requirements and failing that in the \emph{could have} requirements.

    In case of multiple tools satisfying all requirements the final recommendation is made
    based on the teams wishes to learn a specific competence in this project.
\end{abstract}

%\listoftodos   %% commented out when creating final document

\section{Problem domain}
\subsection{Description}

The customer's request is to build a graphical chip editor for Network on chip
designs that uses existing C++ program to interface with and that runs on all
defined platforms: Linux, ms Windows and Mac. This implies part of the solution
is a user interface tool that works from graphical representation of chip
networks. This document is the result of gathering information and analysing
existing user interface tools that we could use for this purpose. Addiditonally
the customer has specified some qualitative requirements like maintainability
and installability.

\subsection{Approach}

The primary goal is that the UI tool suffices: it does not have to be perfect.
All being equal the team will choose from the candidates on the basis of
preference.

The number of available user interface toolkits is large.  Moreover the number
of features each toolkit has, is large. This warrants careful consideration
of toolkits where we start from the project's requirements and use the team's 
preference to break any ties.

\subsection{Search}

Normally ones searches literature for information on subjects in domain
analysis.  For toolkits this may not be the most efficient or in fact the most
effective way of finding the best toolkit. The problem with literature is that
it is constantly behind on the current, actual state of affairs. Even though
information on the internet in general is less reliable than literature
generally is, in this case it forms a better basis to find a more complete list
of toolkits. 

\section{Selection criteria}

Table (\todo{cite}) contains a list of the main requirements for the design tool.
These requirements are not directly applicable, but do generate criteria which we 
can use to find the most appropriate toolkits available. In order to do the team 
justice, any ties\footnote{toolkits with equal qualification as far as requirements are concerned} need to be broken using the team members preferences. Table \todo{cite} 
shows a list of the team's preferences.

Table \todo{cite} shows the implications of the requirements resulting in \todo{X} 
candidates. 
of the requirements.


\section{Requirements for design tool}

Table \ref{tab: appl-requirements} summarizes the high level requirements for
the chip design tool. Table \ref{tab: team-competence-req} shows the team
members' requirements in terms of competence goals. Each requirement may
transfer as full requirement to the UI tool, partially or not at all. Table
\ref{tab: uitool-requirements} summarizes the relationship of the project's
requirements with the user interface toolkit.

If multiple tools satisfy the complete set of requirements, then
team preference will rule the recommendation. Ultimately the customer
and the team decide depending on the recommendation and any external factors
like an other domain analysis.

\subsection{Free software requirement}

The first requirement $M0$ was specified to ensure the free availability of the
ui tool both now and in the future. The source of
this requirement is a remark from the Open University instructions
``Hulpmiddelen en bronnen'' (``tools and sources'') with the following text:

\vspace{1em}

\begin{tabular}[t]{ll}
\begin{minipage}{.45\textwidth}

	``Wij gaan ervan uit dat de producten van het team uiteindelijk veelal als
	open source beschikbaar gesteld worden, dus iedereen mag in principe de
	producten inzien. In de overeenkomst met de opdrachtgever dient op het punt
	van de rechten duidelijkheid gegeven te worden.''

\end{minipage}

&

\begin{minipage}{.45\textwidth}

	``We assume open source availability of the team's products so that the
	products are publicly available. The contract with the customer should be
	clear on this point.''

\end{minipage}

\end{tabular}

\vspace{1em}

\noindent Demanding free software follows from this specification and the fact
that non-free software could potentially lead to non-availability of the
software due to license fees or the author retracting the software. We specify
free software as a must-have requirement to be ``clear on this point''.

\subsection{The project and product requirements}

\begin{center}
    \small\sf
    \begin{tabular}{|l|p{7em}|p{23em}|}
	\hline
	        & \multicolumn{2}{c|}{\sf\em\large requirement class}\\\hline
	{}      & \multicolumn{1}{c|}{\bf req name} & \multicolumn{1}{c|}{\bf req desc}\\\hline
	M	& \multicolumn{2}{c|}{\sf\emph{\large Must have requirements}}
		\\\hline
	M0	& Free software  & The application is free software as defined by the FSF.
		\\\hline
	M1	& Cross platform & The application must run equally well on the
				    defined platforms with respect to all relevant features.
				    Linux, Mac and MS Windows are the defined platforms.
		\\\hline
	M2	& C++ integration & The application must integrate seamlessly with
				the existing C++ code for analysis and verification tools.
		\\\hline
	M3	& 2D drawing & The application must be able to draw items on a
				canvas and treat graphical objects with composite
				behaviour or properties as a primitive object.
		\\\hline
	M4	& GUI widgets & The application must be able to create GUI widgets
				on all defined platforms.
		\\\hline
	M5      & xmas primitives & The application must be able to draw the 8 xmas
				    primitives.
		\\\hline
	M6	& xmas macros     & The application must be able to define and draw
				    a macro of xmas primitives.
		\\\hline
	S	& \multicolumn{2}{c|}{\sf\emph{\large Should have requirements}}
		\\\hline
	S1	& expandability & the application should be expandable in terms of
				analysis and verification modules, xmas-primitives and
				network reporting.
		\\\hline
	S2	& installability & The application should be easy to install.
		\\\hline
	S3	& maintainability & The application should be as easily maintainable as possible.
		\\\hline
	S4	& plugability	& The application should use plugins for analysis and
				verification modules.
		\\\hline
	S5	& performance	& The application should be performant enough for the GUI
				    interface to be ``snappy''
		\\\hline
	C	& \multicolumn{2}{c|}{\sf\emph{\large Could have requirements}}
		\\\hline
	N	& \multicolumn{2}{c|}{\sf\emph{\large Nice to have requirements}}
		\\\hline
	N1	& install util	& It would be nice to have the application provide
				installable packages to the users.
		\\\hline
    \end{tabular}
    \captionof{table}{High level requirements for the design tool}
    \label{tab: appl-requirements}
\end{center}

\subsection{Team preferences}

%%%%% input from team competence tex file (a symbolic link) %%%%%
%%
%%  must be input: include is not permitted from an included file
%%
%%
%% This file contains the competence requirements for the team members.
%% It is part of other documents.
%%

The team has explicit wishes concerning improving or learning competences. Table
\ref{tab:team-competence-req} specifies these wishes per team member.

\begin{center}
    \begin{tabular}{cllp{17em}}
        \hline
        & {\bf Prio} & {\bf Competence}    & {\bf Goal}\\
        \hline
        \multicolumn{4}{c}{\sf\emph{Guus}}\\
        \hline
		GM0 & Must have    & Free software & Prefer to build free software.\\
        GM1 & Must have    & UI tool       & Learning to work with a cross platform UI tool like QT or GTK+.\\
        GS1 & Should have  & Agile		   & Learning to work in an agile environment.\\
        GS2 & Should have  & C++		   & Learning to program C++ (standard 2011).\\
	GC1 & Could have   & versioncontrol	   & Experiencing and learning teamwork with versioncontrol.\\
        GC2 & Could have   & Requirements Eng.   & Learning to apply Requirements engineering.\\
        \hline
        \multicolumn{4}{c}{\sf\emph{Stefan}}\\
        \hline
        SS1 & Should have  & Agile		   & Learning to work in agile environment, like DAD.\\
        SS2 & Should have  & Online Agile tools  & Learning to work agile tools.\\
        SS3 & Should have  & distributed team    & Learning to work in a distributed team.\\
        SC1 & Could have   & Non Microsoft tools & Learning to work with platform independent tools.\\
        \hline
        \multicolumn{4}{c}{\sf\emph{Jeroen}}\\
        \hline
                     &                     & \\
        \hline
    \end{tabular}
    \captionof{table}{Competence wishes of the team}
    \label{tab:team-competence-req}
\end{center}

%%%%%% input from team competence tex file (a symbolic link) %%%%%

\section{Requirements for UI toolkit}

Using the requirements for the application plus the competence requirements of
my team, the impact of all these demands is summarized in table \ref{tab:
uitool-requirements}. Only requirements directly relevant to the ui toolkit
have consequential requirements formulated.

In table \ref{tab: uitool-requirements} the term ``support for plugins'' is
meant to be a weak requirement in the sense that it does not hinder the design
of plugins. The ability to support plugins directly is a nice to have
requirement. The derived requirements ``ease of use'' and ``ease of learning''
are important requirements for the people maintaining the system.

Each requirement has a min-factor that reflects the type of requirement : 1.0
for a must-have requirements, 0.9 for should-have and 0.8 for a could-have
requirement. However, the factor could be less if the importance for the ui
tool is diminished.

The team requirements have a type downgraded by 0.1 indicating less importance
than the product requirements.

\paragraph{Remark.} The term ``ui toolkit'' may indicate one set of tools (like qt) or 
a multiple sets (like one for 2D and one for GUI elements).

\begin{center}
    \begin{longtable}{ll||ccp{13em}}
	{\bf id: requirement}     & {\bf impact} & {\bf id } & {\bf min-factor} & {\bf UI requirement}\\\hline\endhead
	\hline \multicolumn{4}{c}{UI toolkit requirements (Continue on next page)}\endfoot
	\hline\endlastfoot
	\hline
		M0: Free software    & full         &  0  & 1.0 & ui toolkit must be free software as defined by FSF.\\
        M1: Cross platform   & full         &  1  & 1.0 & ui toolkit runs equally well on the defined platforms for all features.\\
        M2: C++ integration  & full         &  2  & 0.7 & ui toolkit does not hinder C++ integration in any way.\\
        M3: 2D drawing       & full         &  3  & 1.0 & ui toolkit has equal 2D features on all defined platforms.\\
        M4: GUI Widgets      & full         &  4  & 1.0 & ui toolkit has relevant GUI widgets on all defined platforms.\\
        M5: xmas primitives  & none         &     &     & \\
        M6: xmas macros      & none         &     &     & \\
        S1: expandability    & limited      &  5  & 0.9 & ui toolkit does not hinder creating plugins.\\
        S2: installability   & limited      &  6  & 0.9 & ui toolkit does not hinder install procedure.\\
        S3: maintainability  & moderate     &  7  & 0.9 & ui toolkit is easy to use.\\
	                         &              &  8  & 0.9 & ui toolkit is easy to learn.\\
	                         &              &  9  & 0.9 & ui toolkit is well documented.\\
	                         &              & 10  & 0.9 & ui toolkit supports observer pattern.\\
	                         &              & 11  & 0.9 & ui toolkit dev. community should have enough mass.\\
	                         &              & 12  & 0.9 & ui toolkit user community should have enough mass.\\
        S4: plugability      & limited      & 13  & 0.9 & ui toolkit should not hinder creating plugins.\\
        S5: performance	     & moderate     & 14  & 0.9 & ui toolkit should support running concurrent processes for modules.\\
	                         &              & 15  & 0.9 & ui toolkit should support observer pattern from independent processes.\\
	                         &              & 16  & 0.9 & ui toolkit should support interruption during concurrent processing.\\
        N1: install util          & none         &     &     & \\\hline
        %--------------------------------------------------------------------------------------------------------------------------
		GM0: Free Software  &              &     &     & Enclosed in requirement M0.\\  
        GM1: UI tool	    &              &     &     & Enclosed in requirement M1.\\
        GS1: Agile	        &              &     &     & \\
        GS2: C++	        & compatible   & 20  & 0.8 & Prefer a ui toolkit that works with C++.\\
        GC1: versioncontrol &              &     &     & \\
        GC2: req. engineering
						    &              &     &     & \\\hline
        %--------------------------------------------------------------------------------------------------------------------------
        SS1: agile	        &              &     &     & \\
        SS2: online agile tools
							&              &     &     & \\
        SS3: distributed team     
							&              &     &     & \\
        SC1: platform independent 
							& compatible	 & 21  & 0.8 & Prefer a platform independent toolkit.\\
    \end{longtable}
    \captionof{table}{UI Toolkit requirements derived.}
    \label{tab: uitool-requirements}
\end{center}



\section{Selected tools versus requirements}

Based on language. C++. Free software.

\begin{center}
    \small\sf
    \begin{tabular}{lccccc|cccc}
	\hline
	{\bf Toolkit} & {\bf Free} & {\bf L-W-M} & {\bf C++}   & {\bf 2D} & {\bf GUI}   
													& \multicolumn{4}{c}{\em Yay or Nay}\\
		{\bf Name}    & {\bf 0} & {\bf 1}        & {\bf 2}     & {\bf 3}  & {\bf 4}     
													& {\bf Yay} & {\bf Part} & {\bf May} & {\bf Nay}\\
        \hline
%%%%%%%%%%%%%%%%%% FS  Plat   C++  2D  GUI  Comm  Yay   Part   May   Nay  
%%%%%%%%%%%%%%%%%%                                      May              
        Cairo     & 1 & 1   & 0.9 & 1 & 0 &  L  &     & Part &     &     \\
        FLTK      & 1 & 1   & 1   & 0 & 1 &     &     & Part &     &     \\
		fpGUI     & 1 & 1   & 0   &   &   &     &     &      &     & Nay \\
	    GTK+	  & 1 & 1   & 1   & 1 & 1 &  L  & Yay &      &     &     \\
	    JUCE      & 1 & 1   & 1   & 1 & 1 &  S  & Yay &      &     &     \\
	Mozilla A.F.  & 1 & 1   & 1   & 1 & 1 &  L  &     &      & May &     \\
	CEGUI         & 1 & 1   & 1   & 1 & 1 &  S  &     &      & May &     \\
	OpenGL        & 1 & 1   & 1   & 1 & 0 &  L  &     &      & May &     \\
	Qt	          & 1 & 1   & 1   & 1 & 1 &  L  & Yay &      &     &     \\
	Tk	          & 1 & 1   & 0.5 & 1 & 1 &  L  &     &      & May &     \\
	Ultimate++    & 1 & 0.6 & 0.7 & 1 & 1 &  M  &     &      &     & Nay \\
	WxWidgets     & 1 & 1   & 1   & 1 & 1 &  L  & Yay &      &     &     \\
	GDK	          & 1 & 1   & 1   & 1 & 0 &  L  &     &      &     & Nay \\
	SDL	          & 1 & 1   & 1   &   &   &     &     &      &     & Nay \\

	\hline
    \end{tabular}
    \captionof{table}{Selected tools against must have product/project requirements}
\end{center}

\paragraph{Cairo: Part} has a wrapper \verb!cairomm! for C++. So C++ integration
is not 100\%. Cairo is popular in the free software communicty for providing
cross-platform support for advanced 2D drawing. Other GUI library (notably
GTK+) already include Cairo.  This library does not contain GUI widgets and so
should be combined with \verb!SomeGUILibrary!. The verdict for Cairo is Part.

\paragraph{FLTK: Part} This is a GUI only library. For 2D a different should be
used and should be combined with \verb!Some2DLibrary!. The verdict is Part.

\paragraph{fpGUI: Nay} is a pascal library based on free pascal (lazarus)
(\cite{Geldenhuys:fpgui}).  The author is also the sole maintainer of the
system. According to \cite{wxwidget:comparison} Lazarus has no C++ integration
to speak of. For this reason we qualify this tool as Nay. 

\paragraph{GTK+: Yay} builds on Cairo and GDK, has a large developer and user
community.  It fullfills the must-haves and the verdict is Yay. It has a
wrapper GTKMM for C++. The verdict is Yay.

\paragraph{JUCE: Yay} A one man project with emphasis on audio
(\cite{juce:juce},\cite{wiki:juce}).  It runs all defined platforms, written in
C++. The user license is dual GPL and commercial.  The verdict according to
requirements is Yay.

\paragraph{Mozilla Application Framework: May} The MAF is platform independent,
very web oriented with a large developer and user community\footnote{All
firefox users are part of the user community}. The 2D drawing support is SVG
support. The MAF library contains much more than is necessary for this tool.
The question is whether this extra is justifiable overhead. The library is
meant to support a subset of the standard GUI frameworks like GTK+, QT and
WxWidgets, aimed at web-programs. According to the requirements the verdict is
a May.

\paragraph{CEGUI: May} This toolkit satisfies the main requirements, but has a
small developer community (\cite{wiki:cegui} and \cite{cegui:cegui}. Verdict is May.
 

\paragraph{OpenGL: May} hardware acceleration and powerful but complex. Low
level library. It does not directly support any GUI widgets. The verdict is
May.

\paragraph{Tk: May} Seems to contain everything necessary for both GUI and 2D
drawing, but C++ integration is not as clear (\cite{wiki:tk}, \cite{tcltk:tk}).
According to the comparison on WxWidgets for C++ better not use Tk
(\cite{wxwidget:comparison}).  The verdict for that reason is May.

\paragraph{Ultimate++: May} The framework seems very complete. It does not
support Mac (\cite{wxwidget:comparison}).  It is not very clear whether this
framework integrates easily into C++, see \cite{u++:ultimate++}. The verdict is
Nay.

\paragraph{WxWidgets: Yay} The framework is complete and has no obvious
disadvantages. According to their comparison to Qt, the two are functional
comparable (\cite{wxwidget:comparison}).  The verdict is Yay.

\paragraph{SDL: May} This is Low level library meant for 2D development. It
does not contain any GUI widgets. It may be usable in combination with other
libraries. The verdict is May.

\paragraph{GDK: Nay} Is a low level library that is ``An intermediate layer
which isolates GTK+ from the details of the windowing system.'' according to
\cite{gnome:gdk3}. In the presence of GTK+ the verdict is Nay.

\paragraph{QT: Yay} has a large developer and user community. According to the
wxwidget comparison with Qt, both are functionally comparable
(\cite{wxwidget:comparison}).


\section{List of cross platform tools}

General source: see \cite{wiki:xplatf}. Left out non-free software or software
not for defined platforms. Due to the amount of toolkits, any obvious disconnect with C++
is also reason to drop a choice\footnote{like Lazarus, that in theory might be able to
support the application, but has no direct integration with C++.}.

\begin{description}
    \item[Cairo] is a library used to provide a vector graphics-based,
		device-independent API for software developers. It is designed
		to provide primitives for 2-dimensional drawing across a number
		of different backends. Cairo is designed to use hardware
		acceleration when available.
		The software is free software with a user license based on GPL
		and MPL.
		\hspace*{\fill}\\Ref: \cite{wiki:cairo}.

    \item[FLTK] The Fast, Light Toolkit (FLTK, pronounced fulltick) is a
		cross-platform graphical control element (GUI)
		library developed by Bill Spitzak and others. Made to
		accommodate 3D graphics programming, it has an interface to
		OpenGL, but it is also suitable for general GUI programming.
		\hspace*{\fill}\\Ref: \cite{wiki:fltk}.

    \item[fpGUI] the Free Pascal GUI toolkit, is a cross-platform
		graphical user interface toolkit developed by Graeme Geldenhuys.
		fpGUI is open source and free software, licensed under a Modified LGPL
		license. The toolkit has been implemented using the Free Pascal
		compiler, meaning it is written in the Object Pascal language.
		fpGUI consists only of graphical widgets or components, and a
		cross-platform 2D drawing library.
		fpGUI is statically linked into programs and is licensed using a
		modified version of LGPL specially designed to allow static linking to
		proprietary programs. The only code you need to make available are
		any changes you made to the fpGUI toolkit - nothing more.
		\hspace*{\fill}\\Ref: \cite{wiki:fpgui}.

    \item[GTK+] (previously GIMP Toolkit, sometimes incorrectly referred to
		as the GNOME Toolkit) is a cross-platform widget toolkit for
		creating graphical user interfaces. It is licensed under the terms
		of the GNU LGPL, allowing both free and proprietary software to use
		it. It is one of the most popular toolkits for the Wayland and
		X11 windowing systems, along with Qt.
		\hspace*{\fill}\\Ref: \cite{wiki:gtk+}.

    \item[JUCE] is a free software, cross-platform C++ application framework, used
		for the development of GUI applications and plug-ins.
		The aim of JUCE is to allow software to be written such that
		the same source code will compile and run identically on Windows,
		Mac OS X and Linux platforms. It supports various development
		environments and compilers, such as GCC, Xcode and Visual Studio.
		\hspace*{\fill}\\Ref: \cite{wiki:juce}.

    \item[Mozilla Application Framework]
		 is a collection of cross-platform software components that
		 make up the Mozilla applications. It was originally known as
		 XPFE, an abbreviation of cross-platform front end.
		 While similar to generic cross-platform application
		 frameworks like GTK+, Qt and wxWidgets, the intent is to
		 provide a subset of cross-platform functionality suitable for
		 building network applications like web browsers, leveraging
		 the cross-platform functionality already built into the Gecko
		 layout engine.
		\hspace*{\fill}\\Ref: \cite{wiki:mozilla_application_framework}.

    \item[OpenGL] is a cross-language, multi-platform application programming
		interface (API) for rendering 2D and 3D vector graphics.
		The API is typically used to interact with a graphics
		processing unit (GPU), to achieve hardware-accelerated
		rendering\footnote{OpenGL is a standard that has libraries based on free
		software and commercial libraries. Programmers never need a license
		to use an OpenGL library.}.
		\hspace*{\fill}\\Ref: \cite{wiki:opengl}.

    \item[Qt]  (/ˈkjuːt/ "cute", or unofficially as Q-T cue-tee) is a cross-platform
		application framework that is widely used for developing application
		software that can be run on various software and hardware platforms
		with little or no change in the codebase, while having the power and
		speed of native applications. Qt is currently being developed both
		by the Qt Company, a subsidiary of Digia, and the Qt Project under
		open-source governance, involving individual developers and firms
		working to advance Qt.

		Digia owns the Qt trademark and copyright. Qt is available with
		both proprietary and open source GPL v3 and LGPL v2 licenses.
		\hspace*{\fill}\\Ref: \cite{wiki:qt}.

    \item[Simple DirectMedia Layer] is a cross-platform software development
		library designed to provide a low level hardware abstraction
		layer to computer hardware components. Software developers
		can use it to write high-performance computer games and other
		multimedia applications that can run on many operating systems
		such as Android, iOS, Linux, Mac OS X, Windows and other platforms.

		SDL manages video, audio, input devices, CD-ROM, threads, shared
		object loading, networking and timers.[5] For 3D graphics it can
		handle an OpenGL or Direct3D context.

		The library is internally written in C and also provides the
		application programming interface in C, with bindings to other
		languages available.[6] It is free and open-source software subject
		to the requirements of the zlib License since version 2.0 and with
		prior versions subject to the GNU Lesser General Public License.
		Because of zlib SDL 2.0 is freely available for static linking
		in commercial closed-source projects, unlike SDL 1.2. SDL is
		extensively used in the industry in both large and small projects.
		Over 700 games, 180 applications, and 120 demos have also been
		posted on the library website.

		It is often believed that SDL is a game engine, but this is not
		true. However, the library is well-suited for building an engine
		on top of it.
		\hspace*{\fill}\\Ref: \cite{wiki:sdl}.
    \item[Tk]	is a free and open-source, cross-platform widget toolkit that
		provides a library of basic elements of GUI widgets for building
		a graphical user interface (GUI) in many different programming
		languages.
		\hspace*{\fill}\\Ref: \cite{wiki:tk}.
    \item[Ultimate++] is a C++ cross-platform development framework which aims
		to reduce the code complexity of typical desktop applications
		by extensively exploiting C++ features.
		\hspace*{\fill}\\Ref: \cite{wiki:ultimate++}.

    \item[WxWidgets] (formerly wxWindows) is a widget toolkit and tools library
		for creating graphical user interfaces (GUIs) for cross-platform
		applications. wxWidgets enables a program's GUI code to compile
		and run on several computer platforms with minimal or no code
		changes. It covers systems such as Microsoft Windows, OS X
		(Carbon and Cocoa), iOS (Cocoa Touch), Linux/Unix (X11, Motif,
		and GTK+), OpenVMS, OS/2 and AmigaOS. A version for embedded
		systems is under development.
		\hspace*{\fill}\\Ref: \cite{wiki:wxwidget}
    \item[GDK]  (GIMP Drawing Kit) is a library that acts as a wrapper around
		the low-level functions provided by the underlying windowing
		and graphics systems. GDK lies between the display server and
		the GTK+ library, handling basic rendering such as drawing
		primitives, raster graphics (bitmaps), cursors, fonts, as well
		as window events and drag-and-drop functionality.

		Like GTK+, GDK is licensed under the GNU Lesser General
		Public License (LGPL).
		\hspace*{\fill}\\Ref: \cite{wiki:gdk}
\end{description}

Not considered are the following platforms:

\begin{center}
    \begin{tabular}{|p{7em}|p{25em}|}
	\hline
	{\bf name} & {\bf reason}\\\hline
	AppearIQ & non-free software. Ref: \cite{appear:appeariq}.\\
	Eclipse & a compiler, not a graphical framework. Ref: \cite{wiki:eclipse}.\\
	GeneXus & non-free software. Also not for linux
		    or mac. Ref: \cite{wiki:genexus}. \\
	Haxe & not a graphics platform. Ref: \cite{wiki:haxe}\\
	Max & not for linux, not free software. Ref: \cite{wiki:max}\\
	Mono & not a graphics environment, emulation
		of C\#. Ref: \cite{wiki:mono}.\\
	MonoCross & aimed at C\#. Ref: \cite{wiki:monocross}.\\
	MoSync & aimed at mobile platforms, no longer maintained. Ref: \cite{wiki:mosync}.\\
	Xojo & non-free software. Ref: \cite{wiki:xojo}.\\
	Smartface & non-free software\footnote{limited edition gratis available}. Ref: \cite{wiki:smartface} and \cite{smartface:license}.\\
	WebDev & aimed at creating websites. Ref: \cite{wiki:webdev}.\\
	WinDev & aimed at data centric apps with forms, works with webdev. Ref: \cite{wiki:windev}.\\
	XPower++ & Insufficient information, looks like an advertisment. Ref: \cite{wiki:xpower++}.\\
	Lazarus & A Pascal development environment. Ref: \cite{wiki:lazarus}.\\
	\hline
    \end{tabular}
    \captionof{table}{Platforms not considered with reason}
\end{center}


\section{Comparison of selected toolkits}
\section{Conclusion of comparison}
\section{Recommendation}
\section{Motivation}

\appendix

\bibliography{da-uitoolkits.bib}
\section{Definitions}
\begin{description}
% \item[analyse module] synoniem met verificatie module
% \item[i18n] Internationalization, referring to translation of menu items, system documentation etc.
% \item[l10n] Localization, referring to country specific settings such as money, numbers, dates etc.
  \item[GPL] Gnu Public License
  \item[LGPL] Lesser Gnu Public License
  \item[MPL] Mozilla Public License
  \item[UI] User Interface
  \item[GUI] Graphical User Interface
  \item[the defined platforms] for the design application Linux, Mac and MS Windows are the
  defined platforms.

\end{description}

\end{document} ;########################### end document ##################################;
