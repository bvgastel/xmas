%%
%% Dit is het hoofddocument: compileer dit met latex of xelatex en je krijgt de gehele pdf
%%
\documentclass[a4paper,11pt,draft]{article}
%
%% option = draft will generate black except with \color, replace images by a rectangle
%% option = final will generate color
%

\usepackage{color}
\usepackage{bold-extra}
\usepackage{footmisc}% granting the ability to use label for a footnote
\usepackage{subfig}
\usepackage{wrapfig}% product wrapfigure and wraptable
\usepackage{array}% additions to tabular
%\usepackage{supertabular}% multiple pages tabular
\usepackage{longtable}% multiple pages table like tabular
\usepackage{rotating}% for the environment sidewaysfigure / sidewaystable
\usepackage[english]{babel}
\usepackage{graphicx}
\usepackage{hyperref}% Load after biblatex
\hypersetup{
    colorlinks = true,
    citecolor = blue,
    linkcolor = blue
}
\usepackage[prependcaption,colorinlistoftodos,obeyFinal,textsize=tiny]{todonotes}% when generating final (documentclass option) skip notes
\usepackage{pdflscape}
\usepackage[a4paper]{geometry}
\usepackage{titlesec}% added to change section headers, see newcommand definition.
\usepackage{boxedminipage}
\usepackage{amssymb}% For \checkmark
\usepackage{pifont}% for \ding{'-code or "-code}

\bibliographystyle{plain}% unsrt, plain, alpha, abbrv

\newcommand{\biburl}[1]{\hspace*{\fill}\\\url{#1} accessed oct/nov 2014}

\author{Guus Bonnema\\member of ABI team 33}
\date{21/10/2014}

\title{\color{blue}Domain analysis for a user interface toolkit}

\setlength\extrarowheight{2pt}% Adds a little space at the top of table rows

%% Document is in subdocumenten gesplitst.

\begin{document}

\selectlanguage{english}
\hyphenation{func-tio-nal}

%%%%%%%%%%%%%%%%%%%%%%%%%%%%%%%%%%%%
\newcommand{\xmas}{x\textsc{mas}}%
\newcommand{\ok}{$\checkmark$}
\newcommand{\w}[1]{\textbf{\textsc{#1}}}
\newcommand\bw[1]{{\color{blue}#1}}

\newcommand{\mybox}[1]{\begin{boxedminipage}[t]{\textwidth}#1\end{boxedminipage}}

%\definecolor{airforceblue}{rgb}{0.36, 0.54, 0.66}%%   This is color in hex #5D8AA8

%%%%%%%%%%%%%%%%%%%%%%%%%%%%%%%%%%%% different section format start %%%%%%%%%%%%%%%%%%%%%%%%%%%%%%%%%
%\newcommand\secformat[1]{%
%    {\fontsize{60}{60}\selectfont\thesection}%
%    \ifthenelse{\equal{\thesection}{}}{}{\quad\rule[-8pt]{2pt}{40pt}\quad}
%    \parbox[b]{.7\textwidth}{\filright\bfseries #1}}%
%\titleformat{\section}[block]
%    {\filright\normalfont\sffamily}{}{0pt}{\secformat}
%\titlespacing*{\section}{0pt}{*3}{*2}[1pc]
%%%%%%%%%%%%%%%%%%%%%%%%%%%%%%%%%%%% different section format end   %%%%%%%%%%%%%%%%%%%%%%%%%%%%%%%%%

\newcommand\smp[1]{%
	\marginpar{\color{blue}\small\bf\textsc#1}
}%
\newcommand\smpp[1]{\smp{#1}#1}


\maketitle

\begin{abstract}

	\noindent{} This document reflects a \bw{domain analysis} for user
	interface tools. It looks for available tools that are free to use, will
	remain available and can do the job required for the design tool of an
	\xmas{} network.  It describes the user interface tool kits found, the
	search approach, the criteria derived from requirements and the team
	preferences, all used in selecting the UI toolkits. The result is a
	motivated recommendation for the project of team 33 to choose a user
	interface toolkit. 
	
	\paragraph{} The document presents the results from abstract to detailed to
	facilitate the reader in understanding the document. This presents
	conclusions first, progressively providing more details. Reading up to and
	including section \ref{sec:phase3listing} leads to full, high level
	understanding.

\end{abstract}

%\listoftodos   %% commented out when creating final document
\newpage
\tableofcontents

\section{Problem domain}
\subsection{Description}

The customer's request is to build a graphical chip editor for Network on chip
designs that uses existing C++ program to interface with and that runs on all
defined platforms: Linux, ms Windows and Mac. This implies part of the solution
is a user interface tool that works from graphical representation of chip
networks. This document is the result of gathering information and analysing
existing user interface tools that we could use for this purpose. Addiditonally
the customer has specified some qualitative requirements like maintainability
and installability.

\subsection{Approach}

The primary goal is that the UI tool suffices: it does not have to be perfect.
All being equal the team will choose from the candidates on the basis of
preference.

The number of available user interface toolkits is large.  Moreover the number
of features each toolkit has, is large. This warrants careful consideration
of toolkits where we start from the project's requirements and use the team's 
preference to break any ties.

\subsection{Search}

Normally ones searches literature for information on subjects in domain
analysis.  For toolkits this may not be the most efficient or in fact the most
effective way of finding the best toolkit. The problem with literature is that
it is constantly behind on the current, actual state of affairs. Even though
information on the internet in general is less reliable than literature
generally is, in this case it forms a better basis to find a more complete list
of toolkits. 

\section{Selection criteria}

Table (\todo{cite}) contains a list of the main requirements for the design tool.
These requirements are not directly applicable, but do generate criteria which we 
can use to find the most appropriate toolkits available. In order to do the team 
justice, any ties\footnote{toolkits with equal qualification as far as requirements are concerned} need to be broken using the team members preferences. Table \todo{cite} 
shows a list of the team's preferences.

Table \todo{cite} shows the implications of the requirements resulting in \todo{X} 
candidates. 
of the requirements.


\section{Search approach}

\subsection{Search approach motivation}

Normally one searches literature for information on science subjects.  For tool
kits this may not be the most efficient or effective way of finding current
tool kits. The problem with literature is that it is constantly behind on the
current state of affairs. Even though information on the internet is generally
less reliable than literature is, in this case it forms a better basis to find
a more complete list of tool kits. Literature does supplement what we find on
the internet.

This analysis is able to determine a list of user interface tool kits starting
from en.widipedia and fanning out through the links wikipedia provides and
searching with a search engine. The dates on the sites provide information on
how actual the data is. Cross referencing sites gives confidence in the
correctness of the information. The next section lays out the criteria we used
to select tool kits.

\subsection{Selection criteria}

Table \ref{tab:appl-requirements} on page~\pageref{tab:appl-requirements}
contains a list of the main requirements for the design tool. These
requirements generate criteria that we can use to find the most appropriate
tool kits available.  Table \ref{tab:team-competence-req} on
page~\pageref{tab:team-competence-req} shows a list of the team's preferences.
Table \ref{tab:uitool-requirements} on page~\pageref{tab:uitool-requirements}
shows the implications of the requirements resulting in specific user interface
requirements. The next section shows what steps we followed and what the
results were.

\subsection{Selection process}

\paragraph{First step} Table \ref{tab:discarded-toolkits} on
page~\pageref{tab:discarded-toolkits} shows the first wave of packages that we
discarded with motivation. 

\paragraph{Second step} Table \ref{tab:compare-main-req} on
page~\pageref{tab:compare-main-req} shows the result of the second wave of
selection. This table shows for each tool kit to what degree it satisfies the
first set of requirements. 

\paragraph{Third step} Table \ref{tab:final-comparison} on
page~\pageref{tab:final-comparison} shows the final comparison. 

\section{Search and Selection}

\subsection{Search approach motivation}

Normally one searches literature for information on science subjects.  For tool
kits this may not be the most efficient or effective way of finding current
tool kits. The problem with literature is that it is constantly behind on the
current state of affairs. Even though information on the internet is generally
less reliable than literature is, in this case it forms a better basis to find
a more complete list of tool kits. Literature does supplement what we find on
the internet.

This analysis is able to determine a list of user interface tool kits starting
from en.widipedia and fanning out through the links wikipedia provides and a
search engine. The dates on the sites provide information on how actual the
data is. Cross referencing sites gives confidence in the correctness of the
information. The next section lays out the criteria we used to select tool
kits.

\subsection{Selection criteria}

Table \ref{tab:appl-requirements} on page~\pageref{tab:appl-requirements}
contains a list of the main requirements for the design tool. These
requirements generate criteria that we can use to find the most appropriate
tool kits available.  Table \ref{tab:team-competence-req} on
page~\pageref{tab:team-competence-req} shows a list of the team's preferences.
Table \ref{tab:uitool-requirements} on page~\pageref{tab:uitool-requirements}
shows the implications of the requirements resulting in specific user interface
requirements. The next section shows what steps I followed and what the
results were.

\subsection{Phase 1: Primary deselection}

The first phase deselects tool kits that do not suffice one
of the important requirements.  Table \ref{tab:discarded-toolkits} on
page~\pageref{tab:discarded-toolkits} shows the tool kits that were deselected
and why.

\subsection{Phase 2: Secondary deselection}

The remaining tool kits were examined in more detail. Each one of these tool
kits finds a summarized description in appendix~\ref{sec:phase2-listing}.  This
results in rating the tool kits according to the main characteristics required
in Table~\ref{tab:compare-main-req} on page~\pageref{tab:compare-main-req}.

Analysis of the results leads to accept the first four for definite 
comparison in the last phase. These tool kits have all necessary
characteristics required. The next three tool kits represent
2D tool kits that could be used in conjunction with the first four.

\subsection{Phase 3: Final selection}

This phase concerns the tool kits : \w{gtk+}, \w{qt}, \w{wxwidgets} and
\w{fltk}. See appendix~\ref{sec:phase2-listing} for a summary of each tool kit
and appendix~\ref{sec:final-listing} for a more detailed description.  Each of
these tool kits satisfies all requirements either autonomously or in
combination with other tool kits. In terms of requirement based selection, the
search is finished.

The next section documents the considertions and the resulting recommendation.


\section{Result and recommendation}
\label{sec:recommendation}

\subsection{Considerations and Motivation}

Considering the goals of the project, and the competence wishes, the team has
different options as a result of different viewpoints.  One option would be to
go with a toolkit that is ``\texttt{mean and lean}'' and add libraries only
where necessary. The other option would be to go with a tool kit that contains
``all++'' and thus is comfortable allbeit a little bloated. Whatever the
choice, it will affect the end product. 

In my recommendation I go with the tool kit \w{fltk} (pronounce
\emph{fulltick}) . The tool kit is targetted at workstations and embedded
hardware. The focus is on being \texttt{mean and lean} and fast and small,
as advertised on the \w{fltk} web site and user forum.

The recommendation is based on this information (also see appendix
\ref{sec:final-listing}) and the personal assessment of the project in next
paragraph. Of course the final decision is up to the team.

The project emphasizes a Network on Chip design, with most user interface
effort going into the 2D drawing interface and the general user interface.
This makes the focus of \w{fltk} on user interface with respect to gui a good
choice. Should we need more emphasis on looks or more complex drawing than
\w{fltk} can provide, then we could substitute \w{OpenGL} or
\w{Cairo}\footnote{Integration of \w{OpenGL} or \w{Cairo} into \w{fltk} should
be trivial}.

\par The argument is solid, but with a shift of viewpoints a different choice
could be valid.  The team should think about what is important in this project
and decide on that basis. See appendix \ref{sec:final-listing} for a summary of
main characteristics of the selected tool kits.

\subsection{Recommendation} 

I recommend using \w{fltk}.



\appendix
\section{Phase 3 Summary of applicable tool kits}

In order to support the final decision, in the following sections I summarize
the main characteristics of the tool kits plus I specify reasons why to choose
or not to choose that specific tool kit.

\subsection{\w{qt}} 

\w{qt} is a complete development environment based on an extension of C++ with
a \w{qt} pre-compiler. It supports many features among which are 2D,
concurrency and memory management. It also has many dependencies as shown in
\cite{qt:qt-dep}.

Main characteristic of \w{qt} is that it is uses a pre-compiler called
\texttt{moc} that extends standard C++ with facilities needed for signal and
slots. The owner is a company \w{Digia} that licenses the \w{qt} toolkit both
as free software and with a commercial license.

The tool kit \w{qt} supports the observer pattern with signals and slots.
Defining an independent observer and observable should also be possible.

\paragraph{Why to choose \w{qt}?} Because \w{qt} runs equally well on many
platforms, including netbooks, and other mobile computers, and because it has
many well known applications running on many platforms. Appealing is also its
simple signal and slot feature and support for it's user interface building is
renowned.  Finally, \w{qt} has an active community and very clear documentation
both online and in books. 

\paragraph{Why not to choose \w{qt}?} Because \w{qt} extends C++, uses macro's
heavily and thus needs an extra pre-processor to build the application. Also,
it provides a lot more than we need for our application.

\subsection{\w{gtkmm}}

\w{gtkmm} combines a graphical user interface with many non gui features like
signals and memory management. For 2D it builds on cairo or opengl. The library
for gtkmm has some dependencies on other libraries that have their own
dependencies as shown in \cite{gtkmm:gtk+-dep}.

Main characteristic of \w{gtkmm} is that it is primarily linux oriented and is
the main development environment for the \w{Gnome} window manager. It uses
standard C++ and additionally collections and other features from \w{Glib}.

The tool kit \w{gtkmm} supports the observer pattern using signals. Defining an
independent observer and observable should also be possible.

\paragraph{Why choose \w{gtkmm}?} Because \w{gtkmm} is C++ even if it is a
wrapper: it does not use macro's heavily. Because it supports cairo natively.
Finally \w{gtkmm} has an active community and clear documentation both online
and in books\footnote{\w{qt} has more literature than \w{gtk+}}.

\paragraph{Why not to choose \w{gtkmm}?} Because \w{gtkmm} is linux oriented
(developments starts from the \w{Gnome} window manager). The primary aim is
unix, not windows or mac.  Also, it provides a lot more than we need for our
application.

\subsection{\w{wxwidget}} 

\w{wxwidget} is a complete cross platform development environment. It is built
as multiple libraries by default. The features cover support for openGL and
concurrent execution. The \w{wxwidget} depends on different libraries for each
platform.

Main characteristic of \w{wxwidget} is that it has a separate underlying
interface to each platform. The result is an application with the look and feel
of the platform. The disadvantage is that applications run differently on
different platforms and may show platform dependent errors. Also, the binaries
of the application are usually bigger than for the other tool kits.

The tool kit \w{wxwidget} does not seem to support the observer pattern
directly, but some examples of how to define one are available like
\cite{wxwidget:observer-example}\footnote{The surrounding text is in Russion,
but the example is clear.}.

\paragraph{Why choose \w{wxwidget}?} Because it is cross platform. It is well
known and used in many applications.
 
\paragraph{Why not to choose \w{wxwidget}?} It produces overweight applications
that could make the application feel bloated.

\subsection{\w{fltk}}

\w{fltk} is a graphical user interface with 2D drawing and nothing more. For
concurrency we look to another cross platform tool kit, like \w{boost}. The
tool kit \w{fltk} depends on the window manager X11 or Wayland or Win32.

Main characteristic of \w{fltk} is that it has no additional facilities.  The
motto "Do one thing and do it right" is one that unix programmers often adhere
to. The binaries for programs using this tool kit are small in comparison to
other tool kits. 

The tool kit \w{fltk} does not support an observer pattern, but one can define
one as stated on the forum at
url{http://groups.google.com/forum/\#!forum/fltkgeneral}.

\paragraph{Why choose \w{fltk}?} Because it is focussed on graphical user
interface.  It does one thing and it does it well. It is easy to learn. It is
easy to combine with other libraries like \w{OpenGL}, \w{Cairo} or \w{boost}.
The community is active and willing to support users of their
library\footnote{I found their support to be both fast and objective (in an
\w{fltk} biased way of course).}. 

\paragraph{Why not to choose \w{fltk}?} Because it does not offer more than
graphical user interface. Also, the code shows it's age in using c-strings in
the API.


\section{Phase 3 Final selection}

\subsection{Phase 3 selection results} 
\label{sec:phase3appendix}

Table \ref{tab:uitool-requirements} shows the tool kit requirements.  The final
comparison takes the remaining tool kit requirements into account.

\vspace{1em}
\begin{minipage}{.95\textwidth}
	\begin{center}
		\small\sf
		\begin{tabular}{|c|p{9em}|p{8em}|cccc|}
			\hline
			{\bf id} & {\bf req}             & {\bf fitness}           & \w{gtkmm} & \w{qt} & \w{wx} & \w{fltk}\\
			\hline
			    7    & Ease of use           & reputation              & +                      & + & + & +\\
			    8    & Ease of learning      & rep \& tutorials        & +                      & + & + & +\\
			    9    & Documentation         & website \& rep          & +                      & + & + & + \\
			    10   & Observer pattern      & signal processing       & +                      & + & + & 
			    +\footnote{using standard C++ or another library like boost\label{fn:c++}} \\
			    14   & Concurrency           & support for concurrency & +                      & + & + & +\footref{fn:c++} \\
			    15   & Concurrent observer   & is it possible?         & +                      & + & + & +\footref{fn:c++} \\
				\hline
		\end{tabular}
		\captionof{table}{Final comparison of selected tool kits}
		\label{tab:final-comparison}
	\end{center}
\end{minipage}

\paragraph{Use and learning.} We can only measure the requirements for ease of
use and learning subjectively. This is an experience result and changes in
time: any tool kit becomes easy to use and learn after enough experience using
it. So the only measure that we could use is reputation, plus tutorials for
ease of learning. 

\paragraph{Use of reputation.} Although subjective, the cumulative criticism
and opinions indicate real quality. Neither of the systems have bad rep on any
of the subjectively measurable qualities (7, 8 and 9). Also, the main website
for the tool kit easily revealed the available documentation. The assumption
that any one of these tool kits lacks in ease of use, learning or in
documentation can be rejected on the basis of experience and reputation. 

\paragraph{Observer pattern.} The observer pattern is important in two ways.
First, all signals from the user interface should relay flawlessly to modules,
even if they execute concurrently. Secondly, any change of status or content in
the modules, even concurrent modules should relay flawlessly to the user
interface thread. Measuring this is a question of checking the docs and asking
around. 


\subsection{Phase 3 selection analysis}

Table \ref{tab:final-comparison} shows that tool kit requirements 7, 8 and 9
are satisfactory for all. The observer pattern and concurrent process
communication is not as simple. In summary, \w{qt}, \w{gtkmm} and \w{WxWdiget}
all suffice without modification. For \w{fltk} we need to use one of the
available cross platform libraries for concurrency. It turns out to be easy
to define ones own observer pattern, or use a library for this like \w{boost}.

As all packages can fulfil the requirements the question is, what distinguishes
these tool kits and on what basis should we choose? The short answer is: they
all suffice. So the choice is no longer one of user interface requirements, but
one of preference. 


\section{Phase 2 Main Selection}

\subsection{Phase 2 selection results}

Table \ref{tab:compare-main-req} shows the remaining tool kits. The column {\sc
Community} indicates the expected support for the tool kit. Among the tool kits
qualified as \w{Maybe} are some unconventional, but powerful frameworks.  The
tool kits qualified as \w{Partial} leave out either the GUI widgets or the 2D
drawing capability. They could still be viable in combination with one of the
other tool kits. In summary, this analysis only considers the categories
\w{Yay} or \w{Partial}.

\begin{center}
    \small\sf
    \begin{tabular}{c|lccc|c|ccc}
	\hline
	\w{nr} & \w{Toolkit} & \w{C++} & \w{2D} & \w{GUI} &    & \multicolumn{3}{c}{\em\bf\sf Yay or Nay} \\
           & {\bf\sf Name}&        &        &         & \w{Community}& \w{Yay} & \w{Part} & \w{Nay} \\
           &	          &        &        &         &              &         & {\bf\sf Maybe}  &  \\
        \hline
%%%%%%%%%%%%%%%%%%  	  C++  2D  GUI  Comm  Yay   Part   Nay
%%%%%%%%%%%%%%%%%%                                  Maybe
1  &	GTK+		& 1   & 1 & 1 &  Active  & Yay &         &     \\
2  &	Qt	      	& 1   & 1 & 1 &  Active  & Yay &         &     \\
3  &	WxWidgets 	& 1   & 1 & 1 &  Active  & Yay &         &     \\
4  &	FLTK      	& 1   & 1 & 1 &  Active  & Yay &         &     \\
\hline
8  &	Cairo     	& 1   & 1 & 0 &  Large  &     & Partial &     \\
9  &	OpenGL 	  	& 1   & 1 & 0 &  Large  &     & Partial &     \\
10 &	SDL			& 1   & 1 & 0 &         &     & Partial &     \\\hline
\hline
5  &	JUCE      	& 1   & 1 & 1 &  Small  &     & Maybe   &     \\
6  &	CEGUI     	& 1   & 1 & 1 &  Active &     & Maybe   &     \\
7  &	Gled		& 1   & 1 & 1 &  Small  &     & Maybe   &     \\
11 &	Mozilla A.F.  	& 1   & 1 & 1 &  Large  &     & Web-oriented   &     \\\hline
12 &	Tk	        & 0.5 & 1 & 1 &  Large  &     &         & Nay \\
13 &	fpGUI     	& 0   &   &   &         &     &         & Nay \\
14 &	GDK       	& 1   & 1 & 0 &  Large  &     &         & Nay \\
\hline
    \end{tabular}
    \captionof{table}{Selected tools against must have product/project requirements}
	\label{tab:compare-main-req}
\end{center}

\subsection{Phase 2 selection analysis}

\paragraph{Tool kits included} The tool kits 1 through 4 satisfy the
requirements.  

The tool kits 8 and 9 satisfy all but one requirement (the GUI) but could still
function if they add sufficient function to a GUI oriented toolkit. 

\paragraph{Tool kits excluded} Tool kit 10 ({\sf SDL}) is not clear on the
supporting community, but does seem to have a following.  The comparison from
\w{WxWidgets} (\cite{wxwidget:comparison}) mentions {\sf SDL} as a viable
addition to \w{WxWidgets}, but oriented towards gaming. The tool kits 12
through 13 do not satisfy the requirements fully, and tool kit 14 (GDK) is low
level and incorporated in {\sf GTK+}.  The Mozilla Application Framework ({\sf
M.A.F.}) is oriented towards the web. The deciding feature is that {\sf MAF}
contains a sub selection of the regular GUI widgets geared towards html and
css.

For \w{fpGUI}, \w{sdl} and \w{Gled} the community activity is not clear from
the home website, so the community is left out in table. The system \w{juce} is
a one man project and aimed at multimedia and gaming.  \w{cegui} is a small
community of developers specifically aimed at gaming, not graphical user
interface.  \w{Gled} is a framework that stems from scientific work and
features the easy distribution of nodes across threads, processes and machines.
See \cite{greed:gled} for a discussion of the scientific goals of the \w{Greed}
project.

\paragraph{Next phase} will contain toolkits $1, 2, 3$ and 4: {\sf GTK+, Qt,
WxWidgets, FLTK} with possible additions of OpenGL or Cairo. 


\section{Requirements for design tool}

Table \ref{tab:appl-requirements} summarizes the high level requirements for
the chip design tool. These are the main base for selection of toolkits. Each 
requirement may transfer as full requirement to the UI tool, partially or 
not at all. 
Table \ref{tab:team-competence-req} shows the team members' requirements in 
terms of competence goals. 
Table \ref{tab:uitool-requirements} summarizes the relationship 
of the project's requirements with the user interface toolkit.

If multiple tools satisfy the complete set of requirements, then
team preference will rule the recommendation. Ultimately the customer
and the team decide depending on the recommendation and any external factors
like an other domain analysis.

\subsection{Justification of free software requirement}

The first requirement $M0$ was specified to ensure the free availability of the
ui tool both now and in the future. The source of
this requirement is a remark from the Open University instructions
``Hulpmiddelen en bronnen'' (``tools and sources'') with the following text:

\vspace{1em}

\begin{tabular}[t]{ll}
\begin{minipage}{.45\textwidth}

	``Wij gaan ervan uit dat de producten van het team uiteindelijk veelal als
	open source beschikbaar gesteld worden, dus iedereen mag in principe de
	producten inzien. In de overeenkomst met de opdrachtgever dient op het punt
	van de rechten duidelijkheid gegeven te worden.''

\end{minipage}

&

\begin{minipage}{.45\textwidth}

	``We assume open source availability of the team's products so that the
	products are publicly available. The contract with the customer should be
	clear on this point.''

\end{minipage}

\end{tabular}

\vspace{1em}

\noindent Demanding free software follows from this specification and the fact
that non-free software could potentially lead to non-availability of the
software due to license fees or the author retracting the software. We specify
free software as a must-have requirement to be ``clear on this point''.

\subsection{The project and product requirements}

\begin{center}
    \small\sf
    \begin{tabular}{|l|p{7em}|p{23em}|}
	\hline
	        & \multicolumn{2}{c|}{\sf\em\large requirement class}\\\hline
	{}      & \multicolumn{1}{c|}{\bf req name} & \multicolumn{1}{c|}{\bf req desc}\\\hline
	M	& \multicolumn{2}{c|}{\sf\emph{\large Must have requirements}}
		\\\hline
	M0	& Free software  & The application is free software as defined by the FSF.
		\\\hline
	M1	& Cross platform & The application must run equally well on the
				    defined platforms with respect to all relevant features.
				    Linux, Mac and MS Windows are the defined platforms.
		\\\hline
	M2	& C++ integration & The application must integrate seamlessly with
				the existing C++ code for analysis and verification tools.
		\\\hline
	M3	& 2D drawing & The application must be able to draw items on a
				canvas and treat graphical objects with composite
				behaviour or properties as a primitive object.
		\\\hline
	M4	& GUI widgets & The application must be able to create GUI widgets
				on all defined platforms.
		\\\hline
	M5      & xmas primitives & The application must be able to draw the 8 xmas
				    primitives.
		\\\hline
	M6	& xmas macros     & The application must be able to define and draw
				    a macro of xmas primitives.
		\\\hline
	S	& \multicolumn{2}{c|}{\sf\emph{\large Should have requirements}}
		\\\hline
	S1	& expandability & the application should be expandable in terms of
				analysis and verification modules, xmas-primitives and
				network reporting.
		\\\hline
	S2	& installability & The application should be easy to install.
		\\\hline
	S3	& maintainability & The application should be as easily maintainable as possible.
		\\\hline
	S4	& plugability	& The application should use plugins for analysis and
				verification modules.
		\\\hline
	S5	& performance	& The application should be performant enough for the GUI
				    interface to be ``snappy''
		\\\hline
	C	& \multicolumn{2}{c|}{\sf\emph{\large Could have requirements}}
		\\\hline
	N	& \multicolumn{2}{c|}{\sf\emph{\large Nice to have requirements}}
		\\\hline
	N1	& install util	& It would be nice to have the application provide
				installable packages to the users.
		\\\hline
    \end{tabular}
    \captionof{table}{High level requirements for the design tool}
    \label{tab:appl-requirements}
\end{center}

\subsection{Team preferences}

%%%%% input from team competence tex file (a symbolic link) %%%%%
%%
%%  must be input: include is not permitted from an included file
%%
%%
%% This file contains the competence requirements for the team members.
%% It is part of other documents.
%%

The team has explicit wishes concerning improving or learning competences. Table
\ref{tab:team-competence-req} specifies these wishes per team member.

\begin{center}
    \begin{tabular}{cllp{17em}}
        \hline
        & {\bf Prio} & {\bf Competence}    & {\bf Goal}\\
        \hline
        \multicolumn{4}{c}{\sf\emph{Guus}}\\
        \hline
		GM0 & Must have    & Free software & Prefer to build free software.\\
        GM1 & Must have    & UI tool       & Learning to work with a cross platform UI tool like QT or GTK+.\\
        GS1 & Should have  & Agile		   & Learning to work in an agile environment.\\
        GS2 & Should have  & C++		   & Learning to program C++ (standard 2011).\\
	GC1 & Could have   & versioncontrol	   & Experiencing and learning teamwork with versioncontrol.\\
        GC2 & Could have   & Requirements Eng.   & Learning to apply Requirements engineering.\\
        \hline
        \multicolumn{4}{c}{\sf\emph{Stefan}}\\
        \hline
        SS1 & Should have  & Agile		   & Learning to work in agile environment, like DAD.\\
        SS2 & Should have  & Online Agile tools  & Learning to work agile tools.\\
        SS3 & Should have  & distributed team    & Learning to work in a distributed team.\\
        SC1 & Could have   & Non Microsoft tools & Learning to work with platform independent tools.\\
        \hline
        \multicolumn{4}{c}{\sf\emph{Jeroen}}\\
        \hline
                     &                     & \\
        \hline
    \end{tabular}
    \captionof{table}{Competence wishes of the team}
    \label{tab:team-competence-req}
\end{center}

%%%%%% input from team competence tex file (a symbolic link) %%%%%

\section{Requirements for UI toolkit}

Using the requirements for the application plus the competence requirements of
my team, the impact of all these demands is summarized in table
\ref{tab:uitool-requirements}. Only requirements directly relevant to the ui
toolkit have consequential requirements formulated.

In table \ref{tab:uitool-requirements} the term ``support for plugins'' is
meant to be a weak requirement in the sense that it does not hinder the design
of plugins. The ability to support plugins directly is a nice to have
requirement. The derived requirements ``ease of use'' and ``ease of learning''
are important requirements for the people maintaining the system.

Each requirement has a min-factor that reflects the type of requirement : 1.0
for a must-have requirements, 0.9 for should-have and 0.8 for a could-have
requirement. However, the factor could be less if the importance for the ui
tool is diminished.

The team requirements have a type downgraded by 0.1 indicating less importance
than the product requirements.

\paragraph{Remark.} The term ``ui toolkit'' may indicate one set of tools (like qt) or
a multiple sets (like one for 2D and one for GUI elements).

\begin{center}
    \begin{longtable}{ll||cp{13em}}
	{\bf id: requirement}     & {\bf impact  } & {\bf id } & {\bf UI requirement}\\\hline\endhead
	\hline \multicolumn{4}{c}{UI toolkit requirements (Continue on next page)}\endfoot
	\hline\endlastfoot
	\hline
		M0: Free software    & full         &  0  & ui toolkit must be free software as defined by FSF.\\
        M1: Cross platform   & full         &  1  & ui toolkit runs equally well on the defined platforms for all features.\\
        M2: C++ integration  & full         &  2  & ui toolkit does not hinder C++ integration in any way.\\
        M3: 2D drawing       & full         &  3  & ui toolkit has equal 2D features on all defined platforms.\\
        M4: GUI Widgets      & full         &  4  & ui toolkit has relevant GUI widgets on all defined platforms.\\
        M5: xmas primitives  & none         &     & \\
        M6: xmas macros      & none         &     & \\
        S1: expandability    & limited      &  5  & ui toolkit does not hinder creating plugins.\\
        S2: installability   & limited      &  6  & ui toolkit does not hinder install procedure.\\
        S3: maintainability  & moderate     &  7  & ui toolkit is easy to use.\\
	                         &              &  8  & ui toolkit is easy to learn.\\
	                         &              &  9  & ui toolkit is well documented.\\
	                         &              & 10  & ui toolkit supports observer pattern.\\
	                         &              & 11  & ui toolkit dev. community should have enough mass.\\
	                         &              & 12  & ui toolkit user community should have enough mass.\\
        S4: plugability      & limited      & 13  & ui toolkit should not hinder creating plugins.\\
        S5: performance	     & moderate     & 14  & ui toolkit should support running concurrent processes for modules.\\
	                         &              & 15  & ui toolkit should support observer pattern from independent processes.\\
	                         &              & 16  & ui toolkit should support interruption during concurrent processing.\\
        N1: install util     & none    		&     & \\\hline
        %--------------------------------------------------------------------------------------------------------------------------
		GM0: Free Software  &              &     & Enclosed in requirement M0.\\
        GM1: UI tool	    &              &     & Enclosed in requirement M1.\\
        GS1: Agile	        &              &     & \\
        GS2: C++	        & compatible   & 20  & Prefer a ui toolkit that works with C++.\\
        GC1: versioncontrol &              &     & \\
        GC2: req. engineering
						    &              &     & \\\hline
        %--------------------------------------------------------------------------------------------------------------------------
        SS1: agile	        &              &     & \\
        SS2: online agile tools
							&              &     & \\
        SS3: distributed team
							&              &     & \\
        SC1: platform independent
							& compatible   & 21  & Prefer a platform independent toolkit.\\
    \end{longtable}
    \captionof{table}{UI Toolkit requirements derived.}
    \label{tab:uitool-requirements}
\end{center}

\section{Phase 0 Requirement Impact}

\subsection{Phase 0 selection results}

Using the requirements for the application plus the competence requirements of
the team, the impact of all these demands is summarized in table
\ref{tab:uitool-requirements}. Only requirements directly relevant to the ui
toolkit have consequential impact formulated.

In table \ref{tab:uitool-requirements} the term ``support for plugins'' is
meant to be a weak requirement in the sense that it does not hinder the design
of plugins. The ability to support plugins directly is a nice to have
requirement. The derived requirements ``ease of use'' and ``ease of learning''
are important requirements for the people maintaining the system.

\paragraph{Remark.} The term ``ui toolkit'' may indicate one set of tools (like
qt) or a multiple sets (like one for 2D and one for GUI elements).

\begin{center}
    \begin{longtable}{ll||cp{21em}}
	{\bf id: requirement}     & {\bf impact  } & {\bf id } & {\bf UI requirement}\\\hline\endhead
	\hline \multicolumn{4}{c}{UI toolkit requirements (Continue on next page)}\endfoot
	\hline\endlastfoot
	\hline
		M0: Free software    & full         &  0  & ui toolkit must be free software as defined by FSF.\\
        M1: Cross platform   & full         &  1  & ui toolkit runs equally well on the defined platforms for all features.\\
        M2: C++ integration  & full         &  2  & ui toolkit does not hinder C++ integration in any way.\\
        M3: 2D drawing       & full         &  3  & ui toolkit has equal 2D features on all defined platforms.\\
        M4: GUI Widgets      & full         &  4  & ui toolkit has relevant GUI widgets on all defined platforms.\\
        M5: xmas primitives  & none         &     & \\
        M6: xmas macros      & none         &     & \\
        S1: expandability    & limited      &  5  & ui toolkit does not hinder creating plugins.\\
        S2: installability   & limited      &  6  & ui toolkit does not hinder install procedure.\\
        S3: maintainability  & moderate     &  7  & ui toolkit is easy to use.\\
	                         &              &  8  & ui toolkit is easy to learn.\\
	                         &              &  9  & ui toolkit is well documented.\\
	                         &              & 10  & ui toolkit supports observer pattern.\\
	                         &              & 11  & ui toolkit dev. community should have enough mass
	                         									and be active in order to inspire confidence 
	                         									in the tool kits future.\\
	                         &              & 12  & ui toolkit user community should have enough mass.
	                         									and be active in order to inspire confidence 
	                         									in the tool kits future.\\
        S4: plugability      & limited      & 13  & ui toolkit should not hinder creating plugins.\\
        S5: performance	     & moderate     & 14  & ui toolkit should support running concurrent processes for modules.\\
	                         &              & 15  & ui toolkit should support observer pattern from independent processes.\\
	                         &              & 16  & ui toolkit should support interruption during concurrent processing.\\
        N1: install util     & none    		&     & \\\hline
        %--------------------------------------------------------------------------------------------------------------------------
		GM0: Free Software  &              &     & Enclosed in requirement M0.\\
        GM1: UI tool	    &              &     & Enclosed in requirement M1.\\
        GS1: Agile	        &              &     & \\
        GS2: C++	        & compatible   & 20  & Prefer a ui toolkit that works with C++.\\
        GC1: versioncontrol &              &     & \\
        GC2: req. engineering
						    &              &     & \\\hline
        %--------------------------------------------------------------------------------------------------------------------------
        SS1: agile	        &              &     & \\
        SS2: online agile tools
							&              &     & \\
        SS3: distributed team
							&              &     & \\
        SC1: platform independent
							& compatible   & 21  & Prefer a platform independent toolkit.\\\hline
        %--------------------------------------------------------------------------------------------------------------------------
		JS1: project        &              &     & \\
        JS2: C++	        & compatible   & 20  & Prefer a ui toolkit that works with C++.\\
		JS3: TDD			&              &     & \\
    \end{longtable}
    \captionof{table}{UI Toolkit requirements derived.}
    \label{tab:uitool-requirements}
\end{center}

\section{Phase 1 Preselection}

The amount of tool kits available is staggering, so the first step is weeding
out the tool kits that do not support one of our defined platforms, are not
free software of just do not interact with C++ sufficiently.  For that reason
this analysis does not consider the following tool kits:

\begin{center}
    \begin{tabular}{|p{7em}|p{25em}|}
	\hline
	{\bf name} & {\bf reason}\\\hline
	AppearIQ & non-free software. Ref: \cite{appear:appeariq}.\\
	Eclipse & a compiler, not a graphical framework. Ref: \cite{wiki:eclipse}.\\
	GeneXus & non-free software. Also not for linux
		    or mac. Ref: \cite{wiki:genexus}. \\
	Haxe & not a graphics platform. Ref: \cite{wiki:haxe}\\
	Max & not for linux, not free software. Ref: \cite{wiki:max}\\
	Mono & not a graphics environment, emulation
		of C\#. Ref: \cite{wiki:mono}.\\
	MonoCross & aimed at C\#. Ref: \cite{wiki:monocross}.\\
	MoSync & aimed at mobile platforms, no longer maintained. Ref: \cite{wiki:mosync}.\\
	Xojo & non-free software. Ref: \cite{wiki:xojo}.\\
	Smartface & non-free software\footnote{limited edition gratis available}. Ref: \cite{wiki:smartface} and \cite{smartface:license}.\\
	WebDev & aimed at creating websites. Ref: \cite{wiki:webdev}.\\
	WinDev & aimed at data centric apps with forms, works with webdev. Ref: \cite{wiki:windev}.\\
	XPower++ & Insufficient information, looks like an advertisment. Ref: \cite{wiki:xpower++}.\\
	Lazarus & A Pascal development environment. Ref: \cite{wiki:lazarus}.\\
	Ultimate++ & Does not support MacOS. Ref: \cite{wiki:ultimate++} and \cite{wxwidget:comparison}.\\
	\hline
    \end{tabular}
    \captionof{table}{Platforms not considered with reason}
	\label{tab:discarded-toolkits}
\end{center}



\section{Phase 2 List of cross platform tool kits}
\label{sec:phase2-listing}

General source: see \cite{wiki:xplatf}. Left out non-free software or software
not for defined platforms. Due to the amount of toolkits, any obvious
disconnect with C++ is also reason to drop a choice\footnote{like Lazarus, that
in theory might be able to support the application, but has no direct
integration with C++.}.

\begin{description}
    \item[Cairo] is a library used to provide a vector graphics-based,
		device-independent API for software developers. It is designed
		to provide primitives for 2-dimensional drawing across a number
		of different backends. Cairo is designed to use hardware
		acceleration when available.
		The software is free software with a user license based on GPL
		and MPL.
		\hspace*{\fill}\\Ref: \cite{wiki:cairo}.

		Cairo has a wrapper \verb!cairomm! for C++.  Cairo is popular in the
		free software communicty for providing cross-platform support for
		advanced 2D drawing. Other GUI library (notably GTK+) already include
		Cairo. This library does not contain GUI widgets and so should be
		combined with \verb!SomeGUILibrary!. The verdict for Cairo is Partial.

    \item[FLTK] The Fast, Light Toolkit (FLTK, pronounced fulltick) is a
		cross-platform graphical control element (GUI)
		library developed by Bill Spitzak and others. Made to
		accommodate 3D graphics programming, it has an interface to
		OpenGL, but it is also suitable for general GUI programming.
		\hspace*{\fill}\\Ref: \cite{wiki:fltk}.

		FLTK is a GUI only library. For 2D a different library should be used.
		The toolkit is not included in the final comparison.

    \item[fpGUI] the Free Pascal GUI toolkit, is a cross-platform
		graphical user interface toolkit developed by Graeme Geldenhuys.
		fpGUI is open source and free software, licensed under a Modified LGPL
		license. The toolkit has been implemented using the Free Pascal
		compiler, meaning it is written in the Object Pascal language.
		fpGUI consists only of graphical widgets or components, and a
		cross-platform 2D drawing library.
		fpGUI is statically linked into programs and is licensed using a
		modified version of LGPL specially designed to allow static linking to
		proprietary programs. The only code you need to make available are
		any changes you made to the fpGUI toolkit - nothing more.
		\hspace*{\fill}\\Ref: \cite{wiki:fpgui}.

		fpGUI is a pascal library based on free pascal (lazarus).  The author
		is also the sole maintainer of the system. According to the WxWidget
		comparison Lazarus has no C++ integration to speak of. For this reason
		this toolkit is not included in the final comparison.
		\hspace*{\fill}\\Ref: \cite{Geldenhuys:fpgui} and
		\cite{wxwidget:comparison}.

	\item[GTK+] (previously GIMP Toolkit, sometimes incorrectly referred to as
		the GNOME Toolkit) is a cross-platform widget toolkit for creating
		graphical user interfaces. It is licensed under the terms of the GNU
		LGPL, allowing both free and proprietary software to use it. It is one
		of the most popular toolkits for the Wayland and X11 windowing systems,
		along with Qt.  \hspace*{\fill}\\Ref: \cite{wiki:gtk+}.

		GTK+ builds on Cairo and GDK and has a large developer and user
		community.  It has a wrapper for C++ (\verb!GTKMM!). It fulfills the
		main requirements and is included in the final comparison.
		\hspace*{\fill}\\Ref: \cite{gtkmm:gtk+}.

    \item[JUCE] is a free software, cross-platform C++ application framework, used
		for the development of GUI applications and plug-ins.
		The aim of JUCE is to allow software to be written such that
		the same source code will compile and run identically on Windows,
		Mac OS X and Linux platforms. It supports various development
		environments and compilers, such as GCC, Xcode and Visual Studio.
		\hspace*{\fill}\\Ref: \cite{wiki:juce}.

		A one man project with emphasis on audio.  It runs all defined
		platforms, written in C++. The user license is dual GPL and commercial.
		The verdict according to requirements is Maybe due to the size of the
		developer community. It is not included in the final comparison.

	\item[CEGUI] Crazy Eddie's GUI System is a free library providing windowing
		and widgets for graphics APIs / engines where such functionality is not
		natively available, or severely lacking. The library is
		object-oriented, written in C++, and targeted at game and application
		developers who should be spending their time creating great games and
		not on building GUI sub-systems!
		\hspace*{\fill}\\Ref: \cite{cegui:getting-started}.

		This toolkit satisfies the main requirements, but has a small developer
		community (\cite{wiki:cegui} and \cite{cegui:cegui}. The verdict for
		this toolkit is Maybe due to the size of the developer community. It is
		not included in the final comparison.

    \item[Mozilla Application Framework]
		 is a collection of cross-platform software components that
		 make up the Mozilla applications. It was originally known as
		 XPFE, an abbreviation of cross-platform front end.
		 While similar to generic cross-platform application
		 frameworks like GTK+, Qt and wxWidgets, the intent is to
		 provide a subset of cross-platform functionality suitable for
		 building network applications like web browsers, leveraging
		 the cross-platform functionality already built into the Gecko
		 layout engine.
		 \hspace*{\fill}\\Ref: \cite{wiki:mozilla_application_framework}.

		 The MAF is platform independent, web oriented with a large developer
		 and user community\footnote{All firefox users are part of the user
		 community}.  The library is meant to support a subset of the standard
		 GUI frameworks like GTK+, QT and WxWidgets, aimed at web-programs. Due
		 to the web orientedness and it's partial support for widgets this
		 toolkit is not selected for the final comparison.

    \item[OpenGL] is a cross-language, multi-platform application programming
		interface (API) for rendering 2D and 3D vector graphics.
		The API is typically used to interact with a graphics
		processing unit (GPU), to achieve hardware-accelerated
		rendering\footnote{OpenGL is a standard that has libraries based on free
		software and commercial libraries. Programmers never need a license
		to use an OpenGL library.}.
		\hspace*{\fill}\\Ref: \cite{wiki:opengl}.

		OpenGL gives hardware acceleration and is a powerful but complex 3D
		user interface toolkit. It is also a low level library and does not
		directly support any GUI widgets. Many GUI libraries support the use of
		an OpenGL library. This toolkit is not selected for final comparison.

    \item[Qt]  (/ˈkjuːt/ "cute", or unofficially as Q-T cue-tee) is a cross-platform
		application framework that is widely used for developing application
		software that can be run on various software and hardware platforms
		with little or no change in the codebase, while having the power and
		speed of native applications. Qt is currently being developed both
		by the Qt Company, a subsidiary of Digia, and the Qt Project under
		open-source governance, involving individual developers and firms
		working to advance Qt.

		Digia owns the Qt trademark and copyright. Qt is available with
		both proprietary and open source GPL v3 and LGPL v2 licenses.
		\hspace*{\fill}\\Ref: \cite{wiki:qt}.

		Qt has a large developer and user community. According
		to the WxWidget comparison with Qt, both are functionally comparable.
		\hspace*{\fill}\\Ref: \cite{wxwidget:comparison}.

    \item[Simple DirectMedia Layer] is a cross-platform software development
		library designed to provide a low level hardware abstraction
		layer to computer hardware components. Software developers
		can use it to write high-performance computer games and other
		multimedia applications that can run on many operating systems
		such as Android, iOS, Linux, Mac OS X, Windows and other platforms.

		SDL manages video, audio, input devices, CD-ROM, threads, shared
		object loading, networking and timers.[5] For 3D graphics it can
		handle an OpenGL or Direct3D context.

		The library is internally written in C and also provides the
		application programming interface in C, with bindings to other
		languages available. It is free and open-source software subject
		to the requirements of the zlib License since version 2.0 and with
		prior versions subject to the GNU Lesser General Public License.
		Because of zlib SDL 2.0 is freely available for static linking
		in commercial closed-source projects, unlike SDL 1.2. SDL is
		extensively used in the industry in both large and small projects.
		Over 700 games, 180 applications, and 120 demos have also been
		posted on the library website.

		It is often believed that SDL is a game engine, but this is not
		true. However, the library is well-suited for building an engine
		on top of it.
		\hspace*{\fill}\\Ref: \cite{wiki:sdl}.

		SDL is a low level library meant for 2D game development. It does not
		contain GUI widgets. It may be usable in combination with other
		libraries. It is not selected for final comparison.

	\item[Tk] is a free and open-source, cross-platform widget toolkit that
		provides a library of basic elements of GUI widgets for building a
		graphical user interface (GUI) in many different programming languages.
		\hspace*{\fill}\\Ref: \cite{wiki:tk}.

		Tk seems to contain everything necessary for both GUI and 2D drawing,
		but C++ integration is not clear (\cite{wiki:tk}, \cite{tcltk:tk}).
		According to the comparison on WxWidgets, for C++ it is better not to
		use Tk.  \hspace*{\fill}\\Ref: \cite{wxwidget:comparison}. 

    \item[Ultimate++] is a C++ cross-platform development framework which aims
		to reduce the code complexity of typical desktop applications
		by extensively exploiting C++ features.
		\hspace*{\fill}\\Ref: \cite{wiki:ultimate++}.

    \item[WxWidgets] (formerly wxWindows) is a widget toolkit and tools library
		for creating graphical user interfaces (GUIs) for cross-platform
		applications. wxWidgets enables a program's GUI code to compile
		and run on several computer platforms with minimal or no code
		changes. It covers systems such as Microsoft Windows, OS X
		(Carbon and Cocoa), iOS (Cocoa Touch), Linux/Unix (X11, Motif,
		and GTK+), OpenVMS, OS/2 and AmigaOS. A version for embedded
		systems is under development.
		\hspace*{\fill}\\Ref: \cite{wiki:wxwidget}

		The framework is complete and has no obvious disadvantages. According
		to their comparison to Qt, the two are functional comparable.
		\hspace*{\fill}\\Ref: \cite{wxwidget:comparison}.  

	\item[Gled]	is a C++ framework for rapid development of applications with
		support for GUI (using FLTK), 3D-graphics and distributed
		computing. It extends the ROOT framework (standard data-analysis
		tool in high-energy physics) with mechanisms for object collection
		management \& serialization, multi-threaded execution, GUI
		auto-generation (object browser \& editor) and dynamic visualization
		(OpenGL). Distributed computing model of Gled is a hierarchy of
		nodes connected via TCP/IP sockets. Gled provides authentication \&
		access control, data exchange, proxying of object collections and
		remote method-call propagation \& execution. Gled can be dynamically
		extended with library sets. Their creation is facilitated by a set
		of scripts for creation of user-code stubs. Simple tasks and
		application configuration can be efficiently done via the
		interactive C++ interpreter (CINT). Gled is used for development of
		programs in high energy physics and as a research tool in
		distributed and grid computing. 
		\hspace*{\fill}\\Ref: \cite{fltk:gled}

		This toolkit satisfies the main requirements, but has a small developer
		community (\cite{gled:gled}). It is not included in the final
		comparison.

    \item[GDK]  (GIMP Drawing Kit) is a library that acts as a wrapper around
		the low-level functions provided by the underlying windowing
		and graphics systems. GDK lies between the display server and
		the GTK+ library, handling basic rendering such as drawing
		primitives, raster graphics (bitmaps), cursors, fonts, as well
		as window events and drag-and-drop functionality.

		Like GTK+, GDK is licensed under the GNU Lesser General
		Public License (LGPL).
		\hspace*{\fill}\\Ref: \cite{wiki:gdk}

		GDK is a low level library that is ``An intermediate layer which
		isolates GTK+ from the details of the windowing system.''.  In the
		presence of GTK+ the toolkit is not included in the final comparison.
		\hspace*{\fill}\\Ref: \cite{gnome:gdk3}.

\end{description}



\bibliography{da-uitoolkits.bib}
\section{Definitions}
\begin{description}
% \item[analyse module] synoniem met verificatie module
% \item[i18n] Internationalization, referring to translation of menu items, system documentation etc.
% \item[l10n] Localization, referring to country specific settings such as money, numbers, dates etc.
  \item[GPL] Gnu Public License
  \item[LGPL] Lesser Gnu Public License
  \item[MPL] Mozilla Public License
  \item[UI] User Interface
  \item[GUI] Graphical User Interface
  \item[the defined platforms] for the design application Linux, Mac and MS Windows are the
  defined platforms.

\end{description}

\end{document} ;########################### end document ##################################;
