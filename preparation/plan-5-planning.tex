%%
%% Dit is een subdocument van het projectplan.
%%
%% items op basis van SE Sommerville p623 (chapter 23.2.1 Project plans)
%%  	sommige items zijn elders in het document reeds opgenomen , zoals bvb risico's
%%

 
%%\subsection{Risico analyse}
%% Een van de risico's is dat de tool, tijdens een analyse of check, een xmas netwerk ten onrechte als fout vrij verklaren. Als gebruikers daaruit beslissen dat het chip ontwerp goed is en dat de chip geproduceerd mag worden, is dit een probleem.


\subsection{Werk indeling}

\begin{enumerate}
\item \underline{\textbf{Inceptie}}
		\begin{itemize}
			\item Meeting met stakeholders.
			\item Het probleem definiëren.
			\item High level requirements verzamelen.
			\item Architectuur oplossingen zoeken.
			\item Planning. 
		\end{itemize}
		\paragraph{Artefacten}
		Deze fase levert een document dat het probleem omschrijft , de high-level requirements, business case ,risiso'c , stakeholders, vision and challenges, mogelijke opties voor de architectuur en een plan.
	
\item \underline{\textbf{Elaboratie en domeinonderzoek}}
	\begin{itemize}
		\item Domein onderzoek (3 onderdelen).
		\item Context onderzoek.
		\item Requirements : Demonstratie van de huidige WickedXmas tool door stakeholder.
		\item Requirements : Op basis van sourcecode van de huidige WickedXmas tool.
		\item Meeting met stakeholders.
	\end{itemize}
	\paragraph{Artefacten}
	Document wordt uitgebreid met de resultaten van het domein onderzoek, eisen uit de observatie, architectuur keuzes en kennis.
	Het team heeft de IDE met tools geïnstalleerd en de skills bijgesteld om  te kunnen starten.   
	
\item \underline{\textbf{Constructie}}

		\begin{itemize}
			\item Iteration 1 : WickedXmas editor en documentatie
			\item Iteration 2 : WickedXmas editor en documentatie
			\item Iteration 3 : WickedXmas editor en documentatie
			\item Iteration 4 : WickedXmas analyse integratie en documentatie
			\item Iteration 5 : WickedXmas analyse integratie en documentatie
		\end{itemize}
		
		\paragraph{Artefacten}
		Ekle iteratie levert een prototype op dat ter evaluatie aan de opdrachtgever aangeboden wordt. Er wordt telkens dat deel van de documentatie
	
	
\item \underline{\textbf{Afronden}}
	\begin{itemize}
		\item Release van de nieuwe WickedXmas tool.
		\item Documentatie van de tool (handleiding).
		\item Presentatie
	\end{itemize}
	\paragraph{Artefacten}
	Een presentatie , documentatie en de nieuwe WickedXmas tool.

\end{enumerate}








