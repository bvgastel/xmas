%%
%% Dit is een subdocument van het projectplan.
%%

\section{Architectuur}

De architectuur levert een gestructureerde indeling van het systeem met als doel om enerzijds de software
toegankelijk te maken voor ontwikkelaars en anderzijds input te leveren voor aanpassingen in de software. Hieronder
een specificatie van de producten die deze activiteit oplevert.

{\tiny
\begin{center}
\begin{tabular}{lp{30em}}
Logical view & De logische structuur in class en object diagrammen.\\
Process view & De dynamische structuur in state
transition diagrams op systeem niveau.\\
Physical view &  De hardware componenten en de verdeling van
software over de hardware componenten.\\
Guidelines \& constraints & De richtlijnen voor bouw,
test, herstructurering en documentatie.\\
\end{tabular}
\end{center}
}

De architectuur wordt uitgewerkt in iteratie 0, voordat de constructie van de software begint. In de latere iteraties
zullen de bovengenoemde producten nog worden verfijnd maar in hoofdlijnen zal de architectuur na iteratie 0 zijn vastgelegd.