%%
%% Dit is een subdocument van het projectplan.
%%

\paragraph{Werkwijze}

%% Deze stappen kloppen niet meer. De eerste paar stappen horen bij 1 stap: Preparation & planning.

\begin{enumerate}
 \item Planning en aanpak
 \item Business case
 \item Risico analyse
 \item Vision en challanges
 \item Requirements
 \item Domeinanalyse
 \item Architectuur (incl GUI framework)
 \item Uitvoering (3 weekse iteraties)
 \begin{enumerate}
  \item a. iteratie 1 -- hoogste risico dempen
  \item b. iteratie 2 -- v1
  \item c. iteratie 3 -- v2
  \item d. iteratie 4 -- v3
  \item e. iteratie 5 -- v4
 \end{enumerate}
\item Afronding en presentatie
\end{enumerate}

\paragraph{De basis principes}
\begin{itemize}
 \item De uitvoering is agile en tijdgedreven.
 \item Elke iteratie heeft een vaste hoeveelheid tijd
 \item Elke iteratie begint met een analyse (requirements engineering, prioriteiten en selectie). Soms wijzigen requirements of komen er nieuwe bij.
 \item Elke iteratie eindigt met een evaluatie
 \item Dit moeten we valideren bij Bernard en Freek. Gebruikers betrokkenheid is
 beperkt tot het begin van de iteratie, de prioriteitsstelling van requirements, bijstelling van
 requirements, nieuwe requirements en uiteindelijk de selectie van wat we gaan bouwen.
 \item Peer review is belangrijk onderdeel van ons proces
 \item Refactoring is standaard onderdeel van process
\end{itemize}

\subsection{Agile aanpak}

\begin{itemize}
 \item Gebruiken git voor versiebeheer van documenten en software
 \item Passen TDD toe (binnen redelijke)
 \item Passen refactoring toe
 \item Elke iteratie:
 \begin{itemize}
   \item Analyse (requirements aanpassen, prioriteitsstelling, selectie)
   \item Ontwikkeling prototype
   \item Evaluatie
 \end{itemize}
\end{itemize}
