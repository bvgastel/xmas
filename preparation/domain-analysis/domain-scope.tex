%%
%% Dit is het hoofddocument: compileer dit met latex of xelatex en je krijgt de gehele pdf
%%
\documentclass[a4paper,11pt]{article}

\usepackage{subfig}
\usepackage{wrapfig}% product wrapfigure and wraptable
\usepackage[dutch]{babel}
\usepackage{graphicx}
\usepackage[colorlinks]{hyperref}
\usepackage[prependcaption,colorinlistoftodos,obeyFinal]{todonotes}% when generating final (documentclass option) skip notes
\usepackage{pgfgantt}
\usepackage{pdflscape}
\usepackage[a4paper]{geometry}

\bibliographystyle{alpha}

\usepackage{hyperref}

\author{Guus Bonnema, Stefan Versluys, Jeroen Kleijn}
\date{24/09/14}

\title{Goals and results of domain analysis}

%% Document is in subdocumenten gesplitst.

\begin{document}

\selectlanguage{dutch}

\newcommand{\xmas}{x\textsc{mas}}%

\maketitle

\section{Introduction}
This document specified the chosen domain analyses that we will perform before starting the development phases
of this project.

\section{Scope definition of toolkit domain analysis}
Student: Guus Bonnema.
\subsection{Requirements for domain analysis}
\begin{itemize}
    \item At least an initial idea of the high level view of the system should be available. The final high level
	    view will depend on the choices made here and for the toolkits.
    \item The high level requirements must be defined and agreed on.
\end{itemize}
\subsection{Goal of analysis}
Find the toolkit combination that optimally satisfies the high level requirements as specified in the project plan.
\subsection{Subjects in scope}
high level architecture. User interface toolkits. IPC toolkits. Concurrency toolkits. Integration toolkits.
Influence on development environment.
\subsection{Subjects out of scope}
database toolkits.
\subsection{Overlapping subjects}
Overlap with integration as many toolkits have approaches to integrating components. Also the constraints that follow
from the domain analysis on integration will influence the resulting decision.
\subsection{Relevance to the project}
The domain analysis results in advice on use of toolkits and on time necessary to master the toolkits.

\section{Scope definition of combinatorial objects}
Student: Stefan Versluys.
\subsection{Requirements for domain analysis}
\begin{itemize}
    \item At least an initial idea of the high level view of the system should be available. The final high level
	    view will depend on the choices made here and for the toolkits.
    \item The high level requirements must be defined and agreed on.
\end{itemize}
\subsection{Goal of analysis}
Find a way of integrating the analysis tools in the design tool, in such a way
that the high level requirements are optimally satisfied. Describe both the functions
and the dynamics of this integration.
\subsection{Subjects in scope}
Integration of tools in the design tool. Types of tools necessary.
high level architecture. Constraints on use of toolkits. Influence on development environment and subsequent constraints.
\subsection{Subjects out of scope}
Specific toolkits. Development environment.
\subsection{Overlapping subjects}
This domain analysis overlaps with toolkit domain analysis. The constraints that follow from this analysis
will influence the choice of toolkit.
\subsection{Relevance to the project}
This analysis results in an advice that will directly influence the applications architecture, its development environment
and the use of toolkits.

\section{Scope definition of combinatoric cycle checker}
Student: Jeroen Kleijn.
\subsection{Requirements for domain analysis}
\begin{itemize}
    \item At least an initial idea of the high level view of the system should be available. The final high level
	    view will depend on the choices made here and for the toolkits.
    \item The high level requirements must be defined and agreed on.
\end{itemize}
\subsection{Goal of analysis}
Integrate the combinatoric cycle checker with the design tool. Determine the specific interface. Determine the common interface
with other tools.
\subsection{Subjects in scope}
Combinatoric cycle checker. Integration.
\subsection{Subjects out of scope}
The design tool.
\subsection{Overlapping subjects}
The choices made in the integration analysis have implementation consequences for this domain analysis and vice versa.
\subsection{Relevance to the project}
This tool is not yet integrated into the design tool.
\note[inline]{is this true?}

\end{document}