\documentclass[a4paper,10pt]{article}
\usepackage[utf8]{inputenc}

%opening
\title{Parkeer document}
\author{Guus Bonnema, Stefan Versluys en Jeroen Kleijn}

\begin{document}

\maketitle

\begin{abstract}

\end{abstract}

\section{Requirements}

\begin{description}
 \item[platform 001] Het ontwerptool moet draaien onder zowel ms Windows, Mac en Linux.
 \item[platform 002] De GUI toolkit moet platform onafhankelijk werken zowel met de
 grafische user interface als met de achterliggende processing en de integratie daarmee.
 \item[platform 003] De documentatie moet op alle platformen te genereren zijn.
 \item[integratie 001] De architectuur van de applicatie moet onderhoudbaarbaar zijn
 \item[integratie 002] De integratie van het ontwerp tool met de analyse tools moet onafhankelijk zijn (kleine koppeling)
 \item[integratie 003] De integratie van het ontwerp tool met de analyse tools moet hecht zijn
 vanuit het perspectief van gebruik.
 \item[integratie 004] Het ontwerptool moet JSON kunnen interpreteren.
 \item[onderhoud 001] De integratie moet gemakkelijk wijzigbaar zijn: voldoende documentatie, goed gestructureerd
 \item[functie 001] Het ontwerp tool ondersteunt het modelleren op basis van de acht primitieven uit het xmas paper \cite{chatterjee-kishinevsky:xmas}.
 \item[functie 002] Het ontwerp tool ondersteunt het controleren van de modellen o.b.v. de analyse tools.
 \item[uitbreiding 001] Het ontwerp tool ondersteunt het toevoegen van een nieuw analyse tool.
 \item[uitbreiding 002] Het ontwerp tool ondersteunt uitbreiding o.b.v. een plug in
\end{description}


\end{document}
