\documentclass[a4paper]{article}


\usepackage[dutch]{babel}
\usepackage[utf8]{inputenc}
\usepackage{amsmath}
\usepackage{graphicx}
\usepackage[colorinlistoftodos]{todonotes}

\title{Ideeën verdere aanpak}

\author{Stefan Versluys}

\date 23 september 2014

\begin{document}
\maketitle


\section{Introductie}

Betreft enkele ideeën om tot een plan van aanpak te komen


\section{wat weten we al ?}

\begin{enumerate}
\item WickedxMas moet Herschreven worden in C++
\item Code moet gestructureerd zijn en zodoende beter onderhoudbaar
\item Documentatie
\item De huidige GUI mag herbruikt worden, functionaliteit OK
\item Analyse d.m.v. algoritmen zoals deadlock check moeten liefst geïntegreerd worden
\item Platform onafhankelijk
\item Qt Creator als IDE en GUI toolkit
\item Git als versie beheer
\item \LaTeX\ documentering
\item Agile ontwikkelmethode
\item Collaborate of Teamviewer als alternatief voor skype
\end{enumerate}

\section{Onbekenden of risico's}

\begin{enumerate}
\item Stakeholders?
\item Source code oude tool?
\item De grafische voorstelling van een xMas netwerk moet dit in C++ herschreven worden of is dit een kant en klare herbruikbare component --> tamelijk complex , verbindingen hebben autorouting!
\item Heeft de analyse tool zoals deadlock reeds een interface, API? 
\item Andere Analyse tools beschikbaar? Onder welke vorm , component? source code?
\item Hoe gaat men tewerk met de WickedxMas tool, demo door Bernard? 
\item sommige analyses zoals liviness vergen extreem veel CPU rekenwerk , soms dagen!
\item de xMas taal is al een vrij grote abstractie van chips of netwerken , heeft het zin om het domein te verkennen voorbij de abstractie, m.a.w. levert ons dat iets op voor WickedxMas?
\item men heeft het over efficientie , slaat dit op analyse zo ja, algoritmen vallen buiten de scope
\end{enumerate}


\section{Haalbare scope (volgens mij)}

\begin{enumerate}
\item Huidige tool , met dezelfde functionaliteit , in C++ ,onderhoudbaar, gestructureerd en gedocumenteerd
\item Interface bedenken en voorzien om analyse tools makkelijk te kunnen integreren in de WickedxMas tool , start/stop , visualisatie/export van de resultaten.
\item Documentatie
\item optioneel : deadlock analyse als voorbeeld component/template, er van uit gaan dat source code of component beschikbaar is
\end{enumerate}

Ik zou het proberen te beperken tot de Wickedxmas tool, interface voor analyse tools en documentatie , we kunnen dit verantwoorden omdat,
C++ niet de meest moderne taal is, het vergt extra aandacht om foutvrije code te schrijven , dus extra tests, debugging en leercurve. IDE's voor modernere talen ondersteunen de ontwikkelaar beter en bieden meer mogelijkheden met nieuwe technologiën en dit vind je niet terug bij C++ . (de OU maakt naar mijn mening steeds de fout om C++ te vergelijken met Java "C++ lijkt op Java" , ik zeg altijd "C++ lijkt op Java maar het lijkt niet te doen wat Java doet!!" Onthou dit goed)
 


\end{document}