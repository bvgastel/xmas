\section{Aanpak}
\subsection{Proceskeuze}

Bij de keuze tussen plangedreven, volledig agile en een hybride aanpak hebben we de
factoren en overwegingen uit figuur \ref{fig: overwegingen} gebruikt.
Het proces dat we gekozen hebben op grond van deze factoren en overwegingen is een hybride
proces waar we voor zover mogelijk de bruikbare technieken uit agile gebruiken. We combineren
planning en iteratief en incrementeel ontwikkelen zodanig dat het voor ons team past.
Voor project management gebruiken we tijdens de fasen voorafgaand aan
de bouw een plan gedreven aanpak. Tijdens de bouw gebruiken we scrum
met sprints van vaste tijdsduur.

Dit plan werkt het idee uit van deze hybride aanpak.
\begin{figure}[ht]
    \fbox{%
    \tiny
    \begin{minipage}[t]{.45\textwidth}
	\begin{enumerate}
	    \item een geografisch gespreid team bestaande uit 3 teamleden
	    \item een vaste tijd voor uitvoering (ca 8 maanden)
	    \item een technisch product met high level requirements bekend
		bij opdrachtgever en geen extern risico
	    \item geen van de teamleden hebben ervaring met agile
	    \item opdrachtgever en begeleider mogen een beperkte tijd besteden aan het project
	    \item het team wil kennismaken met agile, maar niet ten koste van effectiviteit
	    van het project
	    \item de detail requirements komen tijdens het bouwen aan de orde. Vooraf verzamelen
	    minder goed te realiseren
	\end{enumerate}
    \end{minipage}
    }%
    \quad
    \fbox{%
    \tiny
    \begin{minipage}[t]{.45\textwidth}
	\begin{description}
	    \item[SDM2] een waterval methode vergt rigide requirements. Hoewel de high level
		requirements bekend zijn, geldt dat niet voor detail requirements. Deze zijn
		mede afhankelijk van het ontwerp tool.

		Verder heeft een waterval aanpak een groter risico op uitloop doordat het
		team vooraf functionaliteit toezegt in een bepaalde tijd te realiseren. Ervaring
		wijst uit, dat uitloop vaker voorkomt dan niet.

		Ten slotte wenst het team ervaring met agile op te doen, voorzover
		dit binnen de doelstellingen van het project past.
	    \item[XP] Volledig agile is om meerdere redenen niet haalbaar (zie hieronder). Voor XP
	    is de vrijwel continue beschikbaarheid van gebruikers noodzakelijk. Ook is pair programming
	    niet uit te voeren met een geografisch gespreid team. Verder hebben de leden geen ervaring
	    met agile projecten. Om deze redenen is het onverstandig een volledig agile proces
	    op starten zoals XP.
	    \item[Hybride] Een hybride aanpak met cherry picking van  methoden en technieken lijkt het
		meeste kans op succes te hebben.
	\end{description}
    \end{minipage}
    }%
    \caption{Welk proces gaan we hanteren?}\label{fig: overwegingen}
\end{figure}

\paragraph{Conclusies}
\begin{description}
\item XP vergt fysieke nabijheid en grote gebruikers betrokkenheid: valt af
\item SDM2 heeft een rigide requirements Engineering process
\item UP: Iteratief met agile constructie komt het dichtst in de buurt:
\begin{enumerate}
 \item Plangedreven voorbereiding
 \item iteratieve sprints tijdens de uitvoering
\end{enumerate}
\end{description}

Scrum is het meest bruikbaar voor project management tijdens de constructie omdat
het gemakkelijk bij agile en bij incrementeel cq iteratief ontwikkelen te gebruiken is.

