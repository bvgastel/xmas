\documentclass[a4paper,11pt,twoside,draft]{article}

\usepackage{ucs}
\usepackage{subfigure}
\usepackage[english]{babel}
\usepackage{graphicx}

\bibliographystyle{alpha}

\usepackage{hyperref}

\author{Stefan Versluys, Jeroen Kleijn}
\date{\today}

\title{WickedxMas Project Planning}

\begin{document}
	


\paragraph{info}
***dit is gebasserd op wat Guus reeds als indeling gezien had + wat we nu al verzameld hebben + wat de ABI/Freek verwacht (op cursussite). Plan beschrijving is zoals in SE boek p623/624, de reden kun je in de introductie vinden.

\section{Introduction}
This project concerns building an expert tool called WickedXmas. To understand the problem very well, it requires a deep domain analysis . This excercise must start before any implementation can start so the first part of this project plan is a plan driven approach while for the construction an agile approach is chosen. 

\paragraph{todo}
***(Moeten we nog eens overleggen met Guus of hij dit ook zo zag , ik heb daar iets over gevonden in het SE boek maar ben nu even de blz kwijt) 

\section{Organisation}
\paragraph{todo}
***hoe zijn wij als team geörganiseerd, wie zijn we en wat zijn onze rollen

\section{Risk analysis}
\paragraph{todo}
***wat zijn de mogelijke project risico's

\section{Hard- and software requirments}
\paragraph{todo}
***wat hebben we allemaal nodig om het ding te bouwen

\section{Work breakdown}

\begin{enumerate}
	\item \underline{\textbf{Startup}}
	\begin{itemize}
		\item Stakeholder meeting.
		\item Define the problem.
		\item Collect high-level requirments.
		\item Explore architectural solutions.
		\item Find tool and IDE options.
		\item Planning. 
	\end{itemize}
	\paragraph{Result}
	This block delivers a document that describes the problem , high-level requirements, business case ,risks , stakeholders, vision and challenges, technical options and a project plan.
	
	\item \underline{\textbf{Research}}
		\begin{itemize}
			\item Domain analysis.
			\item Context research.
			\item Find requirements based on using the current product tool.
			\item Find requirements based on sourcecode of the current product tool.
			\item Collect requirements.
			\item Stakeholder meeting.
		\end{itemize}
	\paragraph{Result}
	Domain research is added to the document, the result is a refined document with new requirements and choices with argumentation. Technical choices are made and the team has gained knowledge of IDE, tools and domain. 
		
	\item \underline{\textbf{Construction}}
	
	\begin{itemize}
		\item Iteration 1 :
			Create *xmas skeletons and GUI design
		\item Iteration 2 :
			Implement *xmas functionality and GUI
		\item Iteration 3 :
			Create *analysis/checkers skeletons and GUI design
		\item Iteration 4 :
			Implement *analysis/checkers functionality and GUI
		\item Iteration 5 :
			Implement additional options (Export,Report,...)
	\end{itemize}
	(*) xmas is the part that enables editing an xmas network, while analysis/checkers concerns the part that analyse and checks such a network.
	
	\paragraph{Result}
	Each iteration takes about 2 to 4 weeks and starts with an analysis , implementation , testing, documenting and a release (prototype). The aim of the last iteration is to deliver a fully working release of the product.
	
	\item \underline{\textbf{Final}}
	\begin{itemize}
		\item Product description
		\item Presentation
	\end{itemize}
	\paragraph{Result}
	Product is ready and presented to the stakeholders.
\end{enumerate}
\section{Schedule}
\paragraph{todo}
***Gantt chart
\section{Monitoring and reporting}
\paragraph{todo}
***hoe gaan we onze doelstellingen bewaken en rapporteren



\end{document} ;########################### end document ##################################;