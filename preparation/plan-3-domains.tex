%%
%% Dit is een subdocument van het projectplan.
%%

\section{Domeinanalyse}

De domeinanalyse ligt op het domein van de klant (Bernard): het geeft een deelprobleem. Met een domeinanalyse licht je
\'e\'en deel van het probleem uit het geheel en bestudeer je dat gedetailleerder en wetenschappelijker dan de andere deelproblemen.

Ervaring wijst uit dat het belangrijk is de domeinanalyses als eerste te doen. Dat ondersteunt het project en voorkomt
de domeinanalyse aan het eind ``nog even te moeten schrijven''.

We kiezen er voor om de meest onzekere delen van het project als eerste doen. Hiermee halen we het grootste risico uit het
project.

De twee belangrijke kenmerken van een domeinanalyse zijn dat het een beperkt deel van het onderwerp is en dat het daadwerkelijk
ondersteunend is aan het project. De valkuil is om een te breed onderwerp te nemen. Hieronder voorbeelden van goede onderwerpen
voor een domeinanalyse.

De uitkomst van een domeinanalyse is een beschrijving van het probleem,
een beschrijving van de alternatieven, een beschrijving van de keuze criteria en een gewogen aanbeveling. Gewogen betekent dat de aanbeveling naar objectieve criteria plaatsvindt.

De domeinanalyse bevat een aanbeveling. De beslissing is een team effort, waarbij de klant (Bernard in dit geval) de doorslaggevende stem heeft.

De beoordeling van elke domeinanalyse is individueel evenals de uitwerking.

\begin{center}
    \begin{tabular}{|p{2.5cm}|p{10cm}|}
    \hline
        {\bf vb}		& {\bf beschrijving} \\\hline
        Integratie		& \textsc{Structureel}: wat wordt de interface met de analyse tools? \\
	met tools 		& \textsc{Dynamisch}: hoe gaan we de analyse tools integreren in het ontwerp tool?\\
				& \textsc{Functioneel}: hoe gaan we de analyse tools integreren in het ontwerp tool?\\\hline
        Platform onafhankelijk UI toolkit & Uitgaande van de high level requirements voor het ontwerp
					    tool, zoals platform onafhankelijkheid, uitbreidbaarheid
					    en hechte integratie: kies een UI toolkit die het
					    beste past bij dit project en dit probleem gebied.\\\hline
        combinatorische cycle checker & je zou de combinatorische cycle checker kunnen bestuderen inclusief de datastructuur die
					deze tool deelt met de andere analysetools en een domeinanalyse kunnen doen
					over hoe deze tool geintegreerd kan worden met jullie tool.\\\hline
         combinatorial objects &  een onderzoek hoe om te gaan met combinatorial objects.
				    Bv hoe grafisch weer te geven, hoe op te slaan in de data structuur\\\hline
    \end{tabular}

\end{center}
