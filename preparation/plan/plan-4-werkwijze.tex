%%
%% Dit is een subdocument van het projectplan.
%%

\subsection{Introductie}
Het project betreft het herbouwen van het WickedXmas tool, omdat de tool gebruikt wordt voor studie en ontwerp van chips is het belangrijk dat deze tool foutloos werkt. Verder betreft het een niet alledaags domein waardoor er extra aandacht noodzakelijk is om het domein te verkennen. De eis dat het tool platformonafhankelijk moet zijn en bij voorkeur in C++ geschreven is , maken dat er tijd voorzien moet worden om de teamleden de nodige ontwikkeltools eigen te maken. 
Er is gekozen voor een globaal plan dat uit vier blokken van activiteiten bestaat , enkel in het derde blok , tijdens de constructie , wordt agile gewerkt.

\subsection{Organisatie}
 De teamleden ontwikkelen gedelokaliseerd en communiceren via internet. Sources worden gecentraliseerd in de cloud met versiebeheer. Het team beschikt over een begeleider welke toeziet op de gang van zaken en geeft op vraag advies. De opdrachtgever en tevens domeinspecialist kan geraadpleegd worden voor specifieke domein vragen en evaluatie van het product.   
 \begin{enumerate}
 	\item Guus Bonnema
 	\begin{itemize}
		\item Rol - Ontwikkelaar
		\item Skills - IT, Java, Linux
	\end{itemize}
 	\item Jeroen Kleijn
 	\begin{itemize}
		\item Rol - Ontwikkelaar
		\item Skills - IT , MS VisualStudio (C\#), C/C++, Java, Linux + Windows
	\end{itemize}
 	\item Stefan Versluys
 	\begin{itemize}
		\item Rol - Ontwikkelaar
		\item Skills - IT , MS VisualStudio (C\#), C/C++, Java, Windows + VxWorks
	\end{itemize}
	\item Freek Verbeek
	\begin{itemize}
		\item Rol - Proces begeleider
	\end{itemize}
	\item Bernard van Gastel
	\begin{itemize}
		\item Rol - Opdrachtgever / Domeinspecialist : begeleid inhoudelijk
	\end{itemize}
	
 \end{enumerate}
 
\subsection{Hardware en software}
\begin{enumerate}
	\item Voor versiebeheer en sources te borgen wordt een centrale Git repo gebruikt.
	\item Communicatie middelen zijn GitHub, gmail, skype , teamviewer.
	\item Platformonafhankelijke IDE met C++ compiler.
	\item Platformonafhankelijke GUI Toolkit.
	\item Componenten voor xmas analyse en checks.
	\item Mac OS , MS Windows, Linux platformen.
\end{enumerate}


\subsection{Werkwijze}
\subsubsection{De basis principes}
\begin{itemize}
 \item De uitvoering is agile en tijdgedreven.
 \item Elke iteratie heeft een vaste hoeveelheid tijd
 \item Elke iteratie begint met een analyse (requirements engineering, prioriteiten en selectie).     \item Elke iteratie eindigt met een evaluatie
\item Peer review is belangrijk onderdeel van ons proces
 \item Refactoring is standaard onderdeel van process
\end{itemize}

\subsubsection{Agile aanpak}
\begin{itemize}
 \item Passen TDD toe (binnen redelijke)
 \item Passen refactoring toe
 \item Elke iteratie:
 \begin{itemize}
   \item Analyse (requirements aanpassen, prioriteitsstelling, selectie)
   \item Ontwikkeling prototype
   \item Evaluatie : Tijdens het constructie blok is het de bedoeling dat elke iteratie een prototype oplevert, deze wordt dan ter evaluatie aan een gebruiker aangeboden. Feedback die voortvloeit uit de evaluatie wordt meegenomen voor de volgende iteratie.
 \end{itemize}
\end{itemize}



