\section{Aanpak}
\subsection{Proceskeuze}

Bij de keuze tussen plangedreven, volledig agile en een hybride aanpak hebben we de
factoren en overwegingen uit figuur \ref{fig: overwegingen} gebruikt.

\begin{figure}[ht]
    \fbox{%
    \tiny
    \begin{minipage}[t]{.45\textwidth}
	\begin{enumerate}
	    \item een geografisch gespreid team bestaande uit 3 teamleden
	    \item een vaste tijd voor uitvoering (ca 8 maanden)
	    \item een technisch product met high level requirements bekend
		bij opdrachtgever en geen extern risico
	    \item geen van de teamleden hebben ervaring met agile
	    \item opdrachtgever en begeleider mogen een beperkte tijd besteden aan het project
	    \item het team wil kennismaken met agile, maar niet ten koste van effectiviteit
	    van het project
	    \item de detail requirements komen tijdens het bouwen aan de orde. Vooraf verzamelen
	    minder goed te realiseren
	\end{enumerate}
    \end{minipage}
    }%
    \quad
    \fbox{%
    \tiny
    \begin{minipage}[t]{.45\textwidth}
	\begin{description}
	    \item[SDM2] een waterval methode vergt rigide requirements. Hoewel de high level
		requirements bekend zijn, geldt dat niet voor detail requirements. Deze zijn
		mede afhankelijk van het ontwerp tool.

		Verder heeft een waterval aanpak een groter risico op uitloop doordat het
		team vooraf functionaliteit toezegt in een bepaalde tijd te realiseren. Ervaring
		wijst uit, dat uitloop vaker voorkomt dan niet.

		Ten slotte wenst het team ervaring met agile op te doen, voorzover
		dit binnen de doelstellingen van het project past.
	    \item[XP] Volledig agile is om meerdere redenen niet haalbaar (zie hieronder). Voor XP
	    is de vrijwel continue beschikbaarheid van gebruikers noodzakelijk. Ook is pair programming
	    niet uit te voeren met een geografisch gespreid team. Verder hebben de leden geen ervaring
	    met agile projecten. Om deze redenen is het onverstandig een volledig agile proces
	    op starten zoals XP.
	    \item[Hybride] Een hybride aanpak met cherry picking van  methoden en technieken lijkt het
		meeste kans op succes te hebben.
	\end{description}
    \end{minipage}
    }%
    \caption{Welk proces gaan we hanteren?}\label{fig: overwegingen}
\end{figure}



\paragraph{Conclusies}
\begin{description}
\item XP valt af omdat het fysieke nabijheid vergt en grote gebruikers betrokkenheid
\item SDM2 heeft een te rigide requirements engineering proces en valt dus ook af.
\item Scrum valt eveneens af omdat het maar goed werkt als een team de nodige agile vaardigheden heeft. Verder is er ook een rol nodig van Scrum Master wat in een klein team al voor een onevenwicht zorgt in taak verdeling.
\item AUP (Agile Unified Process) is een vereenvoudigde versie van RUP, het sluit goed aan bij de visie van ons team maar wordt sinds 2006 niet meer ondersteund.
\item DAD (Disciplined Agile Delivery) is de verderzetting van AUP door Scott Ambler, DAD is een ``people-first, learning-oriented hybrid agile approach to IT'' . DAD is een raamwerk en de life-cycle bestaat uit drie fasen , de elaboratie fase wordt als een constructie gezien. In tegenstelling tot XP, Scrum en andere waar de focus voornamelijk op software ontwikkeling ligt, is bij DAD het ganse traject van belang.
Overzicht waar DAD voor staat
%%http://www.ambysoft.com/books/dad.html
	\begin{enumerate}
		\item People first : Self-disciplined, Self-organizing, Self-aware
		\item Learning oriented : domain learning, process learning, technical.
		\item Agile : enhances the values and principles of the Agile Manifesto.
		\item Hybrid : adopts and tailors strategies from a variety of sources.
		\item IT solution focused :  provide real business value to your stakeholders.
		\item Full delivery lifecycle : from the beginning of a project to the release of the solution into production.
		\item Goals driven : focused on the right things at the right time
		\item Risks and value driven : attack the risks before they attack you.
		\item Enterprise aware
	\end{enumerate}


DAD lijkt voor ons als team en voor het project het meest geschikt, het stelt
``learning oriented'' voorop en houdt in dat domeinstudie en hoe je de stakeholders het best bedient even belangrijk is dan het ontwikkelen van software. Het hanteert zachte mijlpalen en met de ``Proven Architecture milestone''  in het begin van de constructie fase zorgt men er voor dat de meeste risico's rond architectuur geweken zijn eer men verder bouwt. Met de zogenaamde ``Work Item''  lijst worden te behandelen items volgens risico en waarde geprioriteerd.

\end{description}





