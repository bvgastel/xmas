\section{Aanpak}
\subsection{Proceskeuze}

Bij de keuze tussen plangedreven, volledig agile en een hybride aanpak hebben we de
factoren en overwegingen uit figuur \ref{fig: overwegingen} gebruikt.

\begin{figure}[ht]
    \fbox{%
    \small\sf
    \begin{minipage}[t]{.45\textwidth}
	\begin{enumerate}
	    \item een geografisch gespreid team
	    \item 8 maanden voor de uitvoering
	    \item High level requirements bekend
		bij opdrachtgever en geen extern risico
	    \item geen van de teamleden hebben ervaring met agile
	    \item opdrachtgever en begeleider mogen een beperkte tijd besteden aan het project
	    \item het team wil kennismaken met agile, maar niet ten koste van effectiviteit
	    van het project
	    \item de detailrequirements komen tijdens het bouwen aan de orde. Vooraf verzamelen
	    is minder goed te realiseren
	\end{enumerate}
    \end{minipage}
    }%
    \quad
    \fbox{%
    \small\sf
    \begin{minipage}[t]{.45\textwidth}
	\begin{description}
	    \item[SDM2] een watervalmethode vergt rigide requirements. Hoewel de high level
		requirements bekend zijn, geldt dat niet voor detailrequirements. Deze zijn
		mede afhankelijk van het ontwerptool.

		Verder heeft een watervalaanpak een groter risico op uitloop doordat het
		team vooraf functionaliteit toezegt in een bepaalde tijd te realiseren. Ervaring
		wijst uit, dat uitloop vaker voorkomt dan niet.

		Ten slotte wenst het team ervaring met agile op te doen, voorzover
		dit binnen de doelstellingen van het project past.
	    \item[XP] Volledig agile is om meerdere redenen niet haalbaar (zie hieronder). Voor XP
	    is de vrijwel continue beschikbaarheid van gebruikers noodzakelijk. Ook is pair programming
	    niet uit te voeren met een geografisch gespreid team. Verder hebben de leden geen ervaring
	    met agile projecten. Om deze redenen is het onverstandig een volledig agile proces
	    op starten zoals XP.
	    \item[Hybride] Een hybride aanpak met cherry picking van methoden en technieken lijkt het
		meeste kans op succes te hebben.
	\end{description}
    \end{minipage}
    }%
    \caption{Welk proces gaan we hanteren?}\label{fig: overwegingen}
\end{figure}



\paragraph{Conclusies}
\begin{description}
\item XP valt af omdat het fysieke nabijheid vergt en grote gebruikers betrokkenheid
\item SDM2 heeft een te rigide requirements engineering proces en valt dus ook af.
\item Scrum valt eveneens af omdat het maar goed werkt als een team de nodige agile vaardigheden heeft.
	Verder is er ook een rol nodig van Scrum Master wat in een klein team al
	voor een onevenwicht zorgt in taakverdeling.
\item AUP (Agile Unified Process) is een vereenvoudigde versie van RUP, het sluit goed aan
	bij de visie van ons team maar wordt sinds 2006 niet meer ondersteund.
\item DAD (Disciplined Agile Delivery) is de verderzetting van AUP door Scott Ambler, DAD
    is een ``people-first, learning-oriented hybrid agile approach to IT'' . DAD is een raamwerk en
    de life-cycle bestaat uit drie fasen , de elaboratiefase wordt als een constructie gezien.
    In tegenstelling tot XP, Scrum en andere waar de focus voornamelijk op software ontwikkeling
    ligt, is bij DAD het ganse traject van belang.
Overzicht waar DAD voor staat
%%http://www.ambysoft.com/books/dad.html
	\begin{enumerate}
		\item People first : Self-disciplined, Self-organizing, Self-aware
		\item Learning oriented : domain learning, process learning, technical learning.
		\item Agile : enhances the values and principles of the Agile Manifesto.
		\item Hybrid : adopts and tailors strategies from a variety of sources.
		\item IT solution focused :  provide real business value to your stakeholders.
		\item Full delivery lifecycle : from the beginning of a project to the release of the solution into production.
		\item Goals driven : focused on the right things at the right time
		\item Risks and value driven : attack the risks before they attack you.
		\item Enterprise aware
	\end{enumerate}

DAD lijkt voor ons als team en voor het project het meest geschikt, het stelt
``learning oriented'' voorop en houdt in dat domeinstudie en hoe je de stakeholders
het best bedient even belangrijk is als het ontwikkelen van software. Het hanteert
zachte mijlpalen en met de ``Proven Architecture milestone''  in het begin van de
constructiefase zorgt men er voor dat de meeste risico's rond architectuur geweken
zijn eer men verder bouwt. Met de zogenaamde ``Work Item''  lijst worden te behandelen
items volgens risico en waarde geprioriteerd.

%%%%%%%%%%%%%%%%%%%%%%%%%%%%%% Aanpassing februari 2015 %%%%%%%%%%%%%%%%%%%%%%%%%%%%
\begin{aanpassing1}\label{aanp1: toelichting}
Naar aanleiding van ontwikkelingen aan de \ou is de intensiviteit van communicatie sinds 
december 2014 sterk afgenomen. In een overleg 24 januari 2015 heeft Bernard het ``hoe en waarom'' 
van deze afname en de consequenties toegelicht. Onder andere heeft Bernard geen tijd meer 
om te programmeren aan de VT omgeving (verificatie tools) en moet hij voor 50\% van zijn tijd 
doceren en deze lessen op korte termijn overnemen van zijn vervroegd pensionerende collega's.
Freek gaat voor 3 maanden naar Amerika en heeft nergens meer tijd voor. De communicatie wordt summier. 
Toch wil Bernard wel graag betrokken blijven bij het project, het kan alleen niet zo intensief meer.

Als team realiseerden we ons dat de agile manier van werken die we hadden gepland niet meer haalbaar was.
We kozen er voor om een afgeslankte versie van agile werken te gebruiken. Intern werken we hetzelfde als 
voorheen met driemaal per week contactmomenten, meer indien nodig. Met de opdrachtgever beperken we de 
communicatie tot het hoognodige waarbij we de relatief kleinere beslissingen zonder overleg nemen. In geval
van twijfel stemmen we af met email. De grotere zaken (stappen in het plan) plannen we in met een skype sessie.

Voor Freek als begeleider zorgen we dat hij van de grotere ontwikkelingen op de hoogte is via email. Hij
krijgt ongeveer \'{e}\'{e}n keer per maand voortgangsrapportage met op hoofdlijnen de voortgang.

Verder passen we de werkwijze aan. Vier van de 6 iteraties vervangen we door 3 stappen. Deze stappen
duren langer, hebben een bepaalde functionaliteit tot doel en een geschatte doorlooptijd. Dit in contrast
met de oorspronkelijke vaste looptijd en variabele functionaliteit. We laten de strict agile aanpak hiermee 
los, maar blijven intern intensief samenwerken. Het grote verschil met andere agile projecten is het 
verminderde contact met de opdrachtgever  en de begeleider. Dit is een realisatie van \een van de 
eerder gespecificeerde risico's.

Een iets kleiner verschil is dat we de vaste tijdslijnen vervangen door geschatte tijdslijnen en 
iets meer vrijheid nemen in het uitvoeren van de stappen. Het terrein is voor alle 3 de studenten 
nieuw en we kunnen vooraf moeilijk de doorlooptijd schatten. De 3 stappen helpen ons om gestructureerd 
naar het doel toe te werken zonder ons te committeren aan een gedetailleerde work breakdown structure.

Omdat we een mooi product willen neerzetten nemen we een iets ruimere doorlooptijd. In plaats van 
eind mei klaar te zijn kiezen we voor eind juni/ begin juli. Met deze opoffering compenseren we het risico
dat ontstaat door langere responsetijden en hebben we net iets meer speling om te stappen te voltooien.
Nadeel is dat de einddatum geen speling meer heeft.
\end{aanpassing1}

\end{description}