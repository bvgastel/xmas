%%
%% Dit is een subdocument van het projectplan.
%%
%%  hier worden tijden en eventueel personen toegekend aan activiteiten
%%

\begin{landscape}
\section{Schedule}
 %% dit onderdeel moet vermoedelijk in het schedule plan

%\pagebreak


%%%%%%%%%%%%%%%%%%%%%%%%%%%%%%%%%%%%%%%%%%%%%%%%%%%%%%%%%%%%%%%%%%%%%
%%%%%%%%%%%%%%%%%%%%%% Aanpassing1 febr 2015 %%%%%%%%%%%%%%%%%%%%%%%%

\newgeometry{top=5mm,left=10mm,bottom=5mm,right=-90mm}

\ganttset{%
	calendar week text={%
		%\pgfcalendarmonthshortname{\startmonth}~\startday, \startyear%
		\calendarweek{\startday}{\startmonth}{\startyear}%
	}%
}

\begin{ganttchart}[
	today={2015-02-07}, today rule/.style={draw=blue, dashed, ultra thick},
	progress=today,
	%hgrid,
	%vgrid,
	x unit=0.9mm,
	y unit chart=9mm,
	inline,
 	group/.append style={draw=black,fill=blue!90},
	bar/.append style={draw=black,fill=green!50},
	milestone/.append style={draw=black, shape=circle, fill=gray, rounded corners=3mm},
	milestone complete /.style={draw=black, shape=circle, fill=green, rounded corners=3mm},
	milestone incomplete /.style={draw=black, shape=circle, fill=gray, rounded corners=3mm},
%	milestone progress label text={},
	time slot format=isodate
]{2014-09-15}{2015-07-12}
	\gantttitlecalendar{year, month=name, week} \\
	\ganttgroup{Inceptie}{2014-09-15}{2014-11-16} \\

	\ganttbar[name=planning]{Planning}{2014-09-15}{2014-10-19} \\
	\ganttmilestone[name=m_planning,progress label text={}]{Stakeholder consensus}{2014-10-19} \\

	\ganttbar[name=domein]{Domein}{2014-10-20}{2014-11-16} \\
	\ganttmilestone[name=m_da,progress label text={}]{Domeinanalyse}{2014-11-16} \\

	\ganttgroup{Constructie}{2014-11-17}{2015-06-14} \\
	\ganttgroup[name=onderzoekscontext]{Onderz context}{2015-03-09}{2015-06-14} \\

	\ganttbar[name=it0]{Iter 0}{2014-11-17}{2014-12-01} \\
	\ganttmilestone[name=m_architecture,progress label text={}]{Proven Architecture}{2014-12-02} \\
	\ganttbar[name=it1]{Iter 1}{2014-12-02}{2014-12-23} \\
	\ganttbar[name=step1]{Step 1}{2014-12-23}{2015-03-08} \\
	\ganttbar[name=step2]{Step 2}{2015-03-09}{2015-05-24} \\
	\ganttbar[name=step3]{Step 3}{2015-05-25}{2015-06-14} \\
	\ganttgroup{Transitie}{2015-06-15}{2015-07-05} \\
	\ganttbar[name=afronding]{Afronding}{2015-06-15}{2015-07-05} \\
	\ganttmilestone{presentation,progress label text={}}{2015-07-05}
	
	\ganttlink[link type=dr]{planning}{m_planning}
	\ganttlink[link type=dr]{m_planning}{domein}
	\ganttlink[link type=dr]{domein}{m_da}
	\ganttlink[link type=dr]{it0}{m_architecture}
	\ganttlink[link type=dr]{m_architecture}{it1}
	\ganttlink[link type=dr]{it1}{step1}
	\ganttlink[link type=dr]{step1}{step2}
	\ganttlink[link type=dr]{step2}{step3}
	\ganttlink[link type=dr]{step3}{afronding}
%	\ganttlink[link type=dr]{onderzoekscontext}{afronding} // % No link for onderzoekscontext 
\end{ganttchart}

%%%%%%%%%%%%%%%%%%%%%% Einde Aanpassing1 febr 2015 %%%%%%%%%%%%%%%%%%%%%%%%
%%%%%%%%%%%%%%%%%%%%%%%%%%%%%%%%%%%%%%%%%%%%%%%%%%%%%%%%%%%%%%%%%%%%%%%%%%%


\newgeometry{top=0.5cm,left=0.5cm,bottom=0.5cm,right=-5cm}
\setganttlinklabel{f-s}{}

\begin{figure}[hp]
{\gray \begin{ganttchart}[hgrid,vgrid,y unit chart=.9cm,
	%today=4,
	inline,
	group/.append style={draw=black,fill=blue!90},
	bar/.append style={draw=black,fill=green!50}
	]{0}{42}
 \gantttitle{2014}{15}
 \gantttitle{2015}{28} \\
 \gantttitlelist{38,...,52}{1}
 \gantttitlelist{1,...,28}{1} \\

 \ganttgroup{Inceptie}{1}{8} \\

 \ganttbar[name=planning]{Planning}{1}{4} \\
 \ganttmilestone[name=m_planning]{Stakeholder consensus}{4} \\

 \ganttbar[name=domein]{Domein}{5}{8} \\
 \ganttmilestone[name=m_da]{Domeinanalyse}{8} \\

 \ganttgroup{Constructie}{9}{28} \\

 \ganttbar[name=it0]{Iter 0}{9}{11} \\
 \ganttmilestone[name=m_architecture]{Proven Architecture}{11} \\
 \ganttbar[name=it1]{Iter 1}{12}{14} \\
 \ganttbar[name=it2]{iter 2}{17}{19} \\
	 	\ganttbar[name=it3]{Iter 3}{20}{22} \\
 		\ganttbar[name=it4]{Iter 4}{23}{25} \\
 		\ganttbar[name=it5]{Iter 5}{26}{28} \\

 \ganttmilestone[name=m_suff_func]{Sufficient functionality}{28} \\

 \ganttgroup{Transitie}{29}{31} \\
 \ganttbar[name=afronding]{Afronding}{30}{32}\\
 \ganttbar[name=onderzoekscontext]{Onderzoekscontext}{20}{29} \\
 \ganttbar[bar/.style={}]{}{32}{37} \\
 \ganttmilestone{Delighted stakeholders}{37}

    \ganttlink[link type=dr]{planning}{m_planning}
    \ganttlink[link type=dr]{m_planning}{domein}
    \ganttlink[link type=dr]{domein}{m_da}
    \ganttlink[link type=dr]{it0}{m_architecture}
    \ganttlink[link type=dr]{m_architecture}{it1}
    \ganttlink[link type=dr]{it1}{it2}	
    \ganttlink[link type=dr]{it2}{it3}
    \ganttlink[link type=dr]{it3}{it4}								
    \ganttlink[link type=dr]{it4}{it5} 							
    
    \ganttlink[link type=dr]{it5}{m_suff_func}						
    \ganttlink[link type=dr]{it2}{onderzoekscontext}
\end{ganttchart}
} % gray
\caption{Globale planning met hoofdfasen en belangrijkste mijlpalen}
\end{figure}

\pagebreak

\begin{figure}[hp]
\begin{ganttchart}[
    hgrid,vgrid,
    y unit chart=.9cm,
    %today=36,
    inline,
    group/.append style={draw=black,fill=blue!90},
    bar/.append style={draw=black,fill=green!50}
    ]{0}{37}
  \gantttitle{2014}{37} \\
  \gantttitle{september}{19}
  \gantttitle{oktober}{18} \\
  \gantttitlelist{12,...,30}{1}
  \gantttitlelist{1,...,18}{1}\\

  \ganttmilestone{Startbijeenkomst}{0.5} \\
  \ganttbar{Teamafspraken}{2}{7} \\
  \ganttbar[bar inline label node/.style={anchor=west},bar inline label anchor=east]{Stakeholder meeting}{8}{8} \\
  \ganttbar{Probleemdefinitie}{9}{27} \\
  \ganttbar{High-level requirements}{9}{27} \\
  \ganttmilestone{Concept planning}{27} \\

  \ganttbar{Verbeteringen}{29}{36} \\
  \ganttmilestone{Stakeholder consensus}{36}

  \ganttlink[link type=dr]{elem0}{elem1}
  \ganttlink[link type=dr]{elem1}{elem2}
  \ganttlink[link type=dr]{elem2}{elem3}
  \ganttlink[link type=dr]{elem2}{elem4}
  \ganttlink[link type=dr]{elem3}{elem5}
  \ganttlink[link type=dr]{elem4}{elem5}
  \ganttlink[link type=dr]{elem5}{elem6}
  \ganttlink[link type=dr]{elem6}{elem7}

\end{ganttchart}
\caption{Detailplanning 'Planning'}
\end{figure}

\pagebreak

\begin{figure}[hp]
\begin{ganttchart}[
    hgrid,vgrid,
    y unit chart=.9cm,
    inline,
    group/.append style={draw=black,fill=blue!90},
    bar/.append style={draw=black,fill=green!50}
    ]{0}{32}
  \gantttitle{Domeinanalyse}{32}
  \ganttnewline
  \gantttitle{2014}{33} \\
  \gantttitle{oktober}{15}
  \gantttitle{november}{18} \\
  \gantttitlelist{17,...,31}{1}
  \gantttitlelist{1,...,18}{1}\\

  \ganttmilestone[name=consensus, milestone inline label node/.style={anchor=west},milestone inline label anchor=east]{Stakeholder consensus}{0.5} \\
  \ganttbar[name=demo,
	bar inline label node/.style={anchor=west},bar inline label anchor=east]{Demonstratie WickedXmas}{2}{2} \\
  %% Domein analyse Guus (GB)
  \ganttbar[name=gb1]{Domeinanalyse GB}{3}{22}
  \ganttbar[name=gb2,bar/.style={fill=gray}]{}{23}{29}
  \ganttlinkedmilestone[name=dagb]{}{30}\\
  %% Domein analyse Jeroen (JK)
  \ganttbar[name=jk1]{}{3}{4}
  \ganttlinkedbar[name=jk2]{Domeinanalyse JK}{9}{27}
  \ganttbar[name=jk3,bar/.style={fill=gray}]{}{28}{29}
  \ganttlinkedmilestone[name=dajk]{}{30}\\
  %% Domein analyse Stefan (SV)
  \ganttbar[name=sv1]{}{3}{7}
  \ganttlinkedbar[name=sv2]{Domeinanalyse SV}{15}{29}
  \ganttlinkedmilestone[name=dasv]{}{30}\\

  \ganttlink[link type=dr]{consensus}{demo}
  \ganttlink[link type=dr]{demo}{gb1}
  \ganttlink[link type=dr]{demo}{jk1}
  \ganttlink[link type=dr]{demo}{sv1}

\end{ganttchart}
\caption{Detailplanning 'Domeinanalyse'}
\end{figure}

%%\restoregeometry
\end{landscape}


\begin{figure}[h]
\begin{tabular}{ll}\hline
{\bf Fase}    & {\bf weken}\\\hline
Planning             & 2,5 \\
Domeinanalyse        & 3 \\
\hline
Iteratie 0           & 3 \\
Iteratie 1           & 3 \\
Iteratie 2           & 3 \\
Iteratie 3           & 3 \\
Iteratie 4           & 3 \\
Iteratie 5           & 3 \\
\hline
Onderzoekscontext     &	2 \\
Afronding	     & 1.5 \\
\hline
Totaal               & 27 \\
\end{tabular}
\caption{Studielast taken}
\end{figure}

\begin{figure}[h]
\begin{tabular}{llp{7cm}}\hline
{\bf Week}    & {\bf Taak}  & {\bf Toelichting}\\\hline
38-42         & Planning    \\
42-45         & Domeinanalyse & 4 weken ingepland i.v.m. vakanties Stefan \& Jeroen \\
46-48         & Iteratie 0    \\
49-51         & Iteratie 1    \\
52-1          &               & vakantieperiode, mogelijk
				individueel (bijvoorbeeld onderzoekscontext) \\
2-4           & Iteratie 2    \\
5-7           & Iteratie 3    \\
8-10          & Iteratie 4    \\
11-13         & Iteratie 5    \\

5-14          & Onderzoekcontext & Uitvoering parallel aan de iteraties \\
15-16         & Afronding     \\

22            & Mijlpaal einde project
\end{tabular}
\caption{Planning taken}
\end{figure}

\begin{itemize}
 \item looptijd project: tot eind mei (36 weken)
 \begin{itemize}
  \item 27 weken effectief voor het project  (27 weken * 15 uur/week = 405 uur)
  \item 3 weken vakantie
  \item 6 weken speling
 \end{itemize}
 \item De detailplanning van een fase wordt telkens in de voorafgaande fase uitgewerkt.
\end{itemize}

\paragraph{Planning iteraties}

 \begin{center}
    \begin{figure}[ht]
	\begin{ganttchart}[
	    hgrid,vgrid,
	    y unit chart=.9cm,
	    inline,
	    group/.append style={draw=black,fill=blue!90},
	    bar/.append style={draw=black,fill=green!50}
	]{0}{22}
	\gantttitle{Planning iteraties}{21} \\
	\gantttitlelist{,1,...,21}{1} \\
	\ganttbar[name=start,inline=false,bar label font=\color{blue}]{Overleg start iteratie}{1}{1} \\
	\ganttbar[name=overleg,inline=false,bar label font=\color{blue}]{Overleg voortgang}{12}{12} \\
	\ganttbar[name=demo,inline=false,bar label font=\color{blue}]{Demonstratie software en release}{19}{19}
	\ganttnewline[thick,blue]
	\ganttbar[name=req]{Requirements}{1}{7} \\
	\ganttbar[name=dev]{Ontwikkeling}{3}{17} \\
	\ganttbar[name=doc]{Documentatie}{10}{18}
	\ganttlinkedmilestone[name=rel]{Release}{21}
	\ganttnewline[thick,blue]
	\ganttbar[name=scriptie]{Scriptie}{19}{21} \\
	\ganttbar[name=onderzoekscontext]{Context}{19}{21}

	\ganttlink[link type=ur]{dev}{demo}
	\ganttlink{doc}{rel}
	\end{ganttchart}
    \caption{Globale planning iteraties}
    \end{figure}
 \end{center}


Elke iteratie omvat een vaste periode van 3 weken.
De exacte verdeling van de beschikbare tijd over de taken verschilt per iteratie.
In de eerste iteraties zal bijvoorbeeld relatief meer tijd worden besteed aan requirements.
Aan het einde van een iteratie wordt een voorlopige release van de software aan de
opdrachtgever verstrekt. De laatste twee dagen zijn gereserveerd voor het werken
aan het scriptieverslag en de onderzoekscontext. In deze periode vindt
ook een evaluatie plaats van de afgelopen iteratie. \\
Voor het vaststellen van de requirements zal regelmatig overleg met Bernard plaats
moeten vinden. Hiervoor is het belangrijk dat in de detailplanning van een iteratie
rekening wordt gehouden met de beschikbaarheid van Bernard.

\pagebreak

\paragraph{Capaciteitsplanning}

\begin{itemize}
 \item Beschikbare tijd
 \begin{itemize}
  \item $\sim$15 uur per week per teamlid
  \item $\sim$30 uur gedurende het hele project voor Freek en $\sim$20 uur voor Bernard
        gereserveerd vanuit het ABI. Freek en Bernard hebben echter aangegeven dat zij graag
        intensief betrokken worden tijdens het project. Zij zullen dus meer tijd willen
        en kunnen vrijmaken. Afgesproken is dat wij op elk moment contact op kunnen nemen.
        Zodra wij te veel van hun tijd in beslag nemen zullen zij dat aangeven.
 \end{itemize}


 \item Vakanties / niet beschikbaar voor project
 \begin{itemize}

  \item Stefan
  \begin{itemize}
   \item 25-10-2014 t/m 31-01-2014 (week 44)
   \item 24-12-2014 t/m 02-01-2015 (week 52-01)
  \end{itemize}

  \item Jeroen
  \begin{itemize}
   \item 21-10-2014 t/m 25-10-2014 (week 43)
  \end{itemize}

  \item Bernard
   \begin{itemize}
    \item 04-10-2014 t/m 14-10-2014 (week 41-42)
    \item 20-10-2014 t/m 27-10-2014 (week 43-44)
   \end{itemize}

 \end{itemize}
\end{itemize}
