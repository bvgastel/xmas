%%
%% Dit is een subdocument van het projectplan.
%%
%%  hier worden tijden en eventueel personen toegekend aan activiteiten
%%

\begin{landscape}
\section{Schedule}
 %% dit onderdeel moet vermoedelijk in het schedule plan

%\pagebreak


%%%%%%%%%%%%%%%%%%%%%%%%%%%%%%%%%%%%%%%%%%%%%%%%%%%%%%%%%%%%%%%%%%%%%
%%%%%%%%%%%%%%%%%%%%%% Aanpassing1 febr 2015 %%%%%%%%%%%%%%%%%%%%%%%%

\begin{aanpassing1}
	Het schedule wijzigt grondig, voor de activiteiten die nog niet klaar zijn. 
	We vervangen 4 van de 6 iteraties door 3 stappen en verlengen de einddatum. Per 
	stap leveren we afgesproken producten op (zie sectie \ref{sec:planning}).
	De communicatie met de opdrachtgever is sinds december summier en dat blijft zo,
	gezien de recente ontwikkelingen bij de \ou en de implicaties voor Bernard 
	en Freek. Om die reden vermoeden wij dat de doorlooptijd wat langer is, hetgeen 
	de aanleiding is voor deze aanpassing.
	
	We hebben ervaren dat we als team een aanlooptijd nodig hebben.
	Om die reden hebben we de eerste stap het langste gemaakt, de tweede iets korter 
	en de derde het kortste. Dit is de tweede reden voor verschuiving van de einddatum.
	
	De derde reden is dat we daarmee wat ademruimte voor het project cre\"{e}ren. 
	We voorzien anders een overspannen maand mei en uitloop naar juni. 
	Een geplande uitloop is beter dan dat ``de omstandigheden''
	dit opleggen aan het team. De keerzijde is dat we geen uitloop hebben ten
	opzichte van ons voornemen om voor de zomervakantie klaar te zijn.
\end{aanpassing1}

\newgeometry{top=5mm,left=10mm,bottom=5mm,right=-90mm}

\ganttset{%
	calendar week text={%
		%\pgfcalendarmonthshortname{\startmonth}~\startday, \startyear%
		\calendarweek{\startday}{\startmonth}{\startyear}%
	}%
}

\setganttlinklabel{f-s}{}

\begin{ganttchart}[%
	today={2015-02-07},%
	today rule/.style={draw=blue, dashed, ultra thick},%
	today label=7 febr 2015,%
	progress=today,%
	%progress label text=\relax,%
	%hgrid,
	%vgrid,
	x unit=0.9mm,%
	y unit chart=9mm,%
	inline,%
 	group/.append style={draw=black,fill=blue!90},%
	bar/.append style={draw=black,fill=green!50},%
	milestone/.append style={draw=black, shape=circle, fill=gray, rounded corners=3mm},%
	milestone complete /.style={draw=black, shape=circle, fill=green, rounded corners=3mm},%
	milestone incomplete /.style={draw=black, shape=circle, fill=gray, rounded corners=3mm},%
%	milestone progress label text={},
	time slot format=isodate%
]{2014-09-15}{2015-07-12}%
	\gantttitlecalendar{year, month=name, week} \\
	\ganttgroup{Inceptie}{2014-09-15}{2014-11-16} \\

	\ganttbar[name=planning]{Planning}{2014-09-15}{2014-10-19} \\
	\ganttmilestone[name=m_planning,progress label text={}]{Stakeholder consensus}{2014-10-19} \\

	\ganttbar[name=domein]{Domein}{2014-10-20}{2014-11-16} \\
	\ganttmilestone[name=m_da,progress label text={}]{Domeinanalyse}{2014-11-16} \\

	\ganttgroup{Constructie}{2014-11-17}{2015-06-14} \\
	\ganttgroup[name=onderzoekscontext]{Onderz context}{2015-03-09}{2015-06-14} \\

	\ganttbar[name=it0]{Iter 0}{2014-11-17}{2014-12-01} \\
	\ganttmilestone[name=m_architecture,progress label text={}]{Proven Architecture}{2014-12-02} \\
	\ganttbar[name=it1]{Iter 1}{2014-12-02}{2014-12-23} \\
	\ganttbar[name=stap1]{Stap 1}{2014-12-23}{2015-03-08} \\
	\ganttbar[name=stap2]{Stap 2}{2015-03-09}{2015-04-26} \\
	\ganttbar[name=stap3]{Stap 3}{2015-04-27}{2015-06-07} \\
	\ganttgroup{Transitie}{2015-06-08}{2015-06-29} \\
	\ganttbar[name=afronding]{Afronding}{2015-06-08}{2015-06-28} \\
	\ganttmilestone[name=m_presentation,progress label text={}]{presentatie}{2015-07-05}
	
	\ganttlink[link type=dr]{planning}{m_planning}
	\ganttlink[link type=dr]{m_planning}{domein}
	\ganttlink[link type=dr]{domein}{m_da}
	\ganttlink[link type=dr]{it0}{m_architecture}
	\ganttlink[link type=dr]{m_architecture}{it1}
	\ganttlink[link type=dr]{it1}{stap1}
	\ganttlink[link type=dr]{stap1}{stap2}
	\ganttlink[link type=dr]{stap2}{stap3}
	\ganttlink[link type=dr]{stap3}{afronding}
%	\ganttlink[link type=dr]{onderzoekscontext}{afronding} // % No link for onderzoekscontext 
\end{ganttchart}

\pagebreak

\begin{figure}[hp]
\begin{ganttchart}[
	today={37}, % the largest value, as it is finished on 07-02-2015 (febr)
	today rule/.style={draw=blue, dashed, ultra thick},
	today label=7 febr 2015,
	progress=today,
	progress label text={},
    %hgrid,vgrid,
    y unit chart=.9cm,
    inline,
    group/.append style={draw=black,fill=blue!90},
    bar/.append style={draw=black,fill=green!50}
    ]{0}{37}
  \gantttitle{2014}{37} \\
  \gantttitle{september}{19}
  \gantttitle{oktober}{18} \\
  \gantttitlelist{12,...,30}{1}
  \gantttitlelist{1,...,18}{1}\\

  \ganttmilestone{Startbijeenkomst}{0.5} \\
  \ganttbar{Teamafspraken}{2}{7} \\
  \ganttbar[bar inline label node/.style={anchor=west},bar inline label anchor=east]{Stakeholder meeting}{8}{8} \\
  \ganttbar{Probleemdefinitie}{9}{27} \\
  \ganttbar{High-level requirements}{9}{27} \\
  \ganttmilestone{Concept planning}{27} \\

  \ganttbar{Verbeteringen}{29}{36} \\
  \ganttmilestone{Stakeholder consensus}{36}

  \ganttlink[link type=dr]{elem0}{elem1}
  \ganttlink[link type=dr]{elem1}{elem2}
  \ganttlink[link type=dr]{elem2}{elem3}
  \ganttlink[link type=dr]{elem2}{elem4}
  \ganttlink[link type=dr]{elem3}{elem5}
  \ganttlink[link type=dr]{elem4}{elem5}
  \ganttlink[link type=dr]{elem5}{elem6}
  \ganttlink[link type=dr]{elem6}{elem7}

\end{ganttchart}
\caption{Detailplanning 'Planning' (gereed bij wijziging op 7 febr 2015)}
\end{figure}

\pagebreak

\begin{figure}[hp]
\begin{ganttchart}[
	today={32}, % the largest value, as it is finished on 07-02-2015 (febr)
	today rule/.style={draw=blue, dashed, ultra thick},
	today label=7 febr 2015,
	progress=today,
	progress label text={},
    %hgrid,vgrid,
    y unit chart=.9cm,
    inline,
    group/.append style={draw=black,fill=blue!90},
    bar/.append style={draw=black,fill=green!50}
    ]{0}{32}
  \gantttitle{Domeinanalyse}{32}
  \ganttnewline
  \gantttitle{2014}{33} \\
  \gantttitle{oktober}{15}
  \gantttitle{november}{18} \\
  \gantttitlelist{17,...,31}{1}
  \gantttitlelist{1,...,18}{1}\\

  \ganttmilestone[name=consensus, milestone inline label node/.style={anchor=west},
  					milestone inline label anchor=east]{Stakeholder consensus}{0.5} \\
  \ganttbar[name=demo,
	bar inline label node/.style={anchor=west},bar inline label anchor=east]{Demonstratie WickedXmas}{2}{2} \\
  %% Domein analyse Guus (GB)
  \ganttbar[name=gb1]{Domeinanalyse GB}{3}{22}
  \ganttbar[name=gb2,bar/.style={fill=gray}]{}{23}{29}
  \ganttlinkedmilestone[name=dagb,progress label text={}]{}{30}\\
  %% Domein analyse Jeroen (JK)
  \ganttbar[name=jk1]{}{3}{4}
  \ganttlinkedbar[name=jk2]{Domeinanalyse JK}{9}{27}
  \ganttbar[name=jk3,bar/.style={fill=gray}]{}{28}{29}
  \ganttlinkedmilestone[name=dajk]{}{30}\\
  %% Domein analyse Stefan (SV)
  \ganttbar[name=sv1]{}{3}{7}
  \ganttlinkedbar[name=sv2]{Domeinanalyse SV}{15}{29}
  \ganttlinkedmilestone[name=dasv]{}{30}\\

  \ganttlink[link type=dr]{consensus}{demo}
  \ganttlink[link type=dr]{demo}{gb1}
  \ganttlink[link type=dr]{demo}{jk1}
  \ganttlink[link type=dr]{demo}{sv1}

\end{ganttchart}
\caption{Detailplanning 'Domeinanalyse' (gereed bij wijziging op 7 febr 2015)}
\end{figure}

%%\restoregeometry
\end{landscape}


\begin{figure}[h]
\begin{tabular}{ll}\hline
{\bf Fase}    & {\bf weken}\\\hline
Planning             & 2,5 \\
Domeinanalyse        & 3 \\
\hline
Iteratie 0           & 3 \\
Iteratie 1           & 3 \\
Stap 1				& 11 \\
Stap 2				& 8 \\
Stap 3				& 6 \\
\hline
Onderzoekscontext     &	 \\
Afronding	     & 3 \\
\hline
Totaal               & 40 \\
\end{tabular}
\caption{Studielast taken}
\end{figure}

\begin{figure}[h]
\begin{tabular}{lllp{7cm}}\hline
{} & {\bf Week}    & {\bf Taak}  & {\bf Toelichting}\\\hline
2014 & 38-42         & Planning    \\
2014 & 42-45         & Domeinanalyse & 4 weken ingepland i.v.m. vakanties Stefan \& Jeroen \\
2014 & 46-48         & Iteratie 0    \\
2014 & 49-51         & Iteratie 1    \\
2014 & 52-1          &               & vakantieperiode, mogelijk
				individueel (bijvoorbeeld onderzoekscontext) \\
2015 & 2-10           & Stap 1    \\
2015 & 11-17           & Stap 2    \\
2015 & 18-22          & Stap 3    \\

2015 & 11-22          & Onderzoekcontext & Uitvoering parallel aan de iteraties \\
2015 & 23-25         & Afronding     \\

2015 & 26            & Mijlpaal einde project (presentatie)
\end{tabular}
\caption{Planning taken}
\end{figure}

\begin{itemize}
 \item looptijd project: tot begin juli (40 weken)
 \begin{itemize}
  \item 40 weken effectief voor het project  (40 weken * 15 uur/week = 600 uur)
  \item 3 weken vakantie
  \item geen speling
 \end{itemize}
\end{itemize}

\pagebreak

\paragraph{Capaciteitsplanning}

\begin{aanpassing1}
	\begin{itemize}
	 \item Beschikbare tijd
	 \begin{itemize}
	  \item $\sim$15 uur per week per teamlid
	  \item {\gray $\sim$30 uur gedurende het hele project voor Freek en $\sim$20 uur voor Bernard
	        gereserveerd vanuit het ABI. Freek en Bernard hebben echter aangegeven dat zij graag
	        intensief betrokken worden tijdens het project. Zij zullen dus meer tijd willen
	        en kunnen vrijmaken. Afgesproken is dat wij op elk moment contact op kunnen nemen.
	        Zodra wij te veel van hun tijd in beslag nemen zullen zij dat aangeven.
	        }% gray
	  \item Bernard heeft weinig tijd door zijn nieuwe verantwoordelijkheden. Freek is voor
	  		3 maanden naar Amerika. We limiteren het contact met Freek tot incidenteel email 
	  		en met Bernard tot email en geplande skype sessies. 
	 \end{itemize}


	 \item Vakanties / niet beschikbaar voor project
	 \begin{itemize}

	  \item Stefan
	  \begin{itemize}
	   \item {\gray 25-10-2014 t/m 31-01-2014 (week 44)}
	   \item {\gray 24-12-2014 t/m 02-01-2015 (week 52-01)}
	  \end{itemize}

	  \item Jeroen
	  \begin{itemize}
	   \item {\gray 21-10-2014 t/m 25-10-2014 (week 43)}
	  \end{itemize}

	  \item Bernard
	   \begin{itemize}
	    \item {\gray 04-10-2014 t/m 14-10-2014 (week 41-42)}
	    \item {\gray 20-10-2014 t/m 27-10-2014 (week 43-44)}
	    \item 09-01-2015 t/m 19-01-2015 (week 2-3)
	   \end{itemize}

	 \end{itemize}
	\end{itemize}
\end{aanpassing1}
