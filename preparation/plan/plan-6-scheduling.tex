%%
%% Dit is een subdocument van het projectplan.
%%
%%  hier worden tijden en eventueel personen toegekend aan activiteiten
%%

\begin{landscape}
\section{Schedule}
 %% dit onderdeel moet vermoedelijk in het schedule plan

%\pagebreak
\newgeometry{top=0.5cm,left=0.5cm,bottom=0.5cm,right=-5cm}

\begin{figure}[hp]
\begin{ganttchart}[
   y unit chart=.9cm,
   today=4
 ]{0}{37}
 \gantttitle{2014}{15}
 \gantttitle{2015}{22} \\
 \gantttitlelist{38,...,52}{1}
 \gantttitlelist{1,...,22}{1} \\

 \ganttgroup{Inceptie}{1}{8} \\
 \ganttbar{Planning}{1}{4} \\

 \ganttmilestone{Stakeholder consensus}{4} \\

 \ganttbar{Domeinanalyse}{5}{8} \\


 \ganttmilestone{Domeinanalyse}{8} \\

 \ganttgroup{Constructie}{9}{28} \\

 \ganttbar{Iteratie 0}{9}{11} \\
 \ganttmilestone{Proven Architecture}{11} \\
 \ganttbar{Iteratie 1}{12}{14} \\
 \ganttbar{Iteratie 2}{17}{19} \\
 \ganttbar{Iteratie 3}{20}{22} \\
 \ganttbar{Iteratie 4}{23}{25} \\
 \ganttbar{Iteratie 5}{26}{28} \\

 \ganttmilestone{Sufficient functionality}{28} \\

 \ganttgroup{Transitie}{29}{31} \\
 \ganttbar{Onderzoekscontext}{20}{29} \\
 \ganttbar{Afronding}{30}{32}
 \ganttbar[bar/.style={fill=gray}]{}{32}{37} \\

 \ganttmilestone{Delighted stakeholders}{37}

 \ganttlink{elem1}{elem2}
 \ganttlink{elem2}{elem3}
 \ganttlink{elem3}{elem4}
 \ganttlink{elem4}{elem6}
 \ganttlink{elem6}{elem7}
 \ganttlink{elem7}{elem8}
 \ganttlink{elem8}{elem9}
 \ganttlink{elem9}{elem10}
 \ganttlink{elem10}{elem11}
 \ganttlink{elem11}{elem12}
 \ganttlink{elem12}{elem13}
 \ganttlink{elem13}{elem16}
 \ganttlink{elem15}{elem16}
 \ganttlink{elem17}{elem18}

\end{ganttchart}
\caption{Globale planning met hoofdfasen en belangrijkste mijlpalen}
\end{figure}

\pagebreak

\begin{figure}[hp]
\begin{ganttchart}[
    today=36
  ]{0}{36}
  \gantttitle{2014}{36} \\
  \gantttitle{september}{18}
  \gantttitle{oktober}{18} \\
  \gantttitlelist{13,...,30}{1}
  \gantttitlelist{1,...,18}{1}\\

  \ganttmilestone{Startbijeenkomst}{0.5} \\
  \ganttbar{Teamafspraken}{2}{7} \\
  \ganttbar{Stakeholder meeting}{8}{8} \\
  \ganttbar{Probleemdefinitie}{9}{27} \\
  \ganttbar{High-level requirements}{9}{27} \\
  \ganttmilestone{Concept planning}{27} \\

  \ganttbar{Verbeteringen}{28}{36} \\
  \ganttmilestone{Stakeholder consensus}{36}

  \ganttlink{elem0}{elem1}
  \ganttlink{elem1}{elem2}
  \ganttlink{elem2}{elem3}
  \ganttlink{elem2}{elem4}
  \ganttlink{elem3}{elem5}
  \ganttlink{elem4}{elem5}
  \ganttlink{elem5}{elem6}
  \ganttlink{elem6}{elem7}

\end{ganttchart}
\caption{Detailplanning 'Planning'}
\end{figure}

\pagebreak

\begin{figure}[hp]
\begin{ganttchart}[]{0}{32}
  \gantttitle{Domeinanalyse}{32}
  \ganttnewline
  \gantttitle{2014}{32} \\
  \gantttitle{oktober}{14}
  \gantttitle{november}{18} \\
  \gantttitlelist{18,...,31}{1}
  \gantttitlelist{1,...,18}{1}\\

  \ganttmilestone{Stakeholder consensus}{0.5} \\
  \ganttbar{Demonstratie WickedXmas}{2}{2} \\
  \ganttbar{Domeinanalyse GB}{3}{22}
  \ganttbar[bar/.style={fill=gray}]{}{23}{29} \\
  \ganttbar{Domeinanalyse JK}{3}{4}
  \ganttbar{}{9}{27}
  \ganttbar[bar/.style={fill=gray}]{}{28}{29} \\
  \ganttbar{Domeinanalyse SV}{3}{7}
  \ganttbar{}{15}{29} \\
  \ganttmilestone{Domeinanalyse}{30}

  \ganttlink{elem0}{elem1}
  \ganttlink{elem1}{elem2}
  \ganttlink{elem1}{elem4}
  \ganttlink{elem1}{elem7}
  \ganttlink{elem3}{elem9}
  \ganttlink{elem6}{elem9}
  \ganttlink{elem8}{elem9}

\end{ganttchart}
\caption{Detailplanning 'Domeinanalyse'}
\end{figure}

%%\restoregeometry
\end{landscape}


\begin{figure}[h]
\begin{tabular}{ll}\hline
{\bf Fase}    & {\bf weken}\\\hline
Planning             & 2,5 \\
Domeinanalyse        & 3 \\
\hline
Iteratie 0           & 3 \\
Iteratie 1           & 3 \\
Iteratie 2           & 3 \\
Iteratie 3           & 3 \\
Iteratie 4           & 3 \\
Iteratie 5           & 3 \\
\hline
Onderzoekscontext     &	2 \\
Afronding	     & 1.5 \\
\hline
Totaal               & 27 \\
\end{tabular}
\caption{Studielast taken}
\end{figure}

\begin{figure}[h]
\begin{tabular}{llp{7cm}}\hline
{\bf Week}    & {\bf Taak}  & {\bf Toelichting}\\\hline
38-42         & Planning    \\
42-45         & Domeinanalyse & 4 weken ingepland i.v.m. vakanties Stefan \& Jeroen \\
46-48         & Iteratie 0    \\
49-51         & Iteratie 1    \\
52-1          &               & vakantieperiode, mogelijk
				individueel (bijvoorbeeld onderzoekscontext) \\
2-4           & Iteratie 2    \\
5-7           & Iteratie 3    \\
8-10          & Iteratie 4    \\
11-13         & Iteratie 5    \\

5-14          & Onderzoekcontext & Uitvoering parallel aan de iteraties \\
15-26         & Afronding     \\

22            & Mijlpaal einde project
\end{tabular}
\caption{Planning taken}
\end{figure}






\begin{itemize}
 \item looptijd project: tot eind mei (36 weken)
 \begin{itemize}
  \item 27 weken effectief voor het project  (27 weken * 15 uur/week = 405 uur)
  \item 3 weken vakantie
  \item 6 weken speling
 \end{itemize}
 \item De detailplanning van een fase wordt telkens in de voorafgaande fase uitgewerkt.
\end{itemize}



\paragraph{Planning iteraties}

\begin{figure}[ht]
 \begin{ganttchart}[
 ]{0}{21}
  \gantttitle{Planning iteraties}{21} \\
  \gantttitlelist{1,...,21}{1} \\
  \ganttbar[bar label font=\color{blue}]{Overleg start iteratie}{1}{1} \\
  \ganttbar{Requirements}{1}{7} \\
  \ganttbar{Ontwikkeling}{3}{19} \\
  \ganttbar{Documentatie}{10}{19} \\
  \ganttbar[bar label font=\color{blue}]{Overleg voortgang}{12}{12} \\
  \ganttmilestone{Release}{19} \\
  \ganttbar[bar label font=\color{blue}]{Demonstratie software}{20}{20} \\
  \ganttbar{Scriptieverslag}{20}{21} \\
  \ganttbar{Onderzoekscontext}{20}{21} \\
  \ganttmilestone{Einde iteratie}{21}
  
  \ganttlink{elem2}{elem5}
  \ganttlink{elem3}{elem5}
  \ganttlink{elem5}{elem6}
 \end{ganttchart}
 \caption{Globale planning iteraties}
\end{figure}


Elke iteratie omvat een vaste periode van 3 weken.
De exacte verdeling van de beschikbare tijd over de taken verschilt per iteratie.
In de eerste iteraties zal bijvoorbeeld relatief meer tijd worden besteed aan requirements.
Aan het einde van een iteratie wordt een voorlopige release van de software aan de
opdrachtgever verstrekt. De laatste twee dagen zijn gereserveerd voor het werken
aan het scriptieverslag en de onderzoekscontext. In deze periode vindt
ook een evaluatie plaats van de afgelopen iteratie. \\
Voor het vaststellen van de requirements zal regelmatig overleg met Bernard plaats
moeten vinden. Hiervoor is het belangrijk dat in de detailplanning van een iteratie
rekening wordt gehouden met de beschikbaarheid van Bernard.

\pagebreak

\paragraph{Capaciteitsplanning}

\begin{itemize}
 \item Beschikbare tijd
 \begin{itemize}
  \item $\sim$15 uur per week per teamlid
  \item $\sim$30 uur gedurende het hele project voor Freek en $\sim$20 uur voor Bernard
 \end{itemize}



 \item vakanties / niet beschikbaar voor project
 \begin{itemize}

  \item Stefan
  \begin{itemize}
   \item 25-10-2014 t/m 31-01-2014 (week 44)
   \item 24-12-2014 t/m 02-01-2015 (week 52-01)
  \end{itemize}

  \item Jeroen
  \begin{itemize}
   \item 21-10-2014 t/m 25-10-2014 (week 43)
  \end{itemize}

  \item Bernard
   \begin{itemize}
    \item 04-10-2014 t/m 14-10-2014 (week 41-42)
    \item 20-10-2014 t/m 27-10-2014 (week 43-44)
   \end{itemize}

 \end{itemize}
\end{itemize}
