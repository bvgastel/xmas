%%
%% This is part of quick-plan.tex
%%
\begin{frame}
    \section{Project}

    Guus Bonnema
\end{frame}

\begin{frame}[label=outline]
    \frametitle{Outline}
    \tableofcontents[pausesections]
\end{frame}



\begin{frame}[label=opdracht]{Overview deel 1}{Project Opdracht}

    \begin{columns}
        \begin{column}<1->{.4\textwidth}
	    Project opdracht

	    \begin{itemize}
		\item business case
		\item vision
		\item Challanges and goals (*)
		\item Succesfactoren (*)
		\item Risico (*)
	    \end{itemize}
        \end{column}
        \begin{column}<2->{.4\textwidth}
	    Project uitvoering

	    \begin{itemize}
		\item iteratief
		\item tijdgedreven
		\item samenwerking $\rightarrow$ rolverdeling
		\item hulpmiddelen (*)
	    \end{itemize}
        \end{column}


    \end{columns}



\end{frame}

\begin{frame}[label=challanges]{Overview deel 2}{Project Uitvoering}

    \begin{columns}[t]
        \begin{column}<1->{.5\textwidth}
	    Uitdagingen
	    \begin{description}
	        \item[Platformen] {\tiny MS Windows, Mac en Linux.}
	        \item[Integratie] {\tiny Verficatie tools.}
	        \item[Onderhoud] {\tiny Velen gaan dit onderhoud plegen.}
		\item[Uitbreiding] {\tiny researchtool $\rightarrow$ uitbreidbaar.}
		\item[Functie] {\tiny als WickedXMAS}
	    \end{description}


        \end{column}
        \begin{column}<2->{.4\textwidth}
	    Succes factoren
	    \begin{itemize}
	        \item {\tiny design tool te gebruiken op alle platformen}
	        \item {\tiny functioneert gelijkwaardig aan WickedXMas}
	    \end{itemize}
	    Groot success
	    \begin{itemize}
	        \item {\tiny validatie en verificatie optioneel, wijzigbaar}
	        \item {\tiny primitieven defini\"{e}ren}
	    \end{itemize}
        \end{column}

    \end{columns}


\end{frame}

\begin{frame}[fragile]{Hoofdrisico's}

    \par{\sf Geografische spreiding \pause $\rightarrow$ direct gevolgen voor communicatie, indirect voor tempo en kwaliteit}
    \pause
    \vspace{3em}
    \par{\sf Stakeholder beschikbaarheid \pause $\rightarrow$ gevolgen voor tempo en kwaliteit}
    \pause
    \vspace{3em}
    \par{\sf Interne risico's \pause $\rightarrow$ gevolgen voor tempo en kwaliteit}

\end{frame}


\begin{frame}[fragile]{Overview deel 1}{Risico en oorzaak}

\begin{center}
    \tiny\sf
    \begin{tabular}{|c|p{23em}|p{13em}|}
	\hline
	1	& \multicolumn{2}{c|}{\sf\emph{\normalsize Vertraging}}
		\\\hline
		& Stakeholders niet beschikbaar wanneer dat gewenst is
		& skype(1),email(2),contact momenten(1a)
		\\\hline
		& Agile methode is nieuw voor teamleden.
		& DAD(8), tools(6a, 6b, 6c)
		\\\hline
		& OU heeft traag support
		& Github (6a)
		\\\hline
		& Geografische spreiding leidt tot communicatie problemen
		    met kwaliteitsvermindering tot gevolg
		& skype(4), coordinatie(5), tools (6a, 6b)
		\\\hline
		& C++ is nog relatief nieuw voor 2 van de 3 programmeurs
		& leren(9), review(10)
		\\\hline
		& Het kost tijd om de bestaande verificatietools en de achterliggende
		    \xmas\ -materie op te nemen.
		& tijd inplannen (11 en onderzoekscontext)
		\\\hline
	2 	& \multicolumn{2}{c|}{\sf\emph{\normalsize Kwaliteit}}
		\\\hline
		& Agile is nieuw voor teamleden
		& agile (7), agile tool (6b)
		\\\hline
		& Structuurverval bij nieuwe features is een natuurlijk
		gevolg
		& Vaak refactoren (3), Architectuur(12)
		\\\hline
		& Geografische spreiding leidt tot communicatie problemen
		met kwaliteitsvermindering tot gevolg
		& skype(4), teamviewer(?), agile tool(6b)
		\\\hline
		& Ervaring met C++ beperkt met mogelijke gevolgen
		voor de kwaliteit (fouten, best practices missen)
		& leren(9), review(10)
		\\\hline
	3	& \multicolumn{2}{c|}{\sf\emph{\normalsize Product}}
		\\\hline
		& Verificatie tools merken een fout niet op met consequenties voor
		  in productie name van chips (Zie open vragen).
		& testen en feedback users, eventueel beta versie
		\\\hline
        \end{tabular}
\end{center}

\end{frame}


\begin{frame}[fragile,label=hulpmiddelen]{Overview deel 2}{Hulpmiddelen}

    Hulpmiddelen

    \begin{columns}[c]
	\begin{column}{10em}
	    \begin{itemize}
		\item<1> git
		\item<1> github
		\item<2> google group
		\item<3> agilefant
	    \end{itemize}
	\end{column}

	\begin{column}{30em}
	    \begin{center}
		\begin{tabular}{l}
		    \includegraphics<1>[height=.25\textheight]{github-example}\\
		    \includegraphics<2>[height=.25\textheight]{google-groups-example}\\
		    \includegraphics<3>[height=.25\textheight]{agilefant-example}\\
		\end{tabular}
	    \end{center}
	\end{column}


    \end{columns}


\end{frame}