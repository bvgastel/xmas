%%
%% This is part of quick-plan.tex
%%
\begin{frame}[label=outline]
    \frametitle{Outline}
    \tableofcontents[pausesections]
\end{frame}

\section{Project}

\begin{frame}{Outline}{Inhoud project}
    \tableofcontents[currentsection]
\end{frame}


\begin{frame}[fragile]{Overview deel 1}{Risico's}

    {\tiny
    \begin{tabular}{|c|c|p{15em}|p{30em}|}
     \hline
     {\bf } & {\bf nr} & {\bf Risico} & {\bf Toelichting} \\\hline
    \ok & 1 & Beschikbaarheid stakeholders  & De opdrachtgever en de begeleider hebben een beperkte tijd
 					    beschikbaar. De andere stakeholders zijn niet beschikbaar.\\\hline\pause
     \ok & 5 & Geografische spreiding & Het team woont te ver uit elkaar om face to face meetings te
 				    organiseren voor overleg tijdens het project.\\\hline\pause
      \ok & 2 & Geen agile ervaring & De teamleden hebben vooral ervaring met waterval projecten.\\\hline\pause
     \ok & 7 & twee onervaren C++ programmeurs & Het team heeft \'e\'en ervaren C++ programmeur.\\\hline\pause
     \ok & 3 & Trage support & De support van de OU lijkt niet al te voortvarend. Dit kan problemen veroorzaken.\\\hline\pause
     \ok & 4 & Tijd nodig om de materie kennis op te doen & Het kost tijd om de bestaande analyse
 					    tools en de achterliggende
 					    xmas materie op te nemen.\\\hline\pause
     \ok & 6 & structuur verval bij ontwikkeling & Bij het toevoegen van functionaliteit is structuur verval een
 				    natuurlijk gevolg.\\\hline\pause
     \ding{"38} & 8 & Fouten in ontwerp & Bij het ontwerpen kunnen fouten ontstaan die de controles niet ontdekken\\\hline

    \end{tabular}
    }

\end{frame}

\begin{frame}[label=opdracht]{Overview deel 1}{Project Opdracht}

    \begin{columns}
        \begin{column}<1->{.4\textwidth}
	    Project opdracht

	    \begin{itemize}
		\item business case
		\item vision
		\item Challanges and goals
		\item Succesfactoren
		\item Risico
	    \end{itemize}
        \end{column}
        \begin{column}<2->{.4\textwidth}
	    Project uitvoering

	    \begin{itemize}
		\item iteratief
		\item tijdgedreven
		\item samenwerking $\rightarrow$ rolverdeling
		\item hulpmiddelen
	    \end{itemize}
        \end{column}


    \end{columns}



\end{frame}

\begin{frame}[label=kenmerken]{Overview deel 2}{Project Uitvoering}


\end{frame}

\begin{frame}[fragile,label=hulpmiddelen]{Overview deel 2}{Hulpmiddelen}

    Hulpmiddelen

    \begin{columns}[c]
	\begin{column}{10em}
	    \begin{itemize}
		\item<1> git
		\item<1> github
		\item<2> google group
		\item<3> agilefant
	    \end{itemize}
	\end{column}

	\begin{column}{30em}
	    \begin{center}
		\begin{tabular}{l}
		    \includegraphics<1>[height=.25\textheight]{github-example}\\
		    \includegraphics<2>[height=.25\textheight]{google-groups-example}\\
		    \includegraphics<3>[height=.25\textheight]{agilefant-example}\\
		\end{tabular}
	    \end{center}
	\end{column}


    \end{columns}



\end{frame}