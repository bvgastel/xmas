%%
%% Dit is een subdocument van het projectplan.
%%

\section{Domeinanalyse}

De domeinanalyse ligt op het domein van de klant: het geeft een deelprobleem.
Met een domeinanalyse licht je \'e\'en deel van het probleem uit het geheel
en bestudeer je dat gedetailleerder en wetenschappelijker dan de andere
deelproblemen.

Ervaring wijst uit dat het belangrijk is de domeinanalyses als eerste te doen.
Dat ondersteunt het project en voorkomt
de domeinanalyse aan het eind ``nog even te moeten schrijven''.

De twee belangrijke kenmerken van een domeinanalyse zijn dat het een beperkt
deel van het onderwerp is en dat het daadwerkelijk
ondersteunend is aan het project. De valkuil is om een te breed onderwerp te
nemen. Hieronder de voorbeelden van goede onderwerpen
voor een domeinanalyse die voor ons als input hebben gediend.

De uitkomsten van een domeinanalyse zijn een beschrijving van het probleem,
een beschrijving van de alternatieven, een beschrijving van de keuze criteria en
een gewogen aanbeveling. Gewogen betekent dat de aanbeveling naar objectieve
criteria plaatsvindt. De beslissing is een team effort, waarbij de klant
de doorslaggevende stem heeft. De uitwerking evenals de beoordeling van
een domeinanalyse is individueel.

In het project kiezen we voor de grootste risico's eerst. Dit zijn met name
de integratie en de user interface met ontwikkelomgeving.
Hiermee halen we het grootste risico uit het project. Zie figuur \ref{fig: domain-analysis}
voor de uitwerking per persoon. De keuze van het team is voor Guus de UI toolkit,
voor Jeroen de integratie (met combinatorische cycle checker) en voor Stefan de
combinatorial objects.

\begin{center}
    \begin{tabular}{|p{13em}|p{22em}|}
    \hline
        {\bf vb}		& {\bf beschrijving} \\\hline

Platform onafhankelijk toolkit &
Uitgaande van de high level requirements voor het ontwerp tool, zoals platform
onafhankelijkheid, uitbreidbaarheid en hechte integratie: kies een user interface
toolkit die het beste past bij dit project en dit probleem gebied.\\\hline

Integratie van verificatietools &
Een analyse van de huidige en mogelijke toekomstige verificatietools. Welke
datastructuren gebruiken zij intern en welke datastructuren kunnen de tools
met elkaar delen? Hoe kunnen de verificatietools worden ge\"integreerd met
de ontwerptool? De combinatorische cycle checker is een goede basis om mee
te beginnen.\\\hline

combinatorial objects &
een onderzoek hoe om te gaan met combinatorial objects. Bv hoe grafisch weer
te geven, hoe op te slaan in de data structuur.\\\hline

    \end{tabular}
\end{center}


\begin{figure}[!h]
    \subfloat[UI toolkit]{%
    \begin{boxedminipage}[b]{.32\textwidth}
	{\small\sf
	    Student: Guus Bonnema.
	    \paragraph{Goal}
	    Find the toolkit combination that optimally satisfies the
	    high level requirements as specified in the project plan.
	    \paragraph{In scope}
	    The high level architecture and the user interface toolkit.
	    Influence on IPC toolkits, concurrency toolkits en integration toolkits.
	    The influence on development environment.
	    \paragraph{Out of scope}
	    Any non UI toolkits are out of scope, unless a clear
	    relationship exists.
	    \paragraph{Overlap}
	    Overlap with integration as many toolkits have approaches to
	    integrating components. Also the constraints that follow
	    from the domain analysis on integration will influence the resulting
	    decision.
	    \paragraph{Results}
	    The domain analysis results in advice on use of toolkits and on time
	    necessary to master the toolkits.
	}% small
    \end{boxedminipage}
    }% subfig
%    \caption{Scope of toolkit domain analysis}
%    \label{fig: da-toolkit}
%\end{figure}
    \subfloat[Combinatorial objects]{%
    \begin{boxedminipage}[b]{.32\textwidth}
	{\small\sf%%%%%%%%%%%%%%%%%%%%%%%%%%
	    Student: Stefan Versluys.
	    \paragraph{Goal}
	    Find a way of how combinatorial objects can be structured and
visualised in such a way that the high level requirements are optimally
satisfied.
	    \paragraph{In scope}
	    The concept of how combinatorial objects its data can be structured
and how these can be visualised.
	    \paragraph{Out of scope}
	    GUI: How to draw or show objects on a screen.
	    Verification: How to exchange data structures with the API.
	    \paragraph{Overlap}
	    It overlaps both , the GUI toolkit which concerns visulatisation and
verification tools which concerns efficient data structures and exchange.
	    \paragraph{Results}
	    This analysis will show what options that are available in contrast
with efficiency and usability.
	}% small %%%%%%%%%%%%%%%%%%%%%%%%%%%%%
    \end{boxedminipage}
    }% subfig
%    \caption{Scope definition of combinatorial objects}
%    \label{fig: da-objects}
%\end{figure}
    \subfloat[Integration of verification tools]{%
    \begin{boxedminipage}[b]{.32\textwidth}
	{\small\sf%%%%%%%%%%%%%%%%%%%%%%%%%%
	    Student: Jeroen Kleijn.
	    \paragraph{Goal} Integrate the verification tools, initially the
	    combinatoric cycle checker, with the design tool. Determine the specific
	    datastructures and programming interfaces shared between verification tools
	    and design tool. Determine communication, integration, extensibility of
	    verification tools.
	    \paragraph{In scope} Integration of verification tools. Combinatoric cycle checker.
	    \paragraph{Out of scope} Visual representation of feedback in the design tool
	    \paragraph{Overlap} The choices made with regards to datastructures have
	    implementation consequences for the way the design tool must provide data to the
	    verification tools and vice versa (feedback).
	    \paragraph{Results} This analysis will provide input to the architecture of the
	    design tool, specifically the integration of verification tools with the design
	    tool.
	}% small %%%%%%%%%%%%%%%%%%%%%%%%%%%%%
    \end{boxedminipage}
    }% subfig

%    \paragraph{Requirements for domain analysis}
    \begin{boxedminipage}[c]{.98\textwidth}
    \begin{itemize}
	\item At least an initial idea of the high level view of the system
		should be available. The final high level
		view will depend on the choices made here and for the toolkits.
	\item The high level requirements must be defined and agreed on.
    \end{itemize}
    \end{boxedminipage}


    \caption{Scope domain analyses}
    \label{fig: domain-analysis}
\end{figure}
