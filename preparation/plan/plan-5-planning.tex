%%
%% Dit is een subdocument van het projectplan.
%%
%% items op basis van SE Sommerville p623 (chapter 23.2.1 Project plans)
%%  	sommige items zijn elders in het document reeds opgenomen,  zoals bvb risico's
%%

\section{Planning}
\subsection{Indeling activiteiten over de drie DAD fasen}
%%zie Sommerville SE 2.4 The RUP (ed9 blz50-51)
\begin{enumerate}
\item \underline{\textbf{Inceptie (2 iteraties)}}
%%http://disciplinedagiledelivery.wordpress.com/2012/11/11/comparing-dad-to-the-rational-unified-process-rup-part-2/
		\begin{itemize}
		\item Iteratie ``Planning''
			\begin{itemize}
			\item Team afspraken.
			\item Stakeholder meeting.
			\item Probleem definitie.
			\item High level requirements.
			\item Planning.
			\end{itemize}
		\paragraph{Artefacten}
		De teamleden hebben de nodige afspraken gemaakt met betrekking tot
		werkwijzen zoals communicatiemiddelen en versiebeheer voor documentatie.
		Deze fase levert een document dat het probleem omschrijft,  de
		high-level requirements, business case, risiso's, stakeholders, vision
		and challenges, een plan van aanpak en een planning voor de eerstvolgende
		iteratie.
		\item Iteratie ``Domeinanalyse''
			\begin{itemize}
			\item Domeinanalyse.
			\item Requirements: Op basis van sourcecode van de huidige WickedXmas tool.
			\item Requirements: Demonstratie van de huidige WickedXmas tool door
			stakeholder.
			\item Initi\"ele architectuur visie
			\end{itemize}
		\paragraph{Artefacten}
		De resultaten van de domeinanalyse en requirements op basis van observatie.
		Er is een visie m.b.t. de initi\"ele architectuur.
		Documentatie.
		\end{itemize}

\item \underline{\textbf{Constructie ({\gray 6 iteraties} 3 stappen + 2 keer onderzoekscontext)}}
	\begin{itemize}
	\item Iteratie 0
		\begin{itemize}
		\item Architectuur : Afbakenen , configureren en testen
		\item documentatie aanpassen
		\end{itemize}
		\paragraph{Artefacten op de ``Proven architecture'' mijlpaal}
		 De Architectuur is bepaald en getest en klaar voor evaluatie, documentatie.
	\item Onderzoekscontext
	\item Iteratie 1
		\begin{itemize}
		\item WickedXmas editor ontwikkelen
		\item documentatie aanpassen
		\end{itemize}
	\item {\bf Stap 1\marginpar{[febr 2015]} Initi\"{e}le integratie}
		Deze stap levert een werkend systeem op, dat de meest simpele netwerken
		kan tekenen, kan opslaan en weer kan laden.
		\begin{itemize}
			\item integratie VT datamodel met user interface
			\item opslaan data in JSON in oorspronkelijke vorm
		\end{itemize}
	\item {\bf Stap 2\marginpar{[febr 2015]} Composite toevoegen}
		Deze stap levert een werkend systeem op, dat een composite kan tekenen en opslaan.
		\begin{itemize}
			\item Uitbreiden VT datamodel met composite objects
			\item Platslaan uitgebreide VT datamodel naar plat VT datamodel
			\item Opslaan datamodel aangepast aan user interface en uitgebreid met composite
		\end{itemize}
	\item {\bf Stap 3\marginpar{[febr 2015]} Achtergrond VT}
		Deze stap levert een werkend systeem op, dat een netwerk kan tekenen met primitives
		en composites, deze kan opslaan en laden, en specifieke VTs op de achtergrond kan
		uitvoeren. Van de VTs toont het textuele meldingen en/of kleurt het foutmelding in de grafiek in.
		\begin{itemize}
			\item Design-VT interface met bestaande C++ verificatie tools
			\item syntax en checker VT integreren met user interface op de achtergrond
			\item VT foutmeldingen tonen (optioneel inkleuren)
		\end{itemize}
		\paragraph{\gray Artefacten}
{\gray		Prototype, documentatie
		\item Onderzoekscontext
	\item Iteratie 2
		\begin{itemize}
		\item WickedXmas editor ontwikkelen
		\item documentatie aanpassen
		\end{itemize}
		}%\gray
		\paragraph{\gray Artefacten}
{\gray		Prototype, documentatie
	\item Iteratie 3
		\begin{itemize}
		\item WickedXmas editor ontwikkelen
		\item documentatie aanpassen
		\end{itemize}
		}% \gray
		\paragraph{\gray Artefacten}
{\gray		Prototype, documentatie
	\item Iteratie 4
		\begin{itemize}
		\item WickedXmas verificatietoolinterface ontwikkelen
		\item documentatie aanpassen
		\end{itemize}
		}%\gray
		\paragraph{\gray Artefacten}
{\gray		Prototype, documentatie
	\item Iteratie 5
		\begin{itemize}
			\item WickedXmas verificatietoolinterface ontwikkelen
			\item documentatie aanpassen
		\end{itemize}
		}%\gray
		\paragraph{Artefacten op de ``Sufficient functionality'' mijlpaal }
		 Een release van de WickedXmas Tool zoals beoogd werd,  documentatie
	\end{itemize}
\item \underline{\textbf{Finale transitie (1 iteratie)}}
	\begin{itemize}
		\item {\gray Finale releases van de nieuwe WickedXmas tool.}
		\item Finale releases van de nieuwe design tool.
		\item {\gray Documentatie bundelen (handleiding).}
		\item Documentatie aanpassen, uitbreiden en bundelen (handleiding).
		\item Onderzoekscontext
		\item Presentatie geven
	\end{itemize}
	\paragraph{Artefacten op de ``Production ready'' mijlpaal}
	 Een presentatie, documentatie en scriptieverslag van de nieuwe {\gray WickedXmas} Design tool.

\end{enumerate}