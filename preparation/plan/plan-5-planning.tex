%%
%% Dit is een subdocument van het projectplan.
%%
%% items op basis van SE Sommerville p623 (chapter 23.2.1 Project plans)
%%  	sommige items zijn elders in het document reeds opgenomen,  zoals bvb risico's
%%

\section{Planning}
\subsection{Indeling activiteiten over de drie DAD fasen}
%%zie Sommerville SE 2.4 The RUP (ed9 blz50-51)
\begin{enumerate}
\item \underline{\textbf{Inceptie (2 iteraties)}}
%%http://disciplinedagiledelivery.wordpress.com/2012/11/11/comparing-dad-to-the-rational-unified-process-rup-part-2/
		\begin{itemize}
		\item Iteratie ``Planning''
			\begin{itemize}
			\item Team afspraken.
			\item Stakeholder meeting.
			\item Probleem definitie.
			\item High level requirements.
			\item Planning.
			\end{itemize}
		\paragraph{Artefacten}
		De teamleden hebben de nodige afspraken gemaakt m.b.t. werkwijzen zoals
		communicatie middelen en versiebeheer voor documentatie. Deze fase levert
		een document dat het probleem omschrijft,  de high-level requirements,
		business case, risiso's, stakeholders, vision and challenges, een plan van
		aanpak en een planning voor de eerst volgende iteratie.
		\item Iteratie ``Domein onderzoek''
			\begin{itemize}
			\item Domein onderzoek.
			\item Requirements: Demonstratie van de huidige WickedXmas tool door
			stakeholder.
			\item Requirements: Op basis van sourcecode van de huidige WickedXmas tool.
			\item Initi\"ele architectuur visie
			\end{itemize}
		\paragraph{Artefacten}
		De resultaten van het domein onderzoek, eisen uit de observatie.
		De architectuur is bepaald en beschreven, het team heeft de IDE met tools
		ge\"{i}nstalleerd en getest.
		\end{itemize}


\item \underline{\textbf{Constructie (6 iteraties)}}
	\begin{itemize}
	\item Iteratie 0
		\begin{itemize}
		\item Modelling,  use cases
		\item Architectuur : afbakening,  onderdelen,  testen, evalueren
		\item documentatie aanpassen
		\end{itemize}
		\paragraph{Artefacten op de ``Proven architecture'' mijlpaal}
		 Architectuur getest en goed bevonden, documentatie
	\item Iteratie 1
		\begin{itemize}
		\item WickedXmas editor ontwikkelen
		\item documentatie aanpassen
		\end{itemize}
		\paragraph{Artefacten}
		Prototype, documentatie
	\item Iteratie 2
		\begin{itemize}
		\item WickedXmas editor ontwikkelen
		\item documentatie aanpassen
		\end{itemize}
		\paragraph{Artefacten}
		Prototype, documentatie
	\item Iteratie 3
		\begin{itemize}
		\item Start onderzoekcontext\footnote{De inhoud en diepgang van de onderzoekscontext is nog vaag. Dat moet na de 2e iteratie
		duidelijker zijn.}
		\item WickedXmas editor ontwikkelen
		\item documentatie aanpassen
		\end{itemize}
		\paragraph{Artefacten}
		Prototype, documentatie
	\item Iteratie 4
		\begin{itemize}
		\item WickedXmas analyse tool interface ontwikkelen
		\item documentatie aanpassen
		\end{itemize}
		\paragraph{Artefacten}
		Prototype, documentatie
	\item Iteratie 5
		\begin{itemize}
		\item WickedXmas analyse tool interface ontwikkelen
		\item documentatie aanpassen
		\end{itemize}
		\paragraph{Artefacten op de ``Sufficient functionality'' mijlpaal }
		 Een release van de WickedXmas Tool zoals beoogd werd,  documentatie
	\end{itemize}

\item \underline{\textbf{Finale transitie (1 iteratie)}}
	\begin{itemize}
		\item Finale releases van de nieuwe WickedXmas tool.
		\item Documentatie bundelen (handleiding).
		\item Onderzoekscontext afwerken.
		\item Presentatie geven
	\end{itemize}
	\paragraph{Artefacten op de ``Production ready'' mijlpaal}
	Een presentatie, documentatie, de nieuwe WickedXmas tool en de onderzoekscontext.

\end{enumerate}