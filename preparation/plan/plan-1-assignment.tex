%%
%% Dit is een subdocument van het projectplan.
%%

\section{Opdracht}

\subsection{Business case}

De projectaanvraag voor dit project van Bernard van Gastel en
Freek Verbeek verwoordt de business case in de projectaanvraag
als volgt.

\begin{quote}
    \small\sf
    ``De Network-on-Chip (NoC) groep van de OU doet onderzoek
    naar nieuwe methoden om NoC ontwerpen (moderne manier van
    ontwerpen van processoren) foutvrij te krijgen met behoud van
    efficiëntie. De NoC groep heeft een aantal tools ter ondersteuning
    van het onderzoek gemaakt (met ondersteuning van vele studenten
    projecten), voornamelijk WickedXmas. Deze tool stelt de
    gebruiker in staat om chip ontwerp te maken/bewerken/genereren
    in de visuele xMAS taal van Intel, en vervolgens een aantal door
    ons ontwikkelde tools op los te laten (bv symbolische analyse,
    deadlock checker, etc).
    De huidige tool heeft een aantal problemen:

    \begin{itemize}
	\item niet modulair opgezet (waardoor uitbreidingen moeizaam gaan)
	\item op Windows API gebaseerd (waardoor de onderzoekers die
	    gebruik maken van Mac het lastig kunnen gebruiken)
	\item moeizame integratie met tools: WickedXmas is nu geschreven
	    in C\#, en lijkt moeilijk te integreren met onze C/C++ tools
	\item geen documentatie
    \end{itemize}

    Deze problemen moeten opgelost worden, danwel door een grote
    refactoring van de bestaande code, danwel door het opnieuw
    bouwen.''
\end{quote}

Omdat dit het afstudeerproject is van drie studenten, zijn er geen financi\"ele overwegingen.
De kosten die er zijn (inzet opdrachtgever en begeleider) vallen onder het afstuderen. Deze kosten
zijn al gedekt als onderdeel van de cursus Afstudeerproject Bachelor Informatica (ABI).

De baten zijn vooral betere onderhoudbaarheid en uitbreidbaarheid en een bredere ondersteuning van platformen.
Deze verbeteringen ondersteunen met name het chipontwerponderzoek aan universiteiten en bedrijven. Dankzij een betere integratie
tussen ontwerptool en verificatietools optimaliseren de onderzoekers hun workflow. Het aantal handelingen
dat zij moeten uitvoeren is minder dan met het huidig tool en de directe feedback in de user interface
versnelt het lokaliseren en verhelpen van fouten in een ontwerp.

\subsection{Vision}\label{sec: vision}

Deze sectie geeft aan wat de opdrachtgever ziet als ideale uitkomst van het project. Het ontwerptool is
bedoeld voor wetenschappers van universiteiten en van bedrijven (o.a. Intel). Onderzoekers op verschillende
platformen kunnen het ontwerptool gemakkelijk downloaden, installeren en gebruiken. Het tool toont op verzoek
tijdens het ontwerp fouten die door de verificatietools zijn gevonden.

Het programma is goed gestructureerd, uit te breiden met nieuwe verificatietools en NoC onderzoekers
kunnen gemakkelijk nieuwe primitieven defini\"eren. Ten slotte kunnen de \xmas\ -onderzoekers onze bestanden
converteren naar een ander formaat zoals Verilog.

\subsection{Stakeholders}

Opdrachtgever is Bernard van Gastel van de Open Universiteit. Begeleider is Freek Verbeek van de Open Universiteit.
De doelgroep voor gebruikers van het ontwerptool bestaat uit medewerkers van universiteiten en onderzoekers van bedrijven zoals
Intel en LLC. Een kleine groep van \xmas\ -onderzoekers van zowel de universiteit als bedrijven gebruiken het ontwerptool
vanuit de optiek van onderzoek naar verbetering van NoC ontwerp. Doelstellingen voor deze groep zijn chip import
en export van bestandsformaten (zoals Verilog) en verificatie van aspecten van correctheid (deadlock, livelock,
syntactische of semantische checks). Bernard en Freek nemen het onderhoud van het tool voor hun rekening.

%% Opmerking: stakeholders splitsen in 2 tabellen. 1 tabel met de doelstelling en toelichting daarop
%%            1 tabel met per stakeholder de project en systeem-belangen en -doelstellingen

\begin{figure}
{\small\sf
\begin{center}
\begin{tabular}{lll}\hline
{\bf Stakeholder}    & {\bf Projectrollen}   & {\bf omgevingsrollen} \\\hline
Bernard van Gastel   & Opdrachtgever         & Onderzoeker, gebruiker, ontwikkelaar\\
Freek Verbeek        & Begeleider            & Onderzoeker, gebruiker, ontwikkelaar \\
Team33               & Ontwikkelaar, student & \\
Univ. medewerkers    &                       & Onderzoeker, onderzoeker NoC, gebruiker \\
Bedrijfsmedewerkers  &                       & Onderzoeker, gebruiker \\
\hline
\end{tabular}
\end{center}
}% end tiny
\caption{Stakeholders en hun rollen}\label{fig:stakeholders}
\end{figure}

{\small\sf
\begin{center}
\begin{tabular}{llllll}\hline
{\bf Stakeholder}    & {\bf Verband}   & {\bf Rol}     & {\bf Freq} & {\bf Belang} & {\bf Invloed}\\\hline
Bernard van Gastel   & Onderzoek NoC  & Opdrachtgever & Hoog       & Hoog   & Hoog \\
                     & Onderhoud      & Opdrachtgever & Laag       & Hoog   &  \\
                     & Gebruik        & Gebruiker     &            & Hoog   & \\
Freek Verbeek        & Onderzoek NoC  & Begeleider    & Hoog       & Hoog   & Hoog\\
                     & Onderhoud      & Opdrachtgever & Laag       & Hoog   & \\
                     & Onderhoud      & Ontwikkelaar  & Laag       & Hoog   & \\
                     & Gebruik        & Gebruiker     &            & Hoog   & \\
Univ. medewerkers    & Onderzoek alg. & Gebruiker     & Laag       & Middel & \\
                     & Onderzoek NoC  & Gebruiker     & Middel     & Hoog   & \\
                     & Gebruik        &               &            & Middel & \\
Bedrijfsonderzoekers & Onderzoek NoC  & -             & Middel     & Hoog   & \\
                     & Gebruik        & Gebruiker     &            & Middel & \\
                     & Installatie    & Gebruiker     &            & Hoog   & \\
Bedrijfsmedewerkers  & Gebruik        & Gebruiker     & Laag       & Middel & \\
                     & Installatie    & Gebruiker     &            & Hoog   & \\
Team33               & Ontwikkeling   & Ontwikkelaar  & Hoog       & Hoog   & Hoog\\
                     & Afstuderen     & Student       &            & Hoog   & Hoog\\
                     & Samenwerking   & Ontwikkelaar  & Hoog       & Hoog   & Hoog\\
\hline
\end{tabular}
\end{center}
}% end tiny

\subsection{Challenges and goals}\label{sec: challenges goals}

Hieronder een lijst van uitdagingen (challenges). Tijdens de voorbereidingsfasen werken we de uitdagingen uit tot high level requirements.

\begin{description}
 \item[platformen] Het onderzoeksteam werkt met Mac OS, ontwikkelteam met MS Windows en Linux, de opdrachtgever en studenten met MS Windows, Linux of Mac OS.
		    Dit genereert requirements voor platformonafhankelijkheid van ontwikkelomgeving, taal en toolkits die we gebruiken.
 \item[integratie] Het systeem bestaat uit meerdere modulen, grofweg aangeduid als ontwerptool (wat ons onderwerp is) en verificatietools (tools die
		    deel gebouwd en deels in ontwikkeling zijn). De uitdaging is een hoge graad van integratie van deze componenten, zo onafhankelijk mogelijk.
		    Dit genereert requirements voor onderhoudbaarheid, componentinterfaces en applicatiestructuur. De verificatietools zijn alle C++ programma's.
		    De interfaces zijn gebaseerd op JSON.
 \item[onderhoud] Over de tijd heen onderhouden vele mensen de software, die elkaar niet spreken. Dit genereert requirements met betrekking tot documentatie,
		    onderhoudbaarheid en toegankelijkheid.
 \item[Uitbreiding] De huidige software is moeilijk uitbreidbaar op de wijze die de opdrachtgever graag wil. De wens is relatief gemakkelijk een nieuwe
		    verificatiemodule te kunnen toevoegen.
 \item[functie] Het ontwerptool ondersteunt de primitieven en hun samenwerking zoals gespecificeerd in het \xmas\ paper \cite{chatterjee-kishinevsky:xmas}.
		Het  huidige ontwerp tool: WickedXMas geeft de richting aan implementatie van de functionaliteit en de interface, maar is
		niet beperkend of maatgevend daarvoor.
\end{description}


\subsection{Succesfactoren}
Het project is een succes als

\begin{itemize}
 \item het ontwerptool een ontwerp kan maken in de \xmas\ taal en de verificatietools kan aanroepen voor controle.
 \item Gebruikers van het tool op de volgende platforms kan draaien:
 \begin{itemize}
    \item MS Windows vanaf versie 7 64 bits (32 bits via emulatie)
    \item Linux 64 bits en 32 bits
    \item Mac OS X (versie?)
 \end{itemize}
\end{itemize}

Het project is een groot succes als het project een succes is en:

\begin{itemize}
 \item De gebruiker kan validatie- en verificatiesoftware aanzetten of uitzetten
 \item De gebruiker kan nieuwe primitieven maken en aankoppelen.
\end{itemize}

Het project is mislukt als het ontwerp tool geen succes is.

\subsection{Risico}

De onderkende risico's zijn vooral projectrisico's. De meeste projectrisico's komen voort uit de
geografische spreiding van het team (i.e. de studenten) en opdrachtgever
en begeleider. Diverse maatregelen zijn dan ook gericht op het verminderen
van de risico's die de spreiding veroorzaakt.

Het gaat om herstructureren dan wel herbouwen van bestaande programmatuur. Daarom zijn er weinig systeemrisico's.
De inhoudelijke kennis is bij de opdrachtgever en de begeleider aanwezig. Het grootste systeemrisico is het kiezen
van een user interface toolkit en de ontwikkelomgeving. Dit onderdeel bepaalt de platformonafhankelijkheid
en heeft invloed op de architectuur. Het plan besteedt apart aandacht aan het onderdeel in een domeinanalyse om
vroegtijdig dit risico weg te nemen.

Zie figuur \ref{fig: risico} voor een overzicht van de risico's die we onderkennen en figuur \ref{fig: risico reductie} voor
een overzicht van de maatregelen die we nemen ter compensatie van risico's. Voor elk risico op \'e\'en na hebben we een
maatregel bedacht. Het risico voor fouten in het ontwerp accepteren we maar zie paragraaf \ref{sec: open-punten} voor
een vraag hierover bij de open punten.

\begin{center}
    \label{fig: risico}
    \sf
    \tablecaption{Relevante risico's voor dit project}
    \tablehead{\hline & \multicolumn{2}{c|}{\bf risico }\\\hline
    {} & \multicolumn{1}{c|}{\bf oorzaak} & \multicolumn{1}{c|}{\bf maatregel}\\\hline}
    \tabletail{\hline \multicolumn{3}{r}{\emph{verder op de volgende pagina}}\\}
    \tablelasttail{\hline \multicolumn{3}{r}{\emph{Einde tabel relevante risico's}}\\}
    \small\sf
    \begin{supertabular}{|c|p{23em}|p{13em}|}
	1	& \multicolumn{2}{c|}{\sf\emph{\large vertraging in het project of de oplevering}}
		\\\hline
		& Stakeholders niet beschikbaar wanneer dat gewenst is
		& skype(1),email(2),contact momenten(1a)
		\\\hline
		& Agile methode is nieuw voor teamleden.
		& DAD(8), tools(6a, 6b, 6c)
		\\\hline
		& OU heeft traag support
		& Github (6a)
		\\\hline
		& Geografische spreiding leidt tot communicatie problemen
		    met kwaliteitsvermindering tot gevolg
		& skype(4), coordinatie(5), tools (6a, 6b)
		\\\hline
		& C++ is nog relatief nieuw voor 2 van de 3 programmeurs
		& leren(9), review(10)
		\\\hline
		& Het kost tijd om de bestaande verificatietools en de achterliggende
		    \xmas\ -materie op te nemen.
		& tijd inplannen (11 en onderzoekscontext)
		\\\hline
	2 	& \multicolumn{2}{c|}{\sf\emph{\large Systeem kwaliteitsafname}}
		\\\hline
		& Agile is nieuw voor teamleden
		& agile (7), agile tool (6b)
		\\\hline
		& Structuurverval bij nieuwe features is een natuurlijk
		gevolg
		& Vaak refactoren (3), Architectuur(12)
		\\\hline
		& Geografische spreiding leidt tot communicatie problemen
		met kwaliteitsvermindering tot gevolg
		& skype(4), teamviewer(?), agile tool(6b)
		\\\hline
		& Ervaring met C++ beperkt met mogelijke gevolgen
		voor de kwaliteit (fouten, best practices missen)
		& leren(9), review(10)
		\\\hline
	3	& \multicolumn{2}{c|}{\sf\emph{\large Product kwaliteitsafname}}
		\\\hline
		& Verificatie tools merken een fout niet op met consequenties voor
		  in productie name van chips\footnote{Zie open vragen}.
		& testen en feedback users, eventueel beta versie
		\\\hline
        \end{supertabular}
\end{center}


\begin{center}
    \label{fig: risico reductie}
    \small\sf
    \sf
    \tablecaption{Maatregelen ter vermindering van risico's}
    \tablehead{\hline {} & \multicolumn{1}{c|}{\bf Maatregel} & \multicolumn{1}{c|}{\bf toelichting}\\\hline}
    \tabletail{\hline \multicolumn{3}{r}{\emph{verder op de volgende pagina}}\\}
    \tablelasttail{\hline \multicolumn{3}{r}{\emph{Einde tabel maatregelen ter vermindering van risico's}}\\}
    \begin{supertabular}{|r|p{17em}|p{20em}|}
    \hline
	{\bf id} & {\bf Maatregel} & {\bf Toelichting} \\\hline
	1 & Gepland overleg via skype & Dit verzekert een minimaal contact met opdrachtgever en begeleider \\\hline
	1a & vaste communicatiemomenten afspreken & Dit vermindert het effect van risioco 1 (stakeholder beschikbaarheid)\\\hline
	2 & Tussentijds contact via email & Dit vult de communicatietijd aan tot wat nodig is. Nadeel is
					een kans op vertraging.\\\hline
	3 & Refactoring & Een refactor na uitvoeren en testen van een taak, levert
					    een goed gestructureerd systeem na elke iteratie.\\\hline
	4 & Skype en chat creatief en vaak gebruiken & Verhoogt de gelijkenis met lokaal samenwerken\\\hline
	5 & Op afspraak tegelijkertijd bouwen & Verhoogt de kans om samen te werken\\\hline
	6 & Ondersteuning zoeken buiten de OU & Een trage ondersteuning kan op kritieke momenten
					    het gehele project vertragen. Door minder op
					    OU support te leunen, vermijden we het risico\\\hline
	6a & Gebruik Github	& Vermijdt risico traag OU support\\\hline
	6b & Gebruik agile tool & Vermindert risico van kwaliteitsafname\\\hline
	6c & Gebruik mailing list & Verbetert communicatie over email\\\hline
	7 & Agile literatuur lezen & Door ons actief in agile te verdiepen, verkleinen we de kans op
				    problemen met het proces\\\hline
	8 & DAD methode met ondersteunend tool kiezen & DAD is beter geschikt voor mensen die minder
	ervaring met agile hebben. Het tool zorgt voor gemakkelijkere toepassing van DAD.\\\hline
	9 & C++ studie doen & Door actief ons C++ 2011 eigen te maken, kunnen we
			    met onze achtergrondkennis van programmeertalen en
			    van Java, het risico op vertraging voor zijn\\\hline
	10 & Review door de ervaren C++ programmeur & Learning on the job onder begeleiding van
				    het teamlid dat C++ ervaring heeft\\\hline
	11 & Voldoende tijd voor voorfase & De ontwikkeling pas starten na de domeinanalyse en het bepalen van de
			    architectuur. Dit beperkt het gevaar van te vroeg implementeren\\\hline
	12 & Architectuur & Vergroot de kans dat de structuur a priori geschikt is voor de applicatie en
			    voor de nieuwe features.\\\hline
    \end{supertabular}
\end{center}