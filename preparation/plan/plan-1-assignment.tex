%%
%% Dit is een subdocument van het projectplan.
%%

\section{Opdracht}

\subsection{Business case}

De projectaanvraag voor dit project van Bernard van Gastel en
Freek Verbeek verwoordt de business case in de projectaanvraag
als volgt.

\begin{quote}
    \tiny
    De Network-on-Chip (NoC) groep van de OU doet onderzoek
    naar nieuwe methoden om NoC ontwerpen (moderne manier van
    ontwerpen van processoren) foutvrij te krijgen met behoud van
    efficiëntie. De NoC groep heeft een aantal tools ter ondersteuning
    van het onderzoek gemaakt (met ondersteuning van vele studenten
    projecten), voornamelijk WickedXmas. Deze tool stelt de
    gebruiker in staat om chip ontwerp te maken/bewerken/genereren
    in de visuele xMAS taal van Intel, en vervolgens een aantal door
    ons ontwikkelde tools op los te laten (bv symbolische analyse,
    deadlock checker, etc).
    De huidige tool heeft een aantal problemen:

    \begin{itemize}
	\item niet modulair opgezet (waardoor uitbreidingen moeizaam gaan)
	\item op Windows API gebaseerd (waardoor de onderzoekers die
	    gebruik maken van Mac het lastig kunnen gebruiken)
	\item moeizame integratie met tools: WickedXmas is nu geschreven
	    in C\#, en lijkt moeilijk te integreren met onze C/C++ tools
	\item geen documentatie
    \end{itemize}

    Deze problemen moeten opgelost worden, danwel door een grote
    refactoring van de bestaande code, danwel door het opnieuw
    bouwen.
\end{quote}

Omdat dit het afstudeerproject is van drie studenten, zijn er geen financi\"ele overwegingen.
De kosten die er zijn (inzet opdrachtgever en begeleider) vallen onder het afstuderen. Deze kosten
zijn al gedekt als onderdeel van de cursus Afstudeer Bachelor Informatica (ABI).

De baten zijn vooral betere onderhoudbaarheid en uitbreidbaarheid en een bredere ondersteuning van platform.
Deze verbeteringen ondersteunen met name het chip ontwerp onderzoek aan universiteiten en bedrijven. Dankzij een betere integratie
tussen ontwerptool en analysetools optimaliseren de onderzoekers hun workflow. Het aantal handelingen
dat zij moeten uitvoeren zijn minder dan met het huidig tool en de directe feedback in de user interface
versnelt het lokaliseren en verhelpen van fouten in een ontwerp.

\subsection{Vision}\label{sec: vision}

Deze sectie geeft aan wat de opdrachtgever ziet als ideale uitkomst van het project. Het ontwerp tool is
bedoeld voor wetenschappers van universiteiten en van bedrijven (o.a. Intel). Onderzoekers op verschillende
platformen kunnen het ontwerp tool gemakkelijk downloaden, installeren en gebruiken. Het tool toont op verzoek
tijdens het ontwerp fouten van verificatie tools.

Het programma is goed gestructureerd, uit te breiden met nieuwe analyses en verificaties en NoC onderzoekers
kunnen gemakkelijk nieuwe primitieven defini\"eren. Ten slotte kunnen de \xmas\ onderzoekers onze bestanden
converteren naar een ander formaat zoals Verilog.

\subsection{Stakeholders}

Opdrachtgever is Bernard van Gastel van de Open Universiteit. Begeleider is Freek Verbeek van de Open Universiteit.
De doelgroep voor gebruikers van het ontwerp tool zijn medewerkers van universiteiten en onderzoekers van bedrijven zoals
Intel en LLC. Een kleine groep van xMAS onderzoekers van zowel de universiteit als bedrijven gebruiken het ontwerp tool
vanuit de optiek van onderzoek naar verbetering van NoC ontwerp. Doelstellingen voor deze groep zijn chip import
en export van bestandsformaten (zoals Verilog) en verificatie van aspecten van correctheid (deadlock, livelock,
syntactische of semantische checks). Bernard en Freek nemen het onderhoud van het tool voor hun rekening.

%% Opmerking: stakeholders splitsen in 2 tabellen. 1 tabel met de doelstelling en toelichting daarop
%%            1 tabel met per stakeholder de project en systeem-belangen en -doelstellingen

\begin{figure}
{\tiny
\begin{center}
\begin{tabular}{lll}\hline
{\bf Stakeholder}    & {\bf Projectrollen}   & {\bf omgevingsrollen} \\\hline
Bernard van Gastel   & Opdrachtgever         & Onderzoeker, gebruiker, ontwikkelaar\\
Freek Verbeek        & Begeleider            & Onderzoeker, gebruiker, ontwikkelaar \\
Team33               & Ontwikkelaar, student & \\
Univ. medewerkers    &                       & Onderzoeker, onderzoeker NoC, gebruiker \\
Bedrijfsmedewerkers  &                       & Onderzoeker, gebruiker \\
\hline
\end{tabular}
\end{center}
}% end tiny
\caption{Stakeholders en hun rollen}\label{fig:stakeholders}
\end{figure}

{\tiny
\begin{center}
\begin{tabular}{llllll}\hline
{\bf Stakeholder}    & {\bf Verband}   & {\bf Rol}     & {\bf Freq} & {\bf Belang} & {\bf Invloed}\\\hline
Bernard van Gastel   & Onderzoek NoC  & Opdrachtgever & Hoog       & Hoog   & Hoog \\
                     & Onderhoud      & Opdrachtgever & Laag       & Hoog   &  \\
                     & Gebruik        & Gebruiker     &            & Hoog   & \\
Freek Verbeek        & Onderzoek NoC  & Begeleider    & Hoog       & Hoog   & Hoog\\
                     & Onderhoud      & Opdrachtgever & Laag       & Hoog   & \\
                     & Onderhoud      & Ontwikkelaar  & Laag       & Hoog   & \\
                     & Gebruik        & Gebruiker     &            & Hoog   & \\
Univ. medewerkers    & Onderzoek alg. & Gebruiker     & Laag       & Middel & \\
                     & Onderzoek NoC  & Gebruiker     & Middel     & Hoog   & \\
                     & Gebruik        &               &            & Middel & \\
Bedrijfsonderzoekers & Onderzoek NoC  & -             & Middel     & Hoog   & \\
                     & Gebruik        & Gebruiker     &            & Middel & \\
                     & Installatie    & Gebruiker     &            & Hoog   & \\
Bedrijfsmedewerkers  & Gebruik        & Gebruiker     & Laag       & Middel & \\
                     & Installatie    & Gebruiker     &            & Hoog   & \\
Team33               & Ontwikkeling   & Ontwikkelaar  & Hoog       & Hoog   & Hoog\\
                     & Afstuderen     & Student       &            & Hoog   & Hoog\\
                     & Samenwerking   & Ontwikkelaar  & Hoog       & Hoog   & Hoog\\
\hline
\end{tabular}
\end{center}
}% end tiny

\subsection{Challenges and goals}\label{sec: challenges goals}

Hieronder een lijst van uitdagingen (challenges). Tijdens de voorbereidingsfasen werken we de uitdagingen uit tot high level requirements.

\begin{description}
 \item[platformen] Het onderzoeksteam werkt met Mac OS, ontwikkelteam met MS Windows en Linux, de opdrachtgever en studenten met MS Windows, Linux of Mac OS.
		    Dit genereert requirements voor platform onafhankelijkheid van ontwikkelomgeving, taal en toolkits die we gebruiken.
 \item[integratie] Het systeem bestaat uit meerdere modulen, grofweg aangeduid als ontwerp tool (wat ons onderwerp is) en analyse tools (tools die nog
		    in ontwikkeling zijn, maar deels al gebouwd). De uitdaging is een hoge graad van integratie van deze componenten, zo onafhankelijk mogelijk.
		    Dit genereert requirements voor onderhoudbaarheid, component interfaces, applicatie structuur. De analyse tools zijn alle C++ programma's.
		    De interfaces zijn gebaseerd op JSON.
 \item[onderhoud] Over de tijd heen onderhouden vele mensen de software, die elkaar niet spreken. Dit genereert requirements met betrekking tot documentatie,
		    onderhoudbaarheid en toegankelijkheid.
 \item[Uitbreiding] De huidige software is moeilijk uitbreidbaar op de wijze die de opdrachtgever graag wil.
 \item[functie] Het ontwerp tool ondersteunt de primitieven en hun samenwerking zoals gespecificeerd in het xmas paper \cite{chatterjee-kishinevsky:xmas}.
		Het  huidige ontwerp tool: WickedXMas geeft de richting aan implementatie van de functionaliteit en de interface, maar is
		niet beperkend of maatgevend daarvoor.
\end{description}


\subsection{Succesfactoren}
Het project is een succes als

\begin{itemize}
 \item het design tool een ontwerp kan maken in de xmas taal en de analyse tools kan aanroepen voor controle.
 \item Gebruikers van het tool op de volgende platforms kan draaien:
 \begin{itemize}
    \item MS Windows vanaf versie 7 64 bits (32 bits via emulatie)
    \item Linux 64 bits en 32 bits
    \item Mac OS X (versie?)
 \end{itemize}
\end{itemize}

Het project is een groot succes als het project een succes is en:

\begin{itemize}
 \item De gebruiker kan validatie en verificatie software aanzetten of uitzetten
 \item De gebruiker kan nieuwe primitieven maken en aankoppelen.
\end{itemize}

Het project is mislukt als het ontwerp tool geen succes is.

\subsection{Risico}

De onderkende risico's zijn vooral project risico's. De meeste project risico's komen voort uit de
geografische spreiding van het team (i.e. de studenten) en opdrachtgever
en begeleider. Diverse maatregelen zijn dan ook gericht op het verminderen
van het risico's die de spreiding veroorzaakt.


Omdat het gaat om herstructureren/herbouwen van bestaande programmatuur zijn er weinig systeem risico's.
Inhoudelijke kennis is bij de opdrachtgever en de begeleider voorhanden. Het grootste systeem risico is het kiezen
van een user interface toolkit en de ontwikkelomgeving. Dit onderdeel bepaalt de platform onafhankelijkheid
en heeft invloed op de architectuur. Het plan besteedt apart aandacht aan dit onderdeel in een domeinanalyse.

Zie figuur \ref{fig: risico} een overzicht van de risico's die we onderkennen en figuur \ref{fig: risico reductie} voor
een overzicht van de maatregelen die we nemen ter compensatie van risico's. Voor elk risico op \'e\'en na hebben we een
maatregel bedacht. De laatste accepteren we\footnote{Zie de open punten, voor een vraag hierover.}.

\begin{figure}[ht]
\begin{center}
\tiny
\begin{tabular}{|c|c|p{20em}|p{30em}|}
\hline
{\bf } & {\bf nr} & {\bf Risico} & {\bf Toelichting} \\\hline
  \ok & 1 & Beschikbaarheid stakeholders  & De opdrachtgever en de begeleider hebben een beperkte tijd
					beschikbaar. De andere stakeholders zijn niet beschikbaar.\\\hline
  \ok & 2 & Geen agile ervaring & De teamleden hebben vooral ervaring met waterval projecten.\\\hline
  \ok & 3 & Trage support & De support van de OU lijkt niet al te voortvarend. Dit kan problemen veroorzaken.\\\hline
  \ok & 4 & Tijd nodig om de materie kennis op te doen & Het kost tijd om de bestaande analyse
					tools en de achterliggende
					xmas materie op te nemen.\\\hline
  \ok & 5 & Geografische spreiding & Het team woont te ver uit elkaar om face to face meetings te
				organiseren voor overleg tijdens het project.\\\hline
  \ok & 6 & structuur verval bij ontwikkeling & Bij het toevoegen van functionaliteit is structuur verval een
				natuurlijk gevolg.\\\hline
  \ok & 7 & twee onervaren C++ programmeurs & Het team heeft \'e\'en ervaren C++ programmeur.\\\hline
  \ding{"38} & 8 & Fouten in ontwerp & Bij het ontwerpen kunnen fouten ontstaan die de controles niet ontdekken\\\hline

\end{tabular}
\end{center}
 \caption{Relevante risico's voor dit project}
 \label{fig: risico}
\end{figure}


\begin{figure}[!ht]
\begin{center}
\tiny
\begin{tabular}{|p{20em}p{30em}c|}
\hline
    {\bf Maatregel} & {\bf Toelichting} & {\bf risico} \\\hline
    Gepland overleg via skype & Dit verzekert een minimaal contact met opdrachtgever en begeleider & 1\\\hline
    Tussentijds contact via email & Dit vult de communicatietijd aan tot wat nodig is. Nadeel is
				    een kans op vertraging. & 1\\\hline
    Regelmatige refactor (per taak) & Een refactor na uitvoeren en testen van een taak, levert
					een goed gestructureerd systeem na elke iteratie. & 6\\\hline
    Skype en chat creatief en vaak gebruiken & Verhoogt de gelijkenis met lokaal samenwerken & 5\\\hline
    Op afspraak tegelijkertijd bouwen & Verhoogt de kans om samen te werken & 5\\\hline
    Ondersteuning zoeken buiten de OU & Een trage ondersteuning kan op kritieke momenten
					het gehele project vertragen. Door minder op
					OU support te leunen, verminderen we die afhankelijkheid & 3\\\hline
    Agile literatuur lezen & Door ons actief in agile te verdiepen, verkleinen we de kans op
				problemen met het proces & 2\\\hline
    C++ studie doen & Door actief ons C++ 2011 eigen te maken, kunnen we
			met onze achtergrond kennis van programmeer talen en
			van Java, het risico op vertraging voor zijn & 7\\\hline
    Review door de ervaren C++ & Learning on the job onder begeleiding van
				het teamlid dat C++ ervaring heeft & 7\\\hline
    Voldoende tijd voor voorfase & De ontwikkeling pas starten na de domein
				analyse en de het bepalen van de architectuur.
				Dit beperkt het gevaar van te vroeg implementeren & 4\\\hline
\end{tabular}
\end{center}
 \caption{Risico reductie}\label{fig: risico reductie}
\end{figure}