%%
%% Dit is een subdocument van het projectplan.
%%

\section{Architectuur}

De architectuur levert een gestructureerde indeling van het systeem met als doel om enerzijds de software
toegankelijk te maken voor ontwikkelaars en anderzijds input te leveren voor aanpassingen in de software. Hieronder
een specificatie van de producten die deze activiteit oplevert.

{\small\sf
\begin{center}
\begin{tabular}{lp{30em}}
Logical view & De logische structuur in class en object diagrammen.\\
Process view & De dynamische structuur in state
transition diagrams op systeem niveau.\\
Physical view &  De hardware componenten en de verdeling van
software over de hardware componenten.\\
Guidelines \& constraints & De richtlijnen voor bouw,
test, herstructurering en documentatie.\\
\end{tabular}
\end{center}
}

Door het toepassen van DAD bekomt men op het einde van de inceptie fase een
initi\"ele visie op de architectuur. In het begin van de constructie wordt dan
volgens een "risico-waarde" prioriteit de architectuur bepaald en bereikt men
op de ``Proven Architecture milestone'' een architectuur die bruikbaar en getest
is. Deze bepaalt de onderliggende componenten en hun interfaces.
In principe wijzigt de architectuur niet meer, tenzij een nieuw requirement
daar aanleiding toe geeft.