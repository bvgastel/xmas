\documentclass[a4paper,11pt,singleside]{article}

\usepackage{ucs}
\usepackage{subfigure}
\usepackage[dutch,english]{babel}
\usepackage{graphicx}

\bibliographystyle{alpha}

\usepackage{hyperref}

\author{Guus Bonnema}
\date{24/09/14}

\title{Plan van aanpak voor xmas ontwerp tool}

%% Document is in subdocumenten gesplitst.

\begin{document}

%%
%% Dit is een subdocument van het projectplan.
%%

\section{Opdracht}

\subsection{Business case}

Uit de projectaanvraag voor dit project:

\begin{quote}
    \tiny
    De Network-on-Chip (NoC) groep van de OU doet onderzoek
    naar nieuwe methoden om NoC ontwerpen (moderne manier van
    ontwerpen van processoren) foutvrij te krijgen met behoud van
    efficiëntie. De NoC groep heeft een aantal tools ter ondersteuning
    van het onderzoek gemaakt (met ondersteuning van vele studenten
    projecten), voornamelijk WickedXmas. Deze tool stelt de
    gebruiker in staat om chip ontwerp te maken/bewerken/genereren
    in de visuele xMAS taal van Intel, en vervolgens een aantal door
    ons ontwikkelde tools op los te laten (bv symbolische analyse,
    deadlock checker, etc).
    De huidige tool heeft een aantal problemen:

    \begin{itemize}
	\item niet modulair opgezet (waardoor uitbreidingen moeizaam gaan)
	\item op Windows API gebaseerd (waardoor de onderzoekers die
	    gebruik maken van Mac het lastig kunnen gebruiken)
	\item moeizame integratie met tools: WickedXmas is nu geschreven
	    in C\#, en lijkt moeilijk te integreren met onze C/C++ tools
	\item geen documentatie
    \end{itemize}

    Deze problemen moeten opgelost worden, danwel door een grote
    refactoring van de bestaande code, danwel door het opnieuw
    bouwen.
\end{quote}


\noindent
De business case voor deze aanvraag bestaat uit bruikbaarheid van de ontwerp module
op verschillende platforms, en een hechtere integratie van het ontwerp tool met de
analyse tools.

De platform onafhankelijkheid bevordert beschikbaarheid van het tool voor een grotere
groep onderzoekers. De integratie en betere modulariteit bevordert de onderhoudbaarheid.

Als het project een platform onafhankelijk tool kan maken, dat goed geintegreerd en gestructureerd
is, dan kan de onderzoeks profiteren van een breder publiek, gemakkelijker aanpassingen doen en gemakkelijker
met nieuwe gewijzigde analyse tools integreren.

De baten kunnen wij helaas niet in financie\"ele zin kwantificeren.

\subsection{Risico}

Zie figuur \ref{fig: risico} een overzicht van de risico's die we onderkennen en figuur \ref{fig: risico reductie} voor
een overzicht van de maatregelen die we nemen ter compensatie van risico's.

TODO: Per risico opnemen of we het accepteren of maatregelen nemen (optioneel).


\begin{figure}[ht]
\begin{center}
\small
\begin{tabular}{cp{10em}p{20em}}
{\bf nr} & {\bf Risico} & {\bf Toelichting} \\
 1 & Beschikbaarheid stakeholders  & De opdrachtgever en de begeleider hebben een beperkte tijd
					beschikbaar. De andere stakeholders zijn niet beschikbaar.\\
 2 & Ervaring waterval & De teamleden hebben vooral ervaring met waterval projecten.\\
 3 & Trage support & De support van de OU lijkt niet al te voortvarend. Dit kan problemen veroorzaken.\\
 4 & Tijd nodig om de materie kennis op te doen & Het kost tijd om de bestaande analyse
					tools en de achterliggende
					xmas materie op te nemen.\\
 5 & Geografische spreiding & Het team woont te ver uit elkaar om face to face meetings te
				organiseren voor overleg tijdens het project.\\
 6 & structuur verval bij ontwikkeling & Bij het toevoegen van functionaliteit is structuur verval een
				natuurlijk gevolg.\\
 7 & twee onervaren C++ programmeurs & Het team heeft \'e\'en ervaren C++ programmeur.\\

\end{tabular}
\end{center}
 \caption{Relevante risico's voor dit project}
 \label{fig: risico}
\end{figure}


\begin{figure}[!ht]
\begin{center}
\small
\begin{tabular}{p{10em}p{20em}c}
    {\bf } & {\bf } & {\bf } \\
    Gepland overleg via skype & Dit verzekert een minimaal contact met opdrachtgever en begeleider & 1\\
    Tussentijds contact via email & Dit vult de communicatietijd aan tot wat nodig is. Nadeel is
				    een kans op vertraging. & 1\\
    Regelmatige refactor (per taak) & Een refactor na uitvoeren en testen van een taak, levert
					een goed gestructureerd systeem na elke iteratie. & 6\\
    Skype en chat creatief en vaak gebruiken & Verhoogt de gelijkenis met lokaal samenwerken & 5\\
    Op afspraak tegelijkertijd bouwen & Verhoogt de kans om samen te werken & 5\\
    Ondersteuning zoeken buiten de OU & Een trage ondersteuning kan op kritieke momenten
					het gehele project vertragen. Door minder op
					OU support te leunen, verminderen we die afhankelijkheid & 3\\
    Agile literatuur lezen & Door ons actief in agile te verdiepen, verkleinen we de kans op
				problemen met het proces & 2\\
    C++ studie doen & Door actief ons C++ 2011 eigen te maken, kunnen we
			met onze achtergrond kennis van programmeer talen en
			van Java, het risico op vertraging voor zijn & 7\\
    Review door de ervaren C++ & Learning on the job onder begeleiding van
				het teamlid dat C++ ervaring heeft & 7\\
\end{tabular}
\end{center}
 \caption{Risico reductie}\label{fig: risico reductie}
\end{figure}


\subsection{Vision}\label{sec: vision}

Dit tool is bedoeld voor wetenschappers van universiteiten en van bedrijven (o.a. Intel). We zien een tool dat
is te downloaden en te installeren op alle relevante platforms. De installatie is eenvoudig\footnote{zie open punten}.

We zien een design tool dat dynamisch controles uitvoert\footnote{zie open punten} zodat elk ontwerp van een NoC correct is
zodra het klaar is.

We zien het programma als goed gedocumenteerd en gemakkelijk te onderhouden en uit te breiden.

\paragraph{Nice to have} We zien de onderzoekers naar \xmas nieuwe primitieven bedenken en toevoegen, nieuwe verificatie tools
bedenken en toevoegen. De onderzoekers kunnen onze bestanden naar een ander formaat converteren zoals Verilog.

\subsection{Stakeholders}

Opdrachtgever is Bernard van Gastel van de Open Universiteit. Begeleider is Freek Verbeek van de Open Universiteit.
De doelgroep voor gebruikers van het ontwerp tool zijn medewerkers van universiteiten en onderzoekers van bedrijven zoals
Intel en LLC. Een kleine groep van xMAS onderzoekers van zowel de universiteit als bedrijven gebruiken vanuit de optiek
van onderzoek naar verbetering van NoC ontwerp het ontwerp tool. Doelstellingen voor deze groep zijn chip import
en export van bestandsformaten (zoals Verilog), verificatie van aspecten van correctheid (deadlock, livelock,
syntactische of semantische checks). Bernard en Freek nemen het onderhoud van het tool voor hun rekening.

%% Opmerking: stakeholders splitsen in 2 tabellen. 1 tabel met de doelstelling en toelichting daarop
%%            1 tabel met per stakeholder de project en systeem-belangen en -doelstellingen

\begin{figure}
{\tiny
\begin{center}
\begin{tabular}{lll}\hline
{\bf Stakeholder}    & {\bf Projectrollen}   & {\bf omgevingsrollen} \\\hline
Bernard van Gastel   & Opdrachtgever         & Onderzoeker \\
                     &                       & Gebruiker \\
                     &                       & Ontwikkelaar \\
Freek Verbeek        & Begeleider            & Onderzoeker \\
                     &                       & Gebruiker \\
                     &                       & Ontwikkelaar \\
Team33               & Ontwikkelaar          & \\
                     & Student               & \\
Univ. medewerkers    &                       & Gebruiker \\
                     &                       & Onderzoeker \\
                     &                       & Onderzoeker NoC \\
Bedrijfsmedewerkers  &                       & Gebruiker \\
                     &                       & Onderzoeker \\
\hline
\end{tabular}
\end{center}
}% end tiny
\caption{Stakeholders en hun rollen}\label{fig:stakeholders}
\end{figure}

{\tiny
\begin{center}
\begin{tabular}{llllll}\hline
{\bf Stakeholder}    & {\bf Verband}   & {\bf Rol}     & {\bf Freq} & {\bf Belang} & {\bf Invloed}\\\hline
Bernard van Gastel   & Onderzoek NoC  & Opdrachtgever & Hoog       & Hoog   & Hoog \\
                     & Onderhoud      & Opdrachtgever & Laag       & Hoog   &  \\
                     & Gebruik        & Gebruiker     & ??         & Hoog   & \\
Freek Verbeek        & Onderzoek NoC  & Begeleider    & Hoog       & Hoog   & Hoog\\
                     & Onderhoud      & Opdrachtgever & Laag       & Hoog   & \\
                     & Onderhoud      & Ontwikkelaar  & Laag       & Hoog   & \\
                     & Gebruik        & Gebruiker     &            & Hoog   & \\
Univ. medewerkers    & Onderzoek alg. & Gebruiker     & Laag       & Middel & \\
                     & Onderzoek NoC  & Gebruiker     & Middel     & Hoog   & \\
                     & Gebruik        &               &            & Middel & \\
Bedrijfsonderzoekers & Onderzoek NoC  & -             & Middel     & Hoog   & \\
                     & Gebruik        & Gebruiker     &            & Middel & \\
                     & Installatie    & Gebruiker     &            & Hoog   & \\
Bedrijfsmedewerkers  & Gebruik        & Gebruiker     & Laag       & Middel & \\
                     & Installatie    & Gebruiker     &            & Hoog   & \\
Team33               & Ontwikkeling   & Ontwikkelaar  & Hoog       & Hoog   & Hoog\\
                     & Afstuderen     & Student       &            & Hoog   & Hoog\\
                     & Samenwerking   & Ontwikkelaar  & Hoog       & Hoog   & Hoog\\
\hline
\end{tabular}
\end{center}
}% end tiny

\subsection{Challanges and goals}\label{sec: challanges goals}

Wat zijn uitdagingen en welke high level requirements vloeien hieruit voort? De uitdagingen en requirements
zijn niet volledig. Tijdens de eerste iteratie werkt het team dit uit.

Hieronder een lijst van uitdagingen (challanges):

\begin{description}
 \item[platformen] Het onderzoeksteam werkt met Mac, ontwikkelteam met ms windows en linux, de opdrachtgever en studenten met ms windows, linux of mac.
		    Dit genereert requirements voor platform onafhankelijkheid van ontwikkelomgeving, taal en toolkits die we gebruiken.
 \item[integratie] Het systeem bestaat uit meerdere modulen, grofweg aangeduid als ontwerp tool (wat ons onderwerp is) en analyse tools (tools die nog
		    in ontwikkeling zijn, maar deels al gebouwd). De uitdaging is een hoge graad van integratie van deze componenten, zo onafhankelijk mogelijk.
		    Dit genereert requirements voor onderhoudbaarheid, component interfaces, applicatie structuur. De analyse tools zijn alle C++ programma's.
		    De interfaces zijn gebaseerd op JSON.
 \item[onderhoud] Over de tijd heen onderhouden vele mensen de software, die elkaar niet spreken. Dit genereert requirements met betrekking tot documentatie,
		    onderhoudbaarheid en toegankelijkheid.
 \item[Uitbreiding] De huidige software is moeilijk uitbreidbaar op de wijze die de opdrachtgever graag wil.
 \item[functie] Het ontwerp tool ondersteunt de primitieven en hun samenwerking zoals gespecificeerd in het xmas paper \cite{chatterjee-kishinevsky:xmas}.
		Het  huidige ontwerp tool: WickedXMas geeft de richting aan implementatie van de functionaliteit en de interface, maar is
		niet beperkend of maatgevend daarvoor.
 \item[rest]	Wat zijn de overige uitdagingen? Welke requirements kan de huidige programmatuur moeilijker aan?
\end{description}

De high level doelstellingen van het project:

\begin{description}
 \item[beschikbaarheid]
 \item[hoge integratie met analyse tools]
 \item[Uitbreidbaar met primitieven]
 \item[Uitbreidbaar met analyse en verificatie tools]
 \item[Design functionaliteit]
 \item[Open punten] Zie einde document.
\end{description}

De doelstellingen leiden tot de high level requirements.

\subsection{Succesfactoren}
Het project is een succes als

\begin{itemize}
 \item het design tool een ontwerp kan maken in de xmas taal en de analyse tools kan aanroepen voor controle.
 \item Gebruikers van het tool op de volgende platforms kan draaien:
 \begin{itemize}
    \item ms Windows vanaf versie 7 64 bits (32 bits via emulatie)
    \item Linux 64 bits en 32 bits
    \item Mac OS X (versie?)
 \end{itemize}
\end{itemize}

Het project is een groot succes als het project een succes is en:

\begin{itemize}
 \item De gebruiker kan validatie en verificatie software aanzetten of uitzetten
 \item De gebruiker kan nieuwe primitieven maken en aankoppelen.
\end{itemize}

Het project is mislukt als het ontwerp tool geen succes is.

%%
%% Dit is een subdocument van het projectplan.
%%

\section{Architectuur}
Aan welke eisen moet de architectuur voldoen?
Op welke platformen draait het?
Is het multi user? Is het client/server? Is er een repository?
Is het taalonafhankelijk?

%%
%% Dit is een subdocument van het projectplan.
%%

\section{Domeinanalyse}

De domeinanalyse ligt op het domein van de klant (Bernard): het geeft een deelprobleem. Met een domeinanalyse licht je
\'e\'en deel van het probleem uit het geheel en bestudeer je dat gedetailleerder en wetenschappelijker dan de andere deelproblemen.

Ervaring wijst uit dat het belangrijk is de domeinanalyses als eerste te doen. Dat ondersteunt het project en voorkomt
de domeinanalyse aan het eind ``nog even te moeten schrijven''.

De twee belangrijke kenmerken van een domeinanalyse zijn dat het een beperkt deel van het onderwerp is en dat het daadwerkelijk
ondersteunend is aan het project. De valkuil is om een te breed onderwerp te nemen. Hieronder de voorbeelden van goede onderwerpen
voor een domeinanalyse die voor ons als input hebben gediend.

De uitkomst van een domeinanalyse is een beschrijving van het probleem,
een beschrijving van de alternatieven, een beschrijving van de keuze criteria en
een gewogen aanbeveling. Gewogen betekent dat de aanbeveling naar objectieve criteria plaatsvindt.
De beslissing is een team effort, waarbij de klant (Bernard in dit geval)
de doorslaggevende stem heeft.

De beoordeling van elke domeinanalyse is individueel evenals de uitwerking.

We kiezen er voor om de meest onzekere delen van het project als eerste doen. Hiermee halen we het grootste risico uit het
project. Zie figuur \ref{fig: domain-analysis} voor de uitwerking per persoon.
De keuze van het team is voor Guus de UI toolkit, voor Jeroen de combinatorische cycle checker en voor Stefan de combinatorial objects.

\begin{center}
    \begin{tabular}{|p{2.5cm}|p{10cm}|}
    \hline
        {\bf vb}		& {\bf beschrijving} \\\hline
        Integratie		& \textsc{Structureel}: wat wordt de interface met de analyse tools? \\
	met tools 		& \textsc{Dynamisch}: hoe gaan we de analyse tools integreren in het ontwerp tool?\\
				& \textsc{Functioneel}: hoe gaan we de analyse tools integreren in het ontwerp tool?\\\hline
        Platform onafhankelijk UI toolkit & Uitgaande van de high level requirements voor het ontwerp
					    tool, zoals platform onafhankelijkheid, uitbreidbaarheid
					    en hechte integratie: kies een UI toolkit die het
					    beste past bij dit project en dit probleem gebied.\\\hline
        combinatorische cycle checker & je zou de combinatorische cycle checker kunnen bestuderen inclusief de datastructuur die
					deze tool deelt met de andere analysetools en een domeinanalyse kunnen doen
					over hoe deze tool geintegreerd kan worden met jullie tool.\\\hline
        combinatorial objects &  een onderzoek hoe om te gaan met combinatorial objects.
				    Bv hoe grafisch weer te geven, hoe op te slaan in de data structuur\\\hline
    \end{tabular}
\end{center}


\begin{figure}[!h]
    \subfloat[UI toolkit]{%
    \begin{boxedminipage}[b]{.32\textwidth}
	{\tiny%%%%%%%%%%%%%%%%%%%%%%%%%%
	    Student: Guus Bonnema.
	    \paragraph{Goal}
	    Find the toolkit combination that optimally satisfies the high level requirements as specified in the project plan.
	    \paragraph{In scope}
	    The high level architecture and the user interface toolkit. Influence on IPC toolkits, concurrency toolkits en integration toolkits.
	    The influence on development environment.
	    \paragraph{Out of scope}
	    Any non UI toolkits are out of scope, unless a clear relationship exists.
	    \paragraph{Overlap}
	    Overlap with integration as many toolkits have approaches to integrating components. Also the constraints that follow
	    from the domain analysis on integration will influence the resulting decision.
	    \paragraph{Results}
	    The domain analysis results in advice on use of toolkits and on time necessary to master the toolkits.
	}% tiny %%%%%%%%%%%%%%%%%%%%%%%%%%%%%
    \end{boxedminipage}
    }% subfig
%    \caption{Scope of toolkit domain analysis}
%    \label{fig: da-toolkit}
%\end{figure}
    \subfloat[Combinatorial objects]{%
    \begin{boxedminipage}[b]{.32\textwidth}
	{\tiny%%%%%%%%%%%%%%%%%%%%%%%%%%
	    Student: Stefan Versluys.
	    \paragraph{Goal}
	    Find a way of integrating the analysis tools in the design tool, in such a way
	    that the high level requirements are optimally satisfied. Describe both the functions
	    and the dynamics of this integration.
	    \paragraph{In scope}
	    Integration of tools in the design tool. Types of tools necessary.
	    high level architecture. Constraints on use of toolkits. Influence on development environment and subsequent constraints.
	    \paragraph{Out of scope}
	    Specific toolkits. Development environment.
	    \paragraph{Overlap}
	    This domain analysis overlaps with toolkit domain analysis. The constraints that follow from this analysis
	    will influence the choice of toolkit.
	    \paragraph{Results}
	    This analysis results in an advice that will directly influence the applications architecture, its development environment
	    and the use of toolkits.
	}% tiny %%%%%%%%%%%%%%%%%%%%%%%%%%%%%
    \end{boxedminipage}
    }% subfig
%    \caption{Scope definition of combinatorial objects}
%    \label{fig: da-objects}
%\end{figure}
    \subfloat[Combinatoric cycle checker]{%
    \begin{boxedminipage}[b]{.32\textwidth}
	{\tiny%%%%%%%%%%%%%%%%%%%%%%%%%%
	    Student: Jeroen Kleijn.
	    \paragraph{Goal} Integrate the combinatoric cycle checker with the design tool. Determine the specific interface. Determine the common interface
	    with other tools.
	    \paragraph{In scope} Combinatoric cycle checker. Integration.
	    \paragraph{Out of scope} The design tool.
	    \paragraph{Overlap} The choices made in the integration analysis have implementation consequences for this domain analysis and vice versa.
	    \paragraph{Results} This tool is not yet integrated into the design tool.
	    \todo[inline,caption={cycle checker}]{is this true?}
	}% tiny %%%%%%%%%%%%%%%%%%%%%%%%%%%%%
    \end{boxedminipage}
    }% subfig

%    \paragraph{Requirements for domain analysis}
    \begin{boxedminipage}[c]{.8\textwidth}
    \begin{itemize}
	\item At least an initial idea of the high level view of the system should be available. The final high level
		view will depend on the choices made here and for the toolkits.
	\item The high level requirements must be defined and agreed on.
    \end{itemize}
    \end{boxedminipage}


    \caption{Scope domain analyses}
    \label{fig: domain-analysis}
\end{figure}


\section{Planning en Aanpak}

%%
%% Dit is een subdocument van het projectplan.
%%

\section{Planning en Aanpak}
\subsection{Introductie}
Het project betreft het herbouwen van het WickedXmas tool, omdat de tool gebruikt wordt voor studie
en ontwerp van chips is het belangrijk dat deze tool gemakkelijk en foutloos werkt op de gespecificeerde platformen.
Verder betreft het een niet alledaags domein waardoor er extra aandacht noodzakelijk is om het domein
te verkennen. De eis dat het tool platformonafhankelijk moet zijn en goed met de bestaande C++ programmatuur
moet integreren, maken dat tools noodzakelijk zijn. Er is tijd voorzien voor de teamleden om zich de ontwikkeltools eigen te maken.
Wij hebben gekozen voor de agile aanpak DAD, het plan is opgebouwd volgens de drie fasen ervan.

\subsection{Organisatie}
 De teamleden ontwikkelen gedelokaliseerd en communiceren via internet. Sources worden gecentraliseerd
 in de cloud met versiebeheer. Het team beschikt over een begeleider welke toeziet op de gang van zaken
 en geeft op vraag advies. De opdrachtgever en tevens domeinspecialist kan geraadpleegd worden voor
 specifieke domein vragen en evaluatie van het product.
 \begin{enumerate}
 	\item Guus Bonnema
 	\begin{itemize}
		\item Rol - Ontwikkelaar
		\item Skills - IT, Java, Linux
	\end{itemize}
 	\item Jeroen Kleijn
 	\begin{itemize}
		\item Rol - Ontwikkelaar
		\item Skills - IT,  MS VisualStudio (C\#), C/C++, Java, Linux + Windows
	\end{itemize}
 	\item Stefan Versluys
 	\begin{itemize}
		\item Rol - Ontwikkelaar
		\item Skills - IT,  MS VisualStudio (C\#), C/C++, Java, Windows + VxWorks
	\end{itemize}
	\item Freek Verbeek
	\begin{itemize}
		\item Rol - Proces begeleider
	\end{itemize}
	\item Bernard van Gastel
	\begin{itemize}
		\item Rol - Opdrachtgever / Domeinspecialist : begeleidt inhoudelijk
	\end{itemize}
 \end{enumerate}

 Tijdens de constructie fasen komen er nog de DAD rollen van product owner en architecture owner bij.
 Wie die rollen gaan vervullen bepalen we tijdens de eerste twee iteraties. De DAD rol van team lead
 voorzien wij momenteel niet nodig te zullen hebben omdat het team daarvoor te klein is.


\subsection{Hardware en software}
\begin{enumerate}
	\item Om versiebeheer en sources te borgen gebruiken wij een centrale Git repo.
	\item Communicatie middelen zijn GitHub, gmail, skype,  teamviewer en mogelijk een tiik om
		het agile werken te ondersteunen.
	\item Platformonafhankelijke IDE voor het ontwerp tool met C++ compiler voor de analyse tools.
	\item Platformonafhankelijke GUI Toolkit.
	\item Componenten voor xmas analyse en checks.
	\item Mac OS,  MS Windows, Linux platformen.
	\item DAD Support Tool (Work Item list, Visualize work , Burn down chart)
\end{enumerate}


\subsection{DAD Ontwikkelmethode}
\subsubsection{Lifecycle}

\begin{sidewaysfigure}[t]
  \includegraphics[width=\textwidth]{dadLifecycleUP2}
  \caption{DAD basic Lifecycle}
\end{sidewaysfigure}
Mijlpalen :
\begin{enumerate}
\item Stakeholder consensus
\item Proven architecture
\item Sufficient functionality
\item Production ready
\item Delighted stakeholders
\end{enumerate}

\subsubsection{Iteratie aanpak}
\begin{itemize}
 \item Analyse : Modellen, requirements,  prioriteiten en selectie
 \item (Test Driven) Development
 \item Refactoring
 \item Peer review
 \item Levert steeds iets bruikbaars op dat ge\"evalueerd kan worden voor feedback.
 \item Planning
 \item Documentatie
\end{itemize}

%%
%% Dit is een subdocument van het projectplan.
%%
%% items op basis van SE Sommerville p623 (chapter 23.2.1 Project plans)
%%  	sommige items zijn elders in het document reeds opgenomen,  zoals bvb risico's
%%

\section{Planning}
\subsection{Indeling activiteiten over de drie DAD fasen}
%%zie Sommerville SE 2.4 The RUP (ed9 blz50-51)
\begin{enumerate}
\item \underline{\textbf{Inceptie (2 iteraties)}}
%%http://disciplinedagiledelivery.wordpress.com/2012/11/11/comparing-dad-to-the-rational-unified-process-rup-part-2/
		\begin{itemize}
		\item Iteratie ``Planning''
			\begin{itemize}
			\item Team afspraken.
			\item Stakeholder meeting.
			\item Probleem definitie.
			\item High level requirements.
			\item Planning.
			\end{itemize}
		\paragraph{Artefacten}
		De teamleden hebben de nodige afspraken gemaakt m.b.t. werkwijzen zoals
		communicatie middelen en versiebeheer voor documentatie. Deze fase levert
		een document dat het probleem omschrijft,  de high-level requirements,
		business case, risiso's, stakeholders, vision and challenges, een plan van
		aanpak en een planning voor de eerst volgende iteratie.
		\item Iteratie ``Domein onderzoek''
			\begin{itemize}
			\item Domein onderzoek.
			\item Requirements: Op basis van sourcecode van de huidige WickedXmas tool.
			\item Requirements: Demonstratie van de huidige WickedXmas tool door
			stakeholder.
			\item Initi\"ele architectuur visie
			\end{itemize}
		\paragraph{Artefacten}
		De resultaten van het domein onderzoek en requirements op basis van observatie.
		Er is een visie m.b.t. de initi\"ele architectuur.
		Documentatie.
		\end{itemize}

\item \underline{\textbf{Constructie (7 iteraties)}}
	\begin{itemize}
	\item Iteratie 0
		\begin{itemize}
		\item Architectuur : Afbakenen , configureren en testen
		\item documentatie aanpassen
		\end{itemize}
		\paragraph{Artefacten op de ``Proven architecture'' mijlpaal}
		 De Architectuur is bepaald en getest en klaar voor evaluatie, documentatie.
	\item Iteratie 1
		\begin{itemize}
		\item WickedXmas editor ontwikkelen
		\item documentatie aanpassen
		\end{itemize}
		\paragraph{Artefacten}
		Prototype, documentatie
	\item Iteratie 2
		\begin{itemize}
		\item WickedXmas editor ontwikkelen
		\item documentatie aanpassen
		\end{itemize}
		\paragraph{Artefacten}
		Prototype, documentatie
	\item Iteratie 3
		\begin{itemize}
		\item WickedXmas editor ontwikkelen
		\item documentatie aanpassen
		\end{itemize}
		\paragraph{Artefacten}
		Prototype, documentatie
	\item Iteratie 4
		\begin{itemize}
		\item WickedXmas analyse tool interface ontwikkelen
		\item documentatie aanpassen
		\end{itemize}
		\paragraph{Artefacten}
		Prototype, documentatie
	\item Iteratie 5
		\begin{itemize}
		\item WickedXmas analyse tool interface ontwikkelen
		\item documentatie aanpassen
		\end{itemize}
		\paragraph{Artefacten op de ``Sufficient functionality'' mijlpaal }
		 Een release van de WickedXmas Tool zoals beoogd werd,  documentatie
	\end{itemize}
\item \underline{\textbf{Finale transitie (1 iteratie)}}
	\begin{itemize}
		\item Finale releases van de nieuwe WickedXmas tool.
		\item Documentatie bundelen (handleiding).
		\item Onderzoekscontext
		\item Presentatie geven
	\end{itemize}
	\paragraph{Artefacten op de ``Production ready'' mijlpaal}
	Een presentatie, documentatie en scriptieverslag de nieuwe WickedXmas tool.

\end{enumerate}
%%
%% Dit is een subdocument van het projectplan.
%%
%%  hier worden tijden en eventueel personen toegekend aan activiteiten
%%


\subsection{Schedule}
 %% dit onderdeel moet vermoedelijk in het schedule plan

\pagebreak
\begin{landscape}
\newgeometry{top=0.5cm,left=0.5cm,bottom=0.5cm,right=-5cm}

\begin{figure}[hp]
\begin{ganttchart}[
   y unit chart=.9cm,
   today=4
 ]{0}{37}
 \gantttitle{2014}{15}
 \gantttitle{2015}{22} \\
 \gantttitlelist{38,...,52}{1}
 \gantttitlelist{1,...,22}{1} \\
 
 \ganttgroup{Inceptie}{1}{8} \\
 \ganttbar{Planning}{1}{4} \\
 
 \ganttmilestone{Planning}{4} \\
 
 \ganttbar{Domeinanalyse}{5}{8} \\
 
 
 \ganttmilestone{Domeinanalyse}{8} \\
 
 \ganttgroup{Constructie}{9}{28} \\
 
 \ganttbar{Iteratie 0}{9}{11} \\
 \ganttmilestone{Proven Architecture}{11} \\
 \ganttbar{Iteratie 1}{12}{14} \\
 \ganttbar{Iteratie 2}{17}{19} \\
 \ganttbar{Iteratie 3}{20}{22} \\
 \ganttbar{Iteratie 4}{23}{25} \\
 \ganttbar{Iteratie 5}{26}{28} \\
 
 \ganttmilestone{Constructie}{28} \\
 
 \ganttgroup{Transitie}{29}{31} \\
 \ganttbar{Onderzoekcontext}{20}{29} \\
 \ganttbar{Afronding}{30}{32}
 \ganttbar[bar/.style={fill=gray}]{}{32}{37} \\
 
 \ganttmilestone{Einde project}{37} 
 
 \ganttlink{elem1}{elem2}
 \ganttlink{elem2}{elem3}
 \ganttlink{elem3}{elem4}
 \ganttlink{elem4}{elem6}
 \ganttlink{elem6}{elem7}
 \ganttlink{elem7}{elem8}
 \ganttlink{elem8}{elem9}
 \ganttlink{elem9}{elem10}
 \ganttlink{elem10}{elem11}
 \ganttlink{elem11}{elem12}
 \ganttlink{elem12}{elem13}
 \ganttlink{elem9}{elem15}
 \ganttlink{elem13}{elem16}
 \ganttlink{elem15}{elem16}
 \ganttlink{elem17}{elem18}
 
\end{ganttchart}
\caption{Globale planning met hoofdfasen en belangrijkste mijlpalen}
\end{figure}

\pagebreak

\begin{figure}[hp]
\begin{ganttchart}[
    today=26
  ]{0}{32}
  \gantttitle{2014}{32} \\
  \gantttitle{september}{18}
  \gantttitle{oktober}{14} \\
  \gantttitlelist{13,...,30}{1}
  \gantttitlelist{1,...,14}{1}\\
  
  \ganttmilestone{Startbijeenkomst}{0.5} \\
  \ganttbar{Teamafspraken}{2}{7} \\
  \ganttbar{Stakeholder meeting}{8}{8} \\
  \ganttbar{Probleemdefinitie}{9}{24} \\
  \ganttbar{High-level requirements}{9}{24} \\
  \ganttmilestone{Concept planning}{27} \\
  
  \ganttbar{Verbeteringen}{28}{32} \\
  \ganttmilestone{Planning}{32}
  
  \ganttlink{elem0}{elem1}
  \ganttlink{elem1}{elem2}
  \ganttlink{elem2}{elem3}
  \ganttlink{elem2}{elem4}
  \ganttlink{elem3}{elem5}
  \ganttlink{elem4}{elem5}
  \ganttlink{elem5}{elem6}
  \ganttlink{elem6}{elem7}
 
\end{ganttchart}
\caption{Detailplanning 'Planning'}
\end{figure}

\pagebreak

\begin{figure}[hp]
\begin{ganttchart}[]{0}{33}
  \gantttitle{Domeinanalyse}{33}
  \ganttnewline
  \gantttitle{2014}{33} \\
  \gantttitle{oktober}{17}
  \gantttitle{november}{16} \\
  \gantttitlelist{14,...,31}{1}
  \gantttitlelist{1,...,14}{1}\\
  
  \ganttmilestone{Planning}{0.5} \\
  \ganttbar{Demonstratie WickedXmas}{2}{2} \\
  \ganttbar{Domeinanalyse GB}{3}{22}
  \ganttbar[bar/.style={fill=gray}]{}{23}{29} \\
  \ganttbar{Domeinanalyse JK}{3}{7}
  \ganttbar{}{13}{27}
  \ganttbar[bar/.style={fill=gray}]{}{28}{29} \\
  \ganttbar{Domeinanalyse SV}{3}{11}
  \ganttbar{}{19}{29} \\
  \ganttmilestone{Domeinanalyse}{30}
  
  \ganttlink{elem0}{elem1}
  \ganttlink{elem1}{elem2}
  \ganttlink{elem1}{elem4}
  \ganttlink{elem1}{elem7}
  \ganttlink{elem3}{elem9}
  \ganttlink{elem6}{elem9}
  \ganttlink{elem8}{elem9}
  
\end{ganttchart}
\end{figure}

%%\restoregeometry
\end{landscape}




\begin{tabular}{ll}\hline
{\bf Fase}    & {\bf weken}\\\hline
Planning             & 2,5 \\
\hline
Domeinanalyse        & 3 \\
Iteratie 0           & 3 \\
\hline
Iteratie 1           & 3 \\
Iteratie 2           & 3 \\
Iteratie 3           & 3 \\
Iteratie 4           & 3 \\
Iteratie 5           & 3 \\
\hline
Onderzoekcontext     &	2 \\
\hline
Afronding	     & 1.5 \\
\hline
Totaal               & 27 \\
\end{tabular}


\begin{itemize}
 \item 27 weken * 15 uur/week = 405 uur
 \item 8 maanden, ongeveer 32 weken beschikbaar, dus 5 weken marge
 \item Na elke iteratie wordt een voorlopige versie van de software aan de opdrachtgever verstrekt
\end{itemize}



\paragraph{Planning iteraties}

\begin{figure}[h]
 \begin{ganttchart}[
 ]{0}{21}
  \gantttitle{Planning iteraties}{21} \\
  \gantttitlelist{1,...,21}{1} \\
  \ganttbar{Requirements}{1}{7} \\
  \ganttbar{Ontwikkeling}{3}{18} \\
  \ganttbar{Documentatie}{10}{18} \\
  \ganttmilestone{Release}{18} \\
  \ganttbar{Scriptieverslag}{19}{20} \\
  \ganttbar{Evaluatie}{19}{20} \\
  \ganttbar{Onderzoekcontext}{19}{20} \\
  \ganttmilestone{Einde iteratie}{21}
 \end{ganttchart}
\end{figure}


Elke iteratie omvat een vaste periode van 3 weken.
In de eerste iteraties zal relatief meer tijd worden besteed aan requirements en
weinig tot geen aan de onderzoekcontext. In latere iteraties komt de onderzoekcontext
juist meer aan bod. Aan het einde van elke iteratie zijn twee dagen gereserveerd
voor het werken aan het scriptieverslag en de onderzoekcontext. In deze periode vindt
ook een evaluatie plaats van de afgelopen iteratie. \\
Voor het vaststellen van de requirements zal regelmatig overleg met Bernard plaats
moeten vinden. Hiervoor is het belangrijk dat in de detailplanning van een iteratie
rekening wordt gehouden met de beschikbaarheid van Bernard.


\todo[inline, color=blue!40, caption=planning]{Nu onderzoekcontext, documentatie en
scriptieverslag (deels) in de iteraties zijn opgenomen is er tijd vrijgekomen in de
planning. Hoe moeten we deze tijd invullen, iteraties langer of een extra iteratie?}




\begin{tabular}{lll}\hline
{\bf Week}    & {\bf Taak}  & {\bf Extra}\\\hline
38-41         & Planning    \\
42-45         & Domeinanalyse & 4 weken ingepland i.v.m. vakanties Stefan \& Jeroen \\
46-48         & Iteratie 0    \\
49-51         & Iteratie 1    \\
52-1          &               & geen iteratie i.v.m. vakantieperiode, mogelijk individueel (onderzoekcontext?) \\
2-4           & Iteratie 2    \\
5-7           & Iteratie 3    \\
8-10          & Iteratie 4    \\
11-13         & Iteratie 5    \\
14-16         &               \\
17-18         & Onderzoekcontext \\
19-20         & Afronding     \\
22            & Mijlpaal einde project

\end{tabular}




\paragraph{Capaciteitsplanning}

\begin{itemize}
 \item Beschikbare tijd
 \begin{itemize}
  \item $\sim$15 uur per week per teamlid
  \item $\sim$30 uur gedurende het hele project voor Freek en $\sim$20 uur voor Bernard
 \end{itemize}

 
 
 \item vakanties / niet beschikbaar voor project
 \begin{itemize}
  
  \item Stefan
  \begin{itemize}
   \item 25-10-2014 t/m 31-01-2014 (week 44)
   \item 24-12-2014 t/m 02-01-2015 (week 52-01)
  \end{itemize}
 
  \item Jeroen
  \begin{itemize}
   \item 21-10-2014 t/m 25-10-2014 (week 43)
  \end{itemize}
 
  \item Bernard
   \begin{itemize}
    \item 04-10-2014 t/m 14-10-2014 (week 41-42)
    \item 20-10-2014 t/m 27-10-2014 (week 43-44)
   \end{itemize}

 \end{itemize}
\end{itemize}


\subsection{Open punten}

\begin{enumerate}
 \item Wat zijn de domeinanalyses precies? Welke onderwerpen? Hoe diepgaand? Hoeveel tijd kost het?
 \item Aan welke eisen moet het scriptieverslag voldoen?
 \item Aan welke eisen moet de presentatie voldoen?
 \item Hoe gaan we agile precies invullen?
 \item Hoe valt Freek in de iteratie tijdens de uitvoering? Welke rol speelt Freek precies tijdens de andere fases?
 \item wat is het belang van i18n voor systeem documentatie?
 \item wat is het belang van l10n voor de software (variabele namen, menu opties)?
 \item moet de communicatie in meerdere talen aanwezig zijn?
\end{enumerate}


\appendix
\section{Definitions}

\begin{description}
 \item[i18n] Internationalization, referring to translation of menu items, system documentation etc.
 \item[l10n] Localization, referring to country specific settings such as money, numbers, dates etc.
\end{description}
\section{Verwijzingen}

\bibliography{plan.bib}
\end{document} ;########################### end document ##################################;
