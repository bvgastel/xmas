\documentclass[a4paper,11pt,twoside,draft]{article}

\usepackage{ucs}
\usepackage{subfigure}
\usepackage[dutch,english]{babel}
\usepackage{graphicx}

\bibliographystyle{alpha}

\usepackage{hyperref}

\author{Guus Bonnema}
\date{24/09/14}

\title{Plan van aanpak voor xmas ontwerp tool}

\begin{document}

\section{Opdracht}



\subsection{Inleiding}

\paragraph{Factoren}

\begin{enumerate}
\item geografisch gespreid team
\item technisch product
\item high level functionele requirements bekend bij opdrachtgever.
\item Kwalitatieve requirements nog niet expliciet
\item 3 teamleden
\item Vaste tijd voor volledige uitvoering (8 maanden +/- 1 maand)
\item Weinig externe risico's
\end{enumerate}



\paragraph{Alternatieve aanpakken}

\begin{enumerate}
\item Volledig plangedreven zoals SDM2 (System Development Method 2)
\item Volledig agile zoals XP (Extreme programming)
\item Hybride aanpak zoals UP (Unified Process)
\item Scrum aanpak (additief aan een agile of hybride aanpak): iteratief
\end{enumerate}

\paragraph{Overwegingen}

\begin{enumerate}
\item Klein team met deels bekende requirements
\item geografische spreiding
\item teamleden hebben geen ervaring met agile methoden
\end{enumerate}

\paragraph{Conclusies}
\begin{description}
\item XP vergt fysieke nabijheid en grote gebruikers betrokkenheid: valt af
\item SDM2 heeft een rigide requirements Engineering process
\item UP: Iteratief met agile constructie komt het dichtst in de buurt:
\begin{enumerate}
 \item Plangedreven voorbereiding
 \item iteratieve sprints tijdens de uitvoering
\end{enumerate}
\end{description}

Scrum is het meest bruikbaar voor project management tijdens de constructie omdat
het gemakkelijk bij agile en bij incrementeel cq iteratief ontwikkelen te gebruiken is.

\subsection{Business case}

Wat maakt deze wijziging nodig? Wie helpt het? Hoe helpt het?
Hoe groot is de benefit?

\subsection{Risicos}

Wat zijn de belangrijkste risicos? Welke moeten we accepteren, en welke wat aan doen?
\begin{enumerate}
	\item Technology
	\begin{itemize}
		\item Open source tools : Availability and support.
		\item Unreliable components used for integration.
		\item Poor compiler or tools with negative impact on efficiency.
	\end{itemize}
	\item People
	\begin{itemize}
		\item Someone leaves the team
		\item Unavailability of stakeholders
		\item Absence of imported skills
		\item Personal equipement becomes unavailable (pc, internet).
	\end{itemize}
	\item Organisation
	\begin{itemize}
		\item Poor or delayed support.
		\item IT infrastructur becomes unavailable.
	\end{itemize}
	\item Tool
	\begin{itemize}
		\item Unstable non commercial development tools.
		\item Poor support for tool integration in IDE.
		\item Availability and reliability of collaboration tools. 
	\end{itemize}
	\item Requirements
	\begin{itemize}
		\item Changes that conflict with chosen architecture.
	\end{itemize}
	\item Estimation
	\begin{itemize}
		\item Complexity of a new development environement. 
		\item Time required to gain knowledge of the domain.
	\end{itemize}
\end{enumerate}

\subsection{Challanges}

Wat zijn de belangrijkst project uitdagingen? Als we dit goed formuleren, dan komen hier
requirements uit voort.

\subsection{Vision}

Wat zien we voor ons als het systeem klaar is?

\subsection{Stakeholders}

Opdrachtgever is Bernard van Gastel van de Open Universiteit. Begeleider is Freek Verbeek van de Open Universiteit.
De doelgroep voor gebruikers van het ontwerp tool zijn medewerkers van universiteiten en onderzoekers van bedrijven zoals
Intel en LLC. Een kleine groep van xMAS onderzoekers van zowel de universiteit als bedrijven gebruiken vanuit de optiek
van onderzoek naar verbetering van NoC ontwerp het ontwerp tool. Doelstellingen voor deze groep zijn chip import
en export van bestandsformaten (zoals Verilog), verificatie van aspecten van correctheid (deadlock, livelock,
syntactische of semantische checks). Bernard en Freek nemen het onderhoud van het tool voor hun rekening.

%% Opmerking: stakeholders splitsen in 2 tabellen. 1 tabel met de doelstelling en toelichting daarop
%%            1 tabel met per stakeholder de project en systeem-belangen en -doelstellingen

\begin{figure}
{\tiny
\begin{center}
\begin{tabular}{lll}\hline
{\bf Stakeholder}    & {\bf Projectrollen}   & {\bf omgevingsrollen} \\\hline
Bernard van Gastel   & Opdrachtgever         & Onderzoeker \\
                     &                       & Gebruiker \\
                     &                       & Ontwikkelaar \\
Freek Verbeek        & Begeleider            & Onderzoeker \\
                     &                       & Gebruiker \\
                     &                       & Ontwikkelaar \\
Team33               & Ontwikkelaar          & \\
                     & Student               & \\
Univ. medewerkers    &                       & Gebruiker \\
                     &                       & Onderzoeker \\
                     &                       & Onderzoeker NoC \\
Bedrijfsmedewerkers  &                       & Gebruiker \\
                     &                       & Onderzoeker \\
\hline
\end{tabular}
\end{center}
}% end tiny
\caption{Stakeholders en hun rollen}\label{fig:stakeholders}
\end{figure}

\begin{figure}
{\tiny
\begin{center}
\fbox{
\begin{tabular}{lcp{30em}}
{\bf Doel}    & {\bf Prio} & {\bf Toelichting}\\
Integratie    & 1          & Een hoofddoelstelling is naadloze
			     integratie van de analyse tools
			     in het ontwerp tool. \\
Onderhoud     & 1          & Een hoofddoelstelling voor dit project
			     is het ontwerptool beter onderhoudbaar
			     te maken dan het huidige tool.\\
Usability     & 1 	   & Een hoofddoelstelling is gebruik te bevorderen.
			     Hieronder valt installatie gemak, gebruikers gemak
			     op het niveau van NoC ontwerpers of onderzoeker\\
Uitbreidbaarheid & 2       & Een subdoelstelling van integratie en onderhoudbaarheid
			     is de aanpasbaarheid van de verificatie algoritmen
			     waar het integratie met het design tool betreft.\\
Onderzoek NoC & 2          & Het design tool is ondersteunend
                             voor NoC onderzoek\\
Onderzoek Alg & 3          & Het design tool is ondersteunend
                             voor algemeen onderzoek\\
\end{tabular}
}% end fbox
\end{center}
}% end tiny
\caption{Doelstellingen met prioriteiten \tiny (1 = must have, 2 = could have, 3 = nice to have)}\label{fig:doelstellingen}
\end{figure}




{\tiny
\begin{center}
\begin{tabular}{llllll}\hline
{\bf Stakeholder}    & {\bf Doelen}   & {\bf Rol}     & {\bf Freq} & {\bf Belang} & {\bf Invloed}\\\hline
Bernard van Gastel   & Onderzoek NoC  & Opdrachtgever & Hoog       & Hoog   & Hoog \\
                     & Onderhoud      & Opdrachtgever & Laag       & Hoog   &  \\
                     & Gebruik        & Gebruiker     & ??         & Hoog   & \\
Freek Verbeek        & Onderzoek NoC  & Begeleider    & Hoog       & Hoog   & Hoog\\
                     & Onderhoud      & Opdrachtgever & Laag       & Hoog   & \\
                     & Onderhoud      & Ontwikkelaar  & Laag       & Hoog   & \\
                     & Gebruik        & Gebruiker     &            & Hoog   & \\
Univ. medewerkers    & Onderzoek alg. & Gebruiker     & Laag       & Middel & \\
                     & Onderzoek NoC  & Gebruiker     & Middel     & Hoog   & \\
                     & Gebruik        &               &            & Middel & \\
Bedrijfsonderzoekers & Onderzoek NoC  & -             & Middel     & Hoog   & \\
                     & Gebruik        & Gebruiker     &            & Middel & \\
                     & Installatie    & Gebruiker     &            & Hoog   & \\
Bedrijfsmedewerkers  & Gebruik        & Gebruiker     & Laag       & Middel & \\
                     & Installatie    & Gebruiker     &            & Hoog   & \\
Team33               & Ontwikkeling   & Ontwikkelaar  & Hoog       & Hoog   & Hoog\\
                     & Afstuderen     & Student       &            & Hoog   & Hoog\\
                     & Samenwerking   & Ontwikkelaar  & Hoog       & Hoog   & Hoog\\
\hline
\end{tabular}
\end{center}
}% end tiny

\subsection{High level Requirements}

Wat zijn uitdagingen en welke high level requirements vloeien hieruit voort? De uitdagingen en requirements
zijn niet volledig. Tijdens de eerste iteratie werkt het team dit uit.

Hieronder een lijst van uitdagingen (challanges):

\begin{description}
 \item[platformen] Het onderzoeksteam werkt met Mac, ontwikkelteam met ms windows en linux, de opdrachtgever en studenten met ms windows, linux of mac.
		    Dit genereert requirements voor platform onafhankelijkheid van ontwikkelomgeving, taal en toolkits die we gebruiken.
 \item[integratie] Het systeem bestaat uit meerdere modulen, grofweg aangeduid als ontwerp tool (wat ons onderwerp is) en analyse tools (tools die nog
		    in ontwikkeling zijn, maar deels al gebouwd). De uitdaging is een hoge graad van integratie van deze componenten, zo onafhankelijk mogelijk.
		    Dit genereert requirements voor onderhoudbaarheid, component interfaces, applicatie structuur. De analyse tools zijn alle C++ programma's.
		    De interfaces zijn gebaseerd op JSON.
 \item[onderhoud] Over de tijd heen onderhouden vele mensen de software, die elkaar niet spreken. Dit genereert requirements met betrekking tot documentatie,
		    onderhoudbaarheid en toegankelijkheid.
 \item[Uitbreiding] De huidige software is moeilijk uitbreidbaar op de wijze die de opdrachtgever graag wil.
 \item[functie] Het ontwerp tool ondersteunt de primitieven en hun samenwerking zoals gespecificeerd in het xmas paper \cite{chatterjee-kishinevsky:xmas}.
		Het  huidige ontwerp tool: WickedXMas geeft de richting aan implementatie van de functionaliteit en de interface, maar is
		niet beperkend of maatgevend daarvoor.
 \item[rest]	Wat zijn de overige uitdagingen? Welke requirements kan de huidige programmatuur moeilijker aan?
\end{description}

High level requirements.

\begin{description}
 \item[platform 001] Het ontwerptool moet draaien onder zowel ms Windows, Mac en Linux.
 \item[platform 002] De GUI toolkit moet platform onafhankelijk werken zowel met de
 grafische user interface als met de achterliggende processing en de integratie daarmee.
 \item[platform 003] De documentatie moet op alle platformen te genereren zijn.
 \item[integratie 001] De architectuur van de applicatie moet onderhoudbaarbaar zijn
 \item[integratie 002] De integratie van het ontwerp tool met de analyse tools moet onafhankelijk zijn (kleine koppeling)
 \item[integratie 003] De integratie van het ontwerp tool met de analyse tools moet hecht zijn
 vanuit het perspectief van gebruik.
 \item[integratie 004] Het ontwerptool moet JSON kunnen interpreteren.
 \item[onderhoud 001] De integratie moet gemakkelijk wijzigbaar zijn: voldoende documentatie, goed gestructureerd
 \item[functie 001] Het ontwerp tool ondersteunt het modelleren op basis van de acht primitieven uit het xmas paper \cite{chatterjee-kishinevsky:xmas}.
 \item[functie 002] Het ontwerp tool ondersteunt het controleren van de modellen o.b.v. de analyse tools.
 \item[uitbreiding 001] Het ontwerp tool ondersteunt het toevoegen van een nieuw analyse tool.
 \item[uitbreiding 002] Het ontwerp tool ondersteunt uitbreiding o.b.v. een plug in
\end{description}



\subsection{Succesfactoren}
Wanneer is het project een succes?
Wanneer is het project mislukt?

\section{Architectuur}
Aan welke eisen moet de architectuur voldoen?
Op welke platformen draait het?
Is het multi user? Is het client/server? Is er een repository?
Is het taalonafhankelijk?

\section{Domeinanalyse}

Welke domeinanalyse doen we en wie doet wat? Hoeveel tijd kunnen er aan besteden?

\section{Aanpak}

\paragraph{Volgorde van fasen}

\begin{enumerate}
 \item Planning en aanpak
 \item Business case
 \item Risico analyse
 \item Vision en challanges
 \item Requirements
 \item Domeinanalyse
 \item Architectuur (incl GUI framework)
 \item Uitvoering (3 weekse iteraties)
 \begin{enumerate}
  \item a. iteratie 1 -- hoogste risico dempen
  \item b. iteratie 2 -- v1
  \item c. iteratie 3 -- v2
  \item d. iteratie 4 -- v3
  \item e. iteratie 5 -- v4
 \end{enumerate}
\item Afronding en presentatie
\end{enumerate}

\paragraph{De basis principes}
\begin{itemize}
 \item De uitvoering is agile en tijdgedreven.
 \item Elke iteratie heeft een vaste hoeveelheid tijd
 \item Elke iteratie begint met een analyse (requirements engineering, prioriteiten en selectie). Soms wijzigen requirements of komen er nieuwe bij.
 \item Elke iteratie eindigt met een evaluatie
 \item Dit moeten we valideren bij Bernard en Freek. Gebruikers betrokkenheid is
 beperkt tot het begin van de iteratie, de prioriteitsstelling van requirements, bijstelling van
 requirements, nieuwe requirements en uiteindelijk de selectie van wat we gaan bouwen.
\end{itemize}

\paragraph{Tijdsindeling}

\begin{description}
 \item[Fase 1] Het bepalen van de toolkit en ontwikkel omgeving. Duur 3 weken.
 \item[Fase 2-4] Business case, risico's, vision en challanges. Duur 4 weken.
 \item[Fase 5-7] Requirements engineering uitsluitend high level requirements. Duur 4 weken.
 \item[Fase 8] De uitvoering in 5 iteraties. Duur 5 x 3 weken is 15 weken
 \item[Fase 9] De afronding en de presentatie. Duur 4 weken.
\end{description}

Netto duur 27 weken, bruto 8 maanden 1 maand speling voor vakantie, ziekte en onvoorzien.

Project start 20 sept 2014, Eind 20 mei 2015. Neem de exacte datum met een korrel zout.


\subsection{Agile aanpak}
------------

\begin{itemize}
 \item Gebruiken git voor versiebeheer van documenten en software
 \item Passen TDD toe (binnen redelijke)
 \item Passen refactoring toe
 \item Elke iteratie:
 \begin{itemize}
   \item Analyse (requirements aanpassen, prioriteitsstelling, selectie)
   \item Ontwikkeling prototype
   \item Evaluatie
 \end{itemize}
\end{itemize}


\subsection{Open punten}

\begin{enumerate}
 \item Wat zijn de domeinanalyses precies? Welke onderwerpen? Hoe diepgaand? Hoeveel tijd kost het?
 \item Aan welke eisen moet het scriptieverslag voldoen?
 \item Aan welke eisen moet de presentatie voldoen?
 \item Hoe gaan we agile precies invullen?
 \item Hoe valt Freek in de iteratie tijdens de uitvoering? Welke rol speelt Freek precies tijdens de andere fases?
 \item wat is het belang van i18n voor systeem documentatie?
 \item wat is het belang van l10n voor de software (variabele namen, menu opties)?
 \item moet de communicatie in meerdere talen aanwezig zijn?
\end{enumerate}


\appendix
\section{Definitions}

\begin{description}
 \item[i18n] Internationalization, referring to translation of menu items, system documentation etc.
 \item[l10n] Localization, referring to country specific settings such as money, numbers, dates etc.
\end{description}
\section{Verwijzingen}

\bibliography{plan.bib}
\end{document} ;########################### end document ##################################;