%%
%% Dit is een subdocument van het projectplan.
%%

\section{Opdracht}

\subsection{Business case}

Uit de projectaanvraag voor dit project:

\begin{quote}
    \tiny
    De Network-on-Chip (NoC) groep van de OU doet onderzoek
    naar nieuwe methoden om NoC ontwerpen (moderne manier van
    ontwerpen van processoren) foutvrij te krijgen met behoud van
    efficiëntie. De NoC groep heeft een aantal tools ter ondersteuning
    van het onderzoek gemaakt (met ondersteuning van vele studenten
    projecten), voornamelijk WickedXmas. Deze tool stelt de
    gebruiker in staat om chip ontwerp te maken/bewerken/genereren
    in de visuele xMAS taal van Intel, en vervolgens een aantal door
    ons ontwikkelde tools op los te laten (bv symbolische analyse,
    deadlock checker, etc).
    De huidige tool heeft een aantal problemen:

    \begin{itemize}
	\item niet modulair opgezet (waardoor uitbreidingen moeizaam gaan)
	\item op Windows API gebaseerd (waardoor de onderzoekers die
	    gebruik maken van Mac het lastig kunnen gebruiken)
	\item moeizame integratie met tools: WickedXmas is nu geschreven
	    in C\#, en lijkt moeilijk te integreren met onze C/C++ tools
	\item geen documentatie
    \end{itemize}

    Deze problemen moeten opgelost worden, danwel door een grote
    refactoring van de bestaande code, danwel door het opnieuw
    bouwen.
\end{quote}


\noindent
De business case voor deze aanvraag bestaat uit bruikbaarheid van de ontwerp module
op verschillende platforms, en een hechtere integratie van het ontwerp tool met de
analyse tools.

De platform onafhankelijkheid bevordert beschikbaarheid van het tool voor een grotere
groep onderzoekers. De integratie en betere modulariteit bevordert de onderhoudbaarheid.

Als het project een platform onafhankelijk tool kan maken, dat goed geintegreerd en gestructureerd
is, dan kan de onderzoeks profiteren van een breder publiek, gemakkelijker aanpassingen doen en gemakkelijker
met nieuwe gewijzigde analyse tools integreren.

De baten kunnen wij helaas niet in financie\"ele zin kwantificeren.

\subsection{Risico}

Zie figuur \ref{fig: risico} een overzicht van de risico's die we onderkennen en figuur \ref{fig: risico reductie} voor
een overzicht van de maatregelen die we nemen ter compensatie van risico's.

TODO: Per risico opnemen of we het accepteren of maatregelen nemen (optioneel).


\begin{figure}[ht]
\begin{center}
\small
\begin{tabular}{cp{10em}p{20em}}
{\bf nr} & {\bf Risico} & {\bf Toelichting} \\
 1 & Beschikbaarheid stakeholders  & De opdrachtgever en de begeleider hebben een beperkte tijd
					beschikbaar. De andere stakeholders zijn niet beschikbaar.\\
 2 & Ervaring waterval & De teamleden hebben vooral ervaring met waterval projecten.\\
 3 & Trage support & De support van de OU lijkt niet al te voortvarend. Dit kan problemen veroorzaken.\\
 4 & Tijd nodig om de materie kennis op te doen & Het kost tijd om de bestaande analyse
					tools en de achterliggende
					xmas materie op te nemen.\\
 5 & Geografische spreiding & Het team woont te ver uit elkaar om face to face meetings te
				organiseren voor overleg tijdens het project.\\
 6 & structuur verval bij ontwikkeling & Bij het toevoegen van functionaliteit is structuur verval een
				natuurlijk gevolg.\\
 7 & twee onervaren C++ programmeurs & Het team heeft \'e\'en ervaren C++ programmeur.\\

\end{tabular}
\end{center}
 \caption{Relevante risico's voor dit project}
 \label{fig: risico}
\end{figure}


\begin{figure}[!ht]
\begin{center}
\small
\begin{tabular}{p{10em}p{20em}c}
    {\bf } & {\bf } & {\bf } \\
    Gepland overleg via skype & Dit verzekert een minimaal contact met opdrachtgever en begeleider & 1\\
    Tussentijds contact via email & Dit vult de communicatietijd aan tot wat nodig is. Nadeel is
				    een kans op vertraging. & 1\\
    Regelmatige refactor (per taak) & Een refactor na uitvoeren en testen van een taak, levert
					een goed gestructureerd systeem na elke iteratie. & 6\\
    Skype en chat creatief en vaak gebruiken & Verhoogt de gelijkenis met lokaal samenwerken & 5\\
    Op afspraak tegelijkertijd bouwen & Verhoogt de kans om samen te werken & 5\\
    Ondersteuning zoeken buiten de OU & Een trage ondersteuning kan op kritieke momenten
					het gehele project vertragen. Door minder op
					OU support te leunen, verminderen we die afhankelijkheid & 3\\
    Agile literatuur lezen & Door ons actief in agile te verdiepen, verkleinen we de kans op
				problemen met het proces & 2\\
    C++ studie doen & Door actief ons C++ 2011 eigen te maken, kunnen we
			met onze achtergrond kennis van programmeer talen en
			van Java, het risico op vertraging voor zijn & 7\\
    Review door de ervaren C++ & Learning on the job onder begeleiding van
				het teamlid dat C++ ervaring heeft & 7\\
\end{tabular}
\end{center}
 \caption{Risico reductie}\label{fig: risico reductie}
\end{figure}


\subsection{Vision}\label{sec: vision}

Dit tool is bedoeld voor wetenschappers van universiteiten en van bedrijven (o.a. Intel). We zien een tool dat
is te downloaden en te installeren op alle relevante platforms. De installatie is eenvoudig\footnote{zie open punten}.

We zien een design tool dat dynamisch controles uitvoert\footnote{zie open punten} zodat elk ontwerp van een NoC correct is
zodra het klaar is.

We zien het programma als goed gedocumenteerd en gemakkelijk te onderhouden en uit te breiden.

\paragraph{Nice to have} We zien de onderzoekers naar \xmas nieuwe primitieven bedenken en toevoegen, nieuwe verificatie tools
bedenken en toevoegen. De onderzoekers kunnen onze bestanden naar een ander formaat converteren zoals Verilog.

\subsection{Stakeholders}

Opdrachtgever is Bernard van Gastel van de Open Universiteit. Begeleider is Freek Verbeek van de Open Universiteit.
De doelgroep voor gebruikers van het ontwerp tool zijn medewerkers van universiteiten en onderzoekers van bedrijven zoals
Intel en LLC. Een kleine groep van xMAS onderzoekers van zowel de universiteit als bedrijven gebruiken vanuit de optiek
van onderzoek naar verbetering van NoC ontwerp het ontwerp tool. Doelstellingen voor deze groep zijn chip import
en export van bestandsformaten (zoals Verilog), verificatie van aspecten van correctheid (deadlock, livelock,
syntactische of semantische checks). Bernard en Freek nemen het onderhoud van het tool voor hun rekening.

%% Opmerking: stakeholders splitsen in 2 tabellen. 1 tabel met de doelstelling en toelichting daarop
%%            1 tabel met per stakeholder de project en systeem-belangen en -doelstellingen

\begin{figure}
{\tiny
\begin{center}
\begin{tabular}{lll}\hline
{\bf Stakeholder}    & {\bf Projectrollen}   & {\bf omgevingsrollen} \\\hline
Bernard van Gastel   & Opdrachtgever         & Onderzoeker \\
                     &                       & Gebruiker \\
                     &                       & Ontwikkelaar \\
Freek Verbeek        & Begeleider            & Onderzoeker \\
                     &                       & Gebruiker \\
                     &                       & Ontwikkelaar \\
Team33               & Ontwikkelaar          & \\
                     & Student               & \\
Univ. medewerkers    &                       & Gebruiker \\
                     &                       & Onderzoeker \\
                     &                       & Onderzoeker NoC \\
Bedrijfsmedewerkers  &                       & Gebruiker \\
                     &                       & Onderzoeker \\
\hline
\end{tabular}
\end{center}
}% end tiny
\caption{Stakeholders en hun rollen}\label{fig:stakeholders}
\end{figure}

{\tiny
\begin{center}
\begin{tabular}{llllll}\hline
{\bf Stakeholder}    & {\bf Verband}   & {\bf Rol}     & {\bf Freq} & {\bf Belang} & {\bf Invloed}\\\hline
Bernard van Gastel   & Onderzoek NoC  & Opdrachtgever & Hoog       & Hoog   & Hoog \\
                     & Onderhoud      & Opdrachtgever & Laag       & Hoog   &  \\
                     & Gebruik        & Gebruiker     & ??         & Hoog   & \\
Freek Verbeek        & Onderzoek NoC  & Begeleider    & Hoog       & Hoog   & Hoog\\
                     & Onderhoud      & Opdrachtgever & Laag       & Hoog   & \\
                     & Onderhoud      & Ontwikkelaar  & Laag       & Hoog   & \\
                     & Gebruik        & Gebruiker     &            & Hoog   & \\
Univ. medewerkers    & Onderzoek alg. & Gebruiker     & Laag       & Middel & \\
                     & Onderzoek NoC  & Gebruiker     & Middel     & Hoog   & \\
                     & Gebruik        &               &            & Middel & \\
Bedrijfsonderzoekers & Onderzoek NoC  & -             & Middel     & Hoog   & \\
                     & Gebruik        & Gebruiker     &            & Middel & \\
                     & Installatie    & Gebruiker     &            & Hoog   & \\
Bedrijfsmedewerkers  & Gebruik        & Gebruiker     & Laag       & Middel & \\
                     & Installatie    & Gebruiker     &            & Hoog   & \\
Team33               & Ontwikkeling   & Ontwikkelaar  & Hoog       & Hoog   & Hoog\\
                     & Afstuderen     & Student       &            & Hoog   & Hoog\\
                     & Samenwerking   & Ontwikkelaar  & Hoog       & Hoog   & Hoog\\
\hline
\end{tabular}
\end{center}
}% end tiny

\subsection{Challanges and goals}\label{sec: challanges goals}

Wat zijn uitdagingen en welke high level requirements vloeien hieruit voort? De uitdagingen en requirements
zijn niet volledig. Tijdens de eerste iteratie werkt het team dit uit.

Hieronder een lijst van uitdagingen (challanges):

\begin{description}
 \item[platformen] Het onderzoeksteam werkt met Mac, ontwikkelteam met ms windows en linux, de opdrachtgever en studenten met ms windows, linux of mac.
		    Dit genereert requirements voor platform onafhankelijkheid van ontwikkelomgeving, taal en toolkits die we gebruiken.
 \item[integratie] Het systeem bestaat uit meerdere modulen, grofweg aangeduid als ontwerp tool (wat ons onderwerp is) en analyse tools (tools die nog
		    in ontwikkeling zijn, maar deels al gebouwd). De uitdaging is een hoge graad van integratie van deze componenten, zo onafhankelijk mogelijk.
		    Dit genereert requirements voor onderhoudbaarheid, component interfaces, applicatie structuur. De analyse tools zijn alle C++ programma's.
		    De interfaces zijn gebaseerd op JSON.
 \item[onderhoud] Over de tijd heen onderhouden vele mensen de software, die elkaar niet spreken. Dit genereert requirements met betrekking tot documentatie,
		    onderhoudbaarheid en toegankelijkheid.
 \item[Uitbreiding] De huidige software is moeilijk uitbreidbaar op de wijze die de opdrachtgever graag wil.
 \item[functie] Het ontwerp tool ondersteunt de primitieven en hun samenwerking zoals gespecificeerd in het xmas paper \cite{chatterjee-kishinevsky:xmas}.
		Het  huidige ontwerp tool: WickedXMas geeft de richting aan implementatie van de functionaliteit en de interface, maar is
		niet beperkend of maatgevend daarvoor.
 \item[rest]	Wat zijn de overige uitdagingen? Welke requirements kan de huidige programmatuur moeilijker aan?
\end{description}

De high level doelstellingen van het project:

\begin{description}
 \item[beschikbaarheid]
 \item[hoge integratie met analyse tools]
 \item[Uitbreidbaar met primitieven]
 \item[Uitbreidbaar met analyse en verificatie tools]
 \item[Design functionaliteit]
 \item[Open punten] Zie einde document.
\end{description}

De doelstellingen leiden tot de high level requirements.

\subsection{Succesfactoren}
Het project is een succes als

\begin{itemize}
 \item het design tool een ontwerp kan maken in de xmas taal en de analyse tools kan aanroepen voor controle.
 \item Gebruikers van het tool op de volgende platforms kan draaien:
 \begin{itemize}
    \item ms Windows vanaf versie 7 64 bits (32 bits via emulatie)
    \item Linux 64 bits en 32 bits
    \item Mac OS X (versie?)
 \end{itemize}
\end{itemize}

Het project is een groot succes als het project een succes is en:

\begin{itemize}
 \item De gebruiker kan validatie en verificatie software aanzetten of uitzetten
 \item De gebruiker kan nieuwe primitieven maken en aankoppelen.
\end{itemize}

Het project is mislukt als het ontwerp tool geen succes is.
