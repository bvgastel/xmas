%%
%% Dit is een onderdeel van abi-instructie.tex
%%
\section{Fase 3a Domeinen \& technieken}
Dit is de eerste deelfase bij de eigenlijke uitvoering van het project.

In fase 2 heeft u globaal vastgesteld wat de opdrachtgever wil; om
aan een oplossing te kunnen gaan werken heeft u requirements nodig
die gedetailleerder en exacter zijn. Een deel van de requirements
worden bepaald vanuit het domein en vanuit de hulpmiddelen die
beschikbaar zijn en van de onderzoekcontext. Daartoe maakt u in
deze fase een studie van het domein en in de volgende fase van
de onderzoekcontext (fase 3b).

Vaak zult u u moeten verdiepen in het domein van de opdracht.
Zelfs bij een bekend domein is het goed om na te gaan of u en de
opdrachtgever hetzelfde verstaan onder verschillende termen. Het is
dus zaak om het domein in kaart te brengen.

Ook in de technieken die u mogelijk zult kunnen gebruiken voor de
oplossing zult u zich moeten verdiepen, om in staat te zijn een
beargumenteerde keuze te maken.

U verdeelt hierbij de werkzaamheden: de teamleden zullen elk een
individueel verdiepend onderzoek doen op verschillende terreinen.

\subsection{Doelstelling}
De domeinanalyse is het onderdeel waarbij de ontwikkelaar van een te maken
softwaresysteem zich verdiept in de omgeving waarin het systeem gebruikt zal
worden.
Onderdelen die in de domeinanalyse beschreven kunnen worden zijn bijvoorbeeld:
\begin{itemize}
    \item belangrijke ontwikkelingen rond één of meer aspecten uit dit domein zoals u
die bijvoorbeeld aan kunt treffen op gespecialiseerde websites, in
wetenschappelijke literatuur of in studieboeken;
    \item algemene en voor de opdracht specifieke kennis van het domein, zoals
belangrijke feiten en (bedrijfs)regels die binnen het domein gelden;
    \item terminologie / jargon binnen het vakgebied;
    \item beschrijving van stakeholders en gebruikers van het systeem;
    \item beschrijvingen van het huidige systeem;
    \item relaties met andere domeinen en organisaties.
\end{itemize}


UML‐diagrammen zoals klassendiagrammen kunnen een belangrijke rol spelen in
domeinbeschrijvingen, zeker wanneer u verwacht dat in het project
objectgeoriënteerde concepten centraal zullen staan. Ook andere diagrammen
kunnen gebruikt worden om het domein duidelijk te maken zoals use-case
diagrammen of sequence diagrammen. Ook bedrijfsregels kunnen een aanvulling
vormen op het domeinmodel en kunnen bv. vastgelegd worden in een natuurlijke
taal of in OCL. Deze diagrammen kunnen aangevuld worden met tekstuele
beschrijvingen maar ook andere beschrijvingsformalismen zijn mogelijk.

Het kan ook zijn dat u een onderliggend mathematisch model uitwerkt dat de basis
vormt voor een algoritme dat u wilt toepassen. De vorm en diepgang dient in
overeenstemming te zijn met de oplossingsrichtingen in dit project.

Met objectgeoriënteerd ontwerpen en de daarbij gebruikte technieken hebt u
kennis kunnen maken in de cursussen Objectgeoriënteerde analyse en ontwerp met
behulp van UML en patterns (T34131) en Software Engineering (T07331).

\subsection{Onderdelen mijlpaalproduct}

Het product van deze mijlpaal kan gezien worden als een wetenschappelijk artikel
``in het klein''. U kunt daarom gebruik maken van de opbouw en schrijfstijl
die beschreven wordt in de mini-module Artikel schrijven. De volgende tabel
kan u helpen een passende presentatievorm voor uw domeinanalyse te vinden.

\begin{center}
\begin{tabular}{|p{7em}|p{23em}|}
\hline
{\bf deelproduct} & {\bf omschrijving}
\\\hline
titelblad & titel, auteur en datum en versie
\\\hline
verantwoording \& vraagstelling &
    \par $\bullet$ kernpunten van het onderzoek
    \par $\bullet$ motivatie \& achtergrond
    \par $\bullet$ vraagstelling
    \par $\bullet$ gebruikte (onderzoek)methoden
    \par $\bullet$ routekaart verslag
\\\hline
context &
    \par $\bullet$ plaats onderzoek in de theorie (literatuur, studiemateriaal)
    \par $\bullet$ plaats onderzoek in de praktijk
\\\hline
bevindingen &
    \par $\bullet$ resultaten
    \par $\bullet$ relevantie voor de vraagstelling
\\\hline
conclusies &
    \par $\bullet$ conclusies in relatie met de achtergronden van de vraagstelling
\\\hline
begrippenlijst &
    \par $\bullet$ belangrijke begrippen die niet tot de standaardkennis van de doelgroep
behoren
\\\hline
literatuurlijst &
    \par $\bullet$ lijst met geraadpleegde literatuur en cursusteksten
\\\hline
bijlagen &
    \par $\bullet$ eventueel aanvullende documenten
\\\hline
\end{tabular}
\end{center}

U hoeft dus niet het proces te beschrijven (de weg naar het eindresultaat).
Vooral het inhoudelijk resultaat dient goed onderbouwd te worden.
Als doelgroep kunt u de andere teamleden nemen. Schrijf uw bevindingen zo op dat
uw teamleden profiteren van de kennis die u heeft opgedaan, zonder dat ze zelf
onderzoek moeten doen.
Elk teamlid maakt een individueel verslag van zijn/haar bevindingen. De omvang
is plusminus  5 pagina's. U neemt de definitieve versie op in het individuele
deel van uw scriptieverslag. De individuele bevindingen samen dekken een kleiner
of groter deel af van de volledige domeinbeschrijving die indien nodig verder
aangevuld wordt in de fase ontwerp en implementatie.

\subsection{Rol begeleider}

De begeleider kan aan het begin van deze fase individuele afspraken met u maken
over vaardigheden waaraan u mogelijk extra aandacht dient te besteden bij het
uitvoeren van de domeinanalyse. Dit kan bv. leiden tot een specifieke opdracht
één of meer mini-modules te bestuderen.
U stuurt uw verslag op naar de begeleider, en u ontvangt commentaar.
Aan de hand van dat commentaar stelt u een definitieve versie op, die u naar de
begeleider stuurt. De begeleider beoordeelt die definitieve versie.

\subsection{Fase 3a Beoordelingscriteria Domeinen en Technieken}
\vspace{1em}
\par{\small\sf
\begin{center}
\begin{tabular}{|p{7em}|p{25em}|}
\hline
\multicolumn{2}{|c|}{\emph{Literatuur zoeken}}\\\hline
{\bf eisen} & {\bf Criteria}\\\hline
Bronnen vinden & Voor het overzicht van het domein en van de mogelijke oplossingstechnieken zijn
		    relevante bronnen gebruikt.
\\\hline
Bronvermelding & Het is in het overzicht duidelijk welke informatie uit welke bron afkomstig is.
\\\hline
\multicolumn{2}{|c|}{\emph{Abstraheren}}\\\hline
{\bf eisen} & {\bf Criteria}\\\hline
Synthese maken & De informatie uit verschillende bronnen is samengebracht tot
		een overzichtelijk geheel.
\\\hline
Relevantie bepalen & Uit de bronnen die u heeft geraadpleegd is gedestilleerd wat relevant is voor
		    het project.
\\\hline
Begrippen & Het gebruik van begrippen is meestal niet consistent tussen verschillende
bronnen. U bent in staat om binnen het project begrippen te definiëren en
consistent te gebruiken.
\\\hline
\multicolumn{2}{|c|}{\emph{Documenteren}}\\\hline
{\bf eisen} & {\bf Criteria}\\\hline
Schrijven voor een publiek & Het overzicht van het domein en de technieken is geschreven met de
		medeteamgenoten en andere ontwikkelaars (bijvoorbeeld van volgende projectteams)
		als doelgroep.
\\\hline
Opbouw & De opbouw van teksten is zo dat de lezer het overzicht houdt.
\\\hline
\multicolumn{2}{|c|}{\emph{Probleemstelling formuleren}}\\\hline
{\bf eisen} & {\bf Criteria}\\\hline
Helder doel & Bij het zoeken van informatie is duidelijk voor welk doel u de informatie zoekt.
\\\hline
\end{tabular}
\end{center}
}% small sf