%%
%% Dit is een onderdeel van abi-instructie.tex
%%
\section{Fase 3d Documentatie}
\subsection{Inleiding}
De documentatie is bedoeld om uw producten te beschrijven voor
de begeleider en de opdrachtgever.

Het project dient technisch voldoende gedocumenteerd te zijn
zodat ontwikkelaars die verder willen gaan met het product
dat u opgeleverd hebt voldoende houvast hebben.

In overleg met de opdrachtgever bepaalt u ook de eisen aan de gebruikersdocumentatie.

\subsection{Doelstelling}
U beschrijft de producten voor de begeleider en de opdrachtgever. U dient zowel
met de opdrachtgever als met de begeleider afspraken te maken over de op te
leveren documentatie voor uw specifiek project. U documenteert het project
zodanig dat ontwikkelaars op basis van uw documentatie de producten verder
kunnen ontwikkelen. Afhankelijk van de wensen van de opdrachtgever maakt u
ook de documentatie voor de eindgebruiker. Een rol kan daarbij bv. spelen
of u een operationeel systeem dient te ontwikkelen of een prototype.

\subsection{Onderdelen mijlpaalproduct}
Naast de documentatie op detailniveau levert u ook plusminus 15 pagina's
gezamenlijke tekst aan met een beschrijving van het eindproduct, het ontwerp en
de implementatie ervan. U laat daarin ook zien op welke manier uw eindproduct
een oplossing geeft voor de vraag van de opdrachtgever.
Deze 15 pagina's neemt u op in uw scriptieverslag.

\subsection{Rol begeleider}
U stelt uw documentatie beschikbaar aan de begeleider, en u ontvangt commentaar.
Aan de hand van dat commentaar stelt u een definitieve versie op, die u naar de
begeleider stuurt. De begeleider beoordeelt die definitieve versie.
U neemt de definitieve versie van het mijlpaaldocument op in uw scriptieverslag.

\tablecaption{Beoordelingscriteria}
\tablehead{\hline {\bf eisen} & {\bf Criteria}\\}
\tabletail{\hline \multicolumn{2}{r}{Continues on next page}\\}
\tablelasttail{\hline}
\par{\tiny
\begin{center}
\begin{supertabular}{|l|p{30em}|}
\hline
\multicolumn{2}{|c|}{\emph{Klantgerichtheid}}\\\hline
Afstemming met de opdrachtgever & De contacten met de opdrachtgever over de
documentatie van het project verlopen zoals vastgelegd in het Plan van aanpak.
\\\hline
Overdracht & De overdracht van het resultaat van het project aan de opdrachtgever
gebeurt op zo’n manier dat de opdrachtgever de resultaten daadwerkelijk kan gaan
gebruiken, binnen de kaders van de gemaakte afspraken (op niveau van een prototype
of een implementatie).
\\\hline
\multicolumn{2}{|c|}{\emph{Schrijven}}\\\hline
Taal & De ingeleverde documenten zijn grammaticaal correct.
\\\hline
Precisie & De gebruikte begrippen worden gedefinieerd of uitgelegd en
worden consequent gebruikt.
\\\hline
Structuur & De indeling van de documenten is helder. Er is samenhang tussen de tekstdelen.
\\\hline
\end{supertabular}
\end{center}
}% tiny