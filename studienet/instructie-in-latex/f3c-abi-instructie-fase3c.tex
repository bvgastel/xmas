%%
%% Dit is een onderdeel van abi-instructie.tex
%%
\section{Fase 3c Ontwerp en Implementatie}
\subsection{Doelstelling}
De uitvoering van het project resulteert in een softwareoplossing
voor de opdrachtgever.

\subsection{Projectfasering}Die software ontwerpt en implementeert u volgens de
overeengekomen fasering, planning en taakverdeling. Denk er daarbij aan dat
testen een belangrijk onderdeel van de implementatie is. De manier waarop u
te werk gaat, bent u overeengekomen tijdens de fase van
taakverdeling en planning. De methode die u hanteert, heeft ook consequenties
voor de documenten en andere artefacten die u zult produceren. Qua doorlooptijd
zal deze fase de meeste tijd in beslag nemen. Goede
communicatie en afstemming is cruciaal om het projectplan tijdig en in
gezamenlijkheid te realiseren. Binnen uw team is een wekelijks overleg aan
te bevelen.
\subsection{Onderdelen mijlpaalproduct}
    U levert zowel aan de opdrachtgever als aan de begeleider een portfolio met:

    \begin{itemize}
	\item het eindproduct plus handleidingen
	\item de diverse ontwerpdocumenten (diagrammen, tekst)
	\item de broncode
	\item de testdocumentatie
	\item gebruikersdocumentatie en documentatie voor ontwikkelaars.
    \end{itemize}

\paragraph{OU ondersteuning} Tijdens de ontwikkeling kunt u gebruik maken van
een testserver en van een subversion repository voor het delen van software-code
en documenten (zie ook bij hulpmiddelen).  Op de subversion server worden de
documenten van een team tenminste 3 jaar bewaard. De virtuele testserver
wordt in principe binnen een tweetal maanden na het beëindigen van het project
opgeruimd tenzij met een team andere afspraken gemaakt worden.

\subsection{Contacten met de opdrachtgever}
    Onderhoud contact met de opdrachtgever en maak heldere afspraken over de
tijdstippen waarop de (deel) producten getest worden en de documentatie
beoordeeld wordt.

\paragraph{Testresultaten}
Spreek ook met de opdrachtgever af op welke manier getest wordt en hoe
de resultaten van de tests verwerkt worden in de volgende ontwikkelfase
wanneer u bij de ontwikkeling van het informatiesysteem in een iteratieve
aanpak hanteert.

\subsection{Rol begeleider}
De begeleider kan tijdens deze fase individuele afspraken met u maken over
vaardigheden waaraan u mogelijk extra aandacht dient te besteden bij het
ontwikkelen van het informatiesysteem. Dit kan bv. leiden tot een specifieke
opdracht één of meer mini-modules te bestuderen.

Net als in de overige fasen houdt u de begeleider tenminste maandelijks op
de hoogte van uw voortgang via een maandverslag (u kunt gebruik maken van een
sjabloon). Ook is aan te bevelen om volgens een vast patroon maandelijks te
overleggen met de begeleider. Verder kunt u de begeleider altijd aanspreken
wanneer er vragen of onduidelijkheden zijn.

De begeleider  geeft feedback op uw werkwijze tijdens het project en
beoordeelt aan het eind van deze fase de uitvoering van het project en de
opgeleverde producten.

De beoordeling van het ontwerp en implementatie zijn in eerste instantie de
verantwoordelijkheid van de begeleider. De opdrachtgever kan hier ook een rol in
spelen maar zal vooral kijken naar de functionaliteit en beoordelen of het
product overeenkomt met wat afgesproken is.

\vspace{1em}
\tablecaption{Fase 3c Beoordelingscriteria Ontwerp en Implementatie}
\tablehead{\hline {\bf eisen} & {\bf Criteria}\\}
\tabletail{\hline \multicolumn{2}{r}{Verder op volgende pagina (fase 3c beoordelingscriteria)}\\}
\tablelasttail{\hline \multicolumn{2}{r}{Einde beoordelingscriteria fase 3c}}
\par{\small\sf
\begin{center}
\begin{supertabular}{|p{7em}|p{23em}|}
\hline
\multicolumn{2}{|c|}{\emph{Contextge\"{o}rienteerd beslissen}}\\\hline
Ont\-werp\-be\-slis\-sing\-en & De beslissingen die zijn genomen om tot een ontwerp te komen zijn valide
    beargumenteerd met behulp van wat naar voren is gekomen tijdens het opstellen
    van de requirements, het overzicht van het domein en technieken, en het consult.
    De voors en tegens van beslissingen zijn expliciet gemaakt.
\\\hline
\multicolumn{2}{|c|}{\emph{Ontwerpen}}\\\hline
Vanuit probleem naar oplossing & Op basis van de beschrijving en analyse van het probleem zijn oplossingen
				eenduidig en helder vastgelegd in de vorm van een ontwerp.
\\\hline
Ontwerp implementatie & Ontwerpbeslissingen die tijdens de implementatie zijn genomen zijn expliciet
			gemaakt.
\\\hline
\multicolumn{2}{|c|}{\emph{Documenteren}}\\\hline
Ontwerp & Het ontwerp is zodanig gedocumenteerd dat medeteamleden, de begeleider en
	    toekomstige ontwikkelaars voldoende inzicht kunnen krijgen in het systeem om de
	    uiteindelijke implementatie te kunnen doorgronden.
\\\hline
Implementatie & De code is zo gedocumenteerd dat anderen er verder aan kunnen werken.
\\\hline
\multicolumn{2}{|c|}{\emph{Modelleren}}\\\hline
Gebruik van diagramtechnieken & De gebruikte diagramtechnieken (zoals UML) zijn op de juiste manier
	    gebruikt. Duidelijk is wat er gemodelleerd is (huidige situatie, toekomstige situatie,
	    te bouwen systeem, en zo voort).
\\\hline
\multicolumn{2}{|c|}{\emph{Gebruiken, toepassen, beheren}}\\\hline
Tools & Tijdens het project is op een professionele manier gebruik gemaakt van
	ontwerp- en ontwikkeltools zoals een umltekentool,javadoc, een IDE en
	dergelijke. Ook is gebruik gemaakt van samenwerkingssoftware voor het
	uitwisselen van documenten zoals een  subversion server.
\\\hline
\multicolumn{2}{|c|}{\emph{Implementeren}}\\\hline
Werkend systeem & Het resultaat van het project is een werkend systeem of prototype.
\\\hline
Testen & Van te voren is op eenduidig vastgelegd hoe het systeem getest gaat worden
	    in de vorm van een testplan. Die testen zijn ook daadwerkelijk uitgevoerd en de
	    resultaten zijn beschreven.
\\\hline
Acceptatietest & Er is door de opdrachtgever een acceptatietest uitgevoerd.
		De resultaten hiervan zijn beschreven.
\\\hline
\multicolumn{2}{|c|}{\emph{Programmeren}}\\\hline
    Gebruik bestaande tools, libraries & Waar nodig zijn bestaande tools of libraries bestudeerd en gebruikt.
\\\hline
\multicolumn{2}{|c|}{\emph{Plan realiseren}}\\\hline
Inrichten project & Er is een samenwerkingsomgeving en werkwijze gebruikt die helpt de
projectdocumenten en resources gestructureerd op te slaan en toegankelijk te
maken.
\\\hline
Aanpassen planning & Tijdens de duur van het project is opgemerkt en gedocumenteerd indien de
uitvoering (bv. tijdens een iteratie) niet volgens planning verliep en is de
planning in overleg aangepast. Wijzigingen in de wensen van de opdrachtgever zijn
beargumenteerd (wel of niet) opgenomen in de planning
\\\hline
\multicolumn{2}{|c|}{\emph{Samenwerken}}\\\hline
Communicatie & Er is een gestructureerde manier gevonden om te communiceren via
    gezamenlijke documenten, bijeenkomsten en dergelijke.
    Dit geldt voor de communicatie binnen het team, met de
    opdrachtgever, de begeleider, de examinator en de consulenten.
\\\hline
Taakverdeling & De taakverdeling die van te voren is opgesteld, is aangepast wanneer blijkt
    dat de zwaarte te veel verschilt of nieuw inzicht is ontstaan mbt inhoud van het
    project.
\\\hline
Afstemmen & Waar nodig zijn duidelijke afspraken gemaakt over de afstemming tussen
    verschillende taken, zoals wanneer er aan verschillende onderdelen van het
    systeem wordt geprogrammeerd. Ook wordt voortdurende afgestemd tussen de
    verantwoordelijken voor ontwerp, programmacode en documentatie.
\\\hline
Ver\-ant\-woor\-de\-lijk\-he\-den & Het is duidelijk bij wie de ver\-ant\-woor\-de\-lijk\-he\-den voor verschillende
    onderdelen en aspecten van het project liggen.
\\\hline
\end{supertabular}
\end{center}
}% small sf